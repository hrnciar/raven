\section{Models}
\label{sec:models}

%%%%%%%%%%%%%%%%%%%%%%%%%%%%%%%%%%%%%%%%%%%%%%%%%%%%%%%%%%%%%%%%%%%%%%%%%%%%%%%%
% If you are confused by the input of this document, please make sure you see
% these defined commands first. There is no point writing the same thing over
% and over and over and over and over again, so these will help us reduce typos,
% by just editing a template sentence or paragraph.

% These should be organized according to whic section they are most often used
% e.g. kernelDescription should go under a heading for SVMs.
% This will take a bit of work, but if later things are added it will make
% finding the appropriate parameters easier.

\renewcommand{\nameDescription}
{
  \xmlAttr{name},
  \xmlDesc{required string attribute}, user-defined identifier of this model.
  \nb As with other objects, this identifier can be used to reference this
  specific entity from other input blocks in the XML.
}


\newcommand{\assemblerAttrDescription}[1]
{
    This XML node needs to contain the attributes:
    \begin{itemize}
      \item \xmlAttr{class}, \xmlDesc{required string attribute}, the main
        ``class'' of the #1.
        %
      \item \xmlAttr{type},  \xmlDesc{required string attribute}, the sub-type of the #1.
        %
    \end{itemize}
}


\newcommand{\aliasSystemDescription}[1]
{
  \xmlNode{alias} \xmlDesc{string, optional field} specifies alias for
  any variable of interest in the input or output space for the #1.
  %
  These aliases can be used anywhere in the RAVEN input to refer to the #1
  variables.
  %
  In the body of this node the user specifies the name of the variable that the model is going to use
  (during its execution).
  %
  The actual alias, usable throughout the RAVEN input, is instead defined in the
  \xmlAttr{variable} attribute of this tag.
  \\The user can specify aliases for both the input and the output space. As sanity check, RAVEN
  requires an additional required attribute \xmlAttr{type}. This attribute can be either ``input'' or ``output''.
  %
  \nb The user can specify as many aliases as needed.
  %
  \default{None}
}

\renewcommand{\specBlock}[2]{
  The specifications of this model must be defined within #1 \xmlNode{#2} XML
  block.
}
%
\renewcommand{\subnodeIntro}
{
  This model can be initialized with the following child:
}
\renewcommand{\subnodesIntro}
{
  This model can be initialized with the following children:
}

\newcommand{\ppType}[2]
{
  In order to use the \textit{#1} PP, the user needs to set the
  \xmlAttr{subType} of a \xmlNode{PostProcessor} node:

  \xmlNode{PostProcessor \xmlAttr{name}=\xmlString{ppName} \xmlAttr{subType}=\xmlString{#2}/}.

   Several sub-nodes are available:
}

\newcommand{\skltype}[2]
{
  In order to use the \textit{#1}, the user needs to set the
  sub-node:

  \xmlNode{SKLtype}\texttt{#2}\xmlNode{/SKLtype}.

  In addition to this XML node, several others are available:
}

\newcommand{\nIterDescriptionA}[1]
{
  \xmlNode{n\_iter}, \xmlDesc{integer, optional field}, is the maximum number of
  iterations.
  %
  \ifthenelse{\equal{#1}{}}{}{\default{#1}}
}

\newcommand{\nIterDescriptionB}[1]
{
  \xmlNode{n\_iter}, \xmlDesc{int, optional field}, specifies the number of
  passes over the training data (aka epochs).
  %
  \ifthenelse{\equal{#1}{}}{}{\default{#1}}
}

\newcommand{\nIterNoChangeDescriptionA}[1]
{
  \xmlNode{n\_iter\_no\_change}, \xmlDesc{integer, optional field}, number of iterations with no improvement to wait before early stopping.
  %
  \ifthenelse{\equal{#1}{}}{}{\default{#1}}
}

\newcommand{\tolDescriptionA}[1]
{
  \xmlNode{tol}, \xmlDesc{float, optional field}, stop the algorithm if the convergence error felt below the tolerance specified here.
  %
  \ifthenelse{\equal{#1}{}}{}{\default{#1}}
}

\newcommand{\tolDescriptionB}[1]
{
  \xmlNode{tol}, \xmlDesc{float, optional field}, specifies the tolerance for
  the optimization: if the updates are smaller than tol, the optimization code
  checks the dual gap for optimality and continues until it is smaller than tol.
  %
  \ifthenelse{\equal{#1}{}}{}{\default{#1}}
}

\newcommand{\tolDescriptionC}[1]
{
  \xmlNode{tol}, \xmlDesc{float, optional field}, specifies the tolerance for
  stopping criteria.
  %
  \ifthenelse{\equal{#1}{}}{}{\default{#1}}
}

\newcommand{\fitInterceptDescription}[1]
{
  \xmlNode{fit\_intercept}, \xmlDesc{boolean, optional field}, determines
  whether to calculate the intercept for this model.
  %
  If set to False, no intercept will be used in the calculations (e.g. data is
  expected to be already centered).
  %
  \ifthenelse{\equal{#1}{}}{}{\default{#1}}
}

\newcommand{\normalizeDescription}[1]
{
  \xmlNode{normalize}, \xmlDesc{boolean, optional field}, if True, the
  regressors X will be normalized before regression.
  %
  \ifthenelse{\equal{#1}{}}{}{\default{#1}}
}

\newcommand{\verDescriptionA}[1]
{
  \xmlNode{verbose}, \xmlDesc{boolean, optional field}, use verbose mode
  when fitting the model.
  %
  \ifthenelse{\equal{#1}{}}{}{\default{#1}}
}

\newcommand{\verDescriptionB}[1]
{
  \xmlNode{verbose}, \xmlDesc{boolean or integer, optional field}, use verbose
  mode when fitting the model.
  %
  \ifthenelse{\equal{#1}{}}{}{\default{#1}}
}

\newcommand{\maxIterDescription}[1]
{
\xmlNode{max\_iter}, \xmlDesc{integer, optional field}, specifies the
  maximum number of iterations.
  %
  \ifthenelse{\equal{#1}{}}{}{\default{#1}}
}

\newcommand{\warmStartDescription}[1]
{
  \xmlNode{warm\_start}, \xmlDesc{boolean, optional field}, when set to
  True, the model will reuse the solution of the previous call to fit as
  initialization, otherwise, it will just erase the previous solution.
  %
  \ifthenelse{\equal{#1}{}}{}{\default{#1}}
}

\newcommand{\positiveDescription}[1]
{
  \xmlNode{positive}, \xmlDesc{boolean, optional field}, when set to True, this
  forces the coefficients to be positive.
  %
  \ifthenelse{\equal{#1}{}}{}{\default{#1}}
}

\newcommand{\precomputeDescription}[1]
{
  \xmlNode{precompute}, \xmlDesc{boolean or string, optional field}, determines
  whether to use a precomputed Gram matrix to speed up calculations.
  %
  If set to `auto,' RAVEN will decide.
  %
  The Gram matrix can also be passed as an argument.
  %
  Available options are [True | False | `auto' | array-like].
  %
  \ifthenelse{\equal{#1}{}}{}{\default{#1}}
}

\newcommand{\nAlphasDescription}[1]
{
  \xmlNode{max\_n\_alphas}, \xmlDesc{integer, optional field}, specifies the
  maximum number of points on the path used to compute the residuals in the
  cross-validation.
  %
  \ifthenelse{\equal{#1}{}}{}{\default{#1}}
}

\newcommand{\shuffleDescription}[1]
{
  \xmlNode{shuffle}, \xmlDesc{boolean, optional field}, specifies whether
  or not the training data should be shuffled after each epoch.
  %
  \ifthenelse{\equal{#1}{}}{}{\default{#1}}
}

\newcommand{\randomStateDescription}[1]
{
  \xmlNode{random\_state}, \xmlDesc{int seed, RandomState instance, or None},
  sets the seed of the pseudo random number generator to use when shuffling the
  data.
  %
  \ifthenelse{\equal{#1}{}}{}{\default{#1}}
}

\newcommand{\solverDescription}%[1]
{
  \xmlNode{solver}, \xmlDesc{\{`auto', `svd', `cholesky', `lsqr',
  `sparse\_cg'\}}, specifies the solver to use in the computational routines:
  \begin{itemize}
    \item `auto' chooses the solver automatically based on the type of data.
    %
    \item `svd' uses a singular value decomposition of X to compute the ridge
    coefficients.
    %
    More stable for singular matrices than `cholesky.'
    %
    \item `cholesky' uses the standard scipy.linalg.solve function to obtain a
    closed-form solution.
    %
    \item `sparse\_cg' uses the conjugate gradient solver as found in
    scipy.sparse.linalg.cg.
    %
    As an iterative algorithm, this solver is more appropriate than
    `cholesky' for large-scale data (possibility to set tol and max\_iter).
    %
    \item `lsqr' uses the dedicated regularized least-squares routine
    scipy.sparse.linalg.lsqr.
    %
    It is the fatest but may not be available in old scipy versions.
    %
    It also uses an iterative procedure.
    %
  \end{itemize}
  All three solvers support both dense and sparse data.
  %
  %\ifthenelse{\equal{#1}{}}{}{\default{#1}}
}

%%%%%%%%%%%%%%%%%%%%%%%%% Common Regression Parameters %%%%%%%%%%%%%%%%%%%%%%%%%
%%%%%%%%%%%%%%%%%%%%%%% End Common Regression Parameters %%%%%%%%%%%%%%%%%%%%%%%

%%%%%%%%%%%%%%%%%%%%%%%%%%%% Common SVM Parameters %%%%%%%%%%%%%%%%%%%%%%%%%%%%%
\newcommand{\CSVMDescription}[1]
{
  \xmlNode{C}, \xmlDesc{float, optional field}, sets the penalty parameter C
  of the error term.
  %
  \ifthenelse{\equal{#1}{}}{}{\default{#1}}
  %
}

\newcommand{\kernelDescription}[1]
{
  \xmlNode{kernel}, \xmlDesc{string, optional}, specifies the kernel type
  to be used in the algorithm.
  %
  It must be one of:
  \begin{itemize}
    \item `linear'
    \item `poly'
    \item `rbf'
    \item `sigmoid'
    \item `precomputed'
    \item a callable object
  \end{itemize}
  %
  If a callable is given it is used to pre-compute the kernel matrix.
  %
  \ifthenelse{\equal{#1}{}}{}{\default{#1}}
  %
}

\newcommand{\degreeDescription}[1]
{
  \xmlNode{degree}, \xmlDesc{int, optional field}, determines the degree
  of the polynomial kernel function (`poly').
  %
  Ignored by all other kernels.
  %
  \ifthenelse{\equal{#1}{}}{}{\default{#1}}
}

\newcommand{\gammaDescription}[1]
{
  \xmlNode{gamma}, \xmlDesc{float, optional field}, sets the
  kernel coefficient for the kernels `rbf,' `poly,' and `sigmoid.'
  %
  If gamma is `auto' then 1/n\_features will be used instead.
  %
  \ifthenelse{\equal{#1}{}}{}{\default{#1}}
}

\newcommand{\coefZeroDescription}[1]
{
  \xmlNode{coef0}, \xmlDesc{float, optional field}, is an independent term in
  kernel function.
  %
  It is only significant in `poly' and `sigmoid.'
  %
  \ifthenelse{\equal{#1}{}}{}{\default{#1}}
}

\newcommand{\probabilityDescription}[1]
{
  \xmlNode{probability}, \xmlDesc{boolean, optional field}, determines whether
  or not to enable probability estimates.
  %
  This must be enabled prior to calling fit, and will slow down that method.
  %
  \ifthenelse{\equal{#1}{}}{}{\default{#1}}
}

\newcommand{\shrinkingDescription}[1]
{
  \xmlNode{shrinking}, \xmlDesc{boolean, optional field}, determines whether or
  not to use the shrinking heuristic.
  %
  \ifthenelse{\equal{#1}{}}{}{\default{#1}}
}

\newcommand{\cacheSizeDescription}[1]
{
  \xmlNode{cache\_size}, \xmlDesc{float, optional field}, specifies the
  size of the kernel cache (in MB).
  %
  \ifthenelse{\equal{#1}{}}{}{\default{#1}}
}

\newcommand{\classWeightDescription}[1]
{
  \xmlNode{class\_weight}, \xmlDesc{{dict, `auto'}, optional}, sets the
  parameter C of class i to class\_weight[i]*C for SVC.
  %
  If not given, all classes are assumed to have weight one.
  %
  The `auto' mode uses the values of y to automatically adjust weights inversely
  proportional to class frequencies.
  %
  \ifthenelse{\equal{#1}{}}{}{\default{#1}}
}

\newcommand{\tolSVMDescription}[1]
{
  \tolDescriptionC{#1}
}

\newcommand{\verSVMDescription}[1]
{
  \verDescriptionA{False}
  %
  \nb This setting takes advantage of a per-process runtime setting in libsvm
  that, if enabled, may not work properly in a multithreaded context.
}

\newcommand{\randomStateSVMDescription}[1]
{
  \xmlNode{random\_state}, \xmlDesc{int seed, RandomState instance, or
  None}, represents the seed of the pseudo random number generator to use when
  shuffling the data for probability estimation.
  %
  \ifthenelse{\equal{#1}{}}{}{\default{#1}}
}

%%%%%%%%%%%%%%%%%%%%%%%%%% End Common SVM Parameters %%%%%%%%%%%%%%%%%%%%%%%%%%%

%%%%%%%%%%%%%%%%%%%%%%%%% Common Multi-Class Parameters %%%%%%%%%%%%%%%%%%%%%%%%
\newcommand{\estimatorDescription}[1]
{
  \xmlNode{estimator}, \xmlDesc{boolean, required field},
  %
  An estimator object implementing fit and one of decision\_function or
  predict\_proba.
  %
  This XML node needs to contain the following attribute:
  \vspace{-5mm}
  \begin{itemize}
    \itemsep0em
    \item \xmlAttr{estimatorType}, \xmlDesc{required string attribute}, this
    attribute is another reduced order mode type that needs to be used for the
    construction of the multi-class algorithms.
    %
    Each sub-sequential node depends on the chosen ROM.
  \end{itemize}
  %
  \ifthenelse{\equal{#1}{}}{}{\default{#1}}
}
%%%%%%%%%%%%%%%%%%%%%%% End Common Multi-Class Parameters %%%%%%%%%%%%%%%%%%%%%%

%%%%%%%%%%%%%%%%%%%%%%%%%% Common Bayesian Parameters %%%%%%%%%%%%%%%%%%%%%%%%%%
\newcommand{\alphaBayesDescription}[1]
{
  \xmlNode{alpha}, \xmlDesc{float, optional field}, specifies an additive
  (Laplace/Lidstone) smoothing parameter (0 for no smoothing).
  %
  \ifthenelse{\equal{#1}{}}{}{\default{#1}}
}

\newcommand{\fitPriorDescription}[1]
{
  \xmlNode{fit\_prior}, \xmlDesc{boolean, required field}, determines whether to
  learn class prior probabilities or not.
  %
  If false, a uniform prior will be used.
  %
  \ifthenelse{\equal{#1}{}}{}{\default{#1}}
}

\newcommand{\classPriorDescription}[1]
{
  \xmlNode{class\_prior}, \xmlDesc{array-like float (n\_classes), optional
  field}, specifies prior probabilities of the classes.
  %
  If specified, the priors are not adjusted according to the data.
  %
  \ifthenelse{\equal{#1}{}}{}{\default{#1}}
}

%%%%%%%%%%%%%%%%%%%%%%%% End Common Bayesian Parameters %%%%%%%%%%%%%%%%%%%%%%%%

%%%%%%%%%%%%%%%%%%%%%%%%%% Common Neighbor Parameters %%%%%%%%%%%%%%%%%%%%%%%%%%
\newcommand{\nNeighborsDescription}[1]
{
  \xmlNode{n\_neighbors}, \xmlDesc{integer, optional field}, specifies the
  number of neighbors to use by default for `k\_neighbors' queries.
  %
  \ifthenelse{\equal{#1}{}}{}{\default{#1}}
}

\newcommand{\radiusDescription}[1]
{
  \xmlNode{radius}, \xmlDesc{float, optional field}, specifies the range of
  parameter space to use by default for `radius\_neighbors' queries.
  %
  \ifthenelse{\equal{#1}{}}{}{\default{#1}}
}

\newcommand{\weightsDescription}[1]
{
  \xmlNode{weights}, \xmlDesc{string, optional field}, specifies the weight
  function used in prediction.
  %
  Possible values:
  \begin{itemize}
    \item \textit{uniform} : uniform weights.
    %
    All points in each neighborhood are weighted equally;
    \item \textit{distance} : weight points by the inverse of their distance.
    %
    In this case, closer neighbors of a query point will have a greater
    influence than neighbors which are further away.
    %
  \end{itemize}
  %
  \ifthenelse{\equal{#1}{}}{}{\default{#1}}
}

\newcommand{\metricDescription}[1]
{
  \xmlNode{metric}, \xmlDesc{string, optional field}, sets the distance metric
  to use for the tree.
  %
  The Minkowski metric with p=2 is equivalent to the standard Euclidean metric.
  %
  \ifthenelse{\equal{#1}{}}{}{\default{#1}}
}

\newcommand{\algorithmDescription}[1]
{
  \xmlNode{algorithm}, \xmlDesc{string, optional field}, specifies the algorithm
  used to compute the nearest neighbors:
  \begin{itemize}
    \item \textit{ball\_tree} will use BallTree.
    \item \textit{kd\_tree} will use KDtree.
    \item \textit{brute} will use a brute-force search.
    \item \textit{auto} will attempt to decide the most appropriate algorithm
    based on the values passed to fit method.
    %
  \end{itemize}
  \nb Fitting on sparse input will override the setting of this parameter, using
  brute force.
  %
  \ifthenelse{\equal{#1}{}}{}{\default{#1}}
}

\newcommand{\leafSizeDescription}[1]
{
  \xmlNode{leaf\_size}, \xmlDesc{integer, optional field}, sets the leaf size
  passed to the BallTree or KDTree.
  %
  This can affect the speed of the construction and query, as well as the memory
  required to store the tree.
  %
  The optimal value depends on the nature of the problem.
  %
  \ifthenelse{\equal{#1}{}}{}{\default{#1}}
}

\newcommand{\pDescription}[1]
{
  \xmlNode{p}, \xmlDesc{integer, optional field}, is a parameter for the
  Minkowski metric.
  %
  When $p = 1$, this is equivalent to using manhattan distance (L1), and
  euclidean distance (L2) for $p = 2$.
  %
  For arbitrary p, minkowski distance (L\_p) is used.
  %
  \ifthenelse{\equal{#1}{}}{}{\default{#1}}
}

\newcommand{\outlierLabelDescription}[1]
{
  \xmlNode{outlier\_label}, \xmlDesc{integer, optional field}, is a label, which
  is given for outlier samples (samples with no neighbors on a given radius).
  %
  If set to None, ValueError is raised, when an outlier is detected.
  %
  \ifthenelse{\equal{#1}{}}{}{\default{#1}}
}

%%%%%%%%%%%%%%%%%%%%%%%% End Common Neighbor Parameters %%%%%%%%%%%%%%%%%%%%%%%%

%%%%%%%%%%%%%%%%%%%%%%%%%%%% Common Tree Parameters %%%%%%%%%%%%%%%%%%%%%%%%%%%%
\newcommand{\criterionDescription}[1]
{
  \xmlNode{criterion}, \xmlDesc{string, optional field}, specifies the function
  used to measure the quality of a split.
  %
  Supported criteria are ``gini'' for the Gini impurity and ``entropy'' for the
  information gain.
  %
  \ifthenelse{\equal{#1}{}}{}{\default{#1}}
}

\newcommand{\criterionDescriptionDT}[1]
{
  \xmlNode{criterion}, \xmlDesc{string, optional field}, specifies the function
  used to measure the quality of a split.
  %
  The only supported criterion is ``mse'' for mean squared error.
  %
  \ifthenelse{\equal{#1}{}}{}{\default{#1}}
}

\newcommand{\splitterDescription}[1]
{
  \xmlNode{splitter}, \xmlDesc{string, optional field}, specifies the strategy
  used to choose the split at each node.
  %
  Supported strategies are ``best'' to choose the best split and ``random'' to
  choose the best random split.
  %
  \ifthenelse{\equal{#1}{}}{}{\default{#1}}
}

\newcommand{\maxFeaturesDescription}[1]
{
  \xmlNode{max\_features}, \xmlDesc{int, float or string, optional field}, sets
  the number of features to consider when looking for the best split:
  \begin{itemize}
    \item If int, then consider max\_features features at each split.
    %
    \item If float, then max\_features is a percentage and int(max\_features *
    n\_features) features are considered at each split.
    %
    \item If ``auto,'' then max\_features=sqrt(n\_features).
    \item If ``sqrt,'' then max\_features=sqrt(n\_features).
    \item If ``log2,'' then max\_features=log2(n\_features).
    \item If None, then max\_features=n\_features.
    %
  \end{itemize}
  \nb The search for a split does not stop until at least one valid partition of
  the node samples is found, even if it requires to effectively inspect more
  than max\_features features.
  %
  \ifthenelse{\equal{#1}{}}{}{\default{#1}}
}

\newcommand{\maxDepthDescription}[1]
{
  \xmlNode{max\_depth}, \xmlDesc{integer, optional field}, determines the
  maximum depth of the tree.
  %
  If None, then nodes are expanded until all leaves are pure or until all leaves
  contain less than min\_samples\_split samples.
  %
  Ignored if max\_samples\_leaf is not None.
  %
  \ifthenelse{\equal{#1}{}}{}{\default{#1}}
}

\newcommand{\minSamplesSplitDescription}[1]
{
  \xmlNode{min\_samples\_split}, \xmlDesc{integer, optional field}, sets the
  minimum number of samples required to split an internal node.
  %
  \ifthenelse{\equal{#1}{}}{}{\default{#1}}
}

\newcommand{\minSamplesLeafDescription}[1]
{
  \xmlNode{min\_samples\_leaf}, \xmlDesc{integer, optional field}, sets the
  minimum number of samples required to be at a leaf node.
  %
  \ifthenelse{\equal{#1}{}}{}{\default{#1}}
}

\newcommand{\maxLeafNodesDescription}[1]
{
  \xmlNode{max\_leaf\_nodes}, \xmlDesc{integer, optional field}, grow a tree
  with max\_leaf\_nodes in best-first fashion.
  %
  Best nodes are defined by relative reduction in impurity.
  %
  If None then unlimited number of leaf nodes.
  %
  If not None then max\_depth will be ignored.
  %
  \ifthenelse{\equal{#1}{}}{}{\default{#1}}
}
%%%%%%%%%%%%%%%%%%%%%%%%%% End Common Tree Parameters %%%%%%%%%%%%%%%%%%%%%%%%%%

%%%%%%%%%%%%%%%%%%%%%% Common Gaussian Process Parameters %%%%%%%%%%%%%%%%%%%%%%
\newcommand{\blankbDescription}[1]
{
  %
  \ifthenelse{\equal{#1}{}}{}{\default{#1}}
}
%%%%%%%%%%%%%%%%%%%% End Common Gaussian Process Parameters %%%%%%%%%%%%%%%%%%%%

%%%%%%%%%%%%%%%%%%%%%% Common Multi-layer Perceptron Parameters %%%%%%%%%%%%%%%%%%%%%%
\newcommand{\hiddenLayerSizesMLPDescription}[1]
{
  \xmlNode{hidden\_layer\_sizes}, \xmlDesc{tuple, length = n\_layers - 2, optional field}, the ith
  element represents the number of neurons in the ith hidden layer.
  %
  \ifthenelse{\equal{#1}{}}{}{\default{#1}}
}

\newcommand{\activationMLPDescription}[1]
{
  \xmlNode{activation}, \xmlDesc{string, optional field}, activation function for the hidden layer.
  \begin{itemize}
    \item \textit{identity}, no-op activation, useful to implement linear bottleneck, returns
      f(x) = x
    \item \textit{logistic}, the logistic sigmoid function, returns f(x) = 1/(1+exp(-x))
    \item \textit{tanh}, the hyperbolic tan function, returns f(x) = tanh(x)
    \item \textit{relu}, the rectified linear function, returns f(x) = max(0, x)
  \end{itemize}

  %
  \ifthenelse{\equal{#1}{}}{}{\default{#1}}
}

\newcommand{\solverMLPDescription}[1]
{
  \xmlNode{solver}, \xmlDesc{string, optional field}, The solver for weight optimization.
  \begin{itemize}
    \item \textit{lbfgs}, is an optimizer in the family of quasi-Newton methods
    \item \textit{sgd}, refers to stochastic gradient descent
    \item \textit{adam}, refers to a stochastic gradient-based optimizer proposed by
      Kingma, Diederik, and Jimmy Ba.
  \end{itemize}
  \nb The default solver \textit{adam} works pretty well on relatively large datasets
  (with thousands of training samples or more) in terms of both training time and validation
  score. For small datasets, however, \textit{lbfgs} can converge faster and perform better.
  %
  \ifthenelse{\equal{#1}{}}{}{\default{#1}}
}


\newcommand{\alphaMLPDescription}[1]
{
  \xmlNode{alpha}, \xmlDesc{float, optional field}, L2 penalty parameter
  %
  \ifthenelse{\equal{#1}{}}{}{\default{#1}}
}

\newcommand{\batchSizeMLPDescription}[1]
{
  \xmlNode{batch\_size}, \xmlDesc{int, optional field}, Size of minibatches for stochastic
  optimizers. If the solver is \textit{lbfgs}, the classifier will not use minibatch.
  \default{min(200, n\_samples)}
  %
  \ifthenelse{\equal{#1}{}}{}{\default{#1}}
}


\newcommand{\learningRateMLPDescription}[1]
{
  \xmlNode{learning\_rate}, \xmlDesc{string, optional field}, Learning rate schedule for weight updates
  \begin{itemize}
    \item \textit{constant}, is a constant learning rate given by \textit{learning\_rate\_init}
    \item \textit{invscaling}, gradually decreases the learning rate at each time step `t' using an
      inverse scaling exponent of `power\_t'. effective\_learning\_rate = learning\_rate\_init / pow(t, power\_t)
    \item \textit{adaptive}, keeps the learning rate constant to `learning\_rate\_init' as long as
      training loss keeps decreasing. Each time two consecutive epochs fail to decrease training loss by at least
       tol, or fail to increase validation score by at least tol if `early\_stopping' is on, the current
       learning rate is divided by 5.
       \nb Only used when \textit{solver = `sgd'}
  \end{itemize}
  %
  \ifthenelse{\equal{#1}{}}{}{\default{#1}}
}

\newcommand{\learningRateInitMLPDescription}[1]
{
  \xmlNode{learning\_rate\_init}, \xmlDesc{float, optional field}, The initial learning rate used. It
  controls the step-size in updating the weights. Only used when solver = `sgd' or `adam'
  %
  \ifthenelse{\equal{#1}{}}{}{\default{#1}}
}


\newcommand{\powerTMLPDescription}[1]
{
  \xmlNode{power\_t}, \xmlDesc{float, optional field}, the exponent for inverse scaling learning rate.
  It is used in updating effective learning rate when the learning\_rate is set to `invscaling'. only
  used when solver = `sgd'
  %
  \ifthenelse{\equal{#1}{}}{}{\default{#1}}
}

\newcommand{\maxIterMLPDescription}[1]
{
  \xmlNode{max\_iter}, \xmlDesc{int, optional field}, maximum number of iterations. The solver iterates until
  convergence (determined by `tol') or this number of iterations. For stochastic solvers (`sgd', `adam'), note
  that this determines the number of epochs (how many times each data point will be used), not the number
  of gradient steps
  %
  \ifthenelse{\equal{#1}{}}{}{\default{#1}}
}



\newcommand{\shuffleMLPDescription}[1]
{
  \xmlNode{shuffle}, \xmlDesc{bool, optional field}, whether to shuffle samples in each iteration. Only used
  when solver = `sgd' or `adam'
  %
  \ifthenelse{\equal{#1}{}}{}{\default{#1}}
}

\newcommand{\randomStateMLPDescription}[1]
{
  \xmlNode{random\_state}, \xmlDesc{int, RandomState instance or None, optional field}
  if int, random\_state is the seed used by the random number generator; If RandomState
  instance, random\_state is the random number generator; If None, the random number
  generator is the RandomState instance used by np.random.
  %
  \ifthenelse{\equal{#1}{}}{}{\default{#1}}
}
\newcommand{\tolMLPDescription}[1]
{
  \xmlNode{tol}, \xmlDesc{float, optional field}, tolerance for optimization. When the loss or
  score is not improving by at least tol for two consecutive iterations, unless learning\_rate
  is set to `adaptive', convergence is considered to be reached and training stops.
  %
  \ifthenelse{\equal{#1}{}}{}{\default{#1}}
}
\newcommand{\verboseMLPDescription}[1]
{
  \xmlNode{verbose}, \xmlDesc{bool, optional field}, whether to print progress messages or stdout
  %
  \ifthenelse{\equal{#1}{}}{}{\default{#1}}
}
\newcommand{\warmStartMLPDescription}[1]
{
  \xmlNode{warm\_start}, \xmlDesc{bool, optional field}, when set to True, reuse the solution of
  previous call to fit as initialization, otherise, just erase the previous solution.
  %
  \ifthenelse{\equal{#1}{}}{}{\default{#1}}
}
\newcommand{\momentumMLPDescription}[1]
{
  \xmlNode{momentum}, \xmlDesc{float, optional field}, momentum for gradient descent update.
  Should be between 0 and 1. Only used when solver = `sgd'.
  %
  \ifthenelse{\equal{#1}{}}{}{\default{#1}}
}
\newcommand{\nesterovsMomentumMLPDescription}[1]
{
  \xmlNode{nesterovs\_momentum}, \xmlDesc{bool, optional field}, whether to use Nesterov's momentum.
  Only used when solver='sgd' and momentum > 0.
  %
  \ifthenelse{\equal{#1}{}}{}{\default{#1}}
}
\newcommand{\earlyStoppingMLPDescription}[1]
{
  \xmlNode{early\_stopping}, \xmlDesc{bool, optional field}, whether to use early stopping to terminate
  training when validation score is not improving. If set to true, it will automatically set aside 10\%
  of training data as validation and terminate training when validation score is not improving by at
  least tol for two consecutive epochs. Only effective when solver=`sgd' or `adam'.
  %
  \ifthenelse{\equal{#1}{}}{}{\default{#1}}
}
\newcommand{\validationFractionMLPDescription}[1]
{
  \xmlNode{validation\_fraction}, \xmlDesc{float, optional field}, the proportion of training data to set
  aside as validation set for early stopping. Must be between 0 and 1. Only used if early\_stopping is True.
  %
  \ifthenelse{\equal{#1}{}}{}{\default{#1}}
}
\newcommand{\epsilonMLPDescription}[1]
{
  \xmlNode{epsilon}, \xmlDesc{float, optional field}, value for numerical stability in adam. Only used
  when solver = `adam'
  %
  \ifthenelse{\equal{#1}{}}{}{\default{#1}}
}

\newcommand{\betaAMLPDescription}[1]
{
  \xmlNode{beta\_1}, \xmlDesc{float, optional field}, exponential decay rate for estimates of first moment
  vector in adam. should be in [0, 1). only used when solver = `adam'
  %
  \ifthenelse{\equal{#1}{}}{}{\default{#1}}
}
\newcommand{\betaBMLPDescription}[1]
{
  \xmlNode{beta\_2}, \xmlDesc{float, optional field}, exponentail decay rate for estimates of second moment
  vector in adam. should be in [0, 1). only used when solver = `adam'
  %
  \ifthenelse{\equal{#1}{}}{}{\default{#1}}
}

%%%%%%%%%%%%%%%%%%%% End Common Multi-layer Perceptron Parameters %%%%%%%%%%%%%%%%%%%%

%%%%%%%%%%%%%%%%%%%%%%%%%%%%%%%%%%%%%%%%%%%%%%%%%%%%%%%%%%%%%%%%%%%%%%%%%%%%%%%%

In RAVEN, \textbf{Models} are important entities.
%
A model is an object that employs a mathematical representation of a
phenomenon, either of a physical or other nature (e.g. statistical operators,
etc.).
%
From a practical point of view, it can be seen, as a ``black box'' that, given
an input, returns an output.
%

RAVEN has a strict classification of the different types of models.
%
Each ``class'' of models is represented by the definition reported above, but it
can be further classified based on its particular functionalities:
\begin{itemize}
  \item \xmlNode{Code} represents an external system code that employs a high
  fidelity physical model.
  \item \xmlNode{Dummy} acts as ``transfer'' tool.
  %
  The only action it performs is transferring the the information in the input
  space (inputs) into the output space (outputs).
  %
  For example, it can be used to check the effect of a sampling strategy, since
  its outputs are the sampled parameters' values (input space) and a counter
  that keeps track of the number of times an evaluation has been requested.
  %
  \item \xmlNode{ROM}, or reduced order model, is a mathematical model trained
  to predict a response of interest of a physical system.
  %
  Typically, ROMs trade speed for accuracy representing a faster, rough estimate
  of the underlying phenomenon.
  %
  The ``training'' process is performed by sampling the response of a physical
  model with respect to variation of its parameters subject to probabilistic
  behavior.
  %
  The results (outcomes of the physical model) of the sampling are fed into
  the algorithm representing the ROM that tunes itself to replicate those
  results.
  \item \xmlNode{ExternalModel}, as its name suggests, is an entity existing
  outside the RAVEN framework that is embedded in the RAVEN code at run time.
  %
  This object allows the user to create a Python module that will be treated as
  a predefined internal model object.
  \item \xmlNode{EnsembleModel} is model that is able to combine \textbf{Code},
  \textbf{ExternalModel} and \textbf{ROM} models. It is aimed to create a chain
  of Models (whose execution order is determined by the Input/Output relationships among them).
  If the relationships among the models evolve in a non-linear system, a Picard's Iteration scheme
  is employed.
  \item \xmlNode{PostProcessor} is a container of all the actions that can
  manipulate and process a data object in order to extract key information,
  such as statistical quantities, clustering, etc.
  %
\end{itemize}
Before analyzing each model in detail, it is important to mention that each
type needs to be contained in the main XML node \xmlNode{Models}, as reported
below:

\textbf{Example:}
\begin{lstlisting}[style=XML]
<Simulation>
  ...
  <Models>
    ...
    <WhatEverModel name='whatever'>
      ...
    </WhatEverModel>
    ...
  </Models>
  ...
</Simulation>
\end{lstlisting}
In the following sub-sections each \textbf{Model} type is fully analyzed and
described.
%
%%%%%%%%%%%%%%%%%%%%%%%%
%%%%%%  Code  Model   %%%%%%
%%%%%%%%%%%%%%%%%%%%%%%%
%<Code name='MyRAVEN' subType='RAVEN'><executable>%FRAMEWORK_DIR%/../RAVEN-%METHOD%</executable></Code>
%<alias variable='internal_variable_name'>Material|Fuel|thermal_conductivity</alias>
\subsection{Code}
\label{subsec:models_code}
The \textbf{Code} model represents an external system
software employing a high fidelity physical model.
%
The link between RAVEN and the driven code is performed at run-time, through
coded interfaces that are the responsible for transferring information from the
code to RAVEN and vice versa.
%
In Section~\ref{sec:existingInterface}, all of the available interfaces are
reported and, for advanced users, Section~\ref{sec:newCodeCoupling} explains how
to couple a new code.


\specBlock{a}{Code}
%
\attrsIntro
%
\vspace{-5mm}
\begin{itemize}
  \itemsep0em
  \item \nameDescription
  \item \xmlAttr{subType}, \xmlDesc{required string attribute}, specifies the
  code that needs to be associated to this Model.
  %
  \nb See Section~\ref{sec:existingInterface} for a list of currently supported
  codes.
  %
\end{itemize}
\vspace{-5mm}

\subnodesIntro
%
\begin{itemize}
  \item \xmlNode{executable} \xmlDesc{string, required field} specifies the path
  of the executable to be used.

  \item \xmlNode{walltime}  \xmlDesc{string, optional field} specifies the maximum
  allowed run time of the code; if the code running time is greater than the specified
  walltime then the code run is stopped. The stopped run is then considered as if it crashed.
  %
  \nb Both absolute and relative path can be used. In addition, the relative path
  to the working directory can also be used.
  %
  \item \xmlNode{preexec} \xmlDesc{string, optional field} specifies the path of
    pre-executable to be used.
  \nb Both absolute and relative path can be used. In addition, the relative path
  to the working directory can also be used.
  %
  \item \aliasSystemDescription{Code}
  %
  \item \xmlNode{clargs} \xmlDesc{string, optional field} allows addition of
  command-line arguments to the execution command.  If the code interface
  specified in \xmlNode{Code} \xmlAttr{subType} does not specify how to
  determine the input file(s), this node must be used to specify them.
  There are several types of \xmlNode{clargs}, based on the \xmlAttr{type}:
  \begin{itemize}
    \item \xmlAttr{type} \xmlDesc{string, required field} specifies the type of
    command-line argument to add.  Options include \xmlString{input},
    \xmlString{output}, \xmlString{prepend}, \xmlString{postpend},
    \xmlString{text}, and \xmlString{python}.
    %
    \item \xmlAttr{arg} \xmlDesc{string, optional field} specifies the flag to
    be used before the entry.  For example, \xmlAttr{arg=}\xmlString{-i} would
    place a \texttt{-i} before the entry in the execution command.  Required for
    the \xmlString{output} \xmlAttr{type}.
    %
    \item \xmlAttr{extension} \xmlDesc{string, optional field} specifies the type
    of file extension to use (for example, \texttt{-i} or \texttt{-o}).  This links the
    \xmlNode{Input} file in the \xmlNode{Step} to this location in the execution
    command.  Required for \xmlString{input} \xmlAttr{type}.
    %
    \item \xmlAttr{delimiter} \xmlDesc{string, optional field} specifies the delimiter
      that is used between the \xmlAttr{arg} and the provided input file with
      the extension given by \xmlAttr{extension}.
      \nb This is currently only used to link the \xmlAttr{arg} and input file. i.e.
      the \xmlAttr{type} should be \xmlString{input} in order to use this feature.
  \end{itemize}
  The execution command is combined in the order \xmlString{prepend}, \xmlNode{python}
  \xmlNode{executable}, \xmlString{input}, \xmlString{output}, \xmlString{text},
  \xmlString{postpend}.  The \xmlString{python} is a special type that puts the name of the python command.
  %
  \item \xmlNode{fileargs} \xmlDesc{string, optional field} like \xmlNode{clargs},
  but allows editing of input files to specify the output filename and/or auxiliary
  file names.
  %
  The location in the input files to edit using these arguments are identified in
  the input file using the prefix-postfix notation, which defaults to
  \texttt{\$RAVEN-var\$} for variable keyword \emph{var}.  The variable keyword
  is then listed in the \xmlNode{fileargs} node in the attribute \xmlAttr{arg} to
  couple it in Raven.
  %
  If the code interface specified in \xmlNode{Code} \xmlAttr{subType}
  does not specify how to name the output file, that must be specified either through
  \xmlNode{clargs} or \xmlNode{filargs}, with \xmlAttr{type} \xmlString{output}.
  The attributes required for \xmlNode{fileargs} are as follows:
  \begin{itemize}
    \item \xmlAttr{type} \xmlDesc{string, required field} specifies the type
    of entry to replace in the file.  Possible values for \xmlNode{fileargs}
    \xmlAttr{type} are \xmlString{input} and \xmlString{output}.
    %
    \item \xmlAttr{arg} \xmlDesc{string, required field} specifies the Raven
    variable with which to replace the file of interest.  This should match
    the entry in the template input file; that is, if \texttt{\$RAVEN-auxinp\$}
    is in the input file, the arg for the corresponding input file should be
    \xmlString{auxinp}.
    %
    \item \xmlAttr{extension} \xmlDesc{string, optional field} specifies the
    extension of the input file that should replace the Raven variable in
    the input file.  This attribute is required for the \xmlString{input} \xmlAttr{type}
    and ignored for the \xmlString{output} \xmlAttr{type}.
    \nb{Currently, there can only be a one-to-one pairing between input files
    and extensions; that is, multiple Raven-editable input files cannot have the
    same extension.}
  \end{itemize}
\end{itemize}
\textbf{Example:}
\begin{lstlisting}[style=XML,morekeywords={subType,name,variable}]
<Simulation>
  ...
  <Models>
    ...
    <Code name='aUserDefinedName' subType='RAVEN_Driven_code'>
      <executable>path_to_executable</executable>
      <alias variable='internal_RAVEN_input_variable_name1' type="input">
         External_Code_input_Variable_Name_1
      </alias>
      <alias variable='internal_RAVEN_input_variable_name2' type='input'>
         External_Code_input_Variable_Name_2
      </alias>
      <alias variable='internal_RAVEN__output_variable_name' type='output'>
         External_Code_output_Variable_Name_2
      </alias>
      <clargs type='prepend' arg='python'/>
      <clargs type='input' arg='-i' extension='.i'/>
      <fileargs type='input' arg='aux' extension='.two'
      <fileargs type='output' arg='out' />
    </Code>
    ...
  </Models>
  ...
</Simulation>
\end{lstlisting}

%%%%%%%%%%%%%%%%%%%%%%%%
%%%%%% Dummy Model  %%%%%%
%%%%%%%%%%%%%%%%%%%%%%%%
\subsection{Dummy}
\label{subsec:models_dummy}
The \textbf{Dummy} model is an object that acts as a pass-through tool.
%
The only action it performs is transferring the information in the input
space (inputs) to the output space (outputs).
%
For example, it can be used to check the effect of a particular sampling
strategy, since its outputs are the sampled parameters' values (input space) and
a counter that keeps track of the number of times an evaluation has been
requested.
%

\specBlock{a}{Dummy}.
%
\attrsIntro
%
\vspace{-5mm}
\begin{itemize}
  \itemsep0em
  \item \nameDescription
  %
  \item \xmlAttr{subType}, \xmlDesc{required string attribute}, this attribute
  must be kept empty.
\end{itemize}
\vspace{-5mm}

\subnodesIntro
%
\begin{itemize}
  \item \aliasSystemDescription{Dummy}
  \\Since the \textbf{Dummy} model represents a transfer function only, the usage of the alias is relatively meaningless.
\end{itemize}

Given a particular \textit{Step} using this model, if this model is linked to
a \textit{Data} with the role of \textbf{Output}, it expects one of the output
parameters will be identified by the keyword ``OutputPlaceHolder'' (see
Section~\ref{sec:steps}).

\textbf{Example:}
\begin{lstlisting}[style=XML,morekeywords={subType}]
<Simulation>
  ...
  <Models>
    ...
    <Dummy name='aUserDefinedName1' subType=''/>

    <Dummy name='aUserDefinedName2' subType=''>
      <alias variable="a_RAVEN_input_variable" type="input">
      another_name_for_this_variable_in_the_model
      </alias>
    </Dummy>
    ...
  </Models>
  ...
</Simulation>
\end{lstlisting}



%%%%%%%%%%%%%%%%%%%%%%
%%%%% ROM Model  %%%%%%%
%%%%%%%%%%%%%%%%%%%%%%
\newcommand{\zNormalizationPerformed}[1]
{
  \textcolor{red}{\\It is important to NOTE that RAVEN uses a Z-score normalization of the training data before
  constructing the \textit{#1} ROM:
\begin{equation}
  \mathit{\mathbf{X'}} = \frac{(\mathit{\mathbf{X}}-\mu )}{\sigma }
\end{equation}
 }
}

\newcommand{\zNormalizationNotPerformed}[1]
{
  \textcolor{red}{
  \\It is important to NOTE that RAVEN does not pre-normalize the training data before
  constructing the \textit{#1} ROM.}
}

\newcommand{\romClusterOption}[0]
{
  \item \xmlNode{Segment}, \xmlDesc{node, optional}, provides an alternative way to build the ROM. When
    this mode is enabled, the subspace of the ROM (e.g. ``time'') will be divided into segments as
    requested, then a distinct ROM will be trained on each of the segments. This is especially helpful if
    during the subspace the ROM representation of the signal changes significantly. For example, if the signal
    is different during summer and winter, then a signal can be divided and a distinct ROM trained on the
    segments. By default, no segmentation occurs.

    To futher enable clustering of the segments, the \xmlNode{Segment} has the following attributes:
    \begin{itemize}
      \item \xmlAttr{grouping}, \xmlDesc{string, optional field} enables the use of ROM subspace clustering in
        addition to segmenting if set to \xmlString{cluster}. If set to \xmlString{segment}, then performs
        segmentation without clustering. If clustering, then an additional node needs to be included in the
        \xmlNode{Segment} node, as described below.
        \default{segment}
    \end{itemize}

    This node takes the following subnodes:
    \begin{itemize}
      \item \xmlNode{subspace}, \xmlDesc{string, required field} designates the subspace to divide. This
        should be the pivot parameter (often ``time'') for the ROM. This node also requires an attribute
        to determine how the subspace is divided, as well as other attributes, described below:
        \begin{itemize}
          \item \xmlAttr{pivotLength}, \xmlDesc{float, optional field}, provides the value in the subspace
            that each segment should attempt to represent, independently of how the data is stored. For
            example, if the subspace has hourly resolution, is measured in seconds, and the desired
            segmentation is daily, the \xmlAttr{pivotLength} would be 86400.
            Either this option or \xmlAttr{divisions} must be provided.
          \item \xmlAttr{divisions}, \xmlDesc{integer, optional field}, as an alternative to
            \xmlAttr{pivotLength}, this attribute can be used to specify how many data points to include in
            each subdivision, rather than use the pivot values. The algorithm will attempt to split the data
            points as equally as possible.
            Either this option or \xmlAttr{pivotLength} must be provided.
          \item \xmlAttr{shift}, \xmlDesc{string, optional field}, governs the way in which the subspace is
            treated in each segment. By default, the subspace retains its actual values for each segment; for
            example, if each segment is 4 hours long, the first segment starts at time 0, the second at 4
            hours, the third at 8 hours, and so forth. Options to change this behavior are \xmlString{zero}
            and \xmlString{first}. In the case of \xmlString{zero}, each segment restarts the pivot with the
            subspace value as 0, shifting all other values similarly. In the example above, the first segment
            would start at 0, the second at 0, and the third at 0, with each ending at 4 hours. Note that the
            pivot values are restored when the ROM is evaluated. Using \xmlString{first}, each segment
            subspace restarts at the value of the first segment. This is useful in the event subspace 0 is not
            a desirable value.
        \end{itemize}
      \item \xmlNode{Classifier}, \xmlDesc{string, optional field} associates a \xmlNode{PostProcessor}
        defined in the \xmlNode{Models} block to this segmentation. If clustering is enabled (see
        \xmlAttr{grouping} above), then this associated Classifier will be used to cluster the segmented ROM
        subspaces. The attributes \xmlAttr{class}=\xmlString{Models} and
        \xmlAttr{type}=\xmlString{PostProcessor} must be set, and the text of this node is the \xmlAttr{name}
        of the requested Classifier. Note this Classifier must be a valid Classifier; not all PostProcessors
        are suitable. For example, see the DataMining PostProcessor subtype Clustering.
      \item \xmlNode{clusterFeatures}, \xmlDesc{string, optional field}, if clustering then delineates
        the fundamental ROM features that should be considered while clustering. The available features are
        ROM-dependent, and an exception is raised if an unrecognized request is given. See individual ROMs
        for options. \default All ROM-specific options.
      \item \xmlNode{evalMode}, \xmlDesc{string, optional field}, one of \xmlString{truncated},
        \xmlString{full}, or \xmlString{clustered}, determines how the evaluations are
        represented, as follows:
        \begin{itemize}
          \item \xmlString{full}, reproduce the full signal using representative cluster segments,
          \item \xmlString{truncated}, reproduce a history containing exactly segment from each
            cluster placed back-to-back, with the \xmlNode{pivotParameter} spanning the clustered
            dimension. Note this will almost surely not be the same length as the original signal;
            information about indexing can be found in the ROM's XML metadata.
          \item \xmlString{clustered}, reproduce a N-dimensional object with the variable
            \texttt{\_ROM\_cluster} as one of the indexes for the ROM's sampled variables. Note that
            in order to use the option, the receiving \xmlNode{DataObject} should be of type
            \xmlNode{DataSet} with one of the indices being \texttt{\_ROM\_cluster}.
        \end{itemize}
     \item \xmlNode{evaluationClusterChoice}, \xmlDesc{string, optional field}, one of \xmlString{first} or
        \xmlString{random}, determines, if \xmlAttr{grouping}$=cluster$, which
        strategy needs to be followed for the evaluation stage. If ``first'', the
        first ROM (representative segmented ROM),in each cluster, is considered to
         be representative of the full space in the cluster (i.e. the evaluation is always performed
         interrogating the first ROM in each cluster); If ``random'', a random ROM, in each cluster,
         is choosen when an evaluation is requested.
	 \nb if ``first'' is used, there is \emph{substantial} memory savings when compared to using
	 ``random''.
         %If ``centroid'', a ROM ``trained" on the centroids
         %information of each cluster is used for the evaluation (\nb ``centroid'' option is not
         %available yet).
         \default{first}
    \end{itemize}
}

\subsection{ROM}
\label{subsec:models_ROM}
A Reduced Order Model (ROM) is a mathematical model consisting of a fast
solution trained to predict a response of interest of a physical system.
%
The ``training'' process is performed by sampling the response of a physical
model with respect to variations of its parameters subject, for example, to
probabilistic behavior.
%
The results (outcomes of the physical model) of the sampling are fed into the
algorithm representing the ROM that tunes itself to replicate those results.
%
RAVEN supports several different types of ROMs, both internally developed and
imported through an external library called ``scikit-learn''~\cite{SciKitLearn}.

Currently in RAVEN, the ROMs are classified into several sub-types that, once chosen,
provide access to several different algorithms.
%
These sub-types are specified in the \xmlAttr{subType} attribute and should be
one of the following:
\begin{itemize}
  \item \xmlString{GaussPolynomialRom}, for both static and time-dependent regression
  \item \xmlString{HDMRRom}, for both static and time-dependent regression
  \item \xmlString{NDinvDistWeight}, for both static and time-dependent regression
  \item \xmlString{NDSpline}, for both static and time-dependent regression
  \item \xmlString{SciKitLearn}, for both static and time-dependent regression and classification
  \item \xmlString{MSR}, for both static and time-dependent regression
  \item \xmlString{ARMA}, for time-dependent stochastic regression (time series generator)
  \item \xmlString{PolyExponential}, for time-dependent regression
  \item \xmlString{DMD}, for time-dependent regression
  \item \xmlString{KerasMLPClassifier}, for deep neuron network
\end{itemize}

\specBlock{a}{ROM}
%
\attrsIntro
%
\vspace{-5mm}
\begin{itemize}
  \itemsep0em
  \item \nameDescription
  \item \xmlAttr{subType}, \xmlDesc{required string attribute}, defines which of
  the sub-types should be used, choosing among the previously reported
  types.
  %
  This choice conditions the subsequent the required and/or optional
  \xmlNode{ROM} sub-nodes.
  %
\end{itemize}
\vspace{-5mm}

In the \xmlNode{ROM} input block, the following XML sub-nodes are required,
independent of the \xmlAttr{subType} specified:
%
\begin{itemize}
  %
   \item \xmlNode{Features}, \xmlDesc{comma separated string, required field},
     specifies the names of the features of this ROM.
   \nb These parameters are going to be requested for the training of this object
    (see Section~\ref{subsec:stepRomTrainer});
    \item \xmlNode{Target}, \xmlDesc{comma separated string, required field},
      contains a comma separated list of the targets of this ROM. These parameters
      are the Figures of Merit (FOMs) this ROM is supposed to predict.
    \nb These parameters are going to be requested for the training of this
    object (see Section \ref{subsec:stepRomTrainer}).
\end{itemize}

If a time-dependent ROM is requested, a \textbf{HistorySet} should be provided.
The temporal vairable specified in the \textbf{HistorySet} should be also listed
as sub-nodes inside \xmlNode{ROM}
%
\begin{itemize}
  \item \xmlNode{pivotParameter}, \xmlDesc{string, optional parameter}, specifies the pivot
    variable (e.g. time, etc) used in the input HistorySet.
    \default{time}
\end{itemize}
%
In addition, if the user wants to use the alias system, the following XML block can be inputted:
\begin{itemize}
  \item \aliasSystemDescription{ROM}
\end{itemize}


The types and meaning of the remaining sub-nodes depend on the sub-type
specified in the attribute \xmlAttr{subType}.

%
Note that if an HistorySet is provided in the training step then a temporal ROM is created, i.e. a ROM that generates not a single value prediction of each element indicated in the  \xmlNode{Target} block but its full temporal profile.
\\
\textcolor{red}{\\\textbf{It is important to NOTE that RAVEN uses a Z-score normalization of the training data before constructing most of the
Reduced Order Models (e.g. most of the SciKitLearn-based ROMs):}}
\begin{equation}
  \mathit{\mathbf{X'}} = \frac{(\mathit{\mathbf{X}}-\mu )}{\sigma }
\end{equation}
\\In the following sections the specifications of each ROM type are reported, highlighting when a \textbf{Z-score normalization} is performed by RAVEN before constructing the ROM or when it is not performed.

%
%%%%% ROM Model - NDspline  %%%%%%%
\subsubsection{NDspline}
\label{subsubsec:NDspline}
The NDspline sub-type contains a single ROM type, based on an $N$-dimensional
spline interpolation/extrapolation scheme.
%
In spline interpolation, the regressor is a special type of piece-wise
polynomial called tensor spline.
%
The interpolation error can be made small even when using low degree polynomials
for the spline.
%
Spline interpolation avoids the problem of Runge's phenomenon, in which
oscillation can occur between points when interpolating using higher degree
polynomials.
%

In order to use this ROM, the \xmlNode{ROM} attribute \xmlAttr{subType} needs to
be \xmlString{NDspline} (see the example below).
%
No further XML sub-nodes are required.
%
\nb This ROM type must be trained from a regular Cartesian grid.
%
Thus, it can only be trained from the outcomes of a grid sampling strategy.

\zNormalizationPerformed{NDspline}

\textbf{Example:}
\begin{lstlisting}[style=XML,morekeywords={name,subType}]
<Simulation>
  ...
  <Models>
    ...
    <ROM name='aUserDefinedName' subType='NDspline'>
       <Features>var1,var2,var3</Features>
       <Target>result1,result2</Target>
     </ROM>
    ...
  </Models>
  ...
</Simulation>
\end{lstlisting}

%%%%% ROM Model - GaussPolynomialRom  %%%%%%%
\subsubsection{pickledROM}
\label{subsubsec:pickledROM}
It is not uncommon for a reduced-order model (ROM) to be created and trained in one RAVEN run, then
serialized to file (\emph{pickled}), then loaded into another RAVEN run to be used as a model.  When this is
the case, a \xmlNode{ROM} with subtype \xmlString{pickledROM} is used to hold the place of the ROM that will
be loaded from file.  The notation for this ROM is much less than a typical ROM; it only requires a name and
its subtype.

Note that when loading ROMs from file, RAVEN will not perform any checks on the expected inputs or outputs of
a ROM; it is expected that a user know at least the I/O of a ROM before trying to use it as a model.
However, RAVEN does require that pickled ROMs be trained before pickling in the first place.

Initially, a pickledROM is not usable.  It cannot be trained or sampled; attempting to do so will raise an
error.  An \xmlNode{IOStep} is used to load the ROM from file, at which point the ROM will have all the same
characteristics as when it was pickled in a previous RAVEN run.

\textbf{Example:}
For this example the ROM has already been created and trained in another RAVEN run, then pickled to a file
called \texttt{rom\_pickle.pk}.  In the example, the file is identified in \xmlNode{Files}, the model is
defined in \xmlNode{Models}, and the model loaded in \xmlNode{Steps}.
{\footnotesize
\begin{lstlisting}[style=XML,morekeywords={name,subType}]
<Simulation>
  ...
  <Files>
    <Input name="rompk" type="">rom_pickle.pk</Input>
  </Files>
  ...
  <Models>
    ...
    <ROM name="myRom" subType="pickledROM"/>
    ...
  </Models>
  ...
  <Steps>
    ...
    <IOStep name="loadROM">
      <Input class="Files" type="">rompk</Input>
      <Output class="Models" type="ROM">myRom</Output>
    </IOStep>
    ...
  </Steps>
  ...
</Simulation>
\end{lstlisting}
}


\subsubsection{GaussPolynomialRom}
\label{subsubsec:GaussPolynomialRom}
The GaussPolynomialRom sub-type contains a single ROM type, based on a
characteristic Gaussian polynomial fitting scheme: generalized polynomial chaos
expansion (gPC).
%
In gPC, sets of polynomials orthogonal with respect to the distribution of uncertainty
are used to represent the original model.  The method converges moments of the original
model faster than Monte Carlo for small-dimension uncertainty spaces ($N<15$).
%
In order to use this ROM, the \xmlNode{ROM} attribute \xmlAttr{subType} needs to
be \xmlString{GaussPolynomialRom} (see the example below).
%
The GaussPolynomialRom is dependent on specific sampling; thus, this ROM cannot be trained unless a
SparseGridCollocation or similar Sampler specifies this ROM in its input and is sampled in a MultiRun step.
%
In addition to the common \xmlNode{Target} and \xmlNode{Features}, this ROM requires
two more nodes and can accept multiple entries of a third optional node.
\begin{itemize}
  \item \xmlNode{IndexSet}, \xmlDesc{string, required field},
  specifies the rules by which to construct multidimensional polynomials.  The options are
  \xmlString{TensorProduct}, \xmlString{TotalDegree},\\
  \xmlString{HyperbolicCross}, and \xmlString{Custom}.
  %
  Total degree is efficient for
  uncertain inputs with a large degree of regularity, while hyperbolic cross is more efficient
  for low-regularity input spaces.
  %
  If \xmlString{Custom} is chosen, the \xmlNode{IndexPoints} is required.
  %
  \item \xmlNode{PolynomialOrder}, \xmlDesc{integer, required field},
  indicates the maximum polynomial order in any one dimension to use in the
  polynomial chaos expansion. \nb If non-equal importance weights are supplied in the optional
  \xmlNode{Interpolation} node, the actual polynomial order in dimensions with high
  importance might exceed this value; however, this value is still used to limit the
  relative overall order.
  %
  \item \xmlNode{SparseGrid},\xmlDesc{string, optional field}, allows specification of the multidimensional
    quadrature construction strategy.  Options are \xmlString{smolyak} and \xmlString{tensor}.  Default is
    \xmlString{smolyak}.
  \item \xmlNode{IndexPoints}, \xmlDesc{list of tuples, required field},
  used to specify the index set points in a \xmlString{Custom} index set.  The tuples are
  entered as comma-seprated values between parenthesis, with each tuple separated by a comma.
  Any amount of whitespace is acceptable.  For example, \xmlNode{IndexPoints}\verb'(0,1),(0,2),(1,1),(4,0)'\xmlNode{/IndexPoints}
  \nb{Using custom index sets
  does not guarantee accurate convergence.}
  %
  \item \xmlNode{Interpolation}, \xmlDesc{string, optional field},
  offers the option to specify quadrature, polynomials, and importance weights for the given
  variable name.  The ROM accepts any number of \xmlNode{Interpolation} nodes up to the
  dimensionality of the input space.  This node accepts several attributes, all of which are
  optional and default to
  the code-defined optimal choices based on the input dimension uncertainty distribution:
  \begin{itemize}
    \item \xmlAttr{quad}, \xmlDesc{string, optional field},
      specifies the quadrature type to use for collocation in this dimension.  The default options
      depend on the uncertainty distribution of the input dimension, as shown in Table
      \ref{tab:gpcCompatible}. Additionally, Clenshaw Curtis quadrature can be used for any
      distribution that doesn't include an infinite bound.
      \default{see Table \ref{tab:gpcCompatible}.}
      \nb For an uncertain distribution aside from the four listed on Table
      \ref{tab:gpcCompatible}, this ROM
      makes use of the uniform-like range of the distribution's CDF to apply quadrature that is
      suited uniform uncertainty (Legendre).  It converges more slowly than the four listed, but are
      viable choices.  Choosing polynomial type Legendre for any non-uniform distribution will
      enable this formulation automatically.
    \item \xmlAttr{poly}, \xmlDesc{string,optional field},
      specifies the interpolating polynomial family to use for the polynomial expansion in this
      dimension.  The default options depend on the quadrature type chosen, as shown in Table
      \ref{tab:gpcCompatible}.  Currently, no polynomials are available outside the
      default. \default{see Table \ref{tab:gpcCompatible}.}
    \item  \xmlAttr{weight}, \xmlDesc{float, optional field},
      delineates the importance weighting of this dimension.  A larger importance weight will
      result in increased resolution for this dimension at the cost of resolution in lower-weighted
      dimensions.  The algorithm normalizes weights at run-time.\default{1}.
  \end{itemize}
  %
\end{itemize}
\begin{table}[htb]
  \centering
  \begin{tabular}{c | c c}
    Unc. Distribution & Default Quadrature & Default Polynomials \\ \hline
    Uniform & Legendre & Legendre \\
    Normal & Hermite & Hermite \\ \hline
    Gamma & Laguerre & Laguerre \\
    Beta & Jacobi & Jacobi \\ \hline
    Other & Legendre* & Legendre*
  \end{tabular}
  \caption{GaussPolynomialRom defaults}
  \label{tab:gpcCompatible}
\end{table}
%
\nb This ROM type must be trained from a collocation quadrature set.
%
Thus, it can only be trained from the outcomes of a SparseGridCollocation sampler.
Also, this ROM must be referenced in the SparseGridCollocation sampler in order to
accurately produce the necessary sparse grid points to train this ROM.

\zNormalizationNotPerformed{GaussPolynomialRom}

\textbf{Example:}
{\footnotesize
\begin{lstlisting}[style=XML,morekeywords={name,subType}]
<Simulation>
  ...
  <Samplers>
    ...
    <SparseGridCollocation name="mySG" parallel="0">
      <variable name="x1">
        <distribution>myDist1</distribution>
      </variable>
      <variable name="x2">
        <distribution>myDist2</distribution>
      </variable>
      <ROM class = 'Models' type = 'ROM' >myROM</ROM>
    </SparseGridCollocation>
    ...
  </Samplers>
  ...
  <Models>
    ...
    <ROM name='myRom' subType='GaussPolynomialRom'>
      <Target>ans</Target>
      <Features>x1,x2</Features>
      <IndexSet>TotalDegree</IndexSet>
      <PolynomialOrder>4</PolynomialOrder>
      <Interpolation quad='Legendre' poly='Legendre' weight='1'>x1</Interpolation>
      <Interpolation quad='ClenshawCurtis' poly='Jacobi' weight='2'>x2</Interpolation>
    </ROM>
    ...
  </Models>
  ...
</Simulation>
\end{lstlisting}
}

When Printing this ROM via a Print OutStream (see \ref{sec:printing}), the available metrics are:
\begin{itemize}
  \item \xmlString{mean}, the mean value of the ROM output within the input space it was trained,
  \item \xmlString{variance}, the variance of the ROM output within the input space it was trained,
  \item \xmlString{samples}, the number of distinct model runs required to construct the ROM,
  \item \xmlString{indices}, the Sobol sensitivity indices (in percent), Sobol total indices, and partial variances,
  \item \xmlString{polyCoeffs}, the polynomial expansion coefficients (PCE moments) of the ROM.  These are
    listed by each polynomial combination, with the polynomial order tags listed in the order of the variables
    shown in the XML print.
\end{itemize}

%%%%% ROM Model - HDMRRom  %%%%%%%
\subsubsection{HDMRRom}
\label{subsubsec:HDMRRom}
The HDMRRom sub-type contains a single ROM type, based on a Sobol decomposition scheme.
%
In Sobol decomposition, also known as high-density model reduction (HDMR, specifically Cut-HDMR),
a model is approximated as as the sum of increasing-complexity interactions.  At its lowest level (order 1), it treats the function as a sum of the reference case plus a functional of each input dimesion separately.  At order 2, it adds functionals to consider the pairing of each dimension with each other dimension.  The benefit to this approach is considering several functions of small input cardinality instead of a single function with large input cardinality.  This allows reduced order models like generalized polynomial chaos (see \ref{subsubsec:GaussPolynomialRom}) to approximate the functionals accurately with few computations runs.
%
In order to use this ROM, the \xmlNode{ROM} attribute \xmlAttr{subType} needs to
be \xmlString{HDMRRom} (see the example below).
%
The HDMRRom is dependent on specific sampling; thus, this ROM cannot be trained unless a
Sobol or similar Sampler specifies this ROM in its input and is sampled in a MultiRun step.
%
In addition to the common \xmlNode{Target} and \xmlNode{Features}, this ROM requires
the same nodes as the GaussPolynomialRom (see \ref{subsubsec:GaussPolynomialRom}.
Additionally, this ROM requires the \xmlNode{SobolOrder} node.
\begin{itemize}
  \item \xmlNode{SobolOrder}, \xmlDesc{integer, required field},
  indicates the maximum cardinality of the input space used in the subset functionals.  For example, order 1
  includes only functionals of each independent dimension separately, while order 2 considers pair-wise
  interactions.
  %
\end{itemize}
\nb This ROM type must be trained from a Sobol decomposition training set.
%
Thus, it can only be trained from the outcomes of a Sobol sampler.
Also, this ROM must be referenced in the Sobol sampler in order to
accurately produce the necessary sparse grid points to train this ROM.
Experience has shown order 2 Sobol decompositions to include the great majority of
  uncertainty in most models.

\zNormalizationNotPerformed{HDMRRom}

\textbf{Example:}
{\footnotesize
\begin{lstlisting}[style=XML,morekeywords={name,subType}]
  <Samplers>
    ...
    <Sobol name="mySobol" parallel="0">
      <variable name="x1">
        <distribution>myDist1</distribution>
      </variable>
      <variable name="x2">
        <distribution>myDist2</distribution>
      </variable>
      <ROM class = 'Models' type = 'ROM' >myHDMR</ROM>
    </Sobol>
    ...
  </Samplers>
  ...
  <Models>
    ...
    <ROM name='myHDMR' subType='HDMRRom'>
      <Target>ans</Target>
      <Features>x1,x2</Features>
      <SobolOrder>2</SobolOrder>
      <IndexSet>TotalDegree</IndexSet>
      <PolynomialOrder>4</PolynomialOrder>
      <Interpolation quad='Legendre' poly='Legendre' weight='1'>x1</Interpolation>
      <Interpolation quad='ClenshawCurtis' poly='Jacobi' weight='2'>x2</Interpolation>
    </ROM>
    ...
  </Models>
\end{lstlisting}
}

When Printing this ROM via an OutStream (see \ref{sec:printing}), the available metrics are:
\begin{itemize}
  \item \xmlString{mean}, the mean value of the ROM output within the input space it was trained,
  \item \xmlString{variance}, the ANOVA-calculated variance of the ROM output within the input space it
    was trained.
  \item \xmlString{samples}, the number of distinct model runs required to construct the ROM,
  \item \xmlString{indices}, the Sobol sensitivity indices (in percent), Sobol total indices, and partial variances.
\end{itemize}

%%%%% ROM Model - MSR  %%%%%%%
\subsubsection{MSR}
\label{subsubsec:MSR}
The MSR sub-type contains a class of ROMs that perform a topological
decomposition of the data into approximately monotonic regions and fits weighted
linear patches to the identified monotonic regions of the input space. Query
points have estimated probabilities that they belong to each cluster. These
probabilities can eitehr be used to give a smooth, weighted prediction based on
the associated linear models, or a hard classification to a particular local
linear model which is then used for prediction. Currently, the probability
prediction can be done using kernel density estimation (KDE) or through a
one-versus-one support vector machine (SVM).
%

In order to use this ROM, the \xmlNode{ROM} attribute \xmlAttr{subType} needs to
be \xmlString{MSR} (see the associated example).
%
\subnodesIntro
%
\begin{itemize}
  \item \xmlNode{persistence}, \xmlDesc{string, optional field}, specifies how
  to define the hierarchical simplification by assigning a value to each local
  minimum and maximum according to the one of the strategy options below:
  \begin{itemize}
    \item \texttt{difference} - The function value difference between the
    extremum and its closest-valued neighboring saddle.
    \item \texttt{probability} - The probability integral computed as the
    sum of the probability of each point in a cluster divided by the count of
    the cluster.
    \item \texttt{count} - The count of points that flow to or from the
    extremum.
    % \item \xmlString{area} - The area enclosed by the manifold that flows to
    % or from the extremum.
  \end{itemize}
  \default{\texttt{difference}}
  \item \xmlNode{gradient}, \xmlDesc{string, optional field}, specifies the
  method used for estimating the gradient, available options are:
  \begin{itemize}
    \item \texttt{steepest}
    %\item \texttt{maxflow} \textit{(disabled)}
  \end{itemize}
  \default{\texttt{steepest}}
  \item \xmlNode{simplification}, \xmlDesc{float, optional field}, specifies the
  amount of noise reduction to apply before returning labels.
  \default{0}
  \item \xmlNode{graph} \xmlDesc{, string, optional field}, specifies the type
  of neighborhood graph used in the algorithm, available options are:
  \begin{itemize}
    \item \texttt{beta skeleton}
    \item \texttt{relaxed beta skeleton}
    \item \texttt{approximate knn}
    %\item \texttt{delaunay} \textit{(disabled)}
  \end{itemize}
  \default{\texttt{beta skeleton}}
  \item \xmlNode{beta}, \xmlDesc{float in the range: (0,2], optional field}, is
  only used when the \xmlNode{graph} is set to \texttt{beta skeleton} or
  \texttt{relaxed beta skeleton}.
  \default{1.0}
  \item \xmlNode{knn}, \xmlDesc{integer, optional field}, is the number of
  neighbors when using the \texttt{approximate knn} for the \xmlNode{graph}
  sub-node and used to speed up the computation of other graphs by using the
  approximate knn graph as a starting point for pruning. -1 means use a fully
  connected graph.
  \default{-1}
  % \item \xmlNode{weighted}, \xmlDesc{boolean, optional}, a flag that specifies
  % whether the regression models should be probability weighted.
  % \default{False}
  \item \xmlNode{partitionPredictor}, \xmlDesc{string, optional}, a flag that
  specifies how the predictions for query point classification should be
  performed. Available options are:
  \begin{itemize}
    \item \texttt{kde}
    \item \texttt{svm}
  \end{itemize}
  \default{kde}
  \item \xmlNode{smooth}, if this node is present, the ROM will blend the
  estimates of all of the local linear models weighted by the probability the
  query point is classified as belonging to that partition of the input space.
  \item \xmlNode{kernel}, \xmlDesc{string, optional field}, this option is only
  used when the \xmlNode{partitionPredictor} is set to \texttt{kde} and
  specifies the type of kernel to use in the kernel density estimation.
  Available options are:
  \begin{itemize}
    \item \texttt{uniform}
    \item \texttt{triangular}
    \item \texttt{gaussian}
    \item \texttt{epanechnikov}
    \item \texttt{biweight} or \texttt{quartic}
    \item \texttt{triweight}
    \item \texttt{tricube}
    \item \texttt{cosine}
    \item \texttt{logistic}
    \item \texttt{silverman}
    \item \texttt{exponential}
  \end{itemize}
  \default{gaussian}
  \item \xmlNode{bandwidth}, \xmlDesc{float or string, optional field}, this
  option is only used when the \xmlNode{partitionPredictor} is set to
  \texttt{kde} and specifies the scale of the fall-off. A higher bandwidth
  implies a smooother blending. If set to \texttt{variable}, then the bandwidth
  will be set to the distance of the $k$-nearest neighbor of the query point
  where $k$ is set by the \xmlNode{knn} parameter.
  \default{1.}
\end{itemize}

\zNormalizationNotPerformed{MSR}

\textbf{Example:}
\begin{lstlisting}[style=XML,morekeywords={name,subType}]
<Simulation>
  ...
  <Models>
    ...
    </ROM>
    <ROM name='aUserDefinedName' subType='MSR'>
       <Features>var1,var2,var3</Features>
       <Target>result1,result2</Target>
       <!-- <weighted>true</weighted> -->
       <simplification>0.0</simplification>
       <persistence>difference</persistence>
       <gradient>steepest</gradient>
       <graph>beta skeleton</graph>
       <beta>1</beta>
       <knn>8</knn>
       <partitionPredictor>kde</partitionPredictor>
       <kernel>gaussian</kernel>
       <smooth/>
       <bandwidth>0.2</bandwidth>
     </ROM>
    ...
  </Models>
  ...
</Simulation>
\end{lstlisting}

%%%%% ROM Model - NDinvDistWeight  %%%%%%%
\subsubsection{NDinvDistWeight}
\label{subsubsec:NDinvDistWeight}
The NDinvDistWeight sub-type contains a single ROM type, based on an
$N$-dimensional inverse distance weighting formulation.
%
Inverse distance weighting (IDW) is a type of deterministic method for
multivariate interpolation with a known scattered set of points.
%
The assigned values to unknown points are calculated via a weighted average of
the values available at the known points.
%

In order to use this Reduced Order Model, the \xmlNode{ROM} attribute
\xmlAttr{subType} needs to be xmlString{NDinvDistWeight} (see the example
below).
%
\subnodeIntro

\begin{itemize}
  \item \xmlNode{p}, \xmlDesc{integer, required field}, must be greater than
  zero and represents the ``power parameter''.
  %
  For the choice of value for \xmlNode{p},it is necessary to consider the degree
  of smoothing desired in the interpolation/extrapolation, the density and
  distribution of samples being interpolated, and the maximum distance over
  which an individual sample is allowed to influence the surrounding ones (lower
  $p$ means greater importance for points far away).
  %
\end{itemize}

\zNormalizationPerformed{NDinvDistWeight}

\textbf{Example:}
\begin{lstlisting}[style=XML,morekeywords={name,subType}]
<Simulation>
  ...
  <Models>
    ...
    <ROM name='aUserDefinedName' subType='NDinvDistWeight'>
      <Features>var1,var2,var3</Features>
      <Target>result1,result2</Target>
      <p>3</p>
     </ROM>
    ...
  </Models>
  ...
</Simulation>
\end{lstlisting}


%%%%% ROM Model - SciKitLearn  %%%%%%%
\subsubsection{SciKitLearn}
\label{subsubsec:SciKitLearn}
The SciKitLearn sub-type represents the container of several ROMs available in
RAVEN through the external library scikit-learn~\cite{SciKitLearn}.
%

In order to use this Reduced Order Model, the \xmlNode{ROM} attribute
\xmlAttr{subType} needs to be \\ \xmlString{SciKitLearn} (i.e.
\xmlAttr{subType}\textbf{\texttt{=}}\xmlString{SciKitLearn}).
%
The specifications of a \xmlString{SciKitLearn} ROM depend on the value assumed
by the following sub-node within the main \xmlNode{ROM} XML node:
\begin{itemize}
  \item \xmlNode{SKLtype}, \xmlDesc{vertical bar (\texttt{|}) separated string,
  required field}, contains a string that represents the ROM type to be used.
  %
  As mentioned, its format is:\\
  \xmlNode{SKLtype}\texttt{mainSKLclass|algorithm}\xmlNode{/SKLtype} where the
  first word (before the ``\texttt{|}'' symbol) represents the main class of
  algorithms, and the second word (after the ``\texttt{|}'' symbol) represents
  the specific algorithm.
  %
\end{itemize}
Based on the \xmlNode{SKLtype} several different algorithms are available.
%
\nb for HistorySet's \xmlString{SciKitLearn} performs the task given in \xmlNode{SKLtype} for
each time step, and only synchronized HistorySet can be used as input to this ROM. For
unsynchronized HistorySet, use \xmlString{HistorySetSync} method in \xmlString{Interfaced}
post-processor to synchronize the input data before using \xmlString{SciKitLearn}.

In the following paragraphs a brief explanation and the input requirements are
reported for each of them.
%
%%%%% ROM Model - SciKitLearn: Linear Models %%%%%%%
\paragraph{Linear Models}
\label{LinearModels}
The LinearModels' algorithms implement generalized linear models.
%
They include Ridge regression, Bayesian regression, lasso, and elastic net
estimators computed with least angle regression and coordinate descent.
%
This class also implements stochastic gradient descent related algorithms.
%
In the following, all of the linear models available in RAVEN are reported.
%
\subparagraph{Linear Model: Automatic Relevance Determination Regression}
\mbox{}
\\The \textit{Automatic Relevance Determination} (ARD) regressor is a
hierarchical Bayesian approach where hyperparameters explicitly represent the
relevance of different input features.
%
These relevance hyperparameters determine the range of variation for the
parameters relating to a particular input, usually by modelling the width of a
zero-mean Gaussian prior on those parameters.
%
If the width of the Gaussian is zero, then those parameters are constrained to
be zero, and the corresponding input cannot have any effect on the predictions,
therefore making it irrelevant.
%
ARD optimizes these hyperparameters to discover which inputs are relevant.
%
\skltype{Automatic Relevance Determination regressor}{linear\_model|ARDRegression}.
\begin{itemize}
  \item \nIterDescriptionA{300}
  \item \tolDescriptionA{1.e-3}
  \item \xmlNode{alpha\_1}, \xmlDesc{float, optional field}, is a shape
  hyperparameter for the Gamma distribution prior over the $\alpha$ parameter.
  \default{ 1.e-6}
  %
  \item \xmlNode{alpha\_2}, \xmlDesc{float, optional field}, inverse scale
  hyperparameter (rate parameter) for the Gamma distribution prior over the
  $\alpha$ parameter.
  \default{ 1.e-6}
  %
  \item \xmlNode{lambda\_1}, \xmlDesc{float, optional field}, shape
  hyperparameter for the Gamma distribution prior over the $\lambda$ parameter.
  \default{ 1.e-6}
  %
  \item \xmlNode{lambda\_2}, \xmlDesc{float, optional field}, inverse scale
  hyperparameter (rate parameter) for the Gamma distribution prior over the
  $\lambda$ parameter.
  \default{ 1.e-6}
  %
  \item \xmlNode{compute\_score}, \xmlDesc{boolean, optional field}, if True,
  compute the objective function at each step of the model.
  \default{False}
  %
  \item \xmlNode{threshold\_lambda}, \xmlDesc{float, optional field}, specifies
  the threshold for removing (pruning) weights with high precision from the
  computation.
  \default{ 1.e+4}
  %
  \item \fitInterceptDescription{True}
  %
  \item \normalizeDescription{False}
  %
  \item \verDescriptionA{False}
\end{itemize}

\zNormalizationNotPerformed{ARDRegression}
%%%%%%%%
\subparagraph{Linear Model: Bayesian ridge regression}
\mbox{}

The \textit{Bayesian ridge regression} estimates a probabilistic model of the
regression problem as described above.
%
The prior for the parameter $w$ is given by a spherical Gaussian:
\begin{equation}
p(w|\lambda) =\mathcal{N}(w|0,\lambda^{-1}\bold{I_{p}})
\end{equation}
The priors over $\alpha$ and $\lambda$ are chosen to be gamma distributions, the
conjugate prior for the precision of the Gaussian.
%
The resulting model is called Bayesian ridge regression, and is similar to the
classical ridge regression.
%
The parameters $w$, $\alpha$, and $\lambda$ are estimated jointly during the fit
of the model.
%
The remaining hyperparameters are the parameters of the gamma priors over
$\alpha$ and $\lambda$.
%
These are usually chosen to be non-informative.
%
The parameters are estimated by maximizing the marginal log likelihood.
%
\skltype{Bayesian ridge regressor}{linear\_model|BayesianRidge}
\begin{itemize}
  \item \nIterDescriptionA{300}
  \item \tolDescriptionA{1.e-3}
  \item \xmlNode{alpha\_1}, \xmlDesc{float, optional field}, is a shape
  hyperparameter for the Gamma distribution prior over the $\alpha$ parameter.
  \default{ 1.e-6}
  %
  \item \xmlNode{alpha\_2}, \xmlDesc{float, optional field}, inverse scale
  hyperparameter (rate parameter) for the Gamma distribution prior over the
  $\alpha$ parameter.
  \default{ 1.e-6}
  %
  \item \xmlNode{lambda\_1}, \xmlDesc{float, optional field}, shape
  hyperparameter for the Gamma distribution prior over the $\lambda$ parameter.
  \default{ 1.e-6}
  %
  \item \xmlNode{lambda\_2}, \xmlDesc{float, optional field}, inverse scale
  hyperparameter (rate parameter) for the Gamma distribution prior over the
  $\lambda$ parameter.
  \default{ 1.e-6}
  %
  \item \xmlNode{compute\_score}, \xmlDesc{boolean, optional field}, if True,
  compute the objective function at each step of the model.
  \default{False}
  %
  \item \fitInterceptDescription{True}
  \item \normalizeDescription{False}
  \item \verDescriptionA{False}
\end{itemize}

\zNormalizationNotPerformed{BayesianRidge}
%%%%%%%
\subparagraph{Linear Model: Elastic Net}
\mbox{}
\\The \textit{Elastic Net} is a linear regression technique with combined L1 and
L2 priors as regularizers.
%
It minimizes the objective function:
\begin{equation}
1/(2*n_{samples}) *||y - Xw||^2_2+alpha*l1\_ratio*||w||_1 + 0.5 *alpha*(1 - l1\_ratio)*||w||^2_2
\end{equation}

\skltype{Elastic Net regressor}{linear\_model|ElasticNet}
\begin{itemize}
  \item \xmlNode{alpha}, \xmlDesc{float, optional field}, specifies a constant
  that multiplies the penalty terms.
  %
  $alpha = 0$ is equivalent to an ordinary least square, solved by the
  \textbf{LinearRegression} object.
  \default{1.0}
  %
  \item \xmlNode{l1\_ratio}, \xmlDesc{float, optional field}, specifies the
  ElasticNet mixing parameter, with $0 <= l1\_ratio <= 1$.
  %
  For $l1\_ratio = 0$ the penalty is an L2 penalty.
  %
  For $l1\_ratio = 1$ it is an L1 penalty.
  %
  For $0 < l1\_ratio < 1$, the penalty is a combination of L1 and L2.
  %
  \default{0.5}
  \item \fitInterceptDescription{True}
  \item \normalizeDescription{False}
  \item \maxIterDescription{1000}
  \item \tolDescriptionB{1.e-4}
  \item \warmStartDescription{False}
  \item \positiveDescription{False}
  %
\end{itemize}
\zNormalizationNotPerformed{ElasticNet}
%%%%%%%%
\subparagraph{Linear Model: Elastic Net CV}
\mbox{}
\\The \textit{Elastic Net CV} is a linear regression similar to the Elastic Net
model but with an iterative fitting along a regularization path.
%
The best model is selected by cross-validation.
%

\skltype{Elastic Net CV regressor}{linear\_model|ElasticNetCV}
\begin{itemize}
  \item \xmlNode{l1\_ratio}, \xmlDesc{float, optional field},
  %
  Float flag between 0 and 1 passed to ElasticNet (scaling between l1 and l2
  penalties).
  %
  For $l1\_ratio = 0$ the penalty is an L2 penalty.
  %
  For $l1\_ratio = 1$ it is an L1 penalty.
  %
  For $0 < l1\_ratio < 1$, the penalty is a combination of L1 and L2 This
  parameter can be a list, in which case the different values are tested by
  cross-validation and the one giving the best prediction score is used.
  %
  Note that a good choice of list of values for $l1\_ratio$ is often to put more
  values close to 1 (i.e. Lasso) and less close to 0 (i.e. Ridge), as in [.1,
  .5, .7, .9, .95, .99, 1].
  %
  \default{0.5}
  \item \xmlNode{eps}, \xmlDesc{float, optional field}, specifies the length of
  the path.
  %
  eps=1e-3 means that $alpha\_min / alpha\_max = 1e-3$.
  %
  \default{0.001}
  \item \xmlNode{n\_alphas}, \xmlDesc{integer, optional field}, is the number of
  alphas along the regularization path used for each $l1\_ratio$.
  %
  \default{100}
  \item \precomputeDescription{'auto'}
  \item \maxIterDescription{1000}
  \item \tolDescriptionB{1.e-4}
  %
  \item \positiveDescription{False}
  %
\end{itemize}
\zNormalizationNotPerformed{ElasticNetCV}
%%%%%%
\subparagraph{Linear Model: Least Angle Regression model}
\mbox{}
\\The \textit{Least Angle Regression model} (LARS) is a regression algorithm for
high-dimensional data.
%
The LARS algorithm provides a means of producing an estimate of which variables
to include, as well as their coefficients, when a response variable is
determined by a linear combination of a subset of potential covariates.
%

\skltype{Least Angle Regression model}{linear\_model|Lars}
\begin{itemize}
  \item \xmlNode{n\_nonzero\_coefs}, \xmlDesc{integer, optional field},
  represents the target number of non-zero coefficients.
  %
  \default{500}
  \item \fitInterceptDescription{True}
  \item \verDescriptionA{False}
  \item \precomputeDescription{'auto'}
  \item \normalizeDescription{True}
  \item \xmlNode{eps}, \xmlDesc{float, optional field}, represents the machine
  precision regularization in the computation of the Cholesky diagonal factors.
  %
  Increase this for very ill-conditioned systems.
  %
  Unlike the \xmlNode{tol} parameter in some iterative optimization-based
  algorithms, this parameter does not control the tolerance of the optimization.
  %
  \default{2.2204460492503131e-16}
  \item \xmlNode{fit\_path}, \xmlDesc{boolean, optional field}, if True the
  full path is stored in the coef\_path\_attribute.
  %
  If you compute the solution for a large problem or many targets, setting
  fit\_path to False will lead to a speedup, especially with a small alpha.
  %
  \default{True}
  %
\end{itemize}
\zNormalizationNotPerformed{Lars}
%%%%%%
\subparagraph{Linear Model: Cross-validated Least Angle Regression model}
\mbox{}
\\The \textit{Cross-validated Least Angle Regression model} is a regression
algorithm for high-dimensional data.
%
It is similar to the LARS method, but the best model is selected by
cross-validation.
%
\skltype{Cross-validated Least Angle Regression model}{linear\_model|LarsCV}
\begin{itemize}
  \item \fitInterceptDescription{True}
  \item \verDescriptionA{False}
  \item \normalizeDescription{True}
  \item \precomputeDescription{'auto'}
  \item \maxIterDescription{500}
  \item \nAlphasDescription{1000}
  \item \xmlNode{eps}, \xmlDesc{float, optional field}, represents the
  machine-precision regularization in the computation of the Cholesky diagonal
  factors.
  %
  Increase this for very ill-conditioned systems.
  %
  Unlike the \textit{tol} parameter in some iterative optimization-based
  algorithms, this parameter does not control the tolerance of the optimization.
  %
  \default{2.2204460492503131e-16}
\end{itemize}
\subparagraph{Linear Model trained with L1 prior as regularizer (aka the Lasso)}
\mbox{}
\\The \textit{Linear Model trained with L1 prior as regularizer (Lasso)} is a
shrinkage and selection method for linear regression.
%
It minimizes the usual sum of squared errors, with a bound on the sum of the
absolute values of the coefficients.
%
\skltype{Linear Model trained with L1 prior as regularizer
  (Lasso)}{linear\_model|Lasso}
\begin{itemize}
  \item \xmlNode{alpha}, \xmlDesc{float, optional field}, sets a constant
  multiplier for the L1 term.
  %
  alpha = 0 is equivalent to an ordinary least square, solved by the
  LinearRegression object.
  %
  For numerical reasons, using alpha = 0 with the Lasso object is not advised
  and you should instead use the LinearRegression object.
  %
  \default{1.0}
  %
  \item \fitInterceptDescription{True}
  \item \normalizeDescription{False}
  \item \precomputeDescription{False}
  \nb For sparse input this option is always True to preserve sparsity.
  \item \maxIterDescription{1000}
  \item \tolDescriptionB{1.e-4}
  \item \warmStartDescription{False}
  \item \positiveDescription{False}
\end{itemize}
\zNormalizationNotPerformed{LarsCV}
\subparagraph{Lasso linear model with iterative fitting along a regularization
  path}
\mbox{}

The \textit{Lasso linear model with iterative fitting along a regularization
path} is an algorithm of the Lasso family, that computes the linear regressor weights,
identifying the regularization path in an iterative fitting (see http://www.jstatsoft.org/v33/i01/paper)

\skltype{Lasso linear model with iterative fitting along a regularization path
regressor}{linear\_model|LassoCV}
\begin{itemize}
  \item \xmlNode{eps}, \xmlDesc{float, optional field}, represents the length of
  the path.
  %
  eps=1e-3 means that alpha\_min / alpha\_max = 1e-3.
  %
  \default{1.0e-3}
  %
  \item \xmlNode{n\_alphas}, \xmlDesc{int, optional field}, sets the number of
  alphas along the regularization path.
  %
  \default{100}
  %
  \item \xmlNode{alphas}, \xmlDesc{numpy array, optional field}, lists the
  locations of the alphas used to compute the models.
  %
  \default{None}
  %
  If None, alphas are set automatically.
  \item \precomputeDescription{'auto'}
  \item \maxIterDescription{1000}
  \item \tolDescriptionB{1.e-4}
  \item \verDescriptionB{False}
  \item \positiveDescription{False}
\end{itemize}
\zNormalizationNotPerformed{LassoCV}
\subparagraph{Lasso model fit with Least Angle Regression}
\mbox{}

\textit{Lasso model fit with Least Angle Regression} (aka Lars)
It is a Linear Model trained with an L1 prior as regularizer.
In order to use the \textit{Least Angle Regression model regressor}, the user needs to set the sub-node
%
\skltype{Least Angle Regression model
regressor}{linear\_model|LassoLars}

\begin{itemize}
  \item \xmlNode{alpha}, \xmlDesc{float, optional field}, specifies a constant
  that multiplies the penalty terms.
  %
  $alpha = 0$ is equivalent to an ordinary least square, solved by the
  \textbf{LinearRegression} object.
  \default{1.0}
  %
  \item \fitInterceptDescription{True}
  \item \verDescriptionB{False}
  \item \normalizeDescription{True}
  \item \precomputeDescription{'auto'}
  \item \maxIterDescription{500}
  \item \xmlNode{eps}, \xmlDesc{float, optional field}, sets the machine
  precision regularization in the computation of the Cholesky diagonal factors.
  %
  Increase this for very ill-conditioned systems.
  %
  \default{2.2204460492503131e-16}
\end{itemize}
\zNormalizationNotPerformed{LassoLars}
\subparagraph{Cross-validated Lasso, using the LARS algorithm}
\mbox{}

The \textit{Cross-validated Lasso, using the LARS algorithm} is a
cross-validated Lasso, using the LARS algorithm.

\skltype{Cross-validated Lasso, using the LARS algorithm
regressor}{linear\_model|LassoLarsCV}

\begin{itemize}
  \item \fitInterceptDescription{True}
  \item \verDescriptionB{False}
  \item \normalizeDescription{True}
  \item \precomputeDescription{'auto'}
  \item \maxIterDescription{500}
  \item \nAlphasDescription{1000}
  \item \xmlNode{eps}, \xmlDesc{float, optional field}, specifies the machine
  precision regularization in the computation of the Cholesky diagonal factors.
  %
  Increase this for very ill-conditioned systems.
  %
  \default{2.2204460492503131e-16}
\end{itemize}
\zNormalizationNotPerformed{LassoLarsCV}
\subparagraph{Lasso model fit with Lars using BIC or AIC for model selection}
\mbox{}

The \textit{Lasso model fit with Lars using BIC or AIC for model selection} is
a Lasso model fit with Lars using BIC or AIC for model selection.
%\maljdan{redundant}
The optimization objective for Lasso is:
$(1 / (2 * n\_samples)) * ||y - Xw||^2_2 + alpha * ||w||_1$
AIC is the Akaike information criterion and BIC is the Bayes information
criterion.
%
Such criteria are useful in selecting the value of the regularization parameter
by making a trade-off between the goodness of fit and the complexity of the
model.
%
A good model explains the data well while maintaining simplicity.
%
\skltype{Lasso model fit with Lars using BIC or AIC for
  model selection regressor}{linear\_model|LassoLarsIC}
\begin{itemize}
  \item \xmlNode{criterion}, \xmlDesc{`bic' | `aic' }, specifies the type of
  criterion to use.
  %
  \default{'aic'}
  %
  \item \fitInterceptDescription{True}
  \item \verDescriptionB{False}
  \item \normalizeDescription{True}
  \item \precomputeDescription{'auto'}
  \item \maxIterDescription{500}
  \item \xmlNode{eps}, \xmlDesc{float, optional field}, represents the machine
  precision regularization in the computation of the Cholesky diagonal factors.
  %
  Increase this for very ill-conditioned systems.
  %
  Unlike the tol parameter in some iterative optimization-based algorithms, this
  parameter does not control the tolerance of the optimization.
  %
  %
  \default{2.2204460492503131e-16}
\end{itemize}
\zNormalizationNotPerformed{LassoLarsIC}
\subparagraph{Ordinary least squares Linear Regression}
\mbox{}

The \textit{Ordinary least squares Linear Regression} is a method for
estimating the unknown parameters in a linear regression model, with the goal of
minimizing the differences between the observed responses in some arbitrary
dataset and the responses predicted by the linear approximation of the data.
%
\skltype{Ordinary least squares Linear
Regressor}{linear\_model|LinearRegression}

\begin{itemize}
  \item \fitInterceptDescription{True}
  \item \normalizeDescription{False}
\end{itemize}
\zNormalizationNotPerformed{LinearRegression}
\subparagraph{Logistic Regression}
\mbox{}
\\The \textit{Logistic Regression} implements L1 and L2 regularized logistic
regression using the liblinear library.
%
It can handle both dense and sparse input.
%
This regressor uses C-ordered arrays or CSR matrices containing 64-bit floats
for optimal performance; any other input format will be converted (and copied).
%
\skltype{Logistic Regressor}{linear\_model|LogisticRegression}
\begin{itemize}
  \item \xmlNode{penalty}, \xmlDesc{string, `l1' or `l2'}, specifies the norm
  used in the penalization.
  %
  \default{'l2'}
  %}
  \item \xmlNode{dual}, \xmlDesc{boolean}, specifies the dual or primal
  formulation.
  %
  Dual formulation is only implemented for the l2 penalty.
  %
  Prefer dual=False when n\_samples $>$ n\_features.
  %
  \default{False}
  %
  \item \xmlNode{C}, \xmlDesc{float, optional field}, is the inverse of the
  regularization strength; must be a positive float.
  %
  Like in support vector machines, smaller values specify stronger
  regularization.
  %
  \default{1.0}
  \item \xmlNode{fit\_intercept}, \xmlDesc{boolean}, specifies if a constant
  (a.k.a. bias or intercept) should be added to the decision function.
  %
  \default{True}
  \item \xmlNode{intercept\_scaling}, \xmlDesc{float, optional field}, when
  self.fit\_intercept is True, instance vector x becomes [x,
  self.intercept\_scaling], i.e. a ``synthetic'' feature with constant value
  equal to intercept\_scaling is appended to the instance vector.
  %
  The intercept becomes intercept\_scaling * synthetic feature
  weight.
  \nb The synthetic feature weight is subject to l1/l2 regularization as are all
  other features.
  %
  To lessen the effect of regularization on synthetic feature weight (and
  therefore on the intercept) intercept\_scaling has to be increased.
  \default{1.0}
  \item \xmlNode{class\_weight}, \xmlDesc{dict, or 'balanced', optional}
  Weights associated with classes in the form \{class\_label: weight\}. If not given, all classes are supposed to have weight one.
  %
  The ``balanced'' mode uses the values of y to automatically adjust weights inversely proportional to class frequencies in the
  input data as n\_samples / (n\_classes * np.bincount(y))
  %
  Note that these weights will be multiplied with sample\_weight (passed through the fit method) if sample\_weight is specified.
  %
  New in version 0.17: class\_weight=’balanced’ instead of deprecated class\_weight=’auto’.
  %
  \default{None}
  %
  \item \randomStateDescription{None}
  \item \tolDescriptionC{0.0001}
\end{itemize}
\zNormalizationPerformed{LogisticRegression}
\subparagraph{Multi-task Lasso model trained with L1/L2 mixed-norm as
  regularizer}
\mbox{}
\\The \textit{Multi-task Lasso model trained with L1/L2 mixed-norm as
  regularizer} is a regressor where the optimization objective for Lasso is:
$(1 / (2 * n\_samples)) * ||Y - XW||^2_{Fro} + alpha * ||W||_{21}$
Where:
$||W||_{21} = \sum_i \sqrt{\sum_j w_{ij}^2}$
i.e. the sum of norm of each row.
%
\skltype{Multi-task Lasso model trained with L1/L2
  mixed-norm as regularizer regressor}{linear\_model|MultiTaskLasso}
\begin{itemize}
  \item \xmlNode{alpha}, \xmlDesc{float, optional field}, sets the constant
  multiplier for the L1/L2 term.
  %
  \default{1.0}
  \item \fitInterceptDescription{True}
  \item \normalizeDescription{False}
  \item \maxIterDescription{1000}
  \item \tolDescriptionB{1.e-4}
  \item \warmStartDescription{False}
\end{itemize}
\zNormalizationNotPerformed{MultiTaskLasso}
\subparagraph{Multi-task Elastic Net model trained with L1/L2 mixed-norm as
  regularizer}
\mbox{}

The \textit{Multi-task Elastic Net model trained with L1/L2 mixed-norm as
  regularizer} is a regressor where the optimization objective for
MultiTaskElasticNet is:
$(1 / (2 * n\_samples)) * ||Y - XW||^{Fro}_2
+ alpha * l1\_ratio * ||W||_{21}
+ 0.5 * alpha * (1 - l1\_ratio) * ||W||_{Fro}^2$
Where:
$||W||_{21} = \sum_i \sqrt{\sum_j w_{ij}^2}$
i.e. the sum of norm of each row.
%
\skltype{Multi-task ElasticNet model trained with L1/L2
  mixed-norm as regularizer regressor}{linear\_model|MultiTaskElasticNet}
\begin{itemize}
  \item \xmlNode{alpha}, \xmlDesc{float, optional field}, represents a constant
  multiplier for the L1/L2 term.
  %
  \default{1.0}
  \item \xmlNode{l1\_ratio}, \xmlDesc{float}, represents the Elastic Net mixing
  parameter, with $0 < l1\_ratio \leq 1$.
  %
  For $l1\_ratio = 0$ the penalty is an L1/L2 penalty.
  %
  For $l1\_ratio = 1$ it is an L1 penalty.
  %
  For $0 < l1\_ratio < 1$, the penalty is a combination of L1/L2
  and L2.
  %
  \default{0.5}
  %
  \item \fitInterceptDescription{True}
  \item \normalizeDescription{False}
  \item \maxIterDescription{}
  \item \tolDescriptionB{1.e-4}
  \item \warmStartDescription{False}
\end{itemize}
\zNormalizationNotPerformed{MultiTaskElasticNet}
\subparagraph{Orthogonal Mathching Pursuit model (OMP)}
\mbox{}

The \textit{Orthogonal Mathching Pursuit model (OMP)} is a type of sparse
approximation which involves finding the ``best matching'' projections of
multidimensional data onto an over-complete dictionary, $D$.
%
\skltype{Orthogonal Mathching Pursuit model (OMP)
regressor}{linear\_model|OrthogonalMatchingPursuit}
\begin{itemize}
  \item \xmlNode{n\_nonzero\_coefs}, \xmlDesc{int, optional field}, represents
  the desired number of non-zero entries in the solution.
  %
  If None, this value is set to 10\% of n\_features.
  %
  \default{None}
  \item \xmlNode{tol}, \xmlDesc{float, optional field}, specifies the maximum
  norm of the residual.
  %
  If not None, overrides n\_nonzero\_coefs.
  %
  \default{None}
  %
  \item \fitInterceptDescription{True}
  \item \normalizeDescription{True}
  \item \xmlNode{precompute}, \xmlDesc{\{True, False, `auto'\}}, specifies
  whether to use a precomputed Gram and Xy matrix to speed up calculations.
  %
  Improves performance when n\_targets or n\_samples is very large.
  %
  \nb If you already have such matrices, you can pass them directly to the
  fit method.
  %
  \default{`auto'}
\end{itemize}
\zNormalizationNotPerformed{OrthogonalMatchingPursuit}
\subparagraph{Cross-validated Orthogonal Mathching Pursuit model (OMP)}
\mbox{}

The \textit{Cross-validated Orthogonal Mathching Pursuit model (OMP)} is a
regressor similar to OMP which has good performance in sparse recovery.
%
\skltype{Cross-validated Orthogonal Mathching Pursuit model (OMP)
regressor}{linear\_model|OrthogonalMatchingPursuitCV}
\begin{itemize}
  \item \fitInterceptDescription{True}
  \item \normalizeDescription{True}
  \item \maxIterDescription{None}
  %
  Maximum number of iterations to perform, therefore maximum features to
  include 10\% of n\_features but at least 5 if available.
  %
  \item \xmlNode{cv}, \xmlDesc{cross-validation generator, optional},
  %
  see sklearn.cross\_validation.
  %
  \default{None}
  \item \verDescriptionB{False}
\end{itemize}
\zNormalizationNotPerformed{OrthogonalMatchingPursuitCV}
\subparagraph{Passive Aggressive Classifier}
\mbox{}
\\The \textit{Passive Aggressive Classifier} is a principled approach to linear
classification that advocates minimal weight updates i.e., the least required
to correctly classify the current training instance.
%
\skltype{Passive Aggressive
Classifier}{linear\_model|PassiveAggressiveClassifier}
\begin{itemize}
  \item \xmlNode{C}, \xmlDesc{float}, specifies the maximum step size
  (regularization).
  %
  \default{1.0}
  %
  \item \fitInterceptDescription{True}
  \item \shuffleDescription{True}
  \item \nIterNoChangeDescriptionA{5}
  \item \randomStateDescription{None}
  \item \verDescriptionB{0}
  \item \xmlNode{loss}, \xmlDesc{string, optional field}, the loss function to
  be used:
  \begin{itemize}
    \item hinge: equivalent to PA-I (http://jmlr.csail.mit.edu/papers/volume7/crammer06a/crammer06a.pdf)
    \item squared\_hinge: equivalent to PA-II (http://jmlr.csail.mit.edu/papers/volume7/crammer06a/crammer06a.pdf)
  \end{itemize}
  %
  \default{'hinge'}
  %
  \item \warmStartDescription{False}
\end{itemize}
\zNormalizationPerformed{PassiveAggressiveClassifier}
\subparagraph{Passive Aggressive Regressor}
\mbox{}
\\The \textit{Passive Aggressive Regressor} is similar to the Perceptron in that
it does not require a learning rate.
%
However, contrary to the Perceptron, this regressor includes a regularization
parameter, $C$.

\skltype{Passive Aggressive Regressor}{linear\_model|PassiveAggressiveRegressor}
\begin{itemize}
  \item \xmlNode{C}, \xmlDesc{float}, sets the maximum step size
  (regularization).
  %
  \default{1.0}
  %
  \item \xmlNode{epsilon}, \xmlDesc{float}, if the difference between the
  current prediction and the correct label is below this threshold, the model is
  not updated.
  %
  \default{0.1}
  %
  \item \fitInterceptDescription{True}
  \item \nIterDescriptionB{5}
  \item \shuffleDescription{True}
  \item \randomStateDescription{None}
  \item \verDescriptionB{0}
  \item \xmlNode{loss}, \xmlDesc{string, optional field}, specifies the loss
  function to be used:
  \begin{itemize}
    \item epsilon\_insensitive: equivalent to PA-I in the reference paper (http://jmlr.csail.mit.edu/papers/volume7
    /crammer06a/crammer06a.pdf).
    \item squared\_epsilon\_insensitive: equivalent to PA-II in the reference paper (http://jmlr.csail.mit.edu/papers
    /volume7/crammer06a/crammer06a.pdf).
  \end{itemize}
  %
  \default{'epsilon\_insensitive'}
  %
  \item \warmStartDescription{False}
\end{itemize}
\zNormalizationPerformed{PassiveAggressiveRegressor}
\subparagraph{Perceptron}
\mbox{}

The \textit{Perceptron} method is an algorithm for supervised classification of
an input into one of several possible non-binary outputs.
%
It is a type of linear classifier, i.e. a classification algorithm that makes
its predictions based on a linear predictor function combining a set of weights
with the feature vector.
%
The algorithm allows for online learning, in that it processes elements in the
training set one at a time.
%
\skltype{Perceptron classifier}{linear\_model|Perceptron}
\begin{itemize}
  \item \xmlNode{penalty}, \xmlDesc{None, `l2' or `l1' or `elasticnet'}, defines
  the penalty (aka regularization term) to be used.
  %
  \default{None}
  %
  \item \xmlNode{alpha}, \xmlDesc{float}, sets the constant multiplier for the
  regularization term if regularization is used.
  %
  \default{0.0001}
  \item \fitInterceptDescription{True}
  \item \nIterNoChangeDescriptionA{5}
  \item \shuffleDescription{True}
  \item \randomStateDescription{0}
  \item \verDescriptionB{0}
  \item \xmlNode{eta0}, \xmlDesc{double, optional field}, defines the constant
  multiplier for the updates.
  %
  \default{1.0}
  %
  \item \xmlNode{class\_weight}, \xmlDesc{dict, \{class\_label: weight\} or “balanced” or None, optional}
  Preset for the class\_weight fit parameter.
  %
  Weights associated with classes. If not given, all classes are supposed to have weight one.
  %
  The “balanced” mode uses the values of y to automatically adjust weights inversely proportional to class
  frequencies in the input data as n\_samples / (n\_classes * np.bincount(y))
  %
  \item \warmStartDescription{False}
\end{itemize}
\zNormalizationPerformed{PassiveAggressiveRegressor}
\subparagraph{Linear least squares with l2 regularization}
\mbox{}
\\The \textit{Linear least squares with l2 regularization} solves a regression
model where the loss function is the linear least squares function and the
regularization is given by the l2-norm.
%
Also known as Ridge Regression or Tikhonov regularization.
%
This estimator has built-in support for multivariate regression (i.e., when y
is a 2d-array of shape [n\_samples, n\_targets]).
%
\skltype{Linear least squares with l2 regularization}{linear\_model|Ridge}
\begin{itemize}
  \item \xmlNode{alpha}, \xmlDesc{float, array-like},
  %
  shape = [n\_targets] Small positive values of alpha improve the
  conditioning of the problem and reduce the variance of the estimates.
  %
  Alpha corresponds to $(2*C)^-1$ in other linear models such as
  LogisticRegression or LinearSVC.
  %
  If an array is passed, penalties are assumed to be specific to the targets.
  %
  Hence they must correspond in number.
  %
  \default{1.0}
  %
  \item \fitInterceptDescription{True}
  \item \nIterNoChangeDescriptionA{5}
  \item \maxIterDescription{determined by scipy.sparse.linalg.}
  \item \normalizeDescription{False}
  \item \solverDescription
  \default{`auto'}
\end{itemize}
\zNormalizationNotPerformed{Ridge}
%TODO document copy_X
%TODO document tol
%TODO document random_state

\subparagraph{Classifier using Ridge regression}
\mbox{}

The \textit{Classifier using Ridge regression} is a classifier based on linear
least squares with l2 regularization.
\skltype{Classifier using Ridge regression}{linear\_model|RidgeClassifier}

\begin{itemize}
  \item \xmlNode{alpha}, \xmlDesc{float}, small positive values of alpha improve
  the conditioning of the problem and reduce the variance of the estimates.
  %
  Alpha corresponds to $(2*C)^-1$ in other linear models such as
  LogisticRegression or LinearSVC.
  %
  \default{1.0}
  %
  \item \xmlNode{class\_weight}, \xmlDesc{dict, optional field}, specifies
  weights associated with classes in the form {class\_label: weight}.
  %
  If not given, all classes are assumed to have weight one.
  %
  \default{None}
  %
  \item \fitInterceptDescription{True}
  \item \maxIterDescription{determined by scipy.sparse.linalg.}
  \item \normalizeDescription{False}
  \item \solverDescription
  \default{`auto'}
  \item \xmlNode{tol}, \xmlDesc{float}, defines the required precision of the
  solution.
  \default{0.001}
\end{itemize}
\zNormalizationNotPerformed{RidgeClassifier}
%TODO document random_state
%TODO document copy_X

\subparagraph{Ridge classifier with built-in cross-validation}
\mbox{}
\\The \textit{Ridge classifier with built-in cross-validation} performs
Generalized Cross-Validation, which is a form of efficient leave-one-out
cross-validation.
%
Currently, only the n\_features $>$ n\_samples case is handled efficiently.
%
\skltype{Ridge classifier with built-in cross-validation
classifier}{linear\_model|RidgeClassifierCV}
\begin{itemize}
  \item \xmlNode{alphas}, \xmlDesc{numpy array of shape [n\_alphas]}, is an
  array of alpha values to try.
  %
  Small positive values of alpha improve the conditioning of the problem and
  reduce the variance of the estimates.
  %
  Alpha corresponds to $(2*C)^{-1}$ in other linear models such as
  LogisticRegression or LinearSVC.
  %
  \default{(0.1, 1.0, 10.0)}
  %
  \item \fitInterceptDescription{True}
  \item \normalizeDescription{False}
  \item \xmlNode{scoring}, \xmlDesc{string, callable or None, optional}, is a
  string (see model evaluation documentation) or a scorer callable object /
  function with signature scorer(estimator, X, y).
  %
  \default{None}
  \item \xmlNode{cv}, \xmlDesc{cross-validation generator, optional},
  %
  If None, Generalized Cross-Validation (efficient leave-one-out) will be used.
  %
  \default{None}
  %
  \item \xmlNode{class\_weight}, \xmlDesc{dic, optional field}, specifies
  weights associated with classes in the form {class\_label:weight}.
  %
  If not given, all classes are supposed to have weight one.
  %
  \default{None}
  %
\end{itemize}
\zNormalizationNotPerformed{RidgeClassifierCV}
\subparagraph{Ridge regression with built-in cross-validation}
\mbox{}

The \textit{Ridge regression with built-in cross-validation} performs
Generalized Cross-Validation, which is a form of efficient leave-one-out
cross-validation.
%
\skltype{Ridge regression with built-in cross-validation regressor}{linear\_model|RidgeCV}
\begin{itemize}
  \item \xmlNode{alphas}, \xmlDesc{numpy array of shape [n\_alphas]}, specifies
  an array of alpha values to try.
  %
  Small positive values of alpha improve the conditioning of the problem and
  reduce the variance of the estimates.
  %
  Alpha corresponds to $(2*C)^{-1}$ in other linear models such as
  LogisticRegression or LinearSVC.
  %
  \default{(0.1, 1.0, 10.0)}
  \item \fitInterceptDescription{True}
  \item \normalizeDescription{False}
  \item \xmlNode{scoring}, \xmlDesc{string, callable or None, optional}, is a
  string (see model evaluation documentation) or a scorer callable object /
  function with signature scorer(estimator, X, y).
  %
  \default{None}
  %
  \item \xmlNode{cv}, \xmlDesc{cross-validation generator, optional field}, if
  None, Generalized Cross-Validation (efficient leave-one-out) will be used.
  %
  \default{None}
  %
  \item \xmlNode{gcv\_mode}, \xmlDesc{\{None, `auto,' `svd,' `eigen'\}, optional
  field}, is a flag indicating which strategy to use when performing Generalized
  Cross-Validation.
  %
  Options are:
	\begin{itemize}
    \item `auto:' use svd if n\_samples > n\_features or when X is a
    sparse matrix, otherwise use eigen
  	\item `svd:' force computation via singular value decomposition of $X$
    (does not work for sparse matrices)
	  \item `eigen:' force computation via eigendecomposition of $X^T X$
	\end{itemize}
	The `auto' mode is the default and is intended to pick the cheaper
  option of the two depending upon the shape and format of the training data.
  %
  \default{None}
  \item \xmlNode{store\_cv\_values}, \xmlDesc{boolean}, is a flag indicating if
  the cross-validation values corresponding to each alpha should be stored in
  the cv\_values\_attribute (see below).
  %
  This flag is only compatible with cv=None (i.e. using Generalized
  Cross-Validation).
  %
  \default{False}
\end{itemize}
\zNormalizationNotPerformed{RidgeCV}
\subparagraph{Linear classifiers (SVM, logistic regression, a.o.) with SGD
training}
\mbox{}

The \textit{Linear classifiers (SVM, logistic regression, a.o.) with SGD
training} implements regularized linear models with stochastic gradient
descent (SGD) learning: the gradient of the loss is estimated for each sample at
a time and the model is updated along the way with a decreasing strength
schedule (aka learning rate).
%
SGD allows minibatch (online/out-of-core) learning, see the partial\_fit method.
%
This implementation works with data represented as dense or sparse arrays of
floating point values for the features.
%
The model it fits can be controlled with the loss parameter; by default, it fits
a linear support vector machine (SVM).
%
The regularizer is a penalty added to the loss function that shrinks model
parameters towards the zero vector using either the squared Euclidean norm L2 or
the absolute norm L1 or a combination of both (Elastic Net).
%
If the parameter update crosses the 0.0 value because of the regularizer, the
update is truncated to 0.0 to allow for learning sparse models and achieves
online feature selection.
%
\skltype{Linear classifiers (SVM, logistic regression, a.o.) with SGD
training}{linear\_model|SGDClassifier}
\begin{itemize}
  \item \xmlNode{loss}, \xmlDesc{str, `hinge,' `log,' `modified\_huber,'
  `squared\_hinge,' `perceptron,' or a regression loss: `squared\_loss,'
  `huber,' `epsilon\_insensitive,' or `squared\_epsilon\_insensitive'},
  %
  dictates the loss function to be used.
  %
  The available options are:
  \begin{itemize}
    \item `hinge' gives a linear SVM.
    \item `log' loss gives logistic regression, a probabilistic classifier.
    \item `modified\_huber' is another smooth loss that brings tolerance to
    outliers as well as probability estimates.
    \item `squared\_hinge' is like hinge but is quadratically penalized.
    \item `perceptron' is the linear loss used by the perceptron algorithm.
  \end{itemize}
  The other losses are designed for regression but can be useful in
  classification as well; see SGDRegressor for a description.
  %
  \default{`hinge'}
  %
  \item \xmlNode{penalty}, \xmlDesc{str, `l2' or `l1' or `elasticnet'}, defines
  the penalty (aka regularization term) to be used.
  %
  `l2' is the standard regularizer for linear SVM models.
  %
  `l1' and `elasticnet' might bring sparsity to the model (feature
  selection) not achievable with `l2.'
  %
  \default{`l2'}
  \item \xmlNode{alpha}, \xmlDesc{float}, is the constant multiplier for the
  regularization term.
  %
  \default{0.0001}
  \item \xmlNode{l1\_ratio}, \xmlDesc{float}, is the Elastic Net mixing
  parameter, with 0 <= l1\_ratio <= 1.
  %
  l1\_ratio=0 corresponds to L2 penalty, l1\_ratio=1 to L1.
  %
  \default{0.15}
  %
  \item \fitInterceptDescription{True}
  \item \nIterNoChangeDescriptionA{5}
  \item \shuffleDescription{True}
  \item \randomStateSVMDescription{None}
  \item \verDescriptionB{0}
  \item \xmlNode{epsilon}, \xmlDesc{float, optional field}, varies meaning
  depending on the value of \xmlNode{loss}. If loss is `huber',
  `epsilon\_insensitive' or `squared\_epsilon\_insensitive' then this is the
  epsilon in the epsilon-insensitive loss functions. For ‘huber’,
  determines the threshold at which it becomes less important to get the
  prediction exactly right. For `epsilon\_insensitive, any differences between
  the current prediction and the correct label are ignored if they are less than
  this threshold.
  %
  \default{0.1}
  %
  \item \xmlNode{learning\_rate}, \xmlDesc{string, optional field}, specifies
  the learning rate:
  \begin{itemize}
    \item `constant:' eta = eta0
    \item `optimal:' eta = 1.0 / (t + t0)
    \item `invscaling:' eta = eta0 / pow(t, power\_t)
  \end{itemize}
  \default{`optimal'}
  %
  \item \xmlNode{eta0}, \xmlDesc{double}, specifies the initial learning rate
  for the `constant' or `invscaling' schedules.
  %
  The default value is 0.0 as eta0 is not used by the default schedule
  `optimal.'
  %
  \default{0.0}
  %
  \item \xmlNode{power\_t}, \xmlDesc{double}, represents the exponent for
  the inverse scaling learning rate.
  %
  \default{0.5}
  %
  \item \xmlNode{class\_weight}, \xmlDesc{dict, {class\_label}}, is the preset
  for the class\_weight fit parameter.
  %
  Weights associated with classes.
  %
  If not given, all classes are assumed to have weight one.
  %
  The ``auto'' mode uses the values of y to automatically adjust weights
  inversely proportional to class frequencies.
  %
  \default{None}
  %
  \item \warmStartDescription{False}
  %
\end{itemize}
\zNormalizationPerformed{SGDClassifier}
%TODO document average

\subparagraph{Linear model fitted by minimizing a regularized empirical loss
with SGD}
\mbox{}
\\The \textit{Linear model fitted by minimizing a regularized empirical loss
with SGD} is a model where SGD stands for Stochastic Gradient Descent: the
gradient of the loss is estimated each sample at a time and the model is updated
along the way with a decreasing strength schedule (aka learning rate).
%
The regularizer is a penalty added to the loss function that shrinks model
parameters towards the zero vector using either the squared euclidean norm L2 or
the absolute norm L1 or a combination of both (Elastic Net).
%
If the parameter update crosses the 0.0 value because of the regularizer, the
update is truncated to 0.0 to allow for learning sparse models and achieving
online feature selection.
%
This implementation works with data represented as dense numpy arrays of
floating point values for the features.
%
\skltype{Linear model fitted by minimizing a regularized empirical loss with SGD}{linear\_model|SGDRegressor}
\begin{itemize}
  \item \xmlNode{loss}, \xmlDesc{str, `squared\_loss,' `huber,'
  `epsilon\_insensitive,' or `squared\_epsilon\_insensitive'}, specifies the
  loss function to be used.
  %
  Defaults to `squared\_loss' which refers to the ordinary least squares fit.
  %
  `huber' modifies `squared\_loss' to focus less on getting outliers correct by
  switching from squared to linear loss past a distance of epsilon.
  %
  `epsilon\_insensitive' ignores errors less than epsilon and is linear past
  that; this is the loss function used in SVR.
  %
  `squared\_epsilon\_insensitive' is the same but becomes squared loss past a
  tolerance of epsilon.
  %
  \default{`squared\_loss'}
  \item \xmlNode{penalty}, \xmlDesc{str, `l2' or `l1' or `elasticnet'}, sets
  the penalty (aka regularization term) to be used.
  %
  Defaults to `l2' which is the standard regularizer for linear SVM models.
  %
  `l1' and `elasticnet' might bring sparsity to the model (feature
  selection) not achievable with `l2'.
  %
  \default{`l2'}
  %
  \item \xmlNode{alpha}, \xmlDesc{float},
  %
  Constant that multiplies the regularization term.
  %
  Defaults to 0.0001
  \item \xmlNode{l1\_ratio}, \xmlDesc{float}, is the Elastic Net mixing
  parameter, with $0 \leq l1\_ratio \leq 1$.
  %
  l1\_ratio=0 corresponds to L2 penalty, l1\_ratio=1 to L1.
  %
  \default{0.15}
  %
  \item \fitInterceptDescription{True}
  \item \nIterNoChangeDescriptionA{5}
  \item \shuffleDescription{True}
  \item \randomStateDescription{None}
  \item \verDescriptionB{0}
  %
  \item \xmlNode{epsilon}, \xmlDesc{float}, sets the epsilon in the
  epsilon-insensitive loss functions; only if loss is `huber,'
  `epsilon\_insensitive,' or `squared\_epsilon\_insensitive.'
  %
  For `huber', determines the threshold at which it becomes less important
  to get the prediction exactly right.
  %
  For epsilon-insensitive, any differences between the current prediction and
  the correct label are ignored if they are less than this threshold.
  %
  \default{0.1}
  %
  \item \xmlNode{learning\_rate}, \xmlDesc{string, optional field},
  Learning rate:
  \begin{itemize}
    \item constant: eta = eta0
    \item optimal: eta = 1.0/(t+t0)
    \item invscaling: eta= eta0 / pow(t, power\_t)
  \end{itemize}
  \default{invscaling}
  \item \xmlNode{eta0}, \xmlDesc{double}, specifies the initial learning rate.
  %
  \default{0.01}
  %
  \item \xmlNode{power\_t}, \xmlDesc{double, optional field}, specifies the
  exponent for inverse scaling learning rate.
  %
  \default{0.25}
  %
  \item \warmStartDescription{False}
  %
\end{itemize}
\zNormalizationPerformed{SGDRegressor}
%TODO document average

%%%%% ROM Model - SciKitLearn: Support Vector Machines %%%%%%%
\paragraph{Support Vector Machines}
\label{SVM}
In machine learning, \textbf{Support Vector Machines} (SVMs, also support vector
networks) are supervised learning models with associated learning algorithms
that analyze data and recognize patterns, used for classification and regression
analysis.
%
Given a set of training examples, each marked as belonging to one of two
categories, an SVM training algorithm builds a model that assigns new examples
into one category or the other, making it a non-probabilistic binary linear
classifier.
%
An SVM model is a representation of the examples as points in space, mapped so
that the examples of the separate categories are divided by a clear gap that is
as wide as possible.
%
New examples are then mapped into that same space and predicted to belong to a
category based on which side of the gap they fall on.
%
In addition to performing linear classification, SVMs can efficiently perform a
non-linear classification using what is called the kernel trick, implicitly
mapping their inputs into high-dimensional feature spaces.
%
\zNormalizationPerformed{SVM-based}

In the following, all the SVM models available in RAVEN are reported.

\subparagraph{Linear Support Vector Classifier}
\mbox{}
\\The \textit{Linear Support Vector Classifier} is similar to SVC with parameter
kernel=`linear', but implemented in terms of liblinear rather than libsvm,
so it has more flexibility in the choice of penalties and loss functions and
should scale better (to large numbers of samples).
%
This class supports both dense and sparse input and the multiclass support is
handled according to a one-vs-the-rest scheme.
%
\skltype{Linear Support Vector Classifier}{svm|LinearSVC}
\begin{itemize}
  \item \CSVMDescription{1.0}
  \item \xmlNode{loss}, \xmlDesc{string, `hinge' or `squared\_hinge'}, specifies the loss
  function.
  %
  `hinge' is the hinge loss (standard SVM) while `squared\_hinge' is the squared hinge
  loss.
  %
  \default{`squared\_hinge'}
  %
  \item \xmlNode{penalty}, \xmlDesc{string, `l1' or `l2'}, specifies the norm
  used in the penalization.
  %
  The `l2' penalty is the standard used in SVC.
  %
  The `l1' leads to coef\_vectors that are sparse.
  %
  \default{`l2'}
  %
  \item \xmlNode{dual}, \xmlDesc{boolean}, selects the algorithm to either solve
  the dual or primal optimization problem.
  %
  Prefer dual=False when n\_samples $>$ n\_features.
  %
  \default{True}
  %
  \item \tolSVMDescription{1e-4}
  %
  \item \xmlNode{multi\_class}, \xmlDesc{string, `ovr' or `crammer\_singer'},
  %
  Determines the multi-class strategy if y contains more than two classes.
  %
  ovr trains n\_classes one-vs-rest classifiers, while
  crammer\_singer optimizes a joint objective over all classes.
  %
  While crammer\_singer is interesting from a theoretical perspective as it is
  consistent, it is seldom used in practice and rarely leads to better accuracy
  and is more expensive to compute.
  %
  If crammer\_singer is chosen, the options loss, penalty and dual
  will be ignored.
  %
  \default{`ovr'}
  %
  \item \fitInterceptDescription{True}
  %
  \item \xmlNode{intercept\_scaling}, \xmlDesc{float, optional field}, when
  True, the instance vector x becomes [x,self.intercept\_scaling], i.e. a
  ``synthetic'' feature with constant value equals to intercept\_scaling is
  appended to the instance vector.
  %
  The intercept becomes intercept\_scaling * synthetic feature
  weight.
  \nb The synthetic feature weight is subject to l1/l2 regularization as are all
  other features.
  %
  To lessen the effect of regularization on the synthetic feature weight (and
  therefore on the intercept) intercept\_scaling has to be increased.
  %
  \default{1}
  %
  \item \classWeightDescription{None}
  \item \verDescriptionB{0}
  %
  \nb This setting takes advantage of a per-process runtime setting in liblinear
  that, if enabled, may not work properly in a multithreaded context.
  %
  \item \randomStateSVMDescription{None}
\end{itemize}

\subparagraph{C-Support Vector Classification}
\mbox{}
\\The \textit{C-Support Vector Classification} is a based on libsvm.
%
The fit time complexity is more than quadratic with the number of samples which
makes it hard to scale to datasets with more than a couple of 10000 samples.
%
The multiclass support is handled according to a one-vs-one scheme.
%
\skltype{C-Support Vector Classifier}{svm|SVC}
\begin{itemize}
  \item \CSVMDescription{1.0}
  \item \kernelDescription{`rbf'}
  \item \degreeDescription{3.0}
  \item \gammaDescription{`auto'}
  \item \coefZeroDescription{0.0}
  \item \probabilityDescription{False}
  \item \shrinkingDescription{True}
  \item \tolSVMDescription{1e-3}
  \item \cacheSizeDescription{}
  \item \classWeightDescription{None}
  \item \verSVMDescription{False}
  \item \maxIterDescription{-1}
    %TODO: Should decision_function_shape be documented?
  \item \randomStateSVMDescription{None}
  %
\end{itemize}

\subparagraph{Nu-Support Vector Classification}
\mbox{}

The \textit{Nu-Support Vector Classification} is similar to SVC but uses a
parameter to control the number of support vectors.
%
The implementation is based on libsvm.
%
\skltype{Nu-Support Vector Classifier}{svm|NuSVC}
\begin{itemize}
  \item \xmlNode{nu}, \xmlDesc{float, optional field}, is an upper bound on the
  fraction of training errors and a lower bound of the fraction of support
  vectors.
  %
  Should be in the interval (0, 1].
  %
  \default{0.5}
  %
  \item \kernelDescription{`rbf'}
  \item \degreeDescription{3}
  \item \gammaDescription{`auto'}
  \item \coefZeroDescription{0.0}
  \item \probabilityDescription{False}
  \item \shrinkingDescription{True}
  \item \tolSVMDescription{1e-3}
  \item \cacheSizeDescription{}
  \item \verSVMDescription{False}
  \item \maxIterDescription{-1}
    %TODO document decision_function_shape
  \item \randomStateSVMDescription{None}
  %
\end{itemize}

\subparagraph{Support Vector Regression}
\mbox{}
\\The \textit{Support Vector Regression} is an epsilon-Support Vector
Regression.
%
The free parameters in this model are C and epsilon.
%
The implementations is a based on libsvm.
%
\skltype{Support Vector Regressor}{svm|SVR}
\begin{itemize}
  \item \CSVMDescription{1.0}
  \item \xmlNode{epsilon}, \xmlDesc{float, optional field}, specifies the
  epsilon-tube within which no penalty is associated in the training loss
  function with points predicted within a distance epsilon from the actual
  value.
  %
  \default{0.1}
  %
  \item \kernelDescription{`rbf'}
  \item \degreeDescription{3.0}
  \item \gammaDescription{`auto'}
  \item \coefZeroDescription{0.0}
  \item \shrinkingDescription{True}
  \item \tolSVMDescription{1e-3}
  \item \cacheSizeDescription{}
  \item \verSVMDescription{False}
  \item \maxIterDescription{-1}
  %
\end{itemize}
 %%%%% ROM Model - SciKitLearn: MultiClass %%%%%%%
\paragraph{Multi Class}
\label{Multiclass}
Multiclass classification means a classification task with more than two
classes; e.g., classify a set of images of fruits which may be oranges, apples,
or pears.
%
Multiclass classification makes the assumption that each sample is assigned to
one and only one label: a fruit can be either an apple or a pear but not both at
the same time.

\zNormalizationNotPerformed{multi-class-based}

%
In the following, all the multi-class models available in RAVEN are reported.
%
%%%%%%%%%
\subparagraph{One-vs-the-rest (OvR) multiclass/multilabel strategy}
\mbox{}

The \textit{One-vs-the-rest (OvR) multiclass/multilabel strategy}, also known
as one-vs-all, consists in fitting one classifier per class.
%
For each classifier, the class is fitted against all the other classes.
%
In addition to its computational efficiency (only n\_classes classifiers are
needed), one advantage of this approach is its interpretability.
%
Since each class is represented by one and one classifier only, it is possible
to gain knowledge about the class by inspecting its corresponding classifier.
%
This is the most commonly used strategy and is a fair default choice.

\skltype{One-vs-the-rest (OvR) multiclass/multilabel classifier}{multiClass|OneVsRestClassifier}
\begin{itemize}
  \item \estimatorDescription{}
\end{itemize}
%Should n_jobs be documented?

%%%%%%%%%%%%
\subparagraph{One-vs-one multiclass strategy}
\mbox{}

The \textit{One-vs-one multiclass strategy} consists in fitting one classifier
per class pair.
%
At prediction time, the class which received the most votes is selected.
%
Since it requires to fit n\_classes * (n\_classes - 1) / 2 classifiers, this
method is usually slower than one-vs-the-rest, due to its O(n\_classes$^2$)
complexity.
%
However, this method may be advantageous for algorithms such as kernel
algorithms which do not scale well with n\_samples.
%
This is because each individual learning problem only involves a small subset of
the data whereas, with one-vs-the-rest, the complete dataset is used n\_classes
times.

\skltype{One-vs-one multiclass classifier}{multiClass|OneVsOneClassifier}
\begin{itemize}
  \item \estimatorDescription{}
\end{itemize}
%Should n_jobs be documented?

%%%%%%%%%%%%%
\subparagraph{Error-Correcting Output-Code multiclass strategy}
\mbox{}
\\The \textit{Error-Correcting Output-Code multiclass strategy} consists in
representing each class with a binary code (an array of 0s and 1s).
%
At fitting time, one binary classifier per bit in the code book is fitted.
%
At prediction time, the classifiers are used to project new points in the class
space and the class closest to the points is chosen.
%
The main advantage of these strategies is that the number of classifiers used
can be controlled by the user, either for compressing the model ($0 < code\_
size < 1$) or for making the model more robust to errors ($code\_ size > 1$).

\skltype{Error-Correcting Output-Code multiclass classifier}{multiClass|OutputCodeClassifier}
\begin{itemize}
  \item \estimatorDescription{}
  \item \xmlNode{code\_size}, \xmlDesc{float, required field}, represents the
  percentage of the number of classes to be used to create the code book.
  %
  A number between 0 and 1 will require fewer classifiers than one-vs-the-rest.
  %
  A number greater than 1 will require more classifiers than one-vs-the-rest.
  %
\end{itemize}
%Should random_state and n_jobs be documented?

%%%%%%%%%%%%%
%\subparagraph{fit a one-vs-the-rest strategy}
%pass
%\subparagraph{Make predictions using the one-vs-the-rest strategy}
%pass
%\subparagraph{ Fit a one-vs-one strategy}
%pass
%\subparagraph{Make predictions using the one-vs-one strategy}
%pass
%\subparagraph{Fit an error-correcting output-code strategy}
%pass
%\subparagraph{Make predictions using the error-correcting output-code strategy}
%pass

 %%%%% ROM Model - SciKitLearn: naiveBayes %%%%%%%
\paragraph{Naive Bayes}
\label{naiveBayes}
Naive Bayes methods are a set of supervised learning algorithms based on
applying Bayes' theorem with the ``naive'' assumption of independence between
every pair of features.
%
Given a class variable y and a dependent feature vector x\_1 through x\_n,
Bayes' theorem states the following relationship:
\begin{equation}
P(y \mid x_1, \dots, x_n) = \frac{P(y) P(x_1, \dots x_n \mid y)}
{P(x_1, \dots, x_n)}
\end{equation}
Using the naive independence assumption that
\begin{equation}
P(x_i | y, x_1, \dots, x_{i-1}, x_{i+1}, \dots, x_n) = P(x_i | y),
\end{equation}
for all i, this relationship is simplified to
\begin{equation}
P(y \mid x_1, \dots, x_n) = \frac{P(y) \prod_{i=1}^{n} P(x_i \mid y)}
{P(x_1, \dots, x_n)}
\end{equation}
Since $P(x_1, \dots, x_n)$ is constant given the input, we can use the following
classification rule:
\begin{equation}
P(y \mid x_1, \dots, x_n) \propto P(y) \prod_{i=1}^{n} P(x_i \mid y)
\Downarrow
\end{equation}
\begin{equation}
\hat{y} = \arg\max_y P(y) \prod_{i=1}^{n} P(x_i \mid y),
\end{equation}
and we can use Maximum A Posteriori (MAP) estimation to estimate $P(y)$ and
$P(x_i \mid y)$; the former is then the relative frequency of class $y$ in the
training set.
%
The different naive Bayes classifiers differ mainly by the assumptions they make
regarding the distribution of $P(x_i \mid y)$.

In spite of their apparently over-simplified assumptions, naive Bayes
classifiers have worked quite well in many real-world situations, famously
document classification and spam filtering.
%
They require a small amount of training data to estimate the necessary
parameters.
%
(For theoretical reasons why naive Bayes works well, and on which types of data
it does, see the references below.)
Naive Bayes learners and classifiers can be extremely fast compared to more
sophisticated methods.
%
The decoupling of the class conditional feature distributions means that each
distribution can be independently estimated as a one dimensional distribution.
%
This in turn helps to alleviate problems stemming from the curse of
dimensionality.

On the flip side, although naive Bayes is known as a decent classifier, it is
known to be a bad estimator, so the probability outputs from predict\_proba are
not to be taken too seriously.
%
In the following, all the Naive Bayes available in RAVEN are reported.
%
%%%%%%%
\subparagraph{Gaussian Naive Bayes}
\mbox{}
\\The \textit{Gaussian Naive Bayes strategy} implements the Gaussian Naive Bayes
algorithm for classification.
%
The likelihood of the features is assumed to be Gaussian:
\begin{equation}
P(x_i \mid y) = \frac{1}{\sqrt{2\pi\sigma^2_y}} \exp\left(-\frac{(x_i -
  \mu_y)^2}{2\sigma^2_y}\right)
\end{equation}
The parameters $\sigma_y$ and $\mu_y$ are estimated using maximum likelihood.

In order to use the \textit{Gaussian Naive Bayes strategy}, the user needs to
set the sub-node:

\xmlNode{SKLtype}\texttt{naiveBayes|GaussianNB}\xmlNode{/SKLtype}.

There are no additional sub-nodes available for this method.
%

\zNormalizationPerformed{GaussianNB}
%%%%%%%%%%%%
\subparagraph{Multinomial Naive Bayes}
\mbox{}
\\The \textit{Multinomial Naive Bayes} implements the naive Bayes algorithm for
multinomially distributed data, and is one of the two classic naive Bayes
variants used in text classification (where the data is typically represented
as word vector counts, although tf-idf vectors are also known to work well in
practice).
%
The distribution is parametrized by vectors $\theta_y =
(\theta_{y1},\ldots,\theta_{yn})$ for each class $y$, where $n$ is the number of
features (in text classification, the size of the vocabulary) and $\theta_{yi}$
is the probability $P(x_i \mid y)$ of feature $i$ appearing in a sample
belonging to class $y$.
%
The parameters $\theta_y$ are estimated by a smoothed version of maximum
likelihood, i.e. relative frequency counting:
\begin{equation}
\hat{\theta}_{yi} = \frac{ N_{yi} + \alpha}{N_y + \alpha n}
\end{equation}
where $N_{yi} = \sum_{x \in T} x_i$ is the number of times feature $i$ appears
in a sample of class y in the training set T, and
$N_{y} = \sum_{i=1}^{|T|} N_{yi}$ is the total count of all features for class
$y$.
%
The smoothing priors $\alpha \ge 0$ account for features not present in the
learning samples and prevents zero probabilities in further computations.
%
Setting $\alpha = 1$ is called Laplace smoothing, while $\alpha < 1$ is called
Lidstone smoothing.
%
\skltype{Multinomial Naive Bayes strategy}{naiveBayes|MultinomialNB}
\begin{itemize}
  \item \alphaBayesDescription{1.0}
  \item \fitPriorDescription{True}
  \item \classPriorDescription{None}
\end{itemize}
\zNormalizationNotPerformed{MultinomialNB}
%%%%%%%%%%%%
\subparagraph{Bernoulli Naive Bayes}
\mbox{}
\\The \textit{Bernoulli Naive Bayes} implements the naive Bayes training and
classification algorithms for data that is distributed according to multivariate
Bernoulli distributions; i.e., there may be multiple features but each one is
assumed to be a binary-valued (Bernoulli, boolean) variable.
%
Therefore, this class requires samples to be represented as binary-valued
feature vectors; if handed any other kind of data, a \textit{Bernoulli Naive
Bayes} instance may binarize its input (depending on the binarize parameter).
%
The decision rule for Bernoulli naive Bayes is based on
\begin{equation}
P(x_i \mid y) = P(i \mid y) x_i + (1 - P(i \mid y)) (1 - x_i)
\end{equation}
which differs from multinomial NB's rule in that it explicitly penalizes the
non-occurrence of a feature $i$ that is an indicator for class $y$, where the
multinomial variant would simply ignore a non-occurring feature.
%
In the case of text classification, word occurrence vectors (rather than word
count vectors) may be used to train and use this classifier.
%
\textit{Bernoulli Naive Bayes} might perform better on some datasets, especially
those with shorter documents.
%
It is advisable to evaluate both models, if time permits.
%
\skltype{Bernoulli Naive Bayes strategy}{naiveBayes|BernoulliNB}
\begin{itemize}
  \item \alphaBayesDescription{1.0}
  \item \xmlNode{binarize}, \xmlDesc{float, optional field},
  %
  Threshold for binarizing (mapping to booleans) of sample features.
  %
  If None, input is presumed to already consist of binary vectors.
  %
  \default{0.0}
  \item \fitPriorDescription{True}
  \item \classPriorDescription{None}
  %
\end{itemize}
\zNormalizationPerformed{BernoulliNB}
%%%%%%%%%%%%%%%%%%%%%%%%%%%%%%%%%%%%%%%%
 %%%%% ROM Model - SciKitLearn: Neighbors %%%%%%%
\paragraph{Neighbors}
\label{Neighbors}

The \textit{Neighbors} class provides functionality for unsupervised and
supervised neighbor-based learning methods.
%
The unsupervised nearest neighbors method is the foundation of many other
learning methods, notably manifold learning and spectral clustering.
%
Supervised neighbors-based learning comes in two flavors: classification for
data with discrete labels, and regression for data with continuous labels.

The principle behind nearest neighbor methods is to find a predefined number of
training samples closest in distance to the new point, and predict the label
from these.
%
The number of samples can be a user-defined constant (k-nearest neighbor
learning), or vary based on the local density of points (radius-based neighbor
learning).
%
The distance can, in general, be any metric measure: standard Euclidean distance
is the most common choice.
%
Neighbor-based methods are known as non-generalizing machine learning methods,
since they simply ``remember'' all of its training data (possibly transformed
into a fast indexing structure such as a Ball Tree or KD Tree.).

\zNormalizationPerformed{Neighbors-based}

In the following, all the Neighbors' models available in RAVEN are reported.
%
%%%%%%%%%%%%%%%
\subparagraph{K Neighbors Classifier}
\mbox{}
\\The \textit{K Neighbors Classifier} is a type of instance-based learning or
non-generalizing learning: it does not attempt to construct a general internal
model, but simply stores instances of the training data.
%
Classification is computed from a simple majority vote of the nearest neighbors
of each point: a query point is assigned the data class which has the most
representatives within the nearest neighbors of the point.
%
It implements learning based on the $k$ nearest neighbors of each query point,
where $k$ is an integer value specified by the user.

\skltype{K Neighbors Classifier}{neighbors|KNeighborsClassifier}
\begin{itemize}
  \item \nNeighborsDescription{5}
  \item \weightsDescription{uniform}
  \item \algorithmDescription{auto}
  \item \leafSizeDescription{30}
  \item \metricDescription{minkowski}
  \item \pDescription{2}
    %TODO document metric_params
    %TODO document n_jobs?
\end{itemize}
%%%%%%%%%%%%%%%
\subparagraph{Radius Neighbors Classifier}
\mbox{}
\\The \textit{Radius Neighbors Classifier} is a type of instance-based learning
or non-generalizing learning: it does not attempt to construct a general
internal model, but simply stores instances of the training data.
%
Classification is computed from a simple majority vote of the nearest neighbors
of each point: a query point is assigned the data class which has the most
representatives within the nearest neighbors of the point.
%
It implements learning based on the number of neighbors within a fixed radius
$r$ of each training point, where $r$ is a floating-point value specified by the
user.

\skltype{Radius Neighbors Classifier}{neighbors|RadiusNeighbors}
\begin{itemize}
  \item \radiusDescription{1.0}
  \item \weightsDescription{uniform}
  \item \algorithmDescription{auto}
  \item \leafSizeDescription{30}
  \item \metricDescription{minkowski}
  \item \pDescription{2}
  \item \outlierLabelDescription{None}
    %TODO document metric_params
\end{itemize}

%%%%%%%%%%%%%%%
\subparagraph{K Neighbors Regressor}
\mbox{}

The \textit{K Neighbors Regressor} can be used in cases where the data labels
are continuous rather than discrete variables.
%
The label assigned to a query point is computed based on the mean of the labels
of its nearest neighbors.
%
It implements learning based on the $k$ nearest neighbors of each query point,
where $k$ is an integer value specified by the user.

\skltype{K Neighbors Regressor}{neighbors|KNeighborsRegressor}
\begin{itemize}
  \item \nNeighborsDescription{5}
  \item \weightsDescription{uniform}
  \item \algorithmDescription{auto}
  \item \leafSizeDescription{30}
  \item \metricDescription{minkowski}
  \item \pDescription{2}
    %TODO document metric_params
    %TODO document n_jobs?
\end{itemize}

%%%%%%%%%%%%%%%
\subparagraph{Radius Neighbors Regressor}
\mbox{}

The \textit{Radius Neighbors Regressor} can be used in cases where the data
labels are continuous rather than discrete variables.
%
The label assigned to a query point is computed based on the mean of the labels
of its nearest neighbors.
%
It implements learning based on the neighbors within a fixed radius $r$ of the
query point, where $r$ is a floating-point value specified by the user.

\skltype{Radius Neighbors Regressor}{neighbors|RadiusNeighborsRegressor}
\begin{itemize}
  \item \radiusDescription{1.0}
  \item \weightsDescription{uniform}
  \item \algorithmDescription{auto}
  \item \leafSizeDescription{30}
  \item \metricDescription{minkowski}
  \item \pDescription{2}
    %TODO document metric_params
\end{itemize}
%%%%%%%%%%%%%%%
\subparagraph{Nearest Centroid Classifier}
\mbox{}

The \textit{Nearest Centroid classifier} is a simple algorithm that represents
each class by the centroid of its members.
%
It also has no parameters to choose, making it a good baseline classifier.
%
It does, however, suffer on non-convex classes, as well as when classes have
drastically different variances, as equal variance in all dimensions is assumed.

\skltype{Nearest Centroid Classifier}{neighbors|NearestCentroid}
\begin{itemize}
  \item \xmlNode{shrink\_threshold}, \xmlDesc{float, optional field}, defines
  the threshold for shrinking centroids to remove features.
  %
  \default{None}
  %
  %TODO document metric
\end{itemize}
%\subparagraph{Ball Tree}
%pass
%\subparagraph{K-D Tree}
%pass


The \textit{Quadratic Discriminant Analysis} is a classifier with a quadratic
decision boundary, generated by fitting class conditional densities to the data
and using Bayes' rule.
%
The model fits a Gaussian density to each class.

\skltype{Quadratic Discriminant Analysis Classifier}{qda|QDA}
\begin{itemize}
  \item \xmlNode{priors}, \xmlDesc{array-like (n\_classes), optional field},
  specifies the priors on the classes.
  %
  \default{None}
  \item \xmlNode{reg\_param}, \xmlDesc{float, optional field}, regularizes the
  covariance estimate as (1-reg\_param)*Sigma +
  reg\_param*Identity(n\_features).
  %
  \default{0.0}
  %
\end{itemize}
\zNormalizationNotPerformed{QDA}
 %%%%% ROM Model - SciKitLearn: Tree %%%%%%%
\paragraph{Tree}
\label{tree}

Decision Trees (DTs) are a non-parametric supervised learning method used for
classification and regression.
%
The goal is to create a model that predicts the value of a target variable by
learning simple decision rules inferred from the data features.
%
\begin{itemize}
  \item Some advantages of decision trees are:
  \item Simple to understand and to interpret.
  %
  Trees can be visualized.
  %
  \item Requires little data preparation.
  %
  Other techniques often require data normalization, dummy variables need to be
  created and blank values to be removed.
  %
  Note however, that this module does not support missing values.
  %
  \item The cost of using the tree (i.e., predicting data) is logarithmic in the
  number of data points used to train the tree.
  %
  \item Able to handle both numerical and categorical data.
  %
  Other techniques are usually specialized in analyzing datasets that have only
  one type of variable.
  %
  \item Able to handle multi-output problems.
  %
  \item Uses a white box model.
  %
  If a given situation is observable in a model, the explanation for the
  condition is easily explained by boolean logic.
  %
  By contrast, in a black box model (e.g., in an artificial neural network),
  results may be more difficult to interpret.
  %
  \item Possible to validate a model using statistical tests.
  %
  That makes it possible to account for the reliability of the model.
  %
  \item Performs well even if its assumptions are somewhat violated by the true
  model from which the data were generated.
  %
\end{itemize}
The disadvantages of decision trees include:
\begin{itemize}
  \item Decision-tree learners can create over-complex trees that do not
  generalise the data well.
  %
  This is called overfitting.
  %
  Mechanisms such as pruning (not currently supported), setting the minimum
  number of samples required at a leaf node or setting the maximum depth of the
  tree are necessary to avoid this problem.
  %
  \item Decision trees can be unstable because small variations in the data
  might result in a completely different tree being generated.
  %
  This problem is mitigated by using decision trees within an ensemble.
  %
  \item The problem of learning an optimal decision tree is known to be
  NP-complete under several aspects of optimality and even for simple concepts.
  %
  Consequently, practical decision-tree learning algorithms are based on
  heuristic algorithms such as the greedy algorithm where locally optimal
  decisions are made at each node.
  %
  Such algorithms cannot guarantee to return the globally optimal decision tree.
  %
  This can be mitigated by training multiple trees in an ensemble learner, where
  the features and samples are randomly sampled with replacement.
  %
  \item There are concepts that are hard to learn because decision trees do not
  express them easily, such as XOR, parity or multiplexer problems.
  %
  \item Decision tree learners create biased trees if some classes dominate.
  %
  It is therefore recommended to balance the dataset prior to fitting with the
  decision tree.
  %
\end{itemize}

\zNormalizationPerformed{tree-based}

In the following, all the tree-based algorithms available in RAVEN are reported.

%%%%%%%%%%%%%%%
\subparagraph{Decision Tree Classifier}
\mbox{}
\\The \textit{Decision Tree Classifier} is a classifier that is based on the
decision tree logic.

\skltype{Decision Tree Classifier}{tree|DecisionTreeClassifier}
\begin{itemize}
  \item \criterionDescription{gini}
  \item \splitterDescription{best}
  \item \maxFeaturesDescription{None}
  \item \maxDepthDescription{None}
  \item \minSamplesSplitDescription{2}
  \item \minSamplesLeafDescription{1}
    %TODO document min_weight_fraction_leaf
  \item \maxLeafNodesDescription{None}
    %TODO document class_weight
    %TODO document random_state
    %TODO document presort
\end{itemize}

%%%%%%%%%%%%%%%%
\subparagraph{Decision Tree Regressor}
\mbox{}
\\The \textit{Decision Tree Regressor} is a Regressor that is based on the
decision tree logic.
%
\skltype{Decision Tree Regressor}{tree|DecisionTreeRegressor}
\begin{itemize}
  \item \criterionDescriptionDT{mse}
  \item \splitterDescription{best}
  \item \maxFeaturesDescription{None}
  \item \maxDepthDescription{None}
  \item \minSamplesSplitDescription{2}
  \item \minSamplesLeafDescription{1}
    %TODO document min_weight_fraction_leaf
  \item \maxLeafNodesDescription{None}
    %TODO document random_state
    %TODO document presort
\end{itemize}

%%%%%%%%%%%%%%%%
\subparagraph{Extra Tree Classifier}
\mbox{}
\\The \textit{Extra Tree Classifier} is an extremely randomized tree classifier.
%
Extra-trees differ from classic decision trees in the way they are built.
%
When looking for the best split to separate the samples of a node into two
groups, random splits are drawn for each of the max\_features randomly selected
features and the best split among those is chosen.
%
When max\_features is set 1, this amounts to building a totally random decision
tree.

\skltype{Extra Tree Classifier}{tree|ExtraTreeClassifier}

\begin{itemize}
  \item \criterionDescription{gini}
  \item \splitterDescription{random}
  \item \maxFeaturesDescription{auto}
  \item \maxDepthDescription{None}
  \item \minSamplesSplitDescription{2}
  \item \minSamplesLeafDescription{1}
    %TODO document min_weight_fraction_leaf
  \item \maxLeafNodesDescription{None}
    %TODO document random_state
    %TODO document class_weight
  %
\end{itemize}

%%%%%%%%%%%%
\subparagraph{Extra Tree Regressor}
\mbox{}

The \textit{Extra Tree Regressor} is an extremely randomized tree regressor.
%
Extra-trees differ from classic decision trees in the way they are built.
%
When looking for the best split to separate the samples of a node into two
groups, random splits are drawn for each of the max\_features randomly selected
features and the best split among those is chosen.
%
When max\_features is set 1, this amounts to building a totally random decision
tree.

\skltype{Extra Tree Regressor}{tree|ExtraTreeRegressor}

\begin{itemize}
  \item \criterionDescriptionDT{mse}
  \item \splitterDescription{random}
  \item \maxFeaturesDescription{auto}
  \item \maxDepthDescription{None}
  \item \minSamplesSplitDescription{2}
  \item \minSamplesLeafDescription{1}
    %TODO document min_weight_fraction_leaf
  \item \maxLeafNodesDescription{None}
    %TODO document random_state
\end{itemize}

%%%%%%%%%%%%%%%%%%%%%%%%%%%%%%%%%%%%%%%%%%%
 %%%%% ROM Model - SciKitLearn: Gaussian Process %%%%%%%
\paragraph{Gaussian Process}
\label{GP}
Gaussian Processes for Machine Learning (GPML) is a generic supervised learning
method primarily designed to solve regression problems.
%
The advantages of Gaussian Processes for Machine Learning are:
\begin{itemize}
  \item The prediction interpolates the observations (at least for regular
  correlation models).
  \item The prediction is probabilistic (Gaussian) so that one can compute
  empirical confidence intervals and exceedance probabilities that might be used
  to refit (online fitting, adaptive fitting) the prediction in some region of
  interest.
  \item Versatile: different linear regression models and correlation models can
  be specified.
  %
  Common models are provided, but it is also possible to specify custom models
  provided they are stationary.
  %
\end{itemize}
The disadvantages of Gaussian Processes for Machine Learning include:
\begin{itemize}
  \item It is not sparse.
  %
  It uses the whole samples/features information to perform the prediction.
  \item It loses efficiency in high dimensional spaces – namely when the
  number of features exceeds a few dozens.
  %
  It might indeed give poor performance and it loses computational efficiency.
  \item Classification is only a post-processing, meaning that one first needs
  to solve a regression problem by providing the complete scalar float precision
  output $y$ of the experiment one is attempting to model.
  %
\end{itemize}

\skltype{Gaussian Process Regressor}{GaussianProcess|GaussianProcess}

\begin{itemize}
  \item \xmlNode{normalize\_y}, \xmlDesc{boolean, optional field}, if True, the
  observations $y$ are centered and reduced w.r.t. means and
  standard deviations estimated from the n\_samples observations provided.
  %
  \default{True}
  \item \xmlNode{optimizer}, \xmlDesc{string, optional field}, specifies the
  optimization algorithm to be used.
  %
  Available optimizers are: 'fmin\_cobyla', 'Welch'.
  %
  \default{fmin\_cobyla}
  \item \xmlNode{random\_state}, \xmlDesc{integer, optional field}, is the seed
  of the internal random number generator.
  %
  \default{None}
  \item \xmlNode{n\_restarts\_optimizer }, \xmlDesc{integer, optional field}, The number of restarts of the optimizer for finding
the kernel’s parameters which maximize the log-marginal likelihood. The first run of the optimizer is performed from th
e kernel’s initial parameters, the remaining ones (if any) from thetas sampled log-uniform randomly from the space of
allowed theta-values. If greater than 0, all bounds must be finite. Note that n\_restarts\_optimizer $== 0$ implies that one run is performed.
  %
  \default{0}
  %
  \item \xmlNode{alpha}, \xmlDesc{float, optional field}, Value added to the diagonal of the kernel matrix during fitting.
 Larger values correspond to increased noise level in the observations. This can also prevent a potential numerical issue
 during fitting, by ensuring that the calculated values form a positive definite matrix. If an array is passed, it must have the
 same number of entries as the data used for fitting and is used as datapoint-dependent noise level. Note that this is
 equivalent to adding a WhiteKernel with c=alpha. Allowing to specify the noise level directly as a parameter is mainly for
 convenience and for consistency with Ridge.
  %
  \default{1e-10}
  %
\end{itemize}

\zNormalizationNotPerformed{GaussianProcessRegressor}

\textbf{Example:}
\begin{lstlisting}[style=XML,morekeywords={name,subType}]
<Simulation>
  ...
  <Models>
    ...
   <ROM name='aUserDefinedName' subType='SciKitLearn'>
     <Features>var1,var2,var3</Features>
     <Target>result1</Target>
     <SKLtype>linear_model|LinearRegression</SKLtype>
     <fit_intercept>True</fit_intercept>
     <normalize>False</normalize>
   </ROM>
    ...
  </Models>
  ...
</Simulation>
\end{lstlisting}


%%%%%%%%%%%%%%%%%%%%%%%%%%%%%%%%%%%%%%%%%%%
 %%%%% ROM Model - SciKitLearn: Neural Network Models %%%%%%%
\paragraph{Neural Network Models}
\label{DNN}
It has been more than 70 years since Warren McCulloch and Water Pitts modeled the first
artificial neural network (ANN) that mimicked the way brains work. These days, deep learning
based on ANN is showing outstanding results for solving a wide variety of robotic tasks in
the areas of perception, planning, localization, and control.
%
\textbf{Multi-layer Perceptron (MLP)} is a supervised learning algorithm that can learn
a non-linear function approximator for either classifcation or regression. It is different
from logistic regression, in that between the input and output layer, there can be one
or more non-linear layers, called hidden layers.
%
The advantages of Multi-layer Perceptron are:
\begin{itemize}
  \item Capability to learn non-linear models
  \item Capability to learn models in real-time (online learning)
\end{itemize}
The disadvantages of Multi-layer Perceptron include:
\begin{itemize}
  \item MLP with hidden layers have a non-convex loss function where there exists more than
    one local minimum. Therefore different random weight initializations can lead to different
    validation accuracy.
  \item MLP requires tuning a number of hyperparameters such as the number of hidden neurons, layers
    and iterations.
  \item MLP is sensitive to feature scaling
\end{itemize}

\zNormalizationPerformed{Multi-layer Perceptron}

In the following, Multi-layer perceptron classification and regression algorithms available in RAVEN are reported.

%%%%%%%%%%%%%%%
\subparagraph{MLPClassifier}
\mbox{}
\\The \textit{MLPClassifier} implements a multi-layer perceptron algorithm that trains using \textbf{Backpropagation}
More precisely, it trains using some form of gradient descent and the gradients are calculated using Backpropagation.
For classification, it minimizes the Cross-Entropy loss function, and it supports multi-class classification.

\skltype{MLPClassifier}{neural\_network|MLPClassifier}
\begin{itemize}
  \item \hiddenLayerSizesMLPDescription{(100,)}
  \item \activationMLPDescription{`relu'}
  \item \solverMLPDescription{`adam'}
  \item \alphaMLPDescription{0.0001}
  \item \batchSizeMLPDescription{`auto'}
  \item \learningRateMLPDescription{`constant'}
  \item \learningRateInitMLPDescription{0.001}
  \item \powerTMLPDescription{0.5}
  \item \maxIterMLPDescription{200}
  \item \shuffleMLPDescription{True}
  \item \randomStateMLPDescription{None}
  \item \tolMLPDescription{1e-4}
  \item \verboseMLPDescription{False}
  \item \warmStartMLPDescription{False}
  \item \momentumMLPDescription{0.9}
  \item \nesterovsMomentumMLPDescription{True}
  \item \earlyStoppingMLPDescription{False}
  \item \validationFractionMLPDescription{0.1}
  \item \betaAMLPDescription{0.9}
  \item \betaBMLPDescription{0.999}
  \item \epsilonMLPDescription{1e-8}
\end{itemize}

%%%%%%%%%%%%%%%
\subparagraph{MLPRegressor}
\mbox{}
\\The \textit{MLPRegressor} implements a multi-layer perceptron algorithm that trains using \textbf{Backpropagation} with
no activation function in the output layer, which can also be seen as using the identity function as activation function.
Therefore, it uses the square error as the loss function, and the output is a set of continuous values.
\textit{MLPRegressor} also supports multi-output regression, in which a sample can have more than one target.

\skltype{MLPRegressor}{neural\_network|MLPRegressor}
\begin{itemize}
  \item \hiddenLayerSizesMLPDescription{(100,)}
  \item \activationMLPDescription{`relu'}
  \item \solverMLPDescription{`adam'}
  \item \alphaMLPDescription{0.0001}
  \item \batchSizeMLPDescription{`auto'}
  \item \learningRateMLPDescription{`constant'}
  \item \learningRateInitMLPDescription{0.001}
  \item \powerTMLPDescription{0.5}
  \item \maxIterMLPDescription{200}
  \item \shuffleMLPDescription{True}
  \item \randomStateMLPDescription{None}
  \item \tolMLPDescription{1e-4}
  \item \verboseMLPDescription{False}
  \item \warmStartMLPDescription{False}
  \item \momentumMLPDescription{0.9}
  \item \nesterovsMomentumMLPDescription{True}
  \item \earlyStoppingMLPDescription{False}
  \item \validationFractionMLPDescription{0.1}
  \item \betaAMLPDescription{0.9}
  \item \betaBMLPDescription{0.999}
  \item \epsilonMLPDescription{1e-8}
\end{itemize}

%%%% ROM Model - SyntheticHistory  %%%%%%%
\subsubsection{SyntheticHistory}
% NOTE TODO: this ROM has the "descr" nodes fully filled in the InputSpecs.
\label{subsubsec:arma}
The SyntheticHistory sub-type uses various Time Series Analysis (TSA) algorithms to characterize and
reproduce synthetic histories. It is a more general implementation of the ARMA ROM. The available
algorithms are discussed in more detail below. The SyntheticHistory ROM uses the TSA algorithms to
characterize then reproduce time series in sequence; for example, if using Fourier then ARMA, the
SyntheticHistory ROM will characterize the Fourier properties using the Fourier TSA algorithm on a
training signal, then send the residual to the ARMA TSA algorithm for characterization. Generating
new signals works in reverse, first generating a signal using the ARMA TSA algorithm then
superimposing the Fourier TSA algorithm.

%
In order to use this Reduced Order Model, the \xmlNode{ROM} attribute
\xmlAttr{subType} needs to be \xmlString{SyntheticHistory} (see the example
below).
%
\subnodeIntro

\begin{itemize}
  \item \xmlNode{Features}, \xmlDesc{comma separated string, required field}, defines the features (e.g., scaling). Note that only
  one feature is allowed for \xmlString{SyntheticHistory} and in current implementation this is used for evaluation only.
  \item \xmlNode{Target}, \xmlDesc{comma separated string, required field}, defines the variable(s) of the
    time series.  Should include the pivot parameter (or Index).
  \item \xmlNode{pivotParameter}, \xmlDesc{string, required field}, defines the pivot variable (e.g., time) that is non-decreasing in
  the input HistorySet.
\end{itemize}

In addition, any number of the following TSA algorithms may be included as subnodes of this ROM:

\begin{itemize}
  %%%%% Wavelet %%%%%
\item \xmlNode{Wavelet} performs a discrete wavelet transform on time-dependent
  data. Note: This TSA module requires pywavelets to be installed within your
  python environment.
  \begin{itemize}
  \item \xmlAttr{target}, \xmlDesc{comma sperated string, require field},
    indicates which target sigals should be trained as part of this ROM using this
    TSA algorithm.
  \item \xmlNode{family}, \xmlDesc{string, required field}, indicates which family
    of wavelets to use.

    There are several possible families to choose from, and most families contain
    more than one variation. For more information regarding the wavelet families,
    refer to the Pywavelets documentation.

    Possible values are:
    \begin{itemize}
    \item \textbf{haar family}: haar
    \item \textbf{db family}: db1, db2, db3, db4, db5, db6, db7, db8, db9, db10, db11,
      db12, db13, db14, db15, db16, db17, db18, db19, db20, db21, db22, db23,
      db24, db25, db26, db27, db28, db29, db30, db31, db32, db33, db34, db35,
      db36, db37, db38
    \item \textbf{sym family}: sym2, sym3, sym4, sym5, sym6, sym7, sym8, sym9, sym10,
      sym11, sym12, sym13, sym14, sym15, sym16, sym17, sym18, sym19, sym20
    \item \textbf{coif family}: coif1, coif2, coif3, coif4, coif5, coif6, coif7, coif8,
      coif9, coif10, coif11, coif12, coif13, coif14, coif15, coif16, coif17
    \item \textbf{bior family}: bior1.1, bior1.3, bior1.5, bior2.2, bior2.4, bior2.6,
      bior2.8, bior3.1, bior3.3, bior3.5, bior3.7, bior3.9, bior4.4, bior5.5,
      bior6.8
    \item \textbf{rbio family}: rbio1.1, rbio1.3, rbio1.5, rbio2.2, rbio2.4, rbio2.6,
      rbio2.8, rbio3.1, rbio3.3, rbio3.5, rbio3.7, rbio3.9, rbio4.4, rbio5.5,
      rbio6.8
    \item \textbf{dmey family}: dmey
    \item \textbf{gaus family}: gaus1, gaus2, gaus3, gaus4, gaus5, gaus6, gaus7, gaus8
    \item \textbf{mexh family}: mexh
    \item \textbf{morl family}: morl
    \item \textbf{cgau family}: cgau1, cgau2, cgau3, cgau4, cgau5, cgau6, cgau7, cgau8
    \item \textbf{shan family}: shan
    \item \textbf{fbsp family}: fbsp
    \item \textbf{cmor family}: cmor
    \end{itemize}
  \end{itemize}
  %%%%% PolynomialRegression %%%%%
\item \xmlNode{PolynomialRegression} fits time-series data using a polynomial function of degree one or greater.

  \xmlNode{PolynomialRegression} has the following attributes:
  \begin{itemize}
    \item \xmlAttr{target}, \xmlDesc{comma seperated string, required field}, indicates which
        target signals should be trained as part of this ROM using this TSA algorithm.
  \end{itemize}

  \xmlNode{PolynomialRegression} has the following subnodes:
  \begin{itemize}
  \item \xmlNode{degree}, \xmlDesc{integer, required field}, indicates which
    degree of polynomial to fit to the presented data.
  \end{itemize}

  %%%%% FOURIER %%%%%
  \item \xmlNode{Fourier} uses regression to fit requested Fourier bases by their amplitudes to
    deterministically match the training signal. Fourier signals are defined with the following form:
    \begin{equation*}
      F_m(t) = C_m\sin\left( \frac{2\pi}{k_m}t + \phi_m\right)
    \end{equation*}
    where $m$ indexes a particular base period $k_m$, $C_m$ is the amplitude of this Fourier base in the
    training signal, and $\phi_m$ is the phase shift of this Fourier base in the training signal.

    \xmlNode{Fourier} has the following parameters:
    \begin{itemize}
      \item \xmlAttr{target}, \xmlDesc{comma seperated string, required field}, indicates which
        target signals should be trained as part of this ROM using this TSA algorithm.
      \item \xmlAttr{seed}, \xmlDesc{integer, optional}, provides a static seed to be used for
        random number generation in this ROM. Unused for the Fourier TSA algorithm.
    \end{itemize}
    \xmlNode{Fourier} further has the following subnodes:
    \begin{itemize}
      \item \xmlNode{periods}, \xmlDesc{comma seperated floats, required field}, indicates which base
        periods whose Fourier reperesentations should be fit to the training signal. For example, in
        a signal with hourly measurements, selecting the period \xmlString{12, 24} would fit the daily
        (24-hour) and half-daily (12-hour) periodic trends.
    \end{itemize}

  %%%%% ARMA %%%%%
  \item \xmlNode{ARMA} characterizes the signal using Auto-Regressive and Moving Average
    coefficients to stochastically fit the training signal.
    The ARMA representation has the following form:
    \begin{equation*}
      A_t = \sum_{i=1}^P \phi_i A_{t-i} + \epsilon_t + \sum_{j=1}^Q \theta_j \epsilon_{t-j},
    \end{equation*}
    where $t$ indicates a discrete time step, $\phi$ are the signal lag (or auto-regressive)
    coefficients, $P$ is the number of signal lag terms to consider, $\epsilon$ is a random noise
    term, $\theta$ are the noise lag (or moving average) coefficients, and $Q$ is the number of
    noise lag terms to consider. The ARMA algorithms are developed in RAVEN using the
    \texttt{statsmodels} Python library.

    \xmlNode{ARMA} has the following parameters:
    \begin{itemize}
      \item \xmlAttr{target}, \xmlDesc{comma seperated string, required field}, indicates which
        target signals should be trained as part of this ROM using this TSA algorithm.
      \item \xmlAttr{seed}, \xmlDesc{integer, optional}, provides a static seed to be used for
        random number generation in this ROM. This applies both to the training and sampling of this
        ROM.
      \item \xmlAttr{reduce\_memory}, \xmlDesc{boolean, optional field}, activates a lower memory
        usage ARMA training. This does tend to result
        in a slightly slower training time, at the benefit of lower memory usage. For
        example, in one 1000-length history test, low memory reduced memory usage by 2.3
        MiB (80\%), but increased training time by 0.4 seconds (20\%). No change in results has been
        observed switching between modes. Note that the ARMA must be
        retrained to change this property; it cannot be applied to serialized ARMAs.
        \default{False}
    \end{itemize}
    \xmlNode{ARMA} further has the following subnodes:
    \begin{itemize}
      \item \xmlNode{SignalLag}, \xmlDesc{integer, required field}, number of signal lag terms to
        include in the autoregression term.
      \item \xmlNode{NoiseLag}, \xmlDesc{integer, required field}, number of noise lag terms to
        include in the moving average term.
    \end{itemize}
\end{itemize}




\textbf{SyntheticHistory Example:}
\begin{lstlisting}[style=XML,morekeywords={name,subType,pivotLength,shift,target,threshold,period,width}]
<Simulation>
  ...
  <Models>
    ...
    <ROM name="synth" subType="SyntheticHistory">
      <Target>signal1, signal2, hour</Target>
      <Features>scaling</Features>
      <pivotParameter>hour</pivotParameter>
      <fourier target="signal1, signal2">
        <periods>12, 24</periods>
      </fourier>
      <arma target="signal1, signal2" seed='42'>
        <SignalLag>2</SignalLag>
        <NoiseLag>3</NoiseLag>
      </arma>
    </ROM>
    ...
  </Models>
  ...
</Simulation>
\end{lstlisting}


%%%% ROM Model - ARMA  %%%%%%%
\subsubsection{ARMA}
\label{subsubsec:arma}
The ARMA sub-type contains a single ROM type, based on an autoregressive moving average time series model with
Fourier signal processing, sometimes referred to as a FARMA.
%
ARMA is a type of time dependent model that characterizes the autocorrelation between time series data. The mathematic description of ARMA is given as
\begin{equation*}
x_t = \sum_{i=1}^p\phi_ix_{t-i}+\alpha_t+\sum_{j=1}^q\theta_j\alpha_{t-j},
\end{equation*}
where $x$ is a vector of dimension $n$, and $\phi_i$ and $\theta_j$ are both $n$ by $n$ matrices. When $q=0$, the above is
autoregressive (AR); when $p=0$, the above is moving average (MA).
When
training an ARMA, the input needs to be a synchronized HistorySet. For unsynchronized data, use PostProcessor methods to
synchronize the data before training an ARMA.

The ARMA model implemented allows an option to use Fourier series to detrend the time series before fitting to ARMA model to
train. The Fourier trend will be stored in the trained ARMA model for data generation. The following equation
describes the detrending
process.
\begin{equation*}
\begin{aligned}
x_t &= y_t - \sum_m\left\{a_m\sin(2\pi f_mt)+b_m\cos(2\pi f_mt)\right\}  \\
&=y_t - \sum_m\ c_m\sin(2\pi f_mt+\phi_m)
\end{aligned}
\end{equation*}

where $1/f_m$ is defined by the user parameter \xmlNode{Fourier}. \nb $a_m$ and $b_m$ will be calculated then transformed to
$c_m$ and $\phi$. The $c_m$ will be stored as \xmlString{amplitude}, and $\phi$ will be stored as \xmlString{phase}.

By default, each target in the training will be considered independent and have an unique ARMA for each
target.  Correlated targets can be specified through the \xmlNode{correlate} node, at which point
the correlated targets will be trained together using a vector ARMA (or VARMA). Due to limitations in
the VARMA, in order to seed samples the VARMA must be trained with the node \xmlNode{seed}, which acts
independently from the global random seed used by other RAVEN entities.

Both the ARMA and VARMA make use of the \texttt{statsmodels} python package.

%
In order to use this Reduced Order Model, the \xmlNode{ROM} attribute
\xmlAttr{subType} needs to be \xmlString{ARMA} (see the example
below).
%
\subnodeIntro

\begin{itemize}
  \item \xmlNode{pivotParameter}, \xmlDesc{string, required field}, defines the pivot variable (e.g., time) that is non-decreasing in
  the input HistorySet.
  \item \xmlNode{Features}, \xmlDesc{comma separated string, required field}, defines the features (e.g., scaling). Note that only
  one feature is allowed for \xmlString{ARMA} and in current implementation this is used for evaluation only.
  \item \xmlNode{Target}, \xmlDesc{comma separated string, required field}, defines the variable(s) of the
    time series.  Should include the pivot parameter (or Index).
  \item \xmlNode{correlate}, \xmlDesc{comma separated string, optional field}, indicates the listed variables
    should be considered as influencing each other, and trained together instead of independently.  This node
    can only be listed once, so all variables that are desired for correlation should be included.  \nb The
    correlated VARMA takes notably longer to train than the independent ARMAs for the same number of targets.
  \item \xmlNode{seed}, \xmlDesc{integer, optional field}, provides seed for VARMA and ARMA sampling.
   Must be provided before training. If no seed is assigned,
   then a random number will be used.

   \item \xmlNode{reseedCopies}, \xmlDesc{boolean, optional field}, if \xmlString{True} then whenever the ARMA is loaded from file, a
    random reseeding will be performed to ensure different histories. \nb If reproducible histories are desired for an ARMA loaded from file,
    \xmlNode{reseedCopies} should be set to \xmlString{False}, and in the \xmlNode{RunInfo} block \xmlNode{batchSize} needs to be 1
    and \xmlNode{internalParallel} should be
     \xmlString{False} for RAVEN runs sampling the trained ARMA model.

If \xmlNode{InternalParallel} is \xmlString{True} and the ARMA has \xmlNode{reseedCopies} as \xmlString{False}, an identical ARMA history will always be provided regardless of how many samples are taken.

If \xmlNode{InternalParallel} is \xmlString{False} and \xmlNode{batchSize} is more than 1, it is not possible to guarantee the order of RNG usage by the separate processes, so it is not possible to guarantee reproducible histories are generated.        \default{True}

  \item \xmlNode{P}, \xmlDesc{integer, optional field}, defines the value of $p$.
  \default{3}
  \item \xmlNode{Q}, \xmlDesc{integer, optional field}, defines the value of $q$.
  \default{3}
  \item \xmlNode{Fourier}, \xmlDesc{comma-separated integers, optional field}, must be positive integers. This defines the
    based period that will be used for Fourier detrending, i.e., this field defines $1/f_m$ in the above equation.
    When this filed is not specified, the ARMA considers no Fourier detrend.
  \item \xmlNode{outTruncation}, \xmlDesc{comma-separated string, optional}, defines whether and how output
    time series are limited in domain. This node has one attribute, \xmlAttr{domain}, whose value can be
    \xmlString{positive} or \xmlString{negative}. The value of this node contains the list of targets to whom
    this domain limitation should be applied. In the event a negative value is discovered in a target whose
    domain is strictly positive, the absolute value of the original negative value will be used instead, and
    similarly for the negative domain.
  \default{None}


   \item \xmlNode{Peaks}, \xmlDesc{node, optional}, designed to estimate the peaks in signals that repeat with some frequency,
   often in periodic data. The \xmlNode{Peaks} node has the following attributes:
  	\begin{itemize}
    		\item \xmlAttr{target}, \xmlDesc{required string attribute}, defines the name of one target (besides the
		pivot parameter) expected to have periodic peaks.
    		\item \xmlAttr{threshold}, \xmlDesc{required float attribute}, user-defined minimum required
		height of peaks (absolute value).
    		\item \xmlAttr{period}, \xmlDesc{required float attribute}, user-defined expected period for target variable.
	\end{itemize}

    	For example,
	\begin{lstlisting}[style=XML,morekeywords={target,threshold,period}]
	<Peaks target='Speed' threshold='0' period='86400'>
	\end{lstlisting}
	This means the \xmlString{Speed} signal is a daily signal, and the minimal height required to define a peak is 0.
    Additionally, the \xmlNode{Peaks} requires the sub-node:
    	\begin{itemize}
      		 \item \xmlNode{window}, \xmlDesc{comma-separated floats, required field},
		 lists the window of time within each period in which a peak should be discovered.
		 The text of this node is the upper and lower boundary of this
		 window \emph{relative to} the start of the period, separated by a comma.
		 User can define the lower bound to be a negative
		 number if the window passes through one side of one period. For example, if the period is 24
		 hours, the window can be -2,2 which is equivalent to 22, 2.
		 This node has one attribute:
		 \begin{itemize}
		  \item \xmlAttr{width}, \xmlDesc{required float attribute} which is the
		 user defined  width of peaks in that window. The width is in the unit of the signal as well.
		 \end{itemize}
		 The number of \xmlNode{window} is unlimited.
       	\end{itemize}




  \item \xmlNode{preserveInputCDF}, \xmlDesc{boolean, optional field}, enables a final transform on sampled
    data coercing it to have the same distribution as the original data. If \xmlString{True}, then every
    sample generated by this ARMA after training will have a distribution of values that conforms within
    numerical accuracy to the original data. This is especially useful when variance is desired not to stretch
    the most extreme events (high or low signal values), but instead the sequence of events throughout this
    history. For example, this transform can preserve the load duration curve for a load signal.
    \default{False}
  \item \xmlNode{ZeroFilter}, \xmlDesc{comma-separated string, optional field}, turns on \emph{zero filtering}
    for the listed targets. Zero filtering is a very specific algorithm, and should not be used without
    understanding its application.  When zero filtering is enabled, the ARMA will remove all the values from
    the training data equal to zero for the target, then train on the remaining data (including Fourier detrending
    if applicable). If the target is set as correlated to another target, the second target will be treated as
    two distinct series: one containing times in which the original target is zero, and one in the remaining
    times. The results from separated ARMAs are recombined after sampling. This can be a methodology for
    treating histories with long zero-value segments punctuated periodically by peaks.
  \item \xmlNode{SpecificFourier}, \xmlDesc{node, optional}, provides a means to specify different Fourier
    decomposition for different target variables.  Values given in the subnodes of this node will supercede
    the defaults set by the  \xmlNode{Fourier} and \xmlNode{FourierOrder} nodes.  This node
    requires the following attribute:
    \begin{itemize}
      \item \xmlAttr{variables}, \xmlDesc{comma-separated list, required field}, lists the variables to whom
        the \xmlNode{SpecificFourier} parameters will apply.
    \end{itemize}
    Additionally, the \xmlNode{SpecificFourier} node takes the following subnodes:
    \begin{itemize}
      \item \xmlNode{periods}, \xmlDesc{comma-separated integers, required field}, lists the (fundamental)
        periodic wavelength of the Fourier decomposition for these variables, as in the \xmlNode{Fourier} general node.
    \end{itemize}
  \item \xmlNode{Multicycle}, \xmlDesc{node, optional}, indicates that each sample of the ARMA should yield
    multiple sequential samples. For example, if an ARMA model is trained to produce a year's worth of data,
    enabling \xmlNode{Multicycle} causes it to produce several successive years of data. Multicycle sampling
    is independent of ROM training, and only changes how samples of the ARMA are created.

    \nb The output of a multicycle ARMA must be stored in a \xmlNode{DataSet}, as the targets will depend
      on both the \xmlNode{pivotParameter} as well as the cycle, \xmlString{Cycle}. The cycle is a second
      \xmlNode{Index} that all targets should depend on, with variable name \xmlString{Cycle}.

    The \xmlNode{Multicycle} node accepts the following subnodes:
    \begin{itemize}
      \item \xmlNode{cycles}, \xmlDesc{integer, required field}, the number of cycles the ARMA should produce
        each time it yields a sample.
      \item \xmlNode{growth}, \xmlDesc{float, optional field}, if provided then the histories produced by
        the ARMA will be increased by the growth factor for successive cycles. This node can be added
        multiple times with different settings for different targets.
        The text of this node is the growth factor in percentage. Some examples are in
        Table~\ref{tab:arma multicycle growth}, where \emph{Growth factor} is the value used in the RAVEN
        input and \emph{Scaling factor} is the value by which the history will be multiplied.
        \begin{table}[h!]
          \centering
          \begin{tabular}{r c l}
            Growth factor & Scaling factor & Description \\ \hline
            50 & 1.5 & growing by 50\% each cycle \\
            -50 & 0.5 & shrinking by 50\% each cycle \\
            150 & 2.5 & growing by 150\% each cycle \\
          \end{tabular}
          \caption{ARMA Growth Factor Examples}
          \label{tab:arma multicycle growth}
        \end{table}

        The \xmlNode{growth} node takes the following attributes as settings:
        \begin{itemize}
          \item \xmlAttr{targets}, \xmlDesc{comma-seperated list, required field}, lists the targets
            in this ARMA that this growth factor should apply to.
          \item \xmlAttr{mode}, \xmlDesc{string, required field}, either \xmlString{linear} or
            \xmlString{exponential}, determines the manner in which the growth factor is applied.

            If \xmlString{linear}, then the scaling factor is $(1+y\cdot g/100)$;

            if \xmlString{exponential}, then the scaling factor is $(1+g/100)^y$;

            where $y$ is the cycle after the first and $g$ is the provided scaling factor.
        \end{itemize}
    \end{itemize}
    \romClusterOption
\end{itemize}

Note that when loading the ARMA model from a serialized file as a \xmlNode{pickledROM},
several nodes can be used to modify the evaluation behavior:
\begin{itemize}
  \item \xmlNode{reseed}, \xmlDesc{integer, optional},
  \item \xmlNode{Multicycle}, \xmlDesc{node, optional}, as \xmlNode{Multicycle} above, allows resetting
    a growth factor and number of cycles to sample.
  \item \xmlNode{clusterEvalMode}, \xmlDesc{string, optional}, one of \xmlString{truncated},
    \xmlString{full}, or \xmlString{clustered}, changes the structure of the samples for Clustered
    Segmented ROMs. These are identical to the options for \xmlNode{evalMode} node under \xmlNode{Segmented}
    as described above.
\end{itemize}
the \xmlNode{seed}
as well as the \xmlNode{Multicycle} nodes can be added to change the behavior of the ARMA ROM.

When using ROM segmentation/clustering, the ARMA provides the following classes of features that can
be used for clustering:
\begin{itemize}
   \item \xmlString{global}, the segment-long mean values of the signal;
   \item \xmlString{fourier}, the fundamental Fourier frequency amplitudes discovered during training;
   \item \xmlString{arma}, the standard deviation, $p$, and $q$ coefficients obtained during training;
   \item \xmlString{peak}, the peak probability, mean, standard deviation, and most probable index discovered during training.
\end{itemize}

\textbf{General ARMA Example:}
\begin{lstlisting}[style=XML,morekeywords={name,subType,pivotLength,shift,target,threshold,period,width}]
<Simulation>
  ...
  <Models>
    ...
    <ROM name='aUserDefinedName' subType='ARMA'>
      <pivotParameter>Time</pivotParameter>
      <Features>scaling</Features>
      <Target>Speed1,Speed2</Target>
      <P>5</P>
      <Q>4</Q>
      <Segment>
        <subspace pivotLength="1296000" shift="first">Time</subspace>
      </Segment>
      <preserveInputCDF>True</preserveInputCDF>
      <Fourier>604800,86400</Fourier>
      <FourierOrder>2, 4</FourierOrder>
      <Peaks target='Speed1' threshold='0.1' period='86400'>
        <window width='14400' >-7200,10800</window>
        <window width='18000' >64800,75600</window>
      </Peaks>
     </ROM>
    ...
  </Models>
  ...
</Simulation>
\end{lstlisting}

%%%% ROM Model - PolyExponential  %%%%%%%
\subsubsection{PolyExponential}
\label{subsubsec:polyexponential}
The PolyExponential sub-type contains a single ROM type, aimed to construct a time-dependent (or any other monotonic
variable) surrogate model based on polynomial sum of exponential term. This surrogate have the form:
%
\begin{equation}
  SM(X,z) = \sum_{i=1}^{N} P_{i}(X) \times \exp ( - Q_{i}(X) \times z )
\end{equation}
where:
\begin{itemize}
  \item $\mathbf{z}$ is the independent  monotonic variable (e.g. time)
  \item $\mathbf{X}$  is the vector of the other independent (parametric) variables  (Features)
  \item $\mathbf{P_{i}}(X)$ is a polynomial of rank M function of the parametric space X
  \item  $\mathbf{Q_{i}}(X)$ is a polynomial of rank M function of the parametric space X
  \item  $\mathbf{N}$ is the number of requested exponential terms.
\end{itemize}
It is crucial to notice that this model is quite suitable for FOMs whose drivers are characterized by an exponential-like behavior.
In addition, it is important to notice that the exponential terms' coefficients are computed running a genetic-algorithm optimization
problem, which is quite slow in case of increasing number of ``numberExpTerms''.
%
In order to use this Reduced Order Model, the \xmlNode{ROM} attribute
\xmlAttr{subType} needs to be set equal to \xmlString{PolyExponential} (see the example
below).
%
\subnodeIntro

\begin{itemize}
  \item \xmlNode{pivotParameter}, \xmlDesc{string, optional field}, defines the pivot variable (e.g., time) that represents the
  independent monotonic variable
  \default{time}
  \item \xmlNode{Features}, \xmlDesc{comma separated string, required field}, defines the features (i.e. input parameters) of this
  model
  \item \xmlNode{Target}, \xmlDesc{comma separated string, required field}, defines output FOMs that are going to be predicted
  \item \xmlNode{numberExpTerms}, \xmlDesc{integer, optional field}, the number of exponential terms to be used ($N$ above)
   \default{3}
  \item \xmlNode{coeffRegressor}, \xmlDesc{string, optional field}, defines which regressor to use for interpolating the
   exponential coefficient. Available are ``spline'',``poly'' and ``nearest''.
    \default{spline}
  \item \xmlNode{polyOrder}, \xmlDesc{integer, optional field}, the polynomial order to be used for interpolating the exponential
  coefficients. Only valid in case of  \xmlNode{coeffRegressor} set to ``poly''.
   \default{2}
  \item \xmlNode{tol}, \xmlDesc{float, optional field}, relative tolerance of the optimization problem (differential evolution optimizer)
   \default{1e-3}
  \item \xmlNode{maxNumberIter}, \xmlDesc{integer, optional field}, maximum number of iterations (generations) for the
  optimization problem  (differential evolution optimizer)
   \default{5000}
\end{itemize}

\textbf{Example:}
\begin{lstlisting}[style=XML,morekeywords={name,subType}]
<Simulation>
  ...
  <Models>
    ...
   <ROM name='PolyExp' subType='PolyExponential'>
     <Target>time,decay_heat, xe135_dens</Target>
     <Features>enrichment,bu</Features>
     <pivotParameter>time</pivotParameter>
     <numberExpTerms>5</numberExpTerms>
     <max_iter>1000000</max_iter>
     <tol>0.000001</tol>
  </ROM>
    ...
  </Models>
  ...
</Simulation>
\end{lstlisting}
Once the ROM is trained (\textbf{Step} \xmlNode{RomTrainer}), its coefficients can be exported into an XML file
via an \xmlNode{OutStream} of type \xmlAttr{Print}. The following variable/parameters can be exported (i.e. \xmlNode{what} node
in \xmlNode{OutStream} of type \xmlAttr{Print}):
\begin{itemize}
  \item \xmlNode{expTerms}, see XML input specifications above, inquired pre-pending the keyword ``output|'' (e.g. output| expTerms)
  \item \xmlNode{coeffRegressor}, see XML input specifications above
  \item \xmlNode{polyOrder}, see XML input specifications above
  \item \xmlNode{features}, see XML input specifications above
  \item \xmlNode{timeScale}, XML node containing the array of the training time steps values
  \item \xmlNode{coefficients}, XML node containing the exponential terms' coefficients for each realization
\end{itemize}


 See the following example:
\begin{lstlisting}[style=XML,morekeywords={name,subType}]
<Simulation>
  ...
  <OutStreams>
    ...
    <Print name = 'dumpAllCoefficients'>
      <type>xml</type>
      <source>PolyExp</source>
      <!--
        here the <what> node is omitted. All the available params/coefficients
        are going to be printed out
      -->
    </Print>
    <Print name = 'dumpSomeCoefficients'>
      <type>xml</type>
      <source>PolyExp</source>
      <what>coefficients,timeScale</what>
    </Print>
    ...
  </OutStreams>
  ...
</Simulation>
\end{lstlisting}

%%%% ROM Model - DMD  %%%%%%%
\subsubsection{DMD}
\label{subsubsec:polyexponential}
The DMD sub-type contains a single ROM type, aimed to construct a time-dependent (or any other monotonic
variable) surrogate model based on Dynamic Mode Decomposition (ref. \cite{Schmid2010DMD} and \cite{Vega2017HODMD}).
This surrogate is aimed to perform a ``dimensionality reduction regression'', where, given time series (or any monotonic-dependent
variable) of data, a set of modes each of which is associated with a fixed oscillation frequency and decay/growth rate is computed
in order to represent the data-set.
%
In order to use this Reduced Order Model, the \xmlNode{ROM} attribute
\xmlAttr{subType} needs to be set equal to \xmlString{DMD} (see the example
below).
%
\subnodeIntro

\begin{itemize}
  \item \xmlNode{dmdType}, \xmlDesc{string, optional field}, the type of Dynamic Mode Decomposition to apply. Available are:
   \begin{itemize}
     \item \textit{dmd}, for classical DMD
     \item \textit{hodmd}, for high order DMD.
   \end{itemize}
   \default{dmd}
  \item \xmlNode{pivotParameter}, \xmlDesc{string, optional field}, defines the pivot variable (e.g., time) that represents the
  independent monotonic variable
  \default{time}
  \item \xmlNode{Features}, \xmlDesc{comma separated string, required field}, defines the features (i.e. input parameters) of this
  model
  \item \xmlNode{Target}, \xmlDesc{comma separated string, required field}, defines output FOMs that are going to be predicted
  \item \xmlNode{rankSVD}, \xmlDesc{integer, optional field}, defines the truncation rank to be used for the SVD.
     Available options are:
     \begin{itemize}
     \item \textit{-1}, no truncation is performed
     \item \textit{0}, optimal rank is internally computed
     \item \textit{>1}, this rank is going to be used for the truncation
   \end{itemize}
   \default{-1}
  \item \xmlNode{energyRankSVD}, \xmlDesc{float, optional field},  energy level ($0.0 < float < 1.0$) used to compute the rank such
    as computed rank is the number of the biggest singular values needed to reach the energy identified by
    \xmlNode{energyRankSVD}. This node has always priority over  \xmlNode{rankSVD}
    \default{None}
  \item \xmlNode{rankTLSQ}, \xmlDesc{integer, optional field}, $int > 0$ that defines the truncation rank to be used for the total
   least square problem. If not inputted, no truncation is applied
   \default{None}
   \item \xmlNode{exactModes}, \xmlDesc{bool, optional field}, True if the exact modes need to be computed (eigenvalues and
   eigenvectors),   otherwise the projected ones (using the left-singular matrix after SVD).
  \default{True}
  \item \xmlNode{optimized}, \xmlDesc{float, optional field}, True if the amplitudes need to be computed minimizing the error
   between the modes and all the time-steps or False, if only the 1st timestep only needs to be considered
   \default{True}

\end{itemize}

\textbf{Example:}
\begin{lstlisting}[style=XML,morekeywords={name,subType}]
<Simulation>
  ...
  <Models>
    ...
   <ROM name='DMD' subType='DMD'>
      <Target>time,totals_watts, xe135_dens</Target>
      <Features>enrichment,bu</Features>
      <dmdType>dmd</dmdType>
      <pivotParameter>time</pivotParameter>
      <rankSVD>0</rankSVD>
      <rankTLSQ>5</rankTLSQ>
      <exactModes>False</exactModes>
      <optimized>True</optimized>
    </ROM
    ...
  </Models>
  ...
</Simulation>
\end{lstlisting}

Once the ROM  is trained (\textbf{Step} \xmlNode{RomTrainer}), its parameters/coefficients can be exported into an XML file
via an \xmlNode{OutStream} of type \xmlAttr{Print}. The following variable/parameters can be exported (i.e. \xmlNode{what} node
in \xmlNode{OutStream} of type \xmlAttr{Print}):
\begin{itemize}
  \item \xmlNode{rankSVD}, see XML input specifications above
  \item \xmlNode{energyRankSVD}, see XML input specifications above
  \item \xmlNode{rankTLSQ}, see XML input specifications above
  \item \xmlNode{exactModes}, see XML input specifications above
  \item \xmlNode{optimized}, see XML input specifications above
  \item \xmlNode{features}, see XML input specifications above
  \item \xmlNode{timeScale}, XML node containing the array of the training time steps values
  \item \xmlNode{dmdTimeScale}, XML node containing the array of time scale in the DMD space (can be used as mapping
  between the  \xmlNode{timeScale} and \xmlNode{dmdTimeScale})
  \item \xmlNode{eigs}, XML node containing the eigenvalues (imaginary and real part)
  \item \xmlNode{amplitudes}, XML node containing the amplitudes (imaginary and real part)
  \item \xmlNode{modes}, XML node containing the dynamic modes (imaginary and real part)
\end{itemize}


 See the following example:
\begin{lstlisting}[style=XML,morekeywords={name,subType}]
<Simulation>
  ...
  <OutStreams>
    ...
    <Print name = 'dumpAllCoefficients'>
      <type>xml</type>
      <source>DMD</source>
      <!--
        here the <what> node is omitted. All the available params/coefficients
        are going to be printed out
      -->
    </Print>
    <Print name = 'dumpSomeCoefficients'>
      <type>xml</type>
      <source>PolyExp</source>
      <what>eigs,amplitudes,modes</what>
    </Print>
    ...
  </OutStreams>
  ...
</Simulation>
\end{lstlisting}

%%%%%%%%%%%%%%%%%%%%%%%%%%%%%%%%%%%%%%%%%%%%%%%%%%%%%
%%%%% ROM Model - TensorFlow-Keras Interface  %%%%%%%
%%%%%%%%%%%%%%%%%%%%%%%%%%%%%%%%%%%%%%%%%%%%%%%%%%%%%
%%%%%% command used for TensorFlow-Keras Deep neural networks %%%%%%%%%%%%%%%%%
\newcommand{\layerNameAttr}[0]
{
    This node require the following attribute:
    \begin{itemize}
      \item \xmlAttr{name}, \xmlDesc{string, required field}, name of this layer. The value will be
        used in \xmlNode{layer\_layout} to construct the fully connected neural network.
    \end{itemize}
}
%%% Arguments for Dense Layer %%%
\newcommand{\activation}[0]
{
      \item \xmlNode{activation}, \xmlDesc{string, optional field}, including
        {`relu', `tanh', `elu', `selu', `softplus', `softsign', `sigmoid', `hard\_sigmoid', `linear', `softmax'}.
        (see~\ref{activationsDNN})
        \default{linear}
}
\newcommand{\dimOut}[0]
{
      \item \xmlNode{dim\_out}, \xmlDesc{positive integer, required except if this layer is used as the last output layer},
        dimensionality of the output space of this layer
}
\newcommand{\useBias}[0]
{
  \item \xmlNode{use\_bias}, \xmlDesc{boolean, optional field}, whether the layer uses a bias vector.
    \default{True}
}
\newcommand{\kernelInitializer}[0]
{
  \item \xmlNode{kernel\_initializer}, \xmlDesc{string, optional field}, initializer for the kernel weights matrix
    (see~\ref{initializersDNN}).
    \default{glorot\_uniform}
}
\newcommand{\biasInitializer}[0]
{
  \item \xmlNode{bias\_initializer}, \xmlDesc{string, optional field}, intializer for the bias vector
    (see ~\ref{initializersDNN}).
    \default{zeros}
}
\newcommand{\kernelRegularizer}[0]
{
  \item \xmlNode{kernel\_regularizer}, \xmlDesc{string, optional field}, regularizer function applied to
    the kernel weights matrix (see ~\ref{regularizersDNN}).
    \default{None}
}
\newcommand{\biasRegularizer}[0]
{
  \item \xmlNode{bias\_regularizer}, \xmlDesc{string, optional field}, regularizer function applied to the bias vector
    (see~\ref{regularizersDNN}).
    \default{None}
}
\newcommand{\activityRegularizer}[0]
{
  \item \xmlNode{activity\_regularizer}, \xmlDesc{string, optional field}, regularizer function applied to the output
    of the layer (its "activation"). (see~\ref{regularizersDNN})
    \default{None}
}
\newcommand{\kernelConstraint}[0]
{
  \item \xmlNode{kernel\_constraint}, \xmlDesc{string, optional field}, constraint function applied to the kernel weights
    matrix (see~\ref{constraintsDNN}).
    \default{None}
}
\newcommand{\biasConstraint}[0]
{
  \item \xmlNode{bias\_constraint}, \xmlDesc{string, optional field}, constraint function applied to the bias vector
    (see ~\ref{constraintsDNN})
    \default{None}
}
%%% Arguments for Dropout Layer %%%
\newcommand{\rate}[0]
{
      \item \xmlNode{rate}, \xmlDesc{float between 0 and 1, optional field}, fraction of the input units to drop.
        \default{0}
}
\newcommand{\noiseShape}[0]
{
      \item \xmlNode{noise\_shape}, \xmlDesc{list of integers, optional field}, 1D integer tensor representing the shape
        of the binary dropout mask that will be multiplied with the input.
        \default{None}
}
\newcommand{\seed}[0]
{
      \item \xmlNode{seed}, \xmlDesc{integer, optional field}, a integer to use as random seed.
        \default{None}
}
%%% Arguments for LSTM Layer %%%
\newcommand{\recurrentActivation}[0]
{
      \item \xmlNode{recurrent\_activation}, \xmlDesc{string, optional field}, activation function to use for the recurrent
        step, including {`relu', `tanh', `elu', `selu', `softplus', `softsign', `sigmoid', `hard\_sigmoid', `linear', `softmax'}.
        \default{hard\_sigmoid}
}
\newcommand{\recurrentInitializer}[0]
{
      \item \xmlNode{recurrent\_initializer}, \xmlDesc{string, optional field}, used for the linear transformation of
        the recurrent state (see ~\ref{initializersDNN}).
        \default{orthogonal}
}
\newcommand{\unitForgetBias}[0]
{
      \item \xmlNode{unit\_forget\_bias}, \xmlDesc{boolean, optional field}, add 1 to the bias of the forget gate at
        initialization if True.
        \default{True}
}
\newcommand{\recurrentRegularizer}[0]
{
      \item \xmlNode{recurrent\_regularizer}, \xmlDesc{string, optional field}, regularizer function applied to the
        \textit{recurrent\_kernel} weights matrix (see ~\ref{regularizersDNN}).
        \default{None}
}
\newcommand{\recurrentConstraint}[0]
{
      \item \xmlNode{recurrent\_constraint}, \xmlDesc{string, optional field}, constraint function applied to the
        \textit{recurrent\_kernel} weights matrix (see ~\ref{constraints}).
        \default{None}
}
\newcommand{\dropout}[0]
{
      \item \xmlNode{dropout}, \xmlDesc{float between 0 and 1, optional field}, fraction of the units to drop for the linear
        transformation of the inputs
        \default{0}
}
\newcommand{\recurrentDropout}[0]
{
      \item \xmlNode{recurrent\_dropout}, \xmlDesc{float between 0 and 1, optional field}, fraction of the units to drop for the linear
        transformation of the recurrent state.
        \default{0}
}
\newcommand{\returnSequence}[0]
{
      \item \xmlNode{return\_sequence}, \xmlDesc{boolean, optional field}, whether to return the last output in the output sequence, or
        full sequence.
        \default{False}
}
\newcommand{\implementation}[0]
{
      \item \xmlNode{implementation}, \xmlDesc{integer, optional field},
        implementation mode, either 1 or 2. Mode 1 will structure its operations as a larger number of smaller dot products and additions,
        whereas mode 2 will batch them into fewer, larger operations. These modes will have different performance profiles on different
        hardware and for different applications.
        \default{1}
}
\newcommand{\returnState}[0]
{
      \item \xmlNode{return\_state}, \xmlDesc{boolean, optional field},
        whether to return the last output in the output sequence, or the full sequence.
        \default{False}
}
\newcommand{\goBackwards}[0]
{
      \item \xmlNode{go\_backwards}, \xmlDesc{boolean, optional field},
        if True, process the input sequence backwards and return the reversed sequence.
        \default{False}
}
\newcommand{\stateful}[0]
{
      \item \xmlNode{stateful}, \xmlDesc{boolean, optional field},
        if True, the last state for each sample at index i in a batch will be used as initial state for the sample
        of index i in the following batch.
        \default{False}
}
\newcommand{\unroll}[0]
{
      \item \xmlNode{unroll}, \xmlDesc{boolean, optional field},
        if True, the network will be unrolled, else a symbolic loop will be used. Unrolling can speed-up a RNN,
        although it tends to be more memory-intensive. Unrolling is only suitable for short sequences.
        \default{False}
}

%%% Arguments for Conv1D Layer %%%

\newcommand{\kernelSize}[0]
{
      \item \xmlNode{kernel\_size}, \xmlDesc{integer or list of integers, required field}, specifying the length
        of the 1D convolution window.
}
\newcommand{\strides}[0]
{
      \item \xmlNode{strides}, \xmlDesc{integer or list of integers, optional field}, pecifying the stride
        length of the convolution. Specifying any stride value not equal 1 is incompatible with specifying any
        dilation\_rate value not equal 1.
        \default{1}
}
\newcommand{\padding}[0]
{
      \item \xmlNode{padding}, \xmlDesc{string, optional field},
        one of "valid", "causal" or "same" (case-insensitive).  "valid" means "no padding".
        "same" results in padding the input such that the output has the same length as the original input.
        "causal" results in causal (dilated) convolutions, e.g. output[t] does not depend on input[t + 1:].
        A zero padding is used such that the output has the same length as the original input. Useful when
        modeling temporal data where the model should not violate the temporal order.
        \default{valid}
}
\newcommand{\dataFormat}[0]
{
      \item \xmlNode{data\_format}, \xmlDesc{string, optional field},
        A string, one of "channels\_last" (default) or "channels\_first". The ordering of the dimensions in the inputs.
        "channels\_last" corresponds to inputs with shape  (batch, steps, channels) (default format for temporal data
        in Keras) while "channels\_first" corresponds to inputs with shape (batch, channels, steps).
        \default{channels\_last}
}
\newcommand{\dilationRate}[0]
{
      \item \xmlNode{dilation\_rate}, \xmlDesc{integer or list of integers, optional field},
        specifying the dilation rate to use for dilated convolution. Currently, specifying any dilation\_rate value
        not equal 1 is incompatible with specifying any strides value not equal 1.
        \default{1}
}

%%% Arguments for Pooling Layer %%%

\newcommand{\poolSize}[0]
{
      \item \xmlNode{pool\_size}, \xmlDesc{integer, required field}, size of the max pooling windows.
        \default{2}
}

\newcommand{\DenseLayer}[0]
{
  \item \xmlNode{Dense}, \xmlDesc{required field}, regular densely-connected neural network layer.
    \layerNameAttr
    In addition, this node also accepts the following subnodes
    \begin{itemize}
        \activation
        \dimOut
        \useBias
        \kernelInitializer
        \biasInitializer
        \kernelRegularizer
        \biasRegularizer
        \activityRegularizer
        \kernelConstraint
        \biasConstraint
    \end{itemize}
}
\newcommand{\DropoutLayer}[0]
{
  \item \xmlNode{Dropout}, \xmlDesc{optional field}, applies Dropout to the input. Dropout consists in
    randomly setting a fraction \xmlNode{rate} of input units to 0 at each update during training time,
    which helps prevent overfitting.
    \layerNameAttr
    In addition, this node also accepts the following subnode
    \begin{itemize}
        \rate
        \noiseShape
        \seed
    \end{itemize}
}

\newcommand{\PoolingLayer}[1]
{

  \item \xmlNode{#1}, \xmlDesc{optional field},
    In addition, this node also accepts the following subnodes
    \begin{itemize}
        \poolSize
        \strides
        \padding
        \dataFormat
    \end{itemize}
}
\newcommand{\GlobalPoolingLayer}[1]
{

  \item \xmlNode{#1}, \xmlDesc{optional field},
    In addition, this node also accepts the following subnodes
    \begin{itemize}
        \dataFormat
    \end{itemize}
}

%%%%%%%%%%%%%%%%%%%%%%%%%%%%%%%%%%%%%%%%%%%%%%%%%%%%%%%%%%%%%%%%%%%%%%%%%%%%%%%

%%%%% ROM Model - TensorFlow-Keras Interface  %%%%%%%
\subsubsection{TensorFlow-Keras Deep Neural Networks}
\label{subsubsec:TFK_DNNs}

\textcolor{red}{\\It is important to NOTE that Python3 is required in order to use these deep neural networks.
If python2 is installed, these ROMs will not be imported by RAVEN, and an error will be raised if the user tries
to use these capabilities.}

\textbf{TensorFlow} is an open source software library for high performance numerical computation. Its flexible architecture
allows easy deployment of computation across a variety of platforms (CPUs, GPUs, TPUs), and from desktops to clusters
of servers to mobile and edge devices. Originally developed by researchers and engineers from the Google Brain team
within Google’s AI organization, it comes with strong support for machine learning and deep learning and the flexible
numerical computation core is used across many other scientific domains.

\textbf{Keras} is a high-level API to build and train deep learning models. It's used for fast prototyping, advanced research,
and production, with three key advantages:
\begin{itemize}
  \item \textit{User friendly}: Keras has a simple, consistent interface optimized for common use cases.
    It provides clear and actionable feedback for user errors.
  \item \textit{Modular and composable}: Keras models are made by connecting configurable building blocks together,
    with few restrictions.
  \item \textit{Easy to extend}: Write custom building blocks to express new ideas for research. Create new layers,
    loss functions, and develop state-of-the-art models.
\end{itemize}

\textbf{tf.keras} is TensorFlow's implementation of the Keras API specification. This is a high-level API to build and train
models that include first-class support for TensorFlow-specific functionality, such as eager \textit{execution},
\textit{tf.data} pipelines, and \textit{Estimators}. \textbf{tf.keras} makes TensorFlow easier to use without sacrificing
flexibility and performance. RAVEN will utilize this high-level API to build and train deep neural networks (DNNs) as ROMs, and
these ROMs can be employed by other RAVEN entities to perform uncertainty quantification, model opimization and data analysis.

Before analyzing each classifier in detail, it is important to mention that each type has a similar syntax. In the
example below, the subnodes that can be included in the main XML node \xmlNode{ROM} are reported:
\textbf{Example:}
\begin{lstlisting}[style=XML,morekeywords={name,subType}]
<Simulation>
  ...
  <Models>
    ...
    <ROM name='aUserDefinedName' subType='whatever'>
      <Features>X,Y</Features>
      <Target>Z</Target>
      <loss>mean_squared_error</loss>
      <metrics>accuracy</metrics>
      <batch_size>4</batch_size>
      <epochs>4</epochs>
      <num_classes>2</num_classes>
      <validation_split>0.25</validation_split>
      <optimizerSetting>
        <optimizer>Adam</optimizer>
        ...
      </optimizerSetting>
      <WhateverLayer1 name="layerName1">
        ...
      </WhateverLayer1>
      ...
      <WhateverLayerN name="layerNameN">
        ...
      </WhateverLayerN>
      <layer_layout>layerName1, ..., layerNameN</layer_layout>
    </ROM>
    ...
  </Models>
  ...
</Simulation>
\end{lstlisting}

As shown in above example, in addition to the common subnodes \xmlNode{Target} and \xmlNode{Features}, the \xmlNode{ROM} of DNNs
can be initialized with the following children:
\begin{itemize}
  \item \xmlNode{loss}, \xmlDesc{string or comma separated string, optional field}, if the model has multiple outputs, you can use a different
    loss metric on each output by passing a list of loss metrics. The value that will be minimized by the model will then
    be the sum of all individual value from each loss metric. Available loss functions include \textit{mean\_squared\_error},
    \textit{mean\_absolute\_error}, \textit{mean\_absolute\_percentage\_error}, \textit{mean\_squared\_logarithmic\_error},
    \textit{squared\_hinge}, \textit{hinge}, \textit{categorical\_hinge}, \textit{logcosh}, \textit{categorical\_crossentropy},
    \textit{sparse\_categorical\_crossentropy}, \textit{binary\_crossentropy}, \textit{kullback\_leibler\_divergence},
    \textit{poisson}, \textit{cosine\_proximity}.
  \default{mean\_squared\_error}
  \item \xmlNode{metrics}, \xmlDesc{string or comma separated string, optional field}, list of metrics to be evaluated by
    the model during training and testing. available metrics include
    \textit{binary\_accuracy}, \textit{categorical\_accuracy}, \textit{sparse\_categorical\_accuracy},
    \textit{top\_k\_categorical\_accuracy}, \textit{sparse\_top\_k\_categorical\_accuracy}.
  \default{accuracy}
  \item \xmlNode{batch\_size}, \xmlDesc{integer, optional field}, number of samples per gradient update.
  \default{20}
  \item \xmlNode{epochs}, \xmlDesc{integer, optional field}, number of epochs to train the model. An epoch
    is an iteration over the entire training data.
  \default{20}
  \item \xmlNode{num\_classes}, \xmlDesc{positive integer, optional field}, dimensionality of the output space of given classifier.
  \default{1}
  \item \xmlNode{validation\_split}, \xmlDesc{float between 0 and 1, optional field}, fraction of the training data to
    be used as validation data.
  \default{0.25}
  \item \xmlNode{plot\_model}, \xmlDesc{boolean, optional field}, if true the DNN model constructed by RAVEN will be
    plotted and stored in the working directory. The file name will be \textit{"ROM name" + "\_" + "model.png"}.
    \nb This capability requires the following libraries, i.e. pydot-ng and graphviz to be installed.
  \default{False}
  \item \xmlNode{optimizerSetting}, \xmlDesc{optional field}, including several subnode depending on the type of
    optimizers.
    \begin{itemize}
      \item \xmlNode{optimizer}, \xmlDesc{string, optional field}, name of optimizer.
    \end{itemize}
    \default{Adam}
    \nb The users can also choose different optimizers to train the ROM. The default algorithm is \textit{Adam}.
    Other available optimizers include:
    \textit{SGD}, \textit{RMSprop}, \textit{Adagrad}, \textit{Adadelta}, \textit{Adamx}, \textit{Nadam}.
    For the detailed information, i.e. the parameters for each optimization, the user can refer to
    \url{https://keras.io/optimizers/}. In raven, the user can use \xmlNode{optimizerSetting} to set the
    parameters of the above optimizer as follows:
    \begin{itemize}
      \item \textbf{Adam}, adam optimizer
        \begin{itemize}
          \item \xmlNode{beta\_1}, \xmlDesc{float, optional field}, $0 < beta < 1$. Generally close to 1.
          \default{0.9}
          \item \xmlNode{beta\_2}, \xmlDesc{float, optional field}, $0 < beta < 1$. Generally close to 1.
          \default{0.999}
          \item \xmlNode{epsilon}, \xmlDesc{float, optional field}, fuzz factor.
          \default{None}
          \item \xmlNode{decay}, \xmlDesc{float, optional field}, learning rate decay over each update.
          \default{0.0}
          \item \xmlNode{lr}, \xmlDesc{float, optional field}, learning rate.
          \default{0.001}
        \end{itemize}
    %
      \item \textbf{SGD}, stochastic gradient descent optimizer.
        \begin{itemize}
          \item \xmlNode{momentum}, \xmlDesc{float, optional field}, $> 0$. Parameter that accelerates SGD in
            the relevant direction and dampens oscillations.
          \default{0.0}
          \item \xmlNode{nesterov}, \xmlDesc{boolean, optional field}, whether to apply Nesterov momentum
          \default{False}
          \item \xmlNode{decay}, \xmlDesc{float, optional field}, learning rate decay over each update.
          \default{0.0}
          \item \xmlNode{lr}, \xmlDesc{float, optional field}, learning rate.
          \default{0.001}
        \end{itemize}
    %
      \item \textbf{RMSprop}, RMSProp optimizer.
        \begin{itemize}
          \item \xmlNode{rho}, \xmlDesc{float, optional field}, $> 0$.
          \default{0.9}
          \item \xmlNode{decay}, \xmlDesc{float, optional field}, learning rate decay over each update.
          \default{0.0}
          \item \xmlNode{lr}, \xmlDesc{float, optional field}, learning rate.
          \default{0.001}
          \item \xmlNode{epsilon}, \xmlDesc{float, optional field}, fuzz factor.
          \default{None}
        \end{itemize}
    %
      \item \textbf{Adagrad}, Adagrad optimizer.
        \begin{itemize}
          \item \xmlNode{decay}, \xmlDesc{float, optional field}, learning rate decay over each update.
          \default{0.0}
          \item \xmlNode{lr}, \xmlDesc{float, optional field}, learning rate.
          \default{0.01}
          \item \xmlNode{epsilon}, \xmlDesc{float, optional field}, fuzz factor.
          \default{None}
        \end{itemize}
    %
      \item \textbf{Adadelta}, Adadelta optimizer.
        \begin{itemize}
          \item \xmlNode{decay}, \xmlDesc{float, optional field}, learning rate decay over each update.
          \default{0.0}
          \item \xmlNode{lr}, \xmlDesc{float, optional field}, learning rate.
          \default{1.0}
          \item \xmlNode{epsilon}, \xmlDesc{float, optional field}, fuzz factor.
          \default{None}
          \item \xmlNode{rho}, \xmlDesc{float, optional field}, $> 0$.
          \default{0.95}
        \end{itemize}
    %
      \item \textbf{Adamax}, Adamax optimizer
        \begin{itemize}
          \item \xmlNode{beta\_1}, \xmlDesc{float, optional field}, $0 < beta < 1$. Generally close to 1.
          \default{0.9}
          \item \xmlNode{beta\_2}, \xmlDesc{float, optional field}, $0 < beta < 1$. Generally close to 1.
          \default{0.999}
          \item \xmlNode{epsilon}, \xmlDesc{float, optional field}, fuzz factor.
          \default{None}
          \item \xmlNode{decay}, \xmlDesc{float, optional field}, learning rate decay over each update.
          \default{0.0}
          \item \xmlNode{lr}, \xmlDesc{float, optional field}, learning rate.
          \default{0.002}
        \end{itemize}
    %
      \item \textbf{Nadam},
        \begin{itemize}
          \item \xmlNode{beta\_1}, \xmlDesc{float, optional field}, $0 < beta < 1$. Generally close to 1.
          \default{0.9}
          \item \xmlNode{beta\_2}, \xmlDesc{float, optional field}, $0 < beta < 1$. Generally close to 1.
          \default{0.999}
          \item \xmlNode{epsilon}, \xmlDesc{float, optional field}, fuzz factor.
          \default{None}
          \item \xmlNode{lr}, \xmlDesc{float, optional field}, learning rate.
          \default{0.002}
        \end{itemize}
  \end{itemize}
  \item \xmlNode{layer\_layout}, \xmlDesc{comma seperated string, required}, the layout/order of layers in the
    deep neural networks. The values in the subnode should be the name of layers defined in layer node, such as
    \xmlNode{Dense}, \xmlNode{Dropout}, and \xmlNode{Conv1D}.
\end{itemize}

\nb The descriptions regarding the \xmlNode{WhateverLayer} node will be introduced in following subsections.
Basically, different classifiers will require different layers.
In addition, most core layers will accept the \xmlNode{activation} subnode (see ~\ref{activationsDNN}).

%%%%% Activation Functions  %%%%%%%
\paragraph{Activation Functions}
\label{activationsDNN}
Activations can either be used through an \xmlNode{Activation} layer, or through the
\xmlNode{activation} argument supported by all forward layers.
Available activations include:
\begin{itemize}
  \item \textit{relu}, the rectified linear unit function, returns $f(x) = max(0, x)$.
  \item \textit{tanh}, the hyperbolic tan function, returns $f(x) = tanh(x)$.
  \item \textit{elu}, exponential linear units try to make the mean activations closer to zero which speeds
    up learning. $f(x) = x$ if $x \ge 0$, otherwise $(exp(x) - 1.)$.
  \item \textit{selu}, scaled exponential linear unit, i.e. $scale * elu(x, alpha)$, where $scale, alpha$
    are pre-defined constants.
  \item \textit{softplus}, a smooth approximation to the rectifier linear unit function, return
    $f(x) = log(1. + exp(x))$.
  \item \textit{softsign}, return $f(x) = \frac{x}{1. + |x|}$.
  \item \textit{sigmoid},return $f(x) = \frac{1.}{1. + exp(-x)}$.
  \item \textit{hard\_sigmoid}, hard sigmoid activation function.
  \item \textit{linear}, i.e. identity.
  \item \textit{softmax}, softmax activation function, return $f(x) = \frac{exp(x_i)}{\sum_i{exp(x_i)}}$
\end{itemize}

%%%%% Initializer Functions  %%%%%%%
\paragraph{Initializer Functions}
\label{initializersDNN}
Initializations define the way to set the initial random weights of TensorFlow-Keras layers. The keyword
arguments used to passing initializers to layers will depend on the layer. Usually it is simply
\xmlNode{kernel\_initializer} and \xmlNode{bias\_initializer}.
Available initializers include:
\begin{itemize}
  \item \textit{Zeros}, generates tensors initialized to 0.
  \item \textit{Ones}, generates tensors initialized to 1.
  \item \textit{Constant}, generates tensors initialized to a constant value.
  \item \textit{RandomNormal}, generates tensors with a normal distribution.
  \item \textit{RandomUniform}, generates tensors with a uniform distribution.
  \item \textit{TruncatedNormal}, generates a truncated normal distribution.
  \item \textit{VarianceScaling}, initializer capable of adapting its scale to the shape of weights.
  \item \textit{Orthogonal}, generates a random orthogonal matrix.
  \item \textit{Identity}, generates the identity matrix.
  \item \textit{lecun\_uniform}, LeCun uniform initializer.
    It draws samples from a uniform distribution within
    $[-limit, limit]$ where \textit{limit} is $sqrt(3/fanIn)$ where \textit{fanIn} is the number of input dimensions
    in the weight tensor.
  \item \textit{glorot\_normal}, Glorot normal initializer.
    It draws samples from a truncated normal distribution
    centered on 0 with $stddev = sqrt(2/(fanIn + fanOut))$ where \textit{fanIn} is the number of input dimensions
    in the weight tensor and \textit{fanOut} is the number of output dimensions in the weight tensors.
  \item \textit{glorot\_uniform}, Glorot uniform initializer.
    It draws samples from a uniform distribution within
    $[-limit, limit]$ where \textit{limit} is $sqrt(6/(fanIn+fanOut))$.
  \item \textit{he\_normal}, He normal initializer.
    It draws samples from a truncated normal distribution
    centered on 0 with $stddev = sqrt(2/fanIn)$.
  \item \textit{lecun\_normal}, LeCun normal initializer.
    It draws samples from a truncated normal distribution
    centered on 0 with $stddev = sqrt(1/fanIn)$.
  \item \textit{he\_uniform}, He uniform variance scaling initializer.
    It draws samples from a uniform distribution within
    $[-limit, limit]$ where \textit{limit} is $sqrt(6/fanIn)$ where \textit{fanIn} is the number of input dimensions
    in the weight tensor.
\end{itemize}

%%%%% Rgularizer Functions  %%%%%%%
\paragraph{Regularizer Functions}
\label{regularizersDNN}
Regularizers allow to apply penalities on layer parameters or layer activity during optimization.
These penalties are incorporated in the loss function that the network optimizes. The exact API
will depend on the layer, but the layers \xmlNode{Dense, Conv1D, Conv2D, and Conv3D} have a
unified API.
Available regularizers include:
\begin{itemize}
  \item \textit{l1}, l1 regularization
  \item \textit{l2}, l2 regularization
  \item \textit{l1\_l2}, l1 and l2 regularization
\end{itemize}

%%%%% Constraint Functions  %%%%%%%
\paragraph{Constraint Functions}
\label{constraintsDNN}
Functions from the \textit{constraint} module allow setting constraints on network parameters during optimization.
Available constraints include:
\begin{itemize}
  \item \textit{MaxNorm}, constrains the weights incident to each hidden unit to have a norm less than or equal to
    a desired value.
  \item \textit{NonNeg}, constrains the weights to be non-negative
  \item \textit{UnitNorm}, constrains the weights incident to each hidden unit to have unit norm.
  \item \textit{MinMaxNorm}, constrains the weights incident to each hidden unit to have the norm between a lower bound
    and an upper bound.
\end{itemize}

%%%%% ROM Model - KerasMLPClassifier  %%%%%%%
\paragraph{KerasMLPClassifier}
\label{KerasMLPClassifier}

Multi-Layer Perceptron (MLP) (or Artificial Neural Network - ANN), a class of feedforward
ANN, can be viewed as a logistic regression classifier where input is first transformed
using a non-linear transformation. This transformation probjects the input data into a
space where it becomes linearly separable. This intermediate layer is referred to as a
\textbf{hidden layer}. An MLP consists of at least three layers of nodes. Except for the
input nodes, each node is a neuron that uses a nonlinear \textbf{activation function}. MLP
utilizes a suppervised learning technique called \textbf{Backpropagation} for training.
Generally, a single hidden layer is sufficient to make MLPs a universal approximator.
However, many hidden layers, i.e. deep learning, can be used to model more complex nonlinear
relationships. The extra layers enable composition of features from lower layers, potentially
modeling complex data with fewer units than a similarly performing shallow network.

\zNormalizationPerformed{KerasMLPClassifier}

In order to use this ROM, the \xmlNode{ROM} attribute \xmlAttr{subType} needs to
be \xmlString{KerasMLPClassifier} (see the example below). This model can be initialized with
the following layers:

\begin{itemize}
  \DenseLayer
  \DropoutLayer
\end{itemize}

\textbf{Example:}
\begin{lstlisting}[style=XML,morekeywords={name,subType}]
<Simulation>
  ...
  <Models>
    ...
    <ROM name='aUserDefinedName' subType='KerasMLPClassifier'>
      <Features>X,Y</Features>
      <Target>Z</Target>
      <loss>mean_squared_error</loss>
      <metrics>accuracy</metrics>
      <batch_size>4</batch_size>
      <epochs>4</epochs>
      <optimizerSetting>
        <beta_1>0.9</beta_1>
        <optimizer>Adam</optimizer>
        <beta_2>0.999</beta_2>
        <epsilon>1e-8</epsilon>
        <decay>0.0</decay>
        <lr>0.001</lr>
      </optimizerSetting>
      <Dense name="layer1">
          <activation>relu</activation>
          <dim_out>15</dim_out>
      </Dense>
      <Dropout name="dropout1">
          <rate>0.2</rate>
      </Dropout>
      <Dense name="layer2">
          <activation>tanh</activation>
          <dim_out>8</dim_out>
      </Dense>
      <Dropout name="dropout2">
          <rate>0.2</rate>
      </Dropout>
      <Dense name="outLayer">
          <activation>sigmoid</activation>
      </Dense>
      <layer_layout>layer1, dropout1, layer2, dropout2, outLayer</layer_layout>
    </ROM>
    ...
  </Models>
  ...
</Simulation>
\end{lstlisting}

%%%%% ROM Model - KerasConvNetClassifier  %%%%%%%
\paragraph{KerasConvNetClassifier}
\label{KerasClassifier}

Convolutional Neural Network (CNN) is a deep learning algorithm which can take in an input image, assign
importance to various objects in the image and be able to differentiate one from the other. The
architecture of a CNN is analogous to that of the connectivity pattern of Neurons in the Human Brain
and was inspired by the organization of the Visual Cortex. Individual neurons respond to stimuli only
in a restricted region of the visual field known as the Receptive Field. A collection of such fields
overlap to cover the entire visual area. CNN is able to successfully capture the spatial and temporal
dependencies in an image through the applicaiton of relevant filters. The architecture performs
a better fitting to the image dataset due to the reduction in the number of parameters involved
and reusability of weights. In other words, the network can be trained to understand the sophistication
of the image better.

\zNormalizationPerformed{KerasConvNetClassifier}

In order to use this ROM, the \xmlNode{ROM} attribute \xmlAttr{subType} needs to
be \xmlString{KerasConvNetClassifier} (see the example below). This model can be initialized with
the following layers:

\begin{itemize}
  \DenseLayer
  \DropoutLayer
  \item \xmlNode{Conv1D}, \xmlDesc{optional field},
    \layerNameAttr
    In addition, this node also accepts the following subnodes
    \begin{itemize}
        \activation
        \dimOut
        \useBias
        \kernelSize
        \strides
        \padding
        \dataFormat
        \dilationRate
        \kernelInitializer
        \biasInitializer
        \kernelRegularizer
        \biasRegularizer
        \activityRegularizer
        \kernelConstraint
        \biasConstraint
    \end{itemize}

  \item \xmlNode{Conv2D}, \xmlDesc{optional field},
    In addition, this node also accepts the following subnodes
    \begin{itemize}
        \activation
        \dimOut
        \useBias
        \kernelSize
        \strides
        \padding
        \dataFormat
        \dilationRate
        \kernelInitializer
        \biasInitializer
        \kernelRegularizer
        \biasRegularizer
        \activityRegularizer
        \kernelConstraint
        \biasConstraint
    \end{itemize}

  \item \xmlNode{Conv3D}, \xmlDesc{optional field},
    In addition, this node also accepts the following subnodes
    \begin{itemize}
        \activation
        \dimOut
        \useBias
        \kernelSize
        \strides
        \padding
        \dataFormat
        \dilationRate
        \kernelInitializer
        \biasInitializer
        \kernelRegularizer
        \biasRegularizer
        \activityRegularizer
        \kernelConstraint
        \biasConstraint
    \end{itemize}

  \item \xmlNode{Flatten}, \xmlDesc{optional field},
    In addition, this node also accepts the following subnodes
    \begin{itemize}
        \dataFormat
    \end{itemize}

  \PoolingLayer{MaxPooling1D}
  \PoolingLayer{MaxPooling2D}
  \PoolingLayer{MaxPooling3D}
  \PoolingLayer{AveragePooling1D}
  \PoolingLayer{AveragePooling2D}
  \PoolingLayer{AveragePooling3D}
  \GlobalPoolingLayer{GlobalMaxPooling1D}
  \GlobalPoolingLayer{GlobalMaxPooling2D}
  \GlobalPoolingLayer{GlobalMaxPooling3D}
  \GlobalPoolingLayer{GlobalAveragePooling1D}
  \GlobalPoolingLayer{GlobalAveragePooling2D}
  \GlobalPoolingLayer{GlobalAveragePooling3D}
\end{itemize}

\textbf{Example:}
\begin{lstlisting}[style=XML,morekeywords={name,subType}]
<Simulation>
  ...
  <Models>
    ...
    <ROM name='aUserDefinedName' subType='KerasConvNetClassifier'>
      <Features>x1,x2</Features>
      <Target>labels</Target>
      <loss>mean_squared_error</loss>
      <metrics>accuracy</metrics>
      <batch_size>1</batch_size>
      <epochs>2</epochs>
      <plot_model>True</plot_model>
      <validation_split>0.25</validation_split>
      <num_classes>1</num_classes>
      <optimizerSetting>
        <beta_1>0.9</beta_1>
        <optimizer>Adam</optimizer>
        <beta_2>0.999</beta_2>
        <epsilon>1e-8</epsilon>
        <decay>0.0</decay>
        <lr>0.001</lr>
      </optimizerSetting>
      <Conv1D name="firstConv1D">
          <activation>relu</activation>
          <strides>1</strides>
          <kernel_size>2</kernel_size>
          <padding>valid</padding>
          <dim_out>32</dim_out>
      </Conv1D>
      <MaxPooling1D name="pooling1">
          <strides>2</strides>
          <pool_size>2</pool_size>
      </MaxPooling1D>
      <Conv1D name="SecondConv1D">
          <activation>relu</activation>
          <strides>1</strides>
          <kernel_size>2</kernel_size>
          <padding>valid</padding>
          <dim_out>32</dim_out>
      </Conv1D>
      <MaxPooling1D name="pooling2">
          <strides>2</strides>
          <pool_size>2</pool_size>
      </MaxPooling1D>
      <Flatten name="flatten">
      </Flatten>
      <Dense name="dense1">
          <activation>relu</activation>
          <dim_out>10</dim_out>
      </Dense>
      <Dropout name="dropout1">
          <rate>0.25</rate>
      </Dropout>
      <Dropout name="dropout2">
          <rate>0.25</rate>
      </Dropout>
      <Dense name="dense2">
          <activation>softmax</activation>
      </Dense>
      <layer_layout>firstConv1D, pooling1, dropout1, SecondConv1D, pooling2, dropout2, flatten, dense1, dense2</layer_layout>
    </ROM>
    ...
  </Models>
  ...
</Simulation>
\end{lstlisting}

%%%%% ROM Model - KerasLSTMClassifier  %%%%%%%
\paragraph{KerasLSTMClassifier and KerasLSTMRegression}
\label{KerasClassifier}

Long Short Term Memory networks (LSTM) are a special kind of recurrent neural network, capable
of learning long-term dependencies. They work tremendously well on a large variety of problems, and
are now widely used. LSTMs are explicity designed to avoid the long-term dependency problem. Remembering
information for long periods of time is practically their default behavior, not something that they
struggle to learn.

LSTM's can be used for either classification (with
\xmlString{KerasLSTMClassifier}) or prediction of values (with
\xmlString{KerasLSTMRegression}).

\zNormalizationPerformed{KerasLSTMClassifier \textup{and} KerasLSTMRegression}

In order to use this ROM, the \xmlNode{ROM} attribute \xmlAttr{subType} needs to
be \xmlString{KerasLSTMClassifier} or \xmlString{KerasLSTMRegression} (see the examples below). This model can be initialized with
the following layers:

\begin{itemize}
  \DenseLayer
  \DropoutLayer
  \item \xmlNode{LSTM}, \xmlDesc{required field}, long short-term memory layer.
    \layerNameAttr
    In addition, this node also accepts the following subnodes
    \begin{itemize}
        \activation
        \dimOut
        \recurrentActivation
        \dropout
        \recurrentDropout
        \returnSequence
        \useBias
        \kernelInitializer
        \recurrentInitializer
        \biasInitializer
        \unitForgetBias
        \kernelRegularizer
        \recurrentRegularizer
        \biasRegularizer
        \activityRegularizer
        \kernelConstraint
        \recurrentConstraint
        \biasConstraint
        \implementation
        \returnState
        \goBackwards
        \stateful
        \unroll
    \end{itemize}
\end{itemize}

\textbf{KerasLSTMClassifier Example:}
\begin{lstlisting}[style=XML,morekeywords={name,subType}]
<Simulation>
  ...
  <Models>
    ...
    <ROM name='aUserDefinedName' subType='KerasLSTMClassifier'>
      <Features>x</Features>
      <Target>y</Target>
      <loss>categorical_crossentropy</loss>
      <metrics>accuracy</metrics>
      <batch_size>1</batch_size>
      <epochs>10</epochs>
      <validation_split>0.25</validation_split>
      <num_classes>26</num_classes>
      <optimizerSetting>
        <beta_1>0.9</beta_1>
        <optimizer>Adam</optimizer>
        <beta_2>0.999</beta_2>
        <epsilon>1e-8</epsilon>
        <decay>0.0</decay>
        <lr>0.001</lr>
      </optimizerSetting>
      <LSTM name="lstm1">
          <activation>tanh</activation>
          <dim_out>32</dim_out>
      </LSTM>
      <LSTM name="lstm2">
          <activation>tanh</activation>
          <dim_out>16</dim_out>
      </LSTM>
      <Dropout name="dropout">
          <rate>0.25</rate>
      </Dropout>
      <Dense name="dense">
          <activation>softmax</activation>
      </Dense>
      <layer_layout>lstm1,lstm2,dropout,dense</layer_layout>
    </ROM>
    ...
  </Models>
  ...
</Simulation>
\end{lstlisting}

\textbf{KerasLSTMRegression Example:}
\begin{lstlisting}[style=XML,morekeywords={name,subType}]
<Simulation>
  ...
  <Models>
    ...
    <ROM name="lstmROM" subType="KerasLSTMRegression">
      <Features>prev_sum, prev_square, prev_square_sum</Features>
      <Target>sum, square</Target>
      <pivotParameter>index</pivotParameter>
      <loss>mean_squared_error</loss>
      <LSTM name="lstm1">
        <dim_out>32</dim_out>
      </LSTM>
      <LSTM name="lstm2">
        <dim_out>16</dim_out>
      </LSTM>
      <Dense name="dense">
      </Dense>
      <layer_layout>lstm1, lstm2, dense</layer_layout>

    </ROM>
    ...
  </Models>
  ...
</Simulation>
\end{lstlisting}




%%%%%%%%%%%%%%%%%%%%%%%%
%%%%%%  External  Model   %%%%%%
%%%%%%%%%%%%%%%%%%%%%%%%
\subsection{External Model}
\label{subsec:models_externalModel}
As the name suggests, an external model is an entity that is embedded in the
RAVEN code at run time.
%
This object allows the user to create a python module that is going to be
treated as a predefined internal model object.
%
In other words, the \textbf{External Model} is going to be treated by RAVEN as a
normal external Code (e.g. it is going to be called in order to compute an
arbitrary quantity based on arbitrary input).
%

The specifications of an External Model must be defined within the XML block
\xmlNode{ExternalModel}.
%
This XML node needs to contain the attributes:

\vspace{-5mm}
\begin{itemize}
  \itemsep0em
  \item \xmlAttr{name}, \xmlDesc{required string attribute}, user-defined name
  of this External Model.
  %
  \nb As with the other objects, this is the name that can be used to refer to
  this specific entity from other input blocks in the XML.
  \item \xmlAttr{subType}, \xmlDesc{required string attribute}, must be kept
  empty.
  \item \xmlAttr{ModuleToLoad}, \xmlDesc{required string attribute}, file name
  with its absolute or relative path.
  %
  \nb If a relative path is specified, the code first checks relative
  to the working directory, then it checks with respect to where the
  user runs the code.  Using the relative path with respect to where the
  code is run is not recommended.
  %
\end{itemize}
\vspace{-5mm}

In order to make the RAVEN code aware of the variables the user is going to
manipulate/use in her/his own python Module, the variables need to be specified
in the \xmlNode{ExternalModel} input block.
%
The user needs to input, within this block, only the variables that RAVEN needs
to be aware of (i.e. the variables are going to directly be used by the code)
and not the local variables that the user does not want to, for example, store
in a RAVEN internal object.
%
These variables are specified within a \xmlNode{variables} block:
\begin{itemize}
  \item \xmlNode{variables}, \xmlDesc{string, required parameter}.
  %
  Comma-separated list of variable names.
  %
  Each variable name needs to match a variable used/defined in the external python
  model.
  %
\end{itemize}

In addition, if the user wants to use the alias system, the following XML block can be inputted:
\begin{itemize}
  \item \aliasSystemDescription{ExternalModel}
\end{itemize}


When the external function variables are defined, at run time, RAVEN initializes
them and tracks their values during the simulation.
%
Each variable defined in the \xmlNode{ExternalModel} block is available in the
module (each method implemented) as a python ``self.''
%

In the External Python module, the user can implement all the methods that are
needed for the functionality of the model, but only the following methods, if
present, are called by the framework:
\begin{itemize}
  \item \texttt{\textbf{def \_readMoreXML}}, \xmlDesc{OPTIONAL METHOD}, can be
  implemented by the user if the XML input that belongs to this External Model
  needs to be extended to contain other information.
  %
  The information read needs to be stored in ``self'' in order to be available
  to all the other methods (e.g. if the user needs to add a couple of newer XML
  nodes with information needed by the algorithm implemented in the ``run''
  method).
  \item \texttt{\textbf{def initialize}}, \xmlDesc{OPTIONAL METHOD}, can
  implement all the actions need to be performed at the initialization stage.
  \item \texttt{\textbf{def createNewInput}}, \xmlDesc{OPTIONAL METHOD}, creates
  a new input with the information coming from the RAVEN framework.
  %
  In this function the user can retrieve the information coming from the RAVEN
  framework, during the employment of a calculation flow, and use them to
  construct a new input that is going to be transferred to the ``run'' method.
  \item \texttt{\textbf{def run}}, \xmlDesc{REQUIRED METHOD}, is the actual
  location where the user needs to implement the model action (e.g. resolution
  of a set of equations, etc.).
  %
  This function is going to receive the Input (or Inputs) generated either by
  the External Model ``createNewInput'' method or the internal RAVEN one.
\end{itemize}

In the following sub-sections, all the methods are going to be analyzed in
detail.

\subsubsection{Method: \texttt{def \_readMoreXML}}
\label{subsubsec:externalReadMoreXML}
As already mentioned, the \textbf{readMoreXML} method can be implemented by the
user if the XML input that belongs to this External Model needs to be extended
to contain other information.
%
The information read needs to be stored in ``self'' in order to be available to
all the other methods (e.g. if the user needs to add a couple of newer XML nodes
with information needed by the algorithm implemented in the ``run'' method).
%
If this method is implemented in the \textbf{External Model}, RAVEN is going to
call it when the node \xmlNode{ExternalModel} is found parsing the XML input
file.
%
The method receives from RAVEN an attribute of type ``xml.etree.ElementTree'',
containing all the sub-nodes and attribute of the XML block \xmlNode{ExternalModel}.
%

Example XML:
\begin{lstlisting}[style=XML,morekeywords={subType,ModuleToLoad}]
<Simulation>
  ...
  <Models>
     ...
    <ExternalModel name='AnExtModule' subType='' ModuleToLoad='path_to_external_module'>
       <variables>sigma,rho,outcome</variables>
       <!--
          here we define other XML nodes RAVEN does not read automatically.
          We need to implement, in the external module 'AnExtModule' the readMoreXML method
        -->
        <newNodeWeNeedToRead>
            whatNeedsToBeRead
        </newNodeWeNeedToRead>
    </ExternalModel>
     ...
  </Models>
  ...
</Simulation>
\end{lstlisting}

Corresponding Python function:
\begin{lstlisting}[language=python]
def _readMoreXML(self,xmlNode):
  # the xmlNode is passed in by RAVEN framework
  # <newNodeWeNeedToRead> is unknown (in the RAVEN framework)
  # we have to read it on our own
  # get the node
  ourNode = xmlNode.find('newNodeWeNeedToRead')
  # get the information in the node
  self.ourNewVariable = ourNode.text
  # end function
\end{lstlisting}


\subsubsection{Method: \texttt{def initialize}}
\label{subsubsec:externalInitialize}
The \textbf{initialize} method can be implemented in the \textbf{External Model}
in order to initialize some variables needed by it.
%
For example, it can be used to compute a quantity needed by the ``run'' method
before performing the actual calculation).
%
If this method is implemented in the \textbf{External Model}, RAVEN is going to
call it at the initialization stage of each ``Step'' (see section
\ref{sec:steps}.
%
RAVEN will communicate, thorough a set of method attributes, all the information
that are generally needed to perform a initialization:
\begin{itemize}
  \item runInfo, a dictionary containing information regarding how the
  calculation is set up (e.g. number of processors, etc.).
  %
  It contains the following attributes:
  \begin{itemize}
    \item \texttt{DefaultInputFile} -- default input file to use
    \item \texttt{SimulationFiles} -- the xml input file
    \item \texttt{ScriptDir} -- the location of the pbs script interfaces
    \item \texttt{FrameworkDir} -- the directory where the framework is located
    \item \texttt{WorkingDir} -- the directory where the framework should be
    running
    \item \texttt{TempWorkingDir} -- the temporary directory where a simulation
    step is run
    \item \texttt{NumMPI} -- the number of mpi process by run
    \item \texttt{NumThreads} -- number of threads by run
    \item \texttt{numProcByRun} -- total number of core used by one run (number
    of threads by number of mpi)
    \item \texttt{batchSize} -- number of contemporaneous runs
    \item \texttt{ParallelCommand} -- the command that should be used to submit
    jobs in parallel (mpi)
    \item \texttt{numNode} -- number of nodes
    \item \texttt{procByNode} -- number of processors by node
    \item \texttt{totalNumCoresUsed} -- total number of cores used by driver
    \item \texttt{queueingSoftware} -- queueing software name
    \item \texttt{stepName} -- the name of the step currently running
    \item \texttt{precommand} -- added to the front of the command that is run
    \item \texttt{postcommand} -- added after the command that is run
    \item \texttt{delSucLogFiles} -- if a simulation (code run) has not failed,
    delete the relative log file (if True)
    \item \texttt{deleteOutExtension} -- if a simulation (code run) has not
    failed, delete the relative output files with the listed extension (comma
    separated list, for example: `e,r,txt')
    \item \texttt{mode} -- running mode, curently the only mode supported is
      mpi (but custom modes can be created)
    \item \textit{expectedTime} -- how long the complete input is expected to
    run
    \item \textit{logfileBuffer} -- logfile buffer size in bytes
  \end{itemize}
  \item inputs, a list of all the inputs that have been specified in the
  ``Step'' using this model.
  %
\end{itemize}
In the following an example is reported:
\begin{lstlisting}[language=python]
def initialize(self,runInfo,inputs):
 # Let's suppose we just need to initialize some variables
  self.sigma = 10.0
  self.rho   = 28.0
  # end function
\end{lstlisting}


\subsubsection{Method: \texttt{def createNewInput}}
\label{subsubsec:externalcreateNewInput}

The \textbf{createNewInput} method can be implemented by the user to create a
new input with the information coming from the RAVEN framework.
%
In this function, the user can retrieve the information coming from the RAVEN
framework, during the employment of a calculation flow, and use them to
construct a new input that is going to be transferred to the ``run'' method.
%
The new input created needs to be returned to RAVEN (i.e. ``return NewInput'').
\\This method expects that the new input is returned in a Python ``dictionary''.
%
RAVEN communicates, thorough a set of method attributes, all the information
that are generally needed to create a new input:
%myInput,samplerType,**Kwargs
\begin{itemize}
  \item \texttt{inputs}, \xmlDesc{python list}, a list of all the inputs that
  have been defined in the ``Step'' using this model.
  \item \texttt{samplerType}, \xmlDesc{string}, the type of Sampler, if a
  sampling strategy is employed; will be None otherwise.
  \item \texttt{Kwargs}, \xmlDesc{dictionary}, a dictionary containing several
  pieces of information (that can change based on the ``Step'' type).
  %
  If a sampling strategy is employed, this dictionary contains another
  dictionary identified by the keyword ``SampledVars'', in which the variables
  perturbed by the sampler are reported.
\end{itemize}
\nb If the ``Step'' that is using this Model has as input(s) an object of main
class type ``DataObjects'' (see Section~\ref{sec:DataObjects}), the internal ``createNewInput''
method is going to convert it in a dictionary of values.
%

Here we present an example:
\begin{lstlisting}[language=python]
def createNewInput(self,inputs,samplerType,**Kwargs):
  # in here the actual createNewInput of the
  # model is implemented
  if samplerType == 'MonteCarlo':
    avariable = inputs['something']*inputs['something2']
  else:
    avariable = inputs['something']/inputs['something2']
  return avariable*Kwargs['SampledVars']['aSampledVar']
\end{lstlisting}

\subsubsection{Method: \texttt{def run}}
\label{subsubsec:externalRun}
As stated previously, the only method that \emph{must} be present in an
External Module is the \textbf{run} function.
%
In this function, the user needs to implement the algorithm that RAVEN will
execute.
%
The \texttt{\textbf{run}} method is generally called after having inquired the
``createNewInput'' method (either the internal or the user-implemented one).
%
The only attribute this method is going to receive is a Python list of inputs
(the inputs coming from the \texttt{createNewInput} method).
%
If the user wants RAVEN to collect the results of this method, the outcomes of
interest need to be stored in ``self.''
%
\nb RAVEN is trying to collect the values of the variables listed only in the
\xmlNode{ExternalModel} XML block.
%

In the following an example is reported:
\begin{lstlisting}[language=python]
def run(self,Input):
  # in here the actual run of the
  # model is implemented
  input = Input[0]
  self.outcome = self.sigma*self.rho*input[``whatEver'']
\end{lstlisting}

%\subsection{Projector}
%\label{sec:models_projector}
%
%Description

%Summary

%Example

%%%%%%%%%%%%%%%%%%%%%%%%%%%%%%%%%%
%%%%%%         PostProcessor         %%%%%%%%%
%%%%%%%%%%%%%%%%%%%%%%%%%%%%%%%%%%
\subsection{PostProcessor}
\label{sec:models_postProcessor}
A Post-Processor (PP) can be considered as an action performed on a set of data
or other type of objects.
%
Most of the post-processors contained in RAVEN, employ a mathematical operation
on the data given as ``input''.
%
RAVEN supports several different types of PPs.

Currently, the following types are available in RAVEN:
\begin{itemize}
  \itemsep0em
  \item \textbf{BasicStatistics}
  \item \textbf{ComparisonStatistics}
  \item \textbf{ImportanceRank}
  \item \textbf{SafestPoint}
  \item \textbf{LimitSurface}
  \item \textbf{LimitSurfaceIntegral}
  \item \textbf{External}
  \item \textbf{TopologicalDecomposition}
  \item \textbf{DataMining}
  \item \textbf{HistorySetDelay}
  \item \textbf{HS2PS}
  \item \textbf{HStoPSOperator}
  \item \textbf{HistorySetSampling}
  \item \textbf{HistorySetSnapShot}
  \item \textbf{HistorySetSync}
  \item \textbf{TypicalHistoryFromHistorySet}
  \item \textbf{dataObjectLabelFilter}
  \item \textbf{Metric}
  \item \textbf{CrossValidation}
  \item \textbf{ValueDuration}
  \item \textbf{FastFourierTransform}
  \item \textbf{SampleSelector}
  \item \textbf{ParetoFrontier}
  \item \textbf{EconomicRatio}
  \item \textbf{Validation}
  %\item \textbf{PrintCSV}
  %\item \textbf{LoadCsvIntoInternalObject}
\end{itemize}

The specifications of these types must be defined within the XML block
\xmlNode{PostProcessor}.
%
This XML node needs to contain the attributes:
\vspace{-5mm}
\begin{itemize}
  \itemsep0em
  \item \xmlAttr{name}, \xmlDesc{required string attribute}, user-defined
  identifier of this post-processor.
  %
  \nb As with other objects, this is the name that can be used to refer to this
  specific entity from other input XML blocks.
  \item \xmlAttr{subType}, \xmlDesc{required string attribute}, defines which of
  the post-processors needs to be used, choosing among the previously reported
  types.
  %
  This choice conditions the subsequent required and/or optional
  \xmlNode{PostProcessor} sub nodes.
  %
\end{itemize}
\vspace{-5mm}

As already mentioned, all the types and meaning of the remaining sub-nodes
depend on the post-processor type specified in the attribute \xmlAttr{subType}.
%
In the following sections the specifications of each type are reported.

%%%%% PP BasicStatistics %%%%%%%
\subsubsection{BasicStatistics}
\label{BasicStatistics}
The \textbf{BasicStatistics} post-processor is the container of the algorithms
to compute many of the most important statistical quantities. It is important to notice that this
post-processor can accept as input both \textit{\textbf{PointSet}} and \textit{\textbf{HistorySet}}
data objects, depending on the type of statistics the user wants to compute:
\begin{itemize}
  \item \textit{\textbf{PointSet}}: Static Statistics;
  \item \textit{\textbf{HistorySet}}: Dynamic Statistics. Depending on a ``pivot parameter'' (e.g. time)
  the post-processor is going to compute the statistics for each value of it (e.g. for each time step).
  In case an \textbf{HistorySet} is provided as Input, the Histories needs to be synchronized (use
    \textit{\textbf{Interfaced}} post-processor of type  \textbf{HistorySetSync}).
\end{itemize}
%
\ppType{BasicStatistics post-processor}{BasicStatistics}
\begin{itemize}
  \item \xmlNode{"metric"}, \xmlDesc{comma separated string or node list, required field},
    specifications for the metric to be calculated.  The name of each node is the requested metric.  There are
    two forms for specifying the requested parameters of the metric.  For scalar values such as
    \xmlNode{expectedValue} and \xmlNode{variance}, the text of the node is a comma-separated list of the
    parameters for which the metric should be calculated.  For matrix values such as \xmlNode{sensitivty} and
    \xmlNode{covariance}, the matrix node requires two sub-nodes, \xmlNode{targets} and \xmlNode{features},
    each of which is a comma-separated list of the targets for which the metric should be calculated, and the
    features for which the metric should be calculated for that target.  See the example below.

    \nb When defining the metrics to use, it is possible to have multiple nodes with the same name.  For
    example, if a problem has inputs $W$, $X$, $Y$, and $Z$, and the responses are $A$, $B$, and $C$, it is possible that
    the desired metrics are the \xmlNode{sensitivity} of $A$ and $B$ to $X$ and $Y$, as well as the
    \xmlNode{sensitivity} of $C$ to $W$ and $Z$, but not the sensitivity of $A$ to $W$.   In this event, two
    copies of the \xmlNode{sensitivity} node are added to the input.  The first has targets $A,B$ and features
    $X,Y$, while the second node has target $C$ and features $W,Z$.  This could reduce some computation effort
    in problems with many responses or inputs.  An example of this is shown below.
  %
  \\ Currently the scalar quantities available for request are:
  \begin{itemize}
    \item \textbf{expectedValue}: expected value or mean
    \item \textbf{minimum}: The minimum value of the samples.
    \item \textbf{maximum}: The maximum value of the samples.
    \item \textbf{median}:  The weighted median of the samples ( $50\%$ weighted percentile). If probablitity weights are not assigned, uniform distribution will be assigned. The median $x_k$ satisfying:
    \begin{equation}
      \sum_{i = 1}^{k - 1} w_i \le 1/2  and \sum_{i = k + 1}^{n} w_i \le 1/2
    \end{equation}
    \item \textbf{variance}: variance
    \item \textbf{sigma}: standard deviation
    \item \textbf{percentile}: the percentile. If this quantity is inputted as \textit{percentile} the $5\%$ and $95\%$ percentile(s) are going to be computed.
                               Otherwise the user can specify this quantity with a parameter \textit{percent='X'}, where the \textit{X} represents the requested
                               percentile (a floating point value between 0.0 and 100.0)
    \item \textbf{variationCoefficient}: coefficient of variation, i.e. \textbf{sigma}/\textbf{expectedValue}. \nb If the \textbf{expectedValue} is zero,
    the \textbf{variationCoefficient} will be \textbf{INF}.
    \item \textbf{skewness}: skewness
    \item \textbf{kurtosis}: excess kurtosis (also known as Fisher's kurtosis)
    \item \textbf{samples}: the number of samples in the data set used to determine the statistics.
  \end{itemize}
  The matrix quantities available for request are:
  \begin{itemize}
    \item \textbf{sensitivity}: matrix of sensitivity coefficients, computed via linear regression method. (\nb The condition number is computed every time this quantity is requsted. If it results
    to be greater then $30$, a multicollinearity problem exists and the sensitivity coefficients
    might be incorrect and a Warning is spooned by the code)
    \item \textbf{covariance}: covariance matrix
    \item \textbf{pearson}: matrix of correlation coefficients
    \item \textbf{NormalizedSensitivity}: matrix of normalized sensitivity
    coefficients. \nb{It is the matrix of normalized VarianceDependentSensitivity}
    \item \textbf{VarianceDependentSensitivity}: matrix of sensitivity coefficients dependent on the variance of the variables
  \end{itemize}
  This XML node needs to contain the attribute:
  \begin{itemize}
    \itemsep0em
    \item \xmlAttr{prefix}, \xmlDesc{required string attribute}, user-defined prefix for the given \textbf{metric}.
      For scalar quantifies, RAVEN will define a variable with name defined as:  ``prefix'' + ``\_'' + ``parameter name''.
      For example, if we define ``mean'' as the prefix for \textbf{expectedValue}, and parameter ``x'', then variable
      ``mean\_x'' will be defined by RAVEN.
      For matrix quantities, RAVEN will define a variable with name defined as: ``prefix'' + ``\_'' + ``target parameter name'' + ``\_'' + ``feature parameter name''.
      For example, if we define ``sen'' as the prefix for \textbf{sensitivity}, target ``y'' and feature ``x'', then
      variable ``sen\_y\_x'' will be defined by RAVEN.
      \nb These variable will be used by RAVEN for the internal calculations. It is also accessible by the user through
      \textbf{DataObjects} and \textbf{OutStreams}.
  \end{itemize}
   %
  \nb If the weights are present in the system then weighted quantities are calculated automatically. In addition, if a matrix quantity is requested (e.g. Covariance matrix, etc.), only the weights in the output space are going to be used for both input and output space (the computation of the joint probability between input and output spaces is not implemented yet).
  \\
  \nb Certain ROMs provide their own statistical information (e.g., those using
  the sparse grid collocation sampler such as: \xmlString{GaussPolynomialRom}
  and \xmlString{HDMRRom}) which can be obtained by printing the ROM to file
  (xml). For these ROMs, computing the basic statistics on data generated from
  one of these sampler/ROM combinations may not provide the information that the
  user expects.
  \\
  In addition, RAVEN will automatically calculate the standard errors on the following scalar quantities:
  \begin{itemize}
    \item \textbf{expectedValue}
    \item \textbf{median}
    \item \textbf{variance}
    \item \textbf{sigma}
    \item \textbf{skewness}
    \item \textbf{kurtosis}
  \end{itemize}
  RAVEN will define a variable with name defined as: ``prefix for given \textbf{metric}'' + ``\_ste\_'' + ``parameter name'' to
  store standard error of given \textbf{metric} with respect to given parameter. This information will be stored in the DataObjects,
  i.e. \textbf{PointSet} and \textbf{HistorySet}, and by default will be printed out in the ``CSV'' output files by the
  \textbf{OutStreams}. Option node \xmlNode{what} can be used in the \textbf{OutStreams} to select the information that
  the users want to print.
  In the case when the users want to store all the calculations results in general \textbf{DataSets}, RAVEN will employ a variable
  with name defined as: ``\textbf{metric}'' + ``\_ste'' to store standard error with respect to all target parameters. An additional
  index ``target'' will added in the \textbf{DataSets} with respect to these variables. All these quantities will be automatically
  computed and stored in the given \textbf{DataSet}, and the users do not need to specify these quantities in their RAVEN input files.
  %
   \item \xmlNode{pivotParameter}, \xmlDesc{string, optional field}, name of the parameter that needs
   to be used for the computation of the Dynamic BasicStatistics (e.g. time). This node needs to
   be inputted just in case an \textbf{HistorySet} is used as Input. It represents the reference
   monotonic variable based on which the statistics is going to be computed (e.g. time-dependent
   statistical moments).
    \default{None}
  %
  \item \xmlNode{biased}, \xmlDesc{string (boolean), optional field}, if \textit{True} biased
  quantities are going to be calculated, if \textit{False} unbiased.
  \default{False}
  %
  \item \xmlNode{dataset}, \xmlDesc{boolean, optional field}, if \textit{True} \xmlString{DataSet}
    will be used to store the calculation results, if \textit{False} \xmlString{PointSet} or \xmlString{HistorySet}
    will be used to store the calculation results.
    \nb The optional \xmlString{DataSet} is added only to this PostProcessor, one can still use the \xmlString{OutStreams}
    to print the variables available in the \textit{DataSet}. The \xmlString{"metric"} names are used as the
    variable names, i.e. variable names listed in \xmlNode{Input} or \xmlNode{Output} in the defined \xmlString{DataSet}.
    In addition, the extra node \xmlNode{Index} is required, and the value for \xmlAttr{var} can be found in the following:
    \begin{itemize}
      \item scalar metrics, such as \xmlNode{expectedValue} and \xmlNode{variance},
        are requested, the index variable \xmlString{targets} will be required.
      \item vector metrics, such as \xmlNode{covariance} and \xmlNode{sensitivity}, are requested, the index variables
        \xmlString{targets} and \xmlString{features} will be required.
      \item If \xmlNode{percentile} is requested, an additional index variable \xmlString{percent} should be added.
      \item when dynamic BasicStatistics (e.g. time) is requested, the index variable \xmlString{time}  will be required.
    \end{itemize}
  \default{False}
  %
\item \xmlNode{multipleFeatures}, \xmlDesc(boolean, optional field), if \textbf{False}, this node can be used when
    the users want to compute sensitivities based on one target variable with respect to one feature variable,
    i.e. the sensitivity calculations are directly computed using the \textbf{Linear Regression} or
    \textbf{Best Linear Predictor} method with single feature. This method can be useful when the input features
    depend on each other. The default value is \textbf{True}, which means the sensitivity calculations are performed
    using \textbf{Linear Regression} or \textbf{Best Linear Predictor} method with multiple features. If the input
    features are not fully correlated, the default value for \xmlNode{multipleFeatures} is always recommanded.
    \nb this node only affects the calculations of metrics such as \xmlNode{sensitivity},
    \xmlNode{VarianceDependentSensitivity} and \xmlNode{NormalizedSensitivity}.
  \default{True}
\end{itemize}
\textbf{Example (Static Statistics):}  This example demonstrates how to request the expected value of
\xmlString{x01} and \xmlString{x02}, along with the sensitivity of both \xmlString{x01} and \xmlString{x02} to
\xmlString{a} and \xmlString{b}.
\begin{lstlisting}[style=XML,morekeywords={name,subType,debug}]
<Simulation>
  ...
  <Models>
    ...
    <PostProcessor name='aUserDefinedName' subType='BasicStatistics' verbosity='debug'>
      <expectedValue prefix='mean'>x01,x02</expectedValue>
      <sensitivity prefix='sen'>
        <targets>x01,x02</targets>
        <features>a,b</features>
      </sensitivity>
    </PostProcessor>
    ...
  </Models>
  ...
</Simulation>
\end{lstlisting}

In this case, the RAVEN variables ``mean\_x01, mean\_x02, sen\_x01\_a, sen\_x02\_a, sen\_x01\_b, sen\_x02\_b''
will be created and accessible for the RAVEN entities \textbf{DataObjects} and \textbf{OutStreams}.

\textbf{Example (Static, multiple matrix nodes):} This example shows how multiple nodes can specify
particular metrics multiple times to include different target/feature combinations.  This postprocessor
calculates the expected value of $A$, $B$, and $C$, as well as the sensitivity of both $A$ and $B$ to $X$ and
$Y$ as well as the sensitivity of $C$ to $W$ and $Z$.
\begin{lstlisting}[style=XML,morekeywords={name,subType,debug}]
<Simulation>
  ...
  <Models>
    ...
    <PostProcessor name='aUserDefinedName' subType='BasicStatistics' verbosity='debug'>
      <expectedValue prefix='mean'>A,B,C</expectedValue>
      <sensitivity prefix='sen1'>
        <targets>A,B</targets>
        <features>x,y</features>
      </sensitivity>
      <sensitivity prefix='sen2'>
        <targets>C</targets>
        <features>w,z</features>
      </sensitivity>
    </PostProcessor>
    ...
  </Models>
  ...
</Simulation>
\end{lstlisting}
\textbf{Example (Dynamic Statistics):}
\begin{lstlisting}[style=XML,morekeywords={name,subType,debug}]
<Simulation>
  ...
  <Models>
    ...
    <PostProcessor name='aUserDefinedNameForDynamicPP' subType='BasicStatistics' verbosity='debug'>
      <expectedValue prefix='mean'>x01,x02</expectedValue>
      <sensitivity prefix='sen'>
        <targets>x01,x02</targets>
        <features>a,b</features>
      </sensitivity>
      <pivotParameter>time</pivotParameter>
    </PostProcessor>
    ...
  </Models>
  ...
  <HistorySet name='basicStatHistorySet'>
    <Output>
      mean_x01,mean_x02,
      sen_x01_a, sen_x01_b,
      sen_x02_a, sen_x02_b
    </Output>
    <options>
      <pivotParameter>time</pivotParameter>
    </options>
  </HistorySet>
</Simulation>
\end{lstlisting}

\textbf{Example (Dumping the results into DataSet):}
\begin{lstlisting}[style=XML,morekeywords={name,subType,debug}]
<Simulation>
  ...
  <Models>
    ...
    <PostProcessor name='aUserDefinedNameForDynamicPP' subType='BasicStatistics' verbosity='debug'>
      <dataset>True</dataset>
      <expectedValue prefix='mean'>x01,x02</expectedValue>
      <sensitivity prefix='sen'>
        <targets>x01,x02</targets>
        <features>a,b</features>
      </sensitivity>
      <pivotParameter>time</pivotParameter>
    </PostProcessor>
    ...
  </Models>
  ...
  <DataObjects>
    <DataSet name='basicStatDataSet'>
      <Output>expectedValue,sensitivity</Output>
      <Index var='time'>expectedValue,sensitivity</Index>
      <Index var='targets'>expectedValue,sensitivity</Index>
      <Index var='features'>sensitivity</Index>
    </DataSet>
  </DataObjects>
</Simulation>
\end{lstlisting}
%%%%% PP ComparisonStatistics %%%%%%%
\subsubsection{ComparisonStatistics}
\label{ComparisonStatistics}
The \textbf{ComparisonStatistics} post-processor computes statistics
for comparing two different dataObjects.  This is an experimental
post-processor, and it will definitely change as it is further
developed.

There are four nodes that are used in the post-processor.

\begin{itemize}
\item \xmlNode{kind}: specifies information to use for comparing the
  data that is provided.  This takes either uniformBins which makes
  the bin width uniform or equalProbability which makes the number
  of counts in each bin equal.  It can take the following attributes:
  \begin{itemize}
  \item \xmlAttr{numBins} which takes a number that directly
    specifies the number of bins
  \item \xmlAttr{binMethod} which takes a string that specifies the
    method used to calculate the number of bins.  This can be either
    square-root or sturges.
  \end{itemize}
\item \xmlNode{compare}: specifies the data to use for comparison.
  This can either be a normal distribution or a dataObjects:
  \begin{itemize}
  \item \xmlNode{data}: This will specify the data that is used.  The
    different parts are separated by $|$'s.
  \item \xmlNode{reference}: This specifies a reference distribution
    to be used.  It takes distribution to use that is defined in the
    distributions block.  A name parameter is used to tell which
    distribution is used.
  \end{itemize}
\item \xmlNode{fz}: If the text is true, then extra comparison
  statistics for using the $f_z$ function are generated.  These take
  extra time, so are not on by default.
\item \xmlNode{interpolation}: This switches the interpolation used
  for the cdf and the pdf functions between the default of quadratic
  or linear.
\end{itemize}

The \textbf{ComparisonStatistics} post-processor generates a variety
of data.  First for each data provided, it calculates bin boundaries,
and counts the numbers of data points in each bin.  From the numbers
in each bin, it creates a cdf function numerically, and from the cdf
takes the derivative to generate a pdf.  It also calculates statistics
of the data such as mean and standard deviation. The post-processor
can generate a CSV file only.

The post-processor uses the generated pdf and cdf function to
calculate various statistics.  The first is the cdf area difference which is:
\begin{equation}
  cdf\_area\_difference = \int_{-\infty}^{\infty}{\|CDF_a(x)-CDF_b(x)\|dx}
\end{equation}
This given an idea about how far apart the two pieces of data are, and
it will have units of $x$.

The common area between the two pdfs is calculated.  If there is
perfect overlap, this will be 1.0, if there is no overlap, this will
be 0.0.  The formula used is:
\begin{equation}
  pdf\_common\_area = \int_{-\infty}^{\infty}{\min(PDF_a(x),PDF_b(x))}dx
\end{equation}

The difference pdf between the two pdfs is calculated.  This is calculated as:
\begin{equation}
  f_Z(z) = \int_{-\infty}^{\infty}f_X(x)f_Y(x-z)dx
\end{equation}
This produces a pdf that contains information about the difference
between the two pdfs.  The mean can be calculated as (and will be
calculated only if fz is true):
\begin{equation}
  \bar{z} = \int_{-\infty}^{\infty}{z f_Z(z)dz}
\end{equation}
The mean can be used to get an signed difference between the pdfs,
which shows how their means compare.

The variance of the difference pdf can be calculated as (and will be
calculated only if fz is true):
\begin{equation}
  var = \int_{-\infty}^{\infty}{(z-\bar{z})^2 f_Z(z)dz}
\end{equation}

The sum of the difference function is calculated if fz is true, and is:
\begin{equation}
  sum = \int_{-\infty}^{\infty}{f_z(z)dz}
\end{equation}
This should be 1.0, and if it is different that
points to approximations in the calculation.


\textbf{Example:}
\begin{lstlisting}[style=XML]
<Simulation>
   ...
   <Models>
      ...
      <PostProcessor name="stat_stuff" subType="ComparisonStatistics">
      <kind binMethod='sturges'>uniformBins</kind>
      <compare>
        <data>OriData|Output|tsin_TEMPERATURE</data>
        <reference name='normal_410_2' />
      </compare>
      <compare>
        <data>OriData|Output|tsin_TEMPERATURE</data>
        <data>OriData|Output|tsout_TEMPERATURE</data>
      </compare>
      </PostProcessor>
      <PostProcessor name="stat_stuff2" subType="ComparisonStatistics">
        <kind numBins="6">equalProbability</kind>
        <compare>
          <data>OriData|Output|tsin_TEMPERATURE</data>
        </compare>
        <Distribution class='Distributions' type='Normal'>normal_410_2</Distribution>
      </PostProcessor>
      ...
   </Models>
   ...
   <Distributions>
      <Normal name='normal_410_2'>
         <mean>410.0</mean>
         <sigma>2.0</sigma>
      </Normal>
   </Distributions>
</Simulation>
\end{lstlisting}

%%%%% PP ImportanceRank %%%%%%%
\subsubsection{ImportanceRank}
\label{ImportanceRank}
The \textbf{ImportanceRank} post-processor is specifically used
to compute sensitivity indices and importance indices with respect to input parameters
associated with multivariate normal distributions. In addition, the user can also request the transformation
matrix and the inverse transformation matrix when the PCA reduction is used.
%
\ppType{ImportanceRank}{ImportanceRank}
%
\begin{itemize}
  \item \xmlNode{what}, \xmlDesc{comma separated string, required field},
  %
  List of quantities to be computed.
  %
  Currently the quantities available are:
  \begin{itemize}
    \item \xmlString{SensitivityIndex}: used to measure the impact of sensitivities on the model.
    \item \xmlString{ImportanceIndex}: used to measure the impact of sensitivities and input uncertainties on the model.
    \item \xmlString{PCAIndex}: the indices of principal component directions, used to measure the impact
    of principal component directions on input covariance matrix.
    \nb \xmlString{PCAIndex} can be only requested when subnode \xmlNode{latent} is defined in \xmlNode{features}.
    \item \xmlString{transformation}: the transformation matrix used to map the latent variables to the manifest variables in the original input space.
    \item \xmlString{InverseTransformation}: the inverse transformation matrix used to map the manifest variables to the latent variables in the transformed space.
    \item \xmlString{ManifestSensitivity}: the sensitivity coefficients of \xmlNode{target} with respect to \xmlNode{manifest} variables defined in \xmlNode{features}.

    \nb In order to request \xmlString{transformation} matrix or \xmlString{InverseTransformation} matrix or \xmlString{ManifestSensitivity},
    the subnodes \xmlNode{latent} and \xmlNode{manifest} under \xmlNode{features} are required (more details can be found in the following).
    %
  \end{itemize}
  %
  \nb For each computed quantity, RAVEN will define a unique variable name so that the data can be accessible by the users
  through RAVEN entities \textbf{DataObjects} and \textbf{OutStreams}. These variable names are defined as follows:
  \begin{itemize}
    \item \xmlString{SensitivityIndex}: `sensitivityIndex' + `\_' + `targetVariableName' + `\_' + `latentFeatureVariableName'
    \item \xmlString{ImportanceIndex}: `importanceIndex' + `\_' + `targetVariableName' + `\_' + `latentFeatureVariableName'
    \item \xmlString{PCAIndex}: `pcaIndex' + `\_' + `latentFeatureVariableName'
    \item \xmlString{transformation}: `transformation' + `\_' + `manifestFeatureVariableName' + `\_' + `latentFeatureVariableName'
    \item \xmlString{InverseTransformation}: `inverseTransformation' + `\_' + `latentFeatureVariableName' + `\_' + `manifestFeatureVariableName'
    \item \xmlString{ManifestSensitivity}: `manifestSensitivity' + `\_' + `targetVariableName' + `\_' + `manifestFeatureVariableName'
  \end{itemize}
  %
  If all the quantities need to be computed, the user can input in the body of \xmlNode{what} the string \xmlString{all}.
  \nb \xmlString{all} equivalent to \xmlString {SensitivityIndex, ImportanceIndex, PCAIndex}.

  Since the transformation and InverseTransformation matrix can be very large, they are not printed with option \xmlString{all}.
  In order to request the transformation matrix (or inverse transformation matrix) from this post processor,
  the user need to specify \xmlString{transformation} or \xmlString{InverseTransformation} in \xmlNode{what}. In addition,
  both  \xmlNode{manifest} and \xmlNode{latent} subnodes are required and should be defined in node \xmlNode{features}. For example, let $\mathbf{L, P}$ represent
  the transformation and inverse transformation matrices, respectively. We will define vectors $\mathbf x$ as manifest variables and vectors $\mathbf y$
  as latent variables. If a absolute covariance matrix is used in given distribution, the following equation will be used:

  $
  \mathbf{\delta x} = \mathbf L * \mathbf y
  $

  $
  \mathbf y = \mathbf P * \mathbf \delta \mathbf x
  $

  If a relative covariance matrix is used in given distribution, the following equation will be used:

  $
  \frac{\mathbf \delta \mathbf x}{\mathbf \mu} = \mathbf L * \mathbf y
  $

  $
  \mathbf y = \mathbf P * {\frac{\mathbf \delta \mathbf x}{\mathbf \mu}}
  $

  where $\mathbf{\delta x}$ denotes the changes in the input vector $\mathbf x$, and $\mathbf \mu$ denotes the mean values of the input vector $\mathbf x$.

  %
  %
  \item \xmlNode{features}, \xmlDesc{XML node, required parameter}, used to specify the information for the input variables.
  In this xml-node, the following xml sub-nodes need to be specified:
    \begin{itemize}
      \item \xmlNode{manifest},\xmlDesc{XML node, optional parameter}, used to indicate the input variables belongs to the original input space.
      It can accept the following child node:
        \begin{itemize}
          \item \xmlNode{variables},\xmlDesc{comma separated string, required field}, lists manifest variables.
          \item \xmlNode{dimensions}, \xmlDesc{comma separated integer, optional field}, lists the dimensions corresponding to the manifest variables.
          If not provided, the dimensions are determined by the order indices of given manifest variables.
        \end{itemize}
      \item \xmlNode{latent},\xmlDesc{XML node, optional parameter}, used to indicate the input variables belongs to the transformed space.
      It can accept the following child node:
        \begin{itemize}
          \item \xmlNode{variables},\xmlDesc{comma separated string, required field}, lists latent variables.
          \item \xmlNode{dimensions}, \xmlDesc{comma separated integer, optional field}, lists the dimensions corresponding to the latent variables.
          If not provided, the dimensions are determined by the order indices of given latent variables.
        \end{itemize}
      \nb At least one of the subnodes, i.e. \xmlNode{manifest} and \xmlNode{latent} needs to be specified.
    \end{itemize}
  %
  \item \xmlNode{targets}, \xmlDesc{comma separated string, required field}, lists output responses.
  %
  \item \xmlNode{mvnDistribution}, \xmlDesc{string, required field}, specifies the
  multivariate normal distribution name. The \xmlNode{MultivariateNormal} node must be present. It requires two attributes:
    \begin{itemize}
      \item \xmlAttr{class}, \xmlDesc{required string attribute}, is the main
        ``class'' the listed object is from, the only acceptable class for
        this post-processor is \xmlString{Distributions};
      \item \xmlAttr{type}, \xmlDesc{required string attribute}, is the type of distributions,
        the only acceptable type is \xmlString{MultivariateNormal}
    \end{itemize}
\end{itemize}
  %
  %
  Here is an example to show the user how to request the transformation matrix, the inverse transformation matrix, the
  manifest sensitivities and other quantities.
  %

\textbf{Example:}
\begin{lstlisting}[style=XML,morekeywords={name,subType,debug}]
<Simulation>
  ...
  <Models>
    ...
    <PostProcessor name='aUserDefinedName' subType='ImportanceRank'>
      <what>SensitivityIndex,ImportanceIndex,Transformation, InverseTransformation,ManifestSensitivity</what>
      <features>
        <manifest>
          <variables>x1,x2</variables>
          <dimensions>1,2</dimensions>
        </manifest>
        <latent>
          <variables>latent1</variables>
          <dimensions>1</dimensions>
        </latent>
      </features>
      <targets>y</targets>
      <mvnDistribution>MVN</mvnDistribution>
    </PostProcessor>
    ...
  </Models>
  ...
</Simulation>
\end{lstlisting}

The calculation results can be accessible via variables ``sensitivityIndex\_y\_latent1, importanceIndex\_y\_latent1,
manifestSensitivity\_y\_x1, manifestSensitivity\_y\_x2, transformation\_x1\_latent1, transformation\_x2\_latent1,
inverseTransformation\_latnet1\_x1, inverseTransformation\_laent1\_x2'' through RAVEN entities \textbf{DataObjects}
and \textbf{OutStreams}.

%%%%% PP SafestPoint %%%%%%%
\subsubsection{SafestPoint}
\label{SafestPoint}
The \textbf{SafestPoint} post-processor provides the coordinates of the farthest
point from the limit surface that is given as an input.
%
The safest point coordinates are expected values of the coordinates of the
farthest points from the limit surface in the space of the ``controllable''
variables based on the probability distributions of the ``non-controllable''
variables.

The term ``controllable'' identifies those variables that are under control
during the system operation, while the ``non-controllable'' variables are
stochastic parameters affecting the system behaviour randomly.

The ``SafestPoint'' post-processor requires the set of points belonging to the
limit surface, which must be given as an input.
%
The probability distributions as ``Assembler Objects'' are required in the
``Distribution'' section for both ``controllable'' and ``non-controllable''
variables.

The sampling method used by the ``SafestPoint'' is a ``value'' or ``CDF'' grid.
%
At present only the ``equal'' grid type is available.

\ppType{Safest Point}{SafestPoint}

\begin{itemize}
  \item \xmlNode{Distribution}, \xmlDesc{Required}, represents the probability
  distributions of the ``controllable'' and ``non-controllable'' variables.
  %
  These are \textbf{Assembler Objects}, each of these nodes must contain 2
  attributes that are used to identify those within the simulation framework:
        \begin{itemize}
    \item \xmlAttr{class}, \xmlDesc{required string attribute}, is the main
    ``class'' the listed object is from.
                \item \xmlAttr{type}, \xmlDesc{required string attribute}, is the object
    identifier or sub-type.
        \end{itemize}
             \item  \xmlNode{outputName}, \xmlDesc{string, required field}, specifies the name of the output variable where the probability is going to be stored.
               \nb This variable name must be listed in the \xmlNode{Output} field of the Output DataObject
        \item \xmlNode{controllable}, \xmlDesc{XML node, required field},  lists the controllable variables.
  %
  Each variable is associated with its name and the two items below:
        \begin{itemize}
                \item \xmlNode{distribution} names the probability distribution associated
    with the controllable variable.
    %
                \item \xmlNode{grid} specifies the \xmlAttr{type}, \xmlAttr{steps}, and
    tolerance of the sampling grid.
    %
        \end{itemize}
        \item \xmlNode{non-controllable}, \xmlDesc{XML node, required field}, lists the non-controllable variables.
  %
  Each variable is associated with its name and the two items below:
        \begin{itemize}
                \item \xmlNode{distribution} names the probability distribution associated
    with the non-controllable variable.
    %
                \item \xmlNode{grid} specifies the \xmlAttr{type}, \xmlAttr{steps}, and
    tolerance of the sampling grid.
    %
                \end{itemize}
\end{itemize}

\textbf{Example:}
\begin{lstlisting}[style=XML,morekeywords={name,subType,class,type,steps}]
<Simulation>
  ...
    <Models>
    ...
    <PostProcessor name='SP' subType='SafestPoint'>
      <Distribution  class='Distributions'  type='Normal'>x1_dst</Distribution>
      <Distribution  class='Distributions'  type='Normal'>x2_dst</Distribution>
      <Distribution  class='Distributions'  type='Normal'>gammay_dst</Distribution>
      <controllable>
        <variable name='x1'>
          <distribution>x1_dst</distribution>
          <grid type='value' steps='20'>1</grid>
        </variable>
        <variable name='x2'>
          <distribution>x2_dst</distribution>
          <grid type='value' steps='20'>1</grid>
        </variable>
      </controllable>
      <non-controllable>
        <variable name='gammay'>
          <distribution>gammay_dst</distribution>
          <grid type='value' steps='20'>2</grid>
        </variable>
      </non-controllable>
    </PostProcessor>
    ...
  </Models>
  ...
</Simulation>
\end{lstlisting}
%%%%% PP LimitSurface %%%%%%%
\subsubsection{LimitSurface}
\label{LimitSurface}
The \textbf{LimitSurface} post-processor is aimed to identify the transition
zones that determine a change in the status of the system (Limit Surface).

\ppType{LimitSurface}{LimitSurface}

\begin{itemize}
  \item \xmlNode{parameters}, \xmlDesc{comma separated string, required field},
  lists the parameters that define the uncertain domain and from which the LS
  needs to be computed.
  \item \xmlNode{tolerance}, \xmlDesc{float, optional field}, sets the absolute
  value (in CDF) of the convergence tolerance.
 %
  This value defines the coarseness of the evaluation grid.
 %
 \default{1.0e-4}
  \item \xmlNode{side}, \xmlDesc{string, optional field}, in this node the user can specify
  which side of the limit surface needs to be computed. Three options are available:
  \\ \textit{negative},  Limit Surface corresponding to the goal function value of ``-1'';
  \\ \textit{positive}, Limit Surface corresponding to the goal function value of ``1'';
  \\ \textit{both}, either positive and negative Limit Surface is going to be computed.
  %
  %
\default{negative}
  % Assembler Objects
  \item \textbf{Assembler Objects} These objects are either required or optional
  depending on the functionality of the Adaptive Sampler.
  %
  The objects must be listed with a rigorous syntax that, except for the xml
  node tag, is common among all the objects.
  %
  Each of these nodes must contain 2 attributes that are used to map those
  within the simulation framework:
   \begin{itemize}
    \item \xmlAttr{class}, \xmlDesc{required string attribute}, is the main
    ``class'' of the listed object.
    %
    For example, it can be ``Models,'' ``Functions,'' etc.
    \item \xmlAttr{type}, \xmlDesc{required string attribute}, is the object
    identifier or sub-type.
    %
    For example, it can be ``ROM,'' ``External,'' etc.
    %
  \end{itemize}
  The \textbf{LimitSurface} post-processor requires or optionally accepts the
  following objects' types:
   \begin{itemize}
    \item \xmlNode{ROM}, \xmlDesc{string, optional field}, body of this xml
    node must contain the name of a ROM defined in the \xmlNode{Models} block
    (see section \ref{subsec:models_ROM}).
    \item \xmlNode{Function}, \xmlDesc{string, required field}, the body of
    this xml block needs to contain the name of an External Function defined
    within the \xmlNode{Functions} main block (see section \ref{sec:functions}).
    %
    This object represents the boolean function that defines the transition
    boundaries.
    %
    This function must implement a method called
    \textit{\_\_residuumSign(self)}, that returns either -1 or 1, depending on
    the system conditions (see section \ref{sec:functions}).
    %
    \end{itemize}
\end{itemize}

\textbf{Example:}
\begin{lstlisting}[style=XML,morekeywords={name,subType,debug,class,type}]
<Simulation>
 ...
 <Models>
  ...
    <PostProcessor name="computeLimitSurface" subType='LimitSurface' verbosity='debug'>
      <parameters>x0,y0</parameters>
      <ROM class='Models' type='ROM'>Acc</ROM>
      <!-- Here, you can add a ROM defined in Models block.
           If it is not Present, a nearest neighbor algorithm
           will be used.
       -->
      <Function class='Functions' type='External'>
        goalFunctionForLimitSurface
      </Function>
    </PostProcessor>
    ...
  </Models>
  ...
</Simulation>
\end{lstlisting}

%%%%% PP LimitSurfaceIntegral %%%%%%%

\subsubsection{LimitSurfaceIntegral}
\label{LimitSurfaceIntegral}
The \textbf{LimitSurfaceIntegral} post-processor is aimed to compute the likelihood (probability) of the event, whose boundaries are
represented by the Limit Surface (either from the LimitSurface post-processor or Adaptive sampling strategies).
The inputted Limit Surface needs to be, in the  \textbf{PostProcess} step, of type  \textbf{PointSet} and needs to contain
both boundary sides (-1.0, +1.0).
%\\ The \textbf{LimitSurfaceIntegral} post-processor accepts as outputs both files (CSV) and/or  \textbf{PointSet}s.
\\ The \textbf{LimitSurfaceIntegral} post-processor accepts as output  \textbf{PointSet}s only.

\ppType{LimitSurfaceIntegral}{LimitSurfaceIntegral}
\begin{itemize}
\item \variableDescription
 \variableChildIntro
 \begin{itemize}
     \item  \xmlNode{outputName}, \xmlDesc{string, required field}, specifies the name of the output variable where the probability is going to be stored.
               \nb This variable name must be listed in the \xmlNode{Output} field of the Output DataObject
    \item   \xmlNode{distribution}, \xmlDesc{string,
               optional field}, name of the distribution that is associated to this variable.
              Its name needs to be contained in the \xmlNode{Distributions} block explained
              in Section \ref{sec:distributions}. If this node is not present, the  \xmlNode{lowerBound}
              and  \xmlNode{upperBound} XML nodes must be inputted. It requires the following two attributes:
            \begin{itemize}
              \item \xmlAttr{class}, \xmlDesc{required string attribute}, is the main
              ``class'' the listed object is from, the only acceptable class for
              this post-processor is \xmlString{Distributions};
              \item \xmlAttr{type}, \xmlDesc{required string attribute}, is the type of distributions,
                i.e. Normal, Uniform.
            \end{itemize}
   \item   \xmlNode{lowerBound}, \xmlDesc{float,
               optional field}, lower limit of integration domain for this dimension (variable).
               If this node is not present, the  \xmlNode{distribution} XML node must be inputted.
   \item   \xmlNode{upperBound}, \xmlDesc{float,
               optional field}, upper limit of integration domain for this dimension (variable).
               If this node is not present, the  \xmlNode{distribution} XML node must be inputted.
  \end{itemize}

    \item  \xmlNode{tolerance}, \xmlDesc{float, optional field}, specifies the tolerance for
               numerical integration confidence.
                \default{1.0e-4}
     \item  \xmlNode{integralType}, \xmlDesc{string, optional field}, specifies the type of integrations that
                need to be used. Currently only MonteCarlo integration is available
                \default{MonteCarlo}
    \item  \xmlNode{computeBounds}, \xmlDesc{bool, optional field},
    activates the computation of the bounding error of the limit
    surface integral ( maximum error in the identification of the
    limit surface location). If True, the bounding error is stored
    in a variable named as \xmlNode{outputName} appending the suffix
    ``\_err''. For example, if \xmlNode{outputName} is
    ``EventProbability'', the bounding error will be stored as
    ``EventProbability\_err'' (this variable name must be listed as
    variable in the output DataObject).
                \default{False}
     \item  \xmlNode{seed}, \xmlDesc{integer, optional field}, specifies the random number generator seed.
                \default{20021986}
     \item  \xmlNode{target}, \xmlDesc{string, optional field}, specifies the target name that represents
                the $f\left ( \bar{x} \right )$ that needs to be integrated.
                \default{last output found in the inputted PointSet}
\end{itemize}

\textbf{Example:}
\begin{lstlisting}[style=XML,morekeywords={name,subType,debug,class,type}]
<Simulation>
 ...
 <Models>
  ...
    <PostProcessor name="LimitSurfaceIntegralDistributions" subType='LimitSurfaceIntegral'>
        <tolerance>0.0001</tolerance>
        <integralType>MonteCarlo</integralType>
        <seed>20021986</seed>
        <target>goalFunctionOutput</target>
        <outputName>EventProbability</outputName>
        <variable name='x0'>
          <distribution>x0_distrib</distribution>
        </variable>
        <variable name='y0'>
          <distribution>y0_distrib</distribution>
        </variable>
    </PostProcessor>
    <PostProcessor name="LimitSurfaceIntegralLowerUpperBounds" subType='LimitSurfaceIntegral'>
        <tolerance>0.0001</tolerance>
        <integralType>MonteCarlo</integralType>
        <seed>20021986</seed>
        <target>goalFunctionOutput</target>
        <outputName>EventProbability</outputName>
        <variable name='x0'>
          <lowerBound>-2.0</lowerBound>
          <upperBound>12.0</upperBound>
        </variable>
        <variable name='y0'>
            <lowerBound>-1.0</lowerBound>
            <upperBound>11.0</upperBound>
        </variable>
    </PostProcessor>
    ...
  </Models>
  ...
</Simulation>
\end{lstlisting}



%%%%% PP External %%%%%%%
\subsubsection{External}
\label{External}
The \textbf{External} post-processor will execute an arbitrary python function
defined externally using the \textit{Functions} interface (see
Section~\ref{sec:functions} for more details).
%

\ppType{External}{External}

\begin{itemize}
  \item \xmlNode{method}, \xmlDesc{comma separated string, required field},
  lists the method names of an external Function that will be computed (each
  returning a post-processing value). \nb New variable names will be defined as:
  ``Function Name in this post-processor'' + ``\_`` + ``variable name in XML
  node \xmlNode{method}''. These new varialbes will be used to store the computed
  values from the list of methods, and can be accessed by the users through RAVEN
  entities \textbf{DataObjects} and \textbf{OutStreams}.
  \item \xmlNode{Function}, \xmlDesc{xml node, required string field}, specifies
  the name of a Function where the \textit{methods} listed above are defined.
  %
  \nb This name should match one of the Functions defined in the
  \xmlNode{Functions} block of the input file.
  %
  The objects must be listed with a rigorous syntax that, except for the XML
  node tag, is common among all the objects.
  %
  Each of these sub-nodes must contain 2 attributes that are used to map them
  within the simulation framework:

   \begin{itemize}
     \item \xmlAttr{class}, \xmlDesc{required string attribute}, is the main
     ``class'' the listed object is from, the only acceptable class for
     this post-processor is \xmlString{Functions};
     \item \xmlAttr{type}, \xmlDesc{required string attribute}, is the object
     identifier or sub-type, the only acceptable type for this post-processor is
     \xmlString{External}.
  \end{itemize}
\end{itemize}

  This Post-Processor accepts as Input/Output both \xmlString{PointSet} and \xmlString{HistorySet}:
   \begin{itemize}
    \item If a \xmlString{PointSet}  is used as Input, the parameters are passed in the external  \xmlString{Function}
  as numpy arrays. The methods' return type must be either a new array or a scalar. In the following it is reported an example
  with two methods, one that returns a scalar and the other one that returns an array:
      \begin{lstlisting}[language=python]
import numpy as np
def sum(self):
  return np.sum(self.aParameterInPointSet)

def sumTwoArraysAndReturnAnotherone(self):
  return self.aParamInPointSet1+self.aParamInPointSet2
      \end{lstlisting}
    \item If a \xmlString{HistorySet}  is used as Input, the parameters are passed in the external  \xmlString{Function}
     as a list of numpy arrays. The methods' return type must be either a new list of arrays (if the Output is another
     \xmlString{HistorySet}), a scalar or a single array (if the  Output is  \xmlString{PointSet} . In the following it
     is reported an example
     with two methods, one that returns a new list of arrays (Output = HistorySet) and the other one that returns an array (Output =
     PointSet):
      \begin{lstlisting}[language=python]
import numpy as np
def newHistorySetParameter(self):
  x = []*len(self.time)
  for history in range(len(self.time)):
    for ts in range(len(self.time[history])):
      if self.time[history][ts] >= 0.001: break
    x[history] = self.x[history][ts:]
  return x

def aNewPointSetParameter(self):
  x = []*len(self.time)
  for history in range(len(self.time)):
    x[history] = self.x[history][-1]
  return x
      \end{lstlisting}
   \end{itemize}

\textbf{Example:}
\begin{lstlisting}[style=XML,morekeywords={subType,debug,name,class,type}]
<Simulation>
  ...
  <Models>
    ...
    <PostProcessor name="externalPP" subType='External' verbosity='debug'>
      <method>Delta,Sum</method>
      <Function class='Functions' type='External'>operators</Function>
        <!-- Here, you can add a Function defined in the
             Functions block. This should be present or
             else RAVEN will not know where to find the
             defined methods. -->
    </PostProcessor>
    ...
  </Models>
  ...
</Simulation>
\end{lstlisting}

\nb The calculation results from this post-processor are stored in the internal variables. These variables
are accessible by the users through RAVEN entities \textbf{DataObjects} and \textbf{OutStreams}. The names
of these variables are defined as: ``Function Name in this post-processor'' + ``\_`` + ``variable name in XML
node \xmlNode{method}''. For example, in previous case, variables ``operators\_Delta'' and ``operators\_Sum''
are defined by RAVEN to store the outputs of this post-processor.

%%%%% PP TopologicalDecomposition %%%%%%%
\subsubsection{TopologicalDecomposition}
\label{TopologicalDecomposition}
The \textbf{TopologicalDecomposition} post-processor will compute an
approximated hierarchical Morse-Smale complex which will add two columns to a
dataset, namely \texttt{minLabel} and \texttt{maxLabel} that can be used to
decompose a dataset.
%

The topological post-processor can also be run in `interactive' mode, that is
by passing the keyword \texttt{interactive} to the command line of RAVEN's
driver.
%
In this way, RAVEN will initiate an interactive UI that allows one to explore
the topological hierarchy in real-time and adjust the simplification setting
before adjusting a dataset. Use in interactive mode will replace the parameter
\xmlNode{simplification} described below with whatever setting is set in the UI
upon exiting it.

In order to use the \textbf{TopologicalDecomposition} post-processor, the user
needs to set the attribute \xmlAttr{subType}:
\xmlNode{PostProcessor \xmlAttr{subType}=\xmlString{TopologicalDecomposition}}.
The following is a list of acceptable sub-nodes:
\begin{itemize}
  \item \xmlNode{graph} \xmlDesc{, string, optional field}, specifies the type
  of neighborhood graph used in the algorithm, available options are:
  \begin{itemize}
    \item \texttt{beta skeleton}
    \item \texttt{relaxed beta skeleton}
    \item \texttt{approximate knn}
    %\item Delaunay \textit{(disabled)}
  \end{itemize}
  \default{\texttt{beta skeleton}}
  \item \xmlNode{gradient}, \xmlDesc{string, optional field}, specifies the
  method used for estimating the gradient, available options are:
  \begin{itemize}
    \item \texttt{steepest}
    %\item \xmlString{maxflow} \textit{(disabled)}
  \end{itemize}
  \default{\texttt{steepest}}
  \item \xmlNode{beta}, \xmlDesc{float in the range: (0,2], optional field}, is
  only used when the \xmlNode{graph} is set to \texttt{beta skeleton} or
  \texttt{relaxed beta skeleton}.
  \default{1.0}
  \item \xmlNode{knn}, \xmlDesc{integer, optional field}, is the number of
  neighbors when using the \xmlString{approximate knn} for the \xmlNode{graph}
  sub-node and used to speed up the computation of other graphs by using the
  approximate knn graph as a starting point for pruning. -1 means use a fully
  connected graph.
  \default{-1}
  \item \xmlNode{weighted}, \xmlDesc{boolean, optional}, a flag that specifies
  whether the regression models should be probability weighted.
  \default{False}
  \item \xmlNode{interactive}, if this node is present \emph{and} the user has
  specified the keyword \texttt{interactive} at the command line, then this will
  initiate a graphical interface for exploring the different simplification
  levels of the topological hierarchy. Upon exit of the graphical interface, the
  specified simplification level will be updated to use the last value of the
  graphical interface before writing any ``output'' results.
  \item \xmlNode{persistence}, \xmlDesc{string, optional field}, specifies how
  to define the hierarchical simplification by assigning a value to each local
  minimum and maximum according to the one of the strategy options below:
  \begin{itemize}
    \item \texttt{difference} - The function value difference between the
    extremum and its closest-valued neighboring saddle.
    \item \texttt{probability} - The probability integral computed as the
    sum of the probability of each point in a cluster divided by the count of
    the cluster.
    \item \texttt{count} - The count of points that flow to or from the
    extremum.
    % \item \xmlString{area} - The area enclosed by the manifold that flows to
    % or from the extremum.
  \end{itemize}
  \default{\texttt{difference}}
  \item \xmlNode{simplification}, \xmlDesc{float, optional field}, specifies the
  amount of noise reduction to apply before returning labels.
  \default{0}
  \item \xmlNode{parameters}, \xmlDesc{comma separated string, required field},
  lists the parameters defining the input space.
  \item \xmlNode{response}, \xmlDesc{string, required field}, is a single
  variable name defining the scalar output space.
\end{itemize}
\textbf{Example:}
\begin{lstlisting}[style=XML,morekeywords={subType}]
<Simulation>
  ...
  <Models>
    ...
    <PostProcessor name="***" subType='TopologicalDecomposition'>
      <graph>beta skeleton</graph>
      <gradient>steepest</gradient>
      <beta>1</beta>
      <knn>8</knn>
      <normalization>None</normalization>
      <parameters>X,Y</parameters>
      <response>Z</response>
      <weighted>true</weighted>
      <simplification>0.3</simplification>
      <persistence>difference</persistence>
    </PostProcessor>
    ...
  <Models>
  ...
<Simulation>
\end{lstlisting}

%%%%% PP DataMining %%%%%%%
\subsubsection{DataMining}
\label{subsubsec:DataMining}

Knowledge discovery in databases (KDD) is the process of discovering
 useful knowledge from a collection of data. This widely used data
mining technique is a process that includes data preparation and
selection, data cleansing, incorporating prior knowledge on data
sets and interpreting accurate solutions from the observed results.
Major KDD application areas include marketing, fraud detection,
telecommunication and manufacturing.

DataMining is the analysis step of the KDD process. The overall of
the data mining process is to extract information from a data set
and transform it into an understandable structure for further use.
The actual data mining task is the
automatic or semi-automatic analysis of large quantities of data
to extract previously unknown, interesting patterns such as groups
 of data records (cluster analysis), unusual records (anomaly
detection), and dependencies (association rule mining).
\\
%
In order to use the \textbf{DataMining} post-processor, the user
needs to set the attribute \xmlAttr{subType}: \\
\\
\xmlNode{PostProcessor \xmlAttr{subType}=
\xmlString{DataMining}}. \\
\\
The following is a list of acceptable sub-nodes:
\begin{itemize}
  \item \xmlNode{KDD} \xmlDesc{string,required field}, the subnodes specifies
  the necessary information for the algorithm to be used in the postprocessor.
  The \xmlNode{KDD} has the required attribute: \xmlAttr{lib}, the name of the
  library the algorithm belongs to. Current algorithms applied in the KDD model
  is based on SciKit-Learn library. Thus currently there is only one library:
  \begin{itemize}
    \item \xmlString{SciKitLearn}
  \end{itemize}
  The \xmlNode{KDD} has the optional attribute: \xmlAttr{labelFeature}, the name
  associated to labels or dimensions generated by the \textbf{DataMining}
  post-processor.
  The default name depends on the type of algorithm employed.
  For clustering and mixture models it is the name of the PostProcessor
  followed by ``Labels'' (e.g., if the name of a clustering PostProcessor is
  ``kMeans'' then the default name associated to the labels is ``kMeansLabels''
  if not specified in the attribute \xmlAttr{labelFeature}).
  For decomposition and manifold models, the default names are the name of the
  PostProcessor followed by ``Dimension'' and an integer identifier beginning
  with 1. (e.g., if the name of a dimensionality reduction PostProcessor is
  ``dr'' and the user specifies 3 components, then the output dataObject will
  have three new outputs named ``drDimension1,'' ``drDimension2,'' and
  ``drDimension3.'').
  \nb The ``Labels'' are automatically added to the output \textbf{DataObjects}. It
  is also accessible by the users using the variable name defined above.
\end{itemize}


\paragraph{SciKitLearn}
\xmlString{SciKitLearn} is based on algorithms in SciKit-Learn library, and it performs data mining over PointSet and HistorySet. Note that for HistorySet's \xmlString{SciKitLearn} performs the task given in \xmlNode{SKLType} (see below) for each time step, and so only synchronized HistorySet can be used as input to this model. For unsynchronized HistorySet, use \xmlString{HistorySetSync} method in \xmlString{Interfaced} post-processor to synchronize the input data before using \xmlString{SciKitLearn}. The rest of this subsection and following subsection is dedicated to the \xmlString{SciKitLearn} library.

The temporal variable for a HistorySet \xmlString{SciKitLearn} is specified in the \xmlNode{pivotParameter} node:
\begin{itemize}
  \item \xmlNode{pivotParameter}, \xmlDesc{string, optional parameter} specifies the pivot variable (e.g., time, etc) in the input HistorySet.
      \default{None}.
\end{itemize}

The algorithm for the dataMining is chosen by the subnode \xmlNode{SKLType} under the parent node
\xmlNode{KDD}. The format is same as in \ref{subsubsec:SciKitLearn}. However, for the completeness
sake, it is repeated here.

The data that are used in the training of the \textbf{DataMining}
postprocessor are suplied with subnode \xmlNode{Features} in the parent node
 \xmlNode{KDD}.


\begin{itemize}
  \item \xmlNode{SKLtype}, \xmlDesc{vertical bar (\texttt{|}) separated string,
  required field}, contains a string that represents the data mining algorithm
  to be used.
  %
  As mentioned, its format is:\\
  \xmlNode{SKLtype}\texttt{mainSKLclass|algorithm}\xmlNode{/SKLtype} where the
  first word (before the ``\texttt{|}'' symbol) represents the main class of
  algorithms, and the second word (after the ``\texttt{|}'' symbol) represents
  the specific algorithm.
  %
  \item \xmlNode{Features}, \xmlDesc{string, required field}, defines the data
  to be used for training the data mining algorithm. It can be:
  \begin{itemize}
	\item the name of the variable in the defined dataObject entity
	\item the location (i.e. input or output). In this case the data mining
        is applied to all the variables in the defined space.
  \end{itemize}
\end{itemize}

The \xmlNode{KDD} node can have either optional or required subnodes depending
 on the dataMining algorithm used. The possible subnodes will be described separately
 for each algorithm below. The time dependent clustering data mining algorithms have a \xmlNode{reOrderStep} option that will try and keep the same labels on the clusters.  The higher the number, the longer the history that the clustering algorithm will look through to maintain the same labeling between time steps.

All the available algorithms are described in the following sections.

\paragraph{Gaussian mixture models}
\label{paragraph:GMM}

A Gaussian mixture model is a probabilistic model that assumes all
 the data points are generated from a mixture of a finite number of
 Gaussian distributions with unknown parameters.
\\
Scikit-learn implements different classes to estimate Gaussian
mixture models, that correspond to different estimation strategies,
 detailed below.

\subparagraph{ GMM classifier} \hfill
\label{subparagraph:GMMClass}

The GMM object implements the expectation-maximization (EM)
algorithm for fitting mixture-of-Gaussian models. The GMM comes with different options
 to constrain the covariance of  the difference classes estimated: spherical, diagonal, tied or
 full covariance.

\skltype{Gaussian Mixture Model}{mixture|GMM}
\begin{itemize}
	\item \xmlNode{n\_components}, \xmlDesc{integer, optional
field} Number of mixture components. \default{1}
	\item \xmlNode{covariance\_type}, \xmlDesc{string, optional
field}, describes the type of covariance parameters to use.
Must be one of ‘spherical’, ‘tied’, ‘diag’, ‘full’. \default{diag}
	\item \xmlNode{random\_state}, \xmlDesc{integer seed or random
 number generator instance, optional field},  A random number
generator instance \default{0 or None}
	\item \xmlNode{min\_covar}, \xmlDesc{float, optional field},
 Floor on the diagonal of the covariance matrix to prevent overfitting.
 \default{1e-3}.
	\item \xmlNode{thresh}, \xmlDesc{float, optional field},
convergence threshold. \default{0.01}
	\item \xmlNode{n\_iter}, \xmlDesc{integer, optional field},
Number of EM iterations to perform. \default{100}
	\item \xmlNode{n\_init}, \xmlDesc{integer, optional},
Number of initializations to perform. the best results is kept.
\default{1}
	\item \xmlNode{init\_params}, \xmlDesc{string, optional
field},  The method used to initialize the weights, the means and the precisions. Must be one of 
 ``kmeans'' (responsibilities are initialized using kmeans) or ``random'' (responsibilities are 
 initialized randomly)
 \default{kmeans}
\end{itemize}

\textbf{Example:}
\begin{lstlisting}[style=XML,morekeywords={subType}]
<Simulation>
  ...
  <Models>
    ...
      <PostProcessor name='PostProcessorName' subType='DataMining'>
          <KDD lib='SciKitLearn'>
              <Features>variableName</Features>
              <SKLtype>mixture|GMM</SKLtype>
              <n_components>2</n_components>
              <covariance_type>spherical</covariance_type>
          </KDD>
      </PostProcessor>
    ...
  <Models>
  ...
<Simulation>
\end{lstlisting}

\subparagraph{ Variational GMM Classifier (VBGMM)} \hfill
\label{subparagraph:VBGMM}

The VBGMM object implements a variant of the Gaussian mixture model
 with variational inference algorithms. The API is identical to GMM.

\skltype{Variational Gaussian Mixture Model}{mixture|VBGMM}
\begin{itemize}
	\item \xmlNode{n\_components}, \xmlDesc{integer, optional
field} Number of mixture components. \default{1}
	\item \xmlNode{covariance\_type}, \xmlDesc{string, optional
field}, describes the type of covariance parameters to use.
Must be one of ‘spherical’, ‘tied’, ‘diag’, ‘full’. \default{diag}
	\item \xmlNode{alpha}, \xmlDesc{float, optional field},
represents the concentration parameter of the dirichlet process.
Intuitively, the Dirichlet Process is as likely to start a new cluster
 for a point as it is to add that point to a cluster with alpha
 elements. A higher alpha means more clusters, as the expected
number of clusters is ${\alpha*log(N)}$. \default{1}.
\end{itemize}

\paragraph{ Clustering }
\label{paragraph:Clustering}

Clustering of unlabeled data can be performed with this subType of
 the DataMining PostProcessor.

An overwiev of the different clustering algorithms is given in
Table\ref{tab:clustering}.

\begin{table}[!htbp]
  \centering
  \caption{Overview of Clustering Methods}
  \label{tab:clustering}
  \begin{tabular}{| L{2.5cm} | L{2.5cm} | L{2.5cm} | L{3.5cm} | L{2.75cm} |} \hline
    {\bf Method name} & {\bf Parameters} & {\bf Scalability} & {\bf
Usecase} & {\bf Geometry (metric used)} \\ \hline
    K-Means  & number of clusters & Very large n\_samples, medium
n\_clusters with MiniBatch code & General-purpose, even cluster size,
flat geometry, not too many clusters & Distances between points
 \\ \hline
    Affinity propagation & damping, sample preference & Not scalable with
n\_samples & Many clusters, uneven cluster size, non-flat geometry &
Graph distance (e.g. nearest-neighbor graph)       \\ \hline
    Mean-shift & bandwidth & Not scalable with n\_samples & Many clusters,
 uneven cluster size, non-flat geometry & Distances between points \\ \hline
    Spectral clustering & number of clusters & Medium n\_samples, small
n\_clusters & Few clusters, even cluster size, non-flat geometry &
Graph distance (e.g. nearest-neighbor graph)       \\ \hline
    Ward hierarchical clustering & number of clusters & Large n\_samples
and n\_clusters & Many clusters, possibly connectivity constraints &
Distances between points       \\ \hline
    Agglomerative clustering & number of clusters, linkage type, distance
 & Large n\_samples and n\_clusters & Many clusters, possibly
connectivity constraints, non Euclidean distances & Any pairwise
distance       \\ \hline
    DBSCAN & neighborhood size & Very large n\_samples, medium n\_clusters
 & Non-flat geometry, uneven cluster sizes & Distances between nearest
 points       \\ \hline
    Gaussian mixtures & many & Not scalable & Flat geometry, good for
 density estimation & Mahalanobis distances to centers       \\ \hline
  \end{tabular}
\end{table}

\FloatBarrier

\subparagraph{K-Means Clustering} \hfill
\label{subparagraph:KMeans}

The KMeans algorithm clusters data by trying to separate samples in n groups
of equal variance, minimizing a criterion known as the inertia or within-cluster
sum-of-squares. This algorithm requires the number of clusters to be specified.
 It scales well to large number of samples and has been used across a large
range of application areas in many different fields

\skltype{ K-Means Clustering}{cluster|KMeans}
\begin{itemize}
	\item \xmlNode{n\_clusters}, \xmlDesc{integer, optional field}
The number of clusters to form as well as the number of centroids to
generate. \default{8}
	\item \xmlNode{max\_iter}, \xmlDesc{integer, optional field},
Maximum number of iterations of the k-means algorithm for a single run.
\default{300}
	\item \xmlNode{n\_init}, \xmlDesc{integer, optional field},
Number of time the k-means algorithm will be run with different centroid
 seeds. The final results will be the best output of n\_init consecutive
 runs in terms of inertia. \default{3}
	\item \xmlNode{init}, \xmlDesc{string, optional},
Method for initialization, k-means++’, ‘random’ or an ndarray:
		\begin{itemize}
			\item ‘k-means++’ : selects initial cluster
centers for k-mean clustering in a smart way to speed up convergence.
			\item ‘random’: choose k observations (rows) at
 random from data for the initial centroids.
			\item If an ndarray is passed, it should be of
 shape (n\_clusters, n\_features) and gives the initial centers.
		\end{itemize}
	\item \xmlNode{precompute\_distances}, \xmlDesc{boolean, optional
field}, Precompute distances (if true faster but takes more memory).
\default{true}
	\item \xmlNode{tol}, \xmlDesc{float, optional field}, Relative
tolerance with regards to inertia to declare convergence. \default{1e-4}
	\item \xmlNode{n\_jobs}, \xmlDesc{integer, optional field}, The number
of jobs to use for the computation. This works by breaking down the pairwise
 matrix into n jobs even slices and computing them in parallel. If -1 all CPUs
 are used. If 1 is given, no parallel computing code is used at all, which is
useful for debugging. For n\_jobs below -1, (n\_cpus + 1 + n\_jobs) are used. Thus
 for n\_jobs = -2, all CPUs but one are used. \default{1}
	\item \xmlNode{random\_state}, \xmlDesc{integer or numpy.RandomState,
 optional field} The generator used to initialize the centers. If an integer
 is given, it fixes the seed. \default{the global numpy random number generator}.
\end{itemize}

\textbf{Example:}
\begin{lstlisting}[style=XML,morekeywords={subType}]
<Simulation>
  ...
  <Models>
    ...
      <PostProcessor name='PostProcessorName' subType='DataMining'>
          <KDD lib='SciKitLearn'>
              <Features>variableName</Features>
              <SKLtype>cluster|KMeans</SKLtype>
              <n_clusters>2</n_clusters>
              <tol>0.0001</tol>
              <init>random</init>
          </KDD>
      </PostProcessor>
    ...
  <Models>
  ...
<Simulation>
\end{lstlisting}


\subparagraph{  Mini Batch K-Means } \hfill
\label{subparagraph:MiniBatch}

The MiniBatchKMeans is a variant of the KMeans algorithm which uses
mini-batches to reduce the computation time, while still attempting
 to optimise the same objective function. Mini-batches are subsets of
 the input data, randomly sampled in each training iteration.

MiniBatchKMeans converges faster than KMeans, but the quality of the
results is reduced. In practice this difference in quality can be
 quite small.

\skltype{ Mini Batch K-Means Clustering}{cluster|MiniBatchKMeans}
\begin{itemize}
	\item \xmlNode{n\_clusters}, \xmlDesc{integer, optional field}
The number of clusters to form as well as the number of centroids to
generate. \default{8}
	\item \xmlNode{max\_iter}, \xmlDesc{integer, optional field},
Maximum number of iterations of the k-means algorithm for a single run.
\default{100}
	\item \xmlNode{max\_no\_improvement}, \xmlDesc{integer, optional
firld}, Control early stopping based on the consecutive number of mini
 batches that does not yield an improvement on the smoothed inertia.
To disable convergence detection based on inertia, set
max\_no\_improvement to None. \default{10}
	\item \xmlNode{tol}, \xmlDesc{float, optional field}, Control
 early stopping based on the relative center changes as measured by a
smoothed, variance-normalized of the mean center squared position changes.
 This early stopping heuristics is closer to the one used for the batch
 variant of the algorithms but induces a slight computational and memory
overhead over the inertia heuristic. To disable convergence detection
based on normalized center change, set tol to 0.0 (default). \default{0.0}
	\item \xmlNode{batch\_size}, \xmlDesc{integer, optional field},
Size of the mini batches. \default{100}
	\item{init\_size}, \xmlDesc{integer, optional field}, Number of
samples to randomly sample for speeding up the initialization
 (sometimes at the expense of accuracy): the only algorithm is initialized
 by running a batch KMeans on a random subset of the data.
\textit{This needs to be larger than k.}, \default{3 * \xmlNode{batch\_size}}
	\item \xmlNode{init}, \xmlDesc{string, optional},
Method for initialization, k-means++’, ‘random’ or an ndarray:
		\begin{itemize}
			\item ‘k-means++’ : selects initial cluster
centers for k-mean clustering in a smart way to speed up convergence.
			\item ‘random’: choose k observations (rows) at
 random from data for the initial centroids.
			\item If an ndarray is passed, it should be of
 shape (n\_clusters, n\_features) and gives the initial centers.
		\end{itemize}
	\item \xmlNode{precompute\_distances}, \xmlDesc{boolean, optional
field}, Precompute distances (if true faster but takes more memory).
\default{true}
	\item \xmlNode{n\_init}, \xmlDesc{integer, optional field},
Number of time the k-means algorithm will be run with different centroid
 seeds. The final results will be the best output of n\_init consecutive
 runs in terms of inertia. \default{3}
	\item \xmlNode{compute\_labels}, \xmlDesc{boolean, optional field},
Compute label assignment and inertia for the complete dataset once the
 minibatch optimization has converged in fit. \default{True}
	\item \xmlNode{random\_state}, \xmlDesc{integer or numpy.RandomState,
 optional field} The generator used to initialize the centers. If an integer
 is given, it fixes the seed. \default{the global numpy random number generator}.
	\item{reassignment\_ratio}, \xmlNode{float, optional field}, Control
the fraction of the maximum number of counts for a center to be reassigned.
A higher value means that low count centers are more easily reassigned, which
 means that the model will take longer to converge, but should converge in a
better clustering. \default{0.01}
\end{itemize}

\subparagraph{Affinity Propagation} \hfill
\label{subparagraph:Affinity}

AffinityPropagation creates clusters by sending messages between pairs of
samples until convergence. A dataset is then described using a small number
of exemplars, which are identified as those most representative of other
samples. The messages sent between pairs represent the suitability for one
sample to be the exemplar of the other, which is updated in response to the
values from other pairs. This updating happens iteratively until convergence,
 at which point the final exemplars are chosen, and hence the final clustering
 is given.

\skltype{ AffinityPropogation Clustering}{cluster|AffinityPropogation}
\begin{itemize}
	\item \xmlNode{damping}, \xmlDesc{float, optional field}, Damping factor
 between 0.5 and 1. \default{0.5}
	\item \xmlNode{convergence\_iter}, \xmlDesc{integer, optional field},
Number of iterations with no change in the number of estimated clusters that
stops the convergence. \default{15}
	\item \xmlNode{max\_iter}, \xmlDesc{integer, optional field}, Maximum
 number of iterations. \default{200}
	\item \xmlNode{copy}, \xmlDesc{boolean, optional field}, Make a copy of
input data or not. \default{True}
	\item \xmlNode{preference}, \xmlDesc{array-like, shape (n\_samples,)
or float, optional field}, Preferences for each point - points with larger
values of preferences are more likely to be chosen as exemplars. The number
of exemplars, ie of clusters, is influenced by the input preferences value.
\default{If the preferences are not passed as arguments, they will be set to the median of
 the input similarities.}
	\item \xmlNode{affinity}, \xmlDesc{string, optional field},Which affinity to use.
 At the moment precomputed and euclidean are supported. euclidean uses the negative squared
euclidean distance between points. \default{``euclidean``}
	\item \xmlNode{verbose}, \xmlDesc{boolean, optional field}, Whether to be verbose.
\default{False}
\end{itemize}

\subparagraph{ Mean Shift } \hfill
\label{subparagraph:MeanShift}

MeanShift clustering aims to discover blobs in a smooth density of samples. It is
 a centroid based algorithm, which works by updating candidates for centroids to be
the mean of the points within a given region. These candidates are then filtered in
a post-processing stage to eliminate near-duplicates to form the final set of centroids.

\skltype{ Mean Shift Clustering}{cluster|MeanShift}
\begin{itemize}
	\item \xmlNode{bandwidth}, \xmlDesc{float, optional field}, Bandwidth used
in the RBF kernel. If not given, the bandwidth is estimated using
\textit{sklearn.cluster.estimate\_bandwidth}; see the documentation for that function for
 hints on scalability.
	\item \xmlNode{seeds}, \xmlDesc{array, shape=[n\_samples, n\_features],
optional field}, Seeds used to initialize kernels. If not set, the seeds are
calculated by \textit{clustering.get\_bin\_seeds} with bandwidth as the grid size and
 default values for other parameters.
	\item \xmlNode{bin\_seeding}, \xmlDesc{boolean, optional field}, If true,
 initial kernel locations are not locations of all points, but rather the
 location of the discretized version of points, where points are binned onto
 a grid whose coarseness corresponds to the bandwidth. Setting this option
to True will speed up the algorithm because fewer seeds will be initialized.
 \default{False} Ignored if seeds argument is not None.
	\item \xmlNode{min\_bin\_freq}, \xmlDesc{integer, optional field},
To speed up the algorithm, accept only those bins with at least min\_bin\_freq
 points as seeds. \default{1}.
	\item \xmlNode{cluster\_all}, \xmlDesc{boolean, optional field}, If true,
 then all points are clustered, even those orphans that are not within any
kernel. Orphans are assigned to the nearest kernel. If false, then orphans
are given cluster label -1. \default{True}
\end{itemize}


\subparagraph{Spectral clustering} \hfill
\label{subparagraph:Spectral}

SpectralClustering does a low-dimension embedding of the affinity matrix between
 samples, followed by a \textit{KMeans} in the low dimensional space. It is
especially efficient if the affinity matrix is sparse and the pyamg module is
installed.

\skltype{Spectral Clustering}{cluster|Spectral}
\begin{itemize}
	\item \xmlNode{n\_clusters}, \xmlDesc{integer, optional field},
	The dimension of the projection subspace.\default{8}
	%
	\item \xmlNode{affinity}, \xmlDesc{string, array-like or callable, optional
	  field}, If a string, this may be one of:
	\begin{itemize}
		\item ‘nearest\_neighbors’,
		\item ‘precomputed’,
		\item ‘rbf’ or
		\item one of the kernels supported by \textit{sklearn.metrics.pairwise\_kernels}.
	\end{itemize}
	Only kernels that produce similarity scores (non-negative values that increase
	with similarity) should be used. This property is not checked by the clustering
	 algorithm. \default{‘rbf’}
	%
	\item \xmlNode{gamma}, \xmlDesc{float, optional field}, Scaling factor of RBF,
	polynomial, exponential $chi^2$ and sigmoid affinity kernel.
	Ignored for $affinity='nearest\_neighbors'$. \default{1.0}
	%
	\item \xmlNode{degree}, \xmlDesc{float, optional field}, Degree of the polynomial
	 kernel. Ignored by other kernels. \default{3}
	%
	\item \xmlNode{coef0}, \xmlDesc{float, optional field}, Zero coefficient for
	polynomial and sigmoid kernels. Ignored by other kernels. \default{1}
	%
	\item \xmlNode{n\_neighbors}, \xmlDesc{integer, optional field}, Number of neighbors
	to use when constructing the affinity matrix using the nearest neighbors method.
	Ignored for affinity='rbf'. \default{10}
	%
	\item \xmlNode{eigen\_solver} \xmlDesc{string, optional field},  The eigenvalue
	decomposition strategy to use:
	\begin{itemize}
		\item None,
		\item ‘arpack’,
		\item ‘lobpcg’, or
		\item ‘amg’
	\end{itemize}
	\nb{AMG requires pyamg to be installed. It can be faster on very large, sparse
	problems, but may also lead to instabilities}
	%
	\item \xmlNode{random\_state}, \xmlDesc{integer seed, RandomState instance,
	 or None, optional field}, A pseudo random number generator used for the
	initialization of the lobpcg eigen vectors decomposition when $eigen_solver == ‘amg’$
	 and by the K-Means initialization. \default{None}
	%
	\item \xmlNode{n\_init}, \xmlDesc{integer, optional field}, Number of time the
	 k-means algorithm will be run with different centroid seeds. The final results
	 will be the best output of n\_init consecutive runs in terms of inertia.
	\default{10}
	%
	\item \xmlNode{eigen\_tol}, \xmlDesc{float, optional field}, Stopping criterion
	 for eigendecomposition of the Laplacian matrix when using arpack eigen\_solver.
	\default{0.0}
	%
	\item \xmlNode{assign\_labels}, \xmlDesc{string, optional field}, The strategy to
	use to assign labels in the embedding space. There are two ways to assign labels
	after the laplacian embedding:
	\begin{itemize}
		\item ‘kmeans’,
		\item ‘discretize’
	\end{itemize}
	 k-means can be applied and is a popular choice. But it can also be sensitive
	to initialization. Discretization is another approach which is less sensitive
	 to random initialization. \default{‘kmeans’}
	%
	\item \xmlNode{kernel\_params}, \xmlDesc{dictionary of string to any, optional
	 field}, Parameters (keyword arguments) and values for kernel passed as
	callable object. Ignored by other kernels. \default{None}
\end{itemize}

\textbf{Notes} \\
If you have an affinity matrix, such as a distance matrix, for which 0 means identical
elements, and high values means very dissimilar elements, it can be transformed in a
similarity matrix that is well suited for the algorithm by applying the Gaussian
 (RBF, heat) kernel:
\begin{equation}
np.exp(- X ** 2 / (2. * delta ** 2))
\end{equation}
Another alternative is to take a symmetric version of the k nearest neighbors
connectivity matrix of the points.
If the \textit{pyamg} package is installed, it is used: this greatly speeds
up computation.

\subparagraph{ DBSCAN Clustering } \hfill
\label{subparagraph:DBSCAN}

The Density-Based Spatial Clustering of Applications with Noise (DBSCAN)
 algorithm views clusters as areas of high density separated by
areas of low density. Due to this rather generic view, clusters found by
DBSCAN can be any shape, as opposed to k-means which assumes that clusters
 are convex shaped.

\skltype{DBSCAN Clustering}{cluster|DBSCAN}
\begin{itemize}
	\item \xmlNode{eps}, \xmlDesc{float, optional field}, The maximum
	distance between two samples for them to be considered as in the
	 same neighborhood. \default{0.5}
	%
	\item \xmlNode{min\_samples}, \xmlDesc{integer, optional field},
	The number of samples in a neighborhood for a point to be
	considered as a core point. \default{5}
	%
	\item \xmlNode{metric}, \xmlDesc{string, or callable, optional field}
	The metric to use when calculating distance between instances in
	 a feature array. If metric is a string or callable, it must be one
	of the options allowed by \textit{metrics.pairwise.calculate\_distance}
	 for its metric parameter. If metric is “precomputed”, X is assumed
	 to be a distance matrix and must be square. \default{'euclidean'}
	%
	\item \xmlNode{random\_state}, \xmlDesc{numpy.RandomState,
	 optional field}, The generator used to initialize the centers.
	\default{numpy.random}.
\end{itemize}

\subparagraph{Agglomerative Clustering } \hfill
\label{subparagraph:agglomerative}

Hierarchical clustering is a general family of clustering algorithms that build nested clusters by merging or splitting them successively.
This hierarchy of clusters is represented as a tree (or dendrogram).
The root of the tree is the unique cluster that gathers all of the samples, the leaves being the clusters with only one sample.
The AgglomerativeClustering object performs a hierarchical clustering using a bottom up approach: each observation starts in its own cluster,
and clusters are successively merged together. The linkage criteria determines the metric used for the merge strategy:
\begin{itemize}
  \item Ward: it minimizes the sum of squared differences within all clusters. It is a variance-minimizing approach and in this sense
is similar to the k-means objective function but tackled with an agglomerative hierarchical approach.
  \item Maximum or complete linkage: it minimizes the maximum distance between observations of pairs of clusters.
  \item Average linkage: it minimizes the average of the distances between all observations of pairs of clusters.
\end{itemize}

AgglomerativeClustering can also scale to large number of samples when it is used jointly with a connectivity matrix,
but is computationally expensive when no connectivity constraints are added between samples: it considers at each step all of the possible merges.

\skltype{Agglomerative Clustering}{cluster|Agglomerative}
\begin{itemize}
  \item \xmlNode{n\_clusters}, \xmlDesc{int, optional field}, The number of clusters to find. \default{2}
  \item \xmlNode{connectivity}, \xmlDesc{array like or callable, optional field}, Connectivity matrix. Defines for each sample the neighboring samples
   following a given structure of the data. This can be a connectivity matrix itself or a callable that transforms the data into a connectivity matrix,
   such as derived from kneighbors graph. Default is None, i.e, the hierarchical clustering algorithm is unstructured. \default{None}
  \item \xmlNode{affinity}, \xmlDesc{string or callable, optional field}, Metric used to compute the linkage. Can be ``euclidean'', ``$l1$'', ``$l2$'',
   ``manhattan'',``cosine'', or ``precomputed''. If linkage is ``ward'', only ``euclidean'' is accepted. \default{euclidean}
%  \item \xmlNode{memory}, \xmlDesc{Instance of joblib.Memory or string, optional field}, Used to cache the output of the computation of the tree.
%   By default, no caching is done. If a string is given, it is the path to the caching directory.
  \item \xmlNode{n\_components}, \xmlDesc{int, optional field}, Number of connected components. If None the number of connected components is estimated
  from the connectivity matrix. NOTE: This parameter is now directly determined from the connectivity matrix and will be removed in $0.18$.
%  \item \xmlNode{compute\_full\_tree}, \xmlDesc{bool or 'auto', optional field}, Stop early the construction of the tree at \xmlNode{n\_clusters}.
%  This is useful to decrease computation time if the number of clusters is not small compared to the number of samples. This option is useful only
%  when specifying a connectivity matrix. Note also that when varying the number of clusters and using caching, it may be advantageous to compute the full tree.
  \item \xmlNode{linkage}, \xmlDesc{{ward,complete,average}, optional field}, Which linkage criterion to use. The linkage criterion determines which distance
  to use between sets of observation. The algorithm will merge the pairs of cluster that minimize this criterion. Ward minimizes the variance of the clusters being merged.
  Average uses the average of the distances of each observation of the two sets. Complete or maximum linkage uses the maximum distances between all observations
  of the two sets.. \default{ward}
%  \item \xmlNode{pooling\_func}, \xmlDesc{callable, optional field}, This combines the values of agglomerated features into a single value, and
%  should accept an array of shape $[M, N]$ and the keyword argument axis=1, and reduce it to an array of size $[M]$. \default{np.mean}
\end{itemize}


\subparagraph{Clustering performance evaluation} \hfill
\label{subparagraph:ClusterPerformance}

Evaluating the performance of a clustering algorithm is not as trivial as
counting the number of errors or the precision and recall of a supervised
 classification algorithm. In particular any evaluation metric should not
 take the absolute values of the cluster labels into account but rather if
 this clustering define separations of the data similar to some ground truth
 set of classes or satisfying some assumption such that members belong to
the same class are more similar that members of different classes according
to some similarity metric.

If the ground truth labels are not known, evaluation must be performed using
 the model itself. The \textbf{Silhouette Coefficient} is an example of
such an evaluation, where a higher Silhouette Coefficient score relates to
 a model with better defined clusters. The Silhouette Coefficient is defined
 for each sample and is composed of two scores:
\begin{enumerate}
	\item The mean distance between a sample and all other points in the
	 same class.
	%
	\item The mean distance between a sample and all other points in the
	 next nearest cluster.
\end{enumerate}

The Silhoeutte Coefficient s for a single sample is then given as:
\begin{equation}
s = \frac{b - a}{max(a, b)}
\end{equation}
The Silhouette Coefficient for a set of samples is given as the mean of the
 Silhouette Coefficient for each sample. In normal usage, the Silhouette
 Coefficient is applied to the results of a cluster analysis.

\begin{description}
	\item[Advantages] \hfill \\
	\begin{itemize}
		\item The score is bounded between -1 for incorrect
		clustering and +1 for highly dense clustering. Scores around
		 zero indicate overlapping clusters.
		\item The score is higher when clusters are dense and well
		 separated, which relates to a standard concept of a cluster.
	\end{itemize}
	\item[Drawbacks] \hfill \\
	The Silhouette Coefficient is generally higher for convex clusters
	than other concepts of clusters, such as density based clusters like
	 those obtained through DBSCAN.
\end{description}

\paragraph{Decomposing signals in components (matrix factorization problems)}
\label{paragraph:Decomposing}
\subparagraph{Principal component analysis (PCA)}
\label{subparagraph:PCA}

\begin{itemize}
	\item \textbf{Exact PCA and probabilistic interpretation} \\
	Linear Dimensionality reduction using Singular Value Decomposition of
	the data and keeping only the most significant singular vectors to
	 project the data to a lower dimensional space.
	\skltype{Exact PCA}{decomposition|PCA}
	\begin{itemize}
		\item \xmlNode{n\_components}, \xmlDesc{integer, None or String,
		optional field}, Number of components to keep. if
		\item \xmlNode{n\_components} is not set all components are kept,
		\default{all components}
		\item \xmlNode{copy}, \xmlDesc{boolean, optional field}, If False,
		 data passed to fit are overwritten and running fit(X).transform(X)
 		will not yield the expected results, use fit\_transform(X) instead.
		\default{True}
		\item \xmlNode{whiten}, \xmlDesc{boolean, optional field}, When True
		the components\_ vectors are divided by n\_samples times singular
		 values to ensure uncorrelated outputs with unit component-wise
		variances. Whitening will remove some information from the transformed
		 signal (the relative variance scales of the components) but can
		sometime improve the predictive accuracy of the downstream estimators
		 by making there data respect some hard-wired assumptions. \default{False}
	\end{itemize}
\textbf{Example:}
\begin{lstlisting}[style=XML,morekeywords={subType}]
<Simulation>
  ...
  <Models>
    ...
      <PostProcessor name='PostProcessorName' subType='DataMining'>
          <KDD lib='SciKitLearn'>
              <Features>variable1,variable2,variable3, variable4,variable5</Features>
              <SKLtype>decomposition|PCA</SKLtype>
              <n_components>2</n_components>
          </KDD>
      </PostProcessor>
    ...
  <Models>
  ...
<Simulation>
\end{lstlisting}


	\item \textbf{Randomized {(Approximate)} PCA} \\
	Linear Dimensionality reduction using Singular Value Decomposition of the data
	and keeping only the most significant singular vectors to project the data to a
	 lower dimensional space.
	\skltype{Randomized PCA}{decomposition|RandomizedPCA}
	\begin{itemize}
		\item \xmlNode{n\_components}, \xmlDesc{interger, None or String,
		optional field}, Number of components to keep. if n\_components is
		not set all components are kept.\default{all components}
		\item \xmlNode{copy}, \xmlDesc{boolean, optional field}, If False,
		 data passed to fit are overwritten and running fit(X).transform(X)
 		will not yield the expected results, use fit\_transform(X) instead.
		\default{True}
		\item \xmlNode{iterated\_power}, \xmlDesc{integer, optional field},
		Number of iterations for the power method. \default{3}
		\item \xmlNode{whiten}, \xmlDesc{boolean, optional field}, When True
		the components\_ vectors are divided by n\_samples times singular
		 values to ensure uncorrelated outputs with unit component-wise
		variances. Whitening will remove some information from the transformed
		 signal (the relative variance scales of the components) but can
		sometime improve the predictive accuracy of the downstream estimators
		 by making there data respect some hard-wired assumptions. \default{False}
		\item \xmlNode{random\_state}, \xmlDesc{int, or Random State instance
		or None, optional field}, Pseudo Random Number generator seed control.
		 If None, use the numpy.random singleton. \default{None}
	\end{itemize}
	\item \textbf{Kernel PCA} \\
	Non-linear dimensionality reduction through the use of kernels.
	\skltype{Kernel PCA}{decomposition|KernelPCA}
	\begin{itemize}
		\item \xmlNode{n\_components}, \xmlDesc{interger, None or String,
		optional field}, Number of components to keep. if n\_components is
		not set all components are kept.\default{all components}
		\item \xmlNode{kernel}, \xmlDesc{string, optional field}, name of
		the kernel to be used, options are:
		\begin{itemize}
			\item linear
			\item poly
			\item rbf
			\item sigmoid
			\item cosine
			\item precomputed
		\end{itemize}
		\default{linear}
		\xmlNode{degree}, \xmlDesc{integer, optional field}, Degree for poly
		 kernels, ignored by other kernels. \default{3}
		\xmlNode{gamma}, \xmlDesc{float, optional field}, Kernel coefficient
		 for rbf and poly kernels, ignored by other kernels. \default{1/n\_features}
		\item \xmlNode{coef0}, \xmlDesc{float, optional field}, independent term in
		 poly and sigmoig kernels, ignored by other kernels.
		\item \xmlNode{kernel\_params}, \xmlDesc{mapping of string to any, optional
		 field}, Parameters (keyword arguments) and values for kernel passed as
		callable object. Ignored by other kernels. \default{3}
		\item{alpha}, \xmlDesc{int, optional field}, Hyperparameter of the ridge
		regression that learns the inverse transform (when fit\_inverse\_transform=True).
		\default{1.0}
		\item \xmlNode{fit\_inverse\_transform}, \xmlDesc{bool, optional field},
		Learn the inverse transform for non-precomputed kernels. (i.e. learn to find
		 the pre-image of a point) \default{False}
		\item \xmlNode{eigen\_solver}, \xmlDesc{string, optional field}, Select eigensolver
		 to use. If n\_components is much less than the number of training samples,
		arpack may be more efficient than the dense eigensolver. Options are:
		\begin{itemize}
			\item auto
			\item dense
			\item arpack
		\end{itemize} \default{False}
		\item{tol}, \xmlDesc{float, optional field}, convergence tolerance for arpack.
		\default{0 (optimal value will be chosen by arpack)}
		\item{max\_iter}, \xmlDesc{int, optional field}, maximum number of iterations
		for arpack. \default{None (optimal value will be chosen by arpack)}
		\item \xmlNode{remove\_zero\_eig}, \xmlDesc{boolean, optional field}, If True,
		 then all components with zero eigenvalues are removed, so that the number of
		 components in the output may be < n\_components (and sometimes even zero due
		 to numerical instability). When n\_components is None, this parameter is
		 ignored and components with zero eigenvalues are removed regardless. \default{True}
	\end{itemize}
	\item \textbf{Sparse PCA} \\
	Finds the set of sparse components that can optimally reconstruct the data. The amount
	of sparseness is controllable by the coefficient of the L1 penalty, given by the
	parameter alpha.
	\skltype{Sparse PCA}{decomposition|SparsePCA}
	\begin{itemize}
		\item \xmlNode{n\_components}, \xmlDesc{integer, optional field}, Number of
		sparse atoms to extract. \default{None}
		\item \xmlNode{alpha}, \xmlDesc{float, optional field}, Sparsity controlling
		 parameter. Higher values lead to sparser components. \default{1.0}
		\item \xmlNode{ridge\_alpha}, \xmlDesc{float, optional field}, Amount of ridge
		 shrinkage to apply in order to improve conditioning when calling the transform
		 method. \default{0.01}
		\item \xmlNode{max\_iter}, \xmlDesc{float, optional field}, maximum number of
		iterations to perform. \default{1000}
		\item \xmlNode{tol}, \xmlDesc{float, optional field}, convergence tolerance.
		\default{1E-08}
		\item \xmlNode{method}, \xmlDesc{string, optional field}, method to use,
		options are:
		\begin{itemize}
			\item lars: uses the least angle regression method to solve the lasso
			 problem (linear\_model.lars\_path)
			\item cd: uses the coordinate descent method to compute the Lasso
			solution (linear\_model.Lasso)
		\end{itemize}
		Lars will be faster if the estimated components are sparse. \default{lars}
		\item \xmlNode{n\_jobs}, \xmlDesc{int, optional field}, number of parallel
		 runs to run. \default{1}
		\item \xmlNode{U\_init}, \xmlDesc{array of shape (n\_samples, n\_components)
		, optional field}, Initial values for the loadings for warm restart scenarios
		\default{None}
		\item \xmlNode{V\_init}, \xmlDesc{array of shape (n\_components, n\_features),
		 optional field}, Initial values for the components for warm restart scenarios
		\default{None}
		\item{verbose}, \xmlDesc{boolean, optional field}, Degree of verbosity of the
		 printed output. \default{False}
		\item{random\_state}, \xmlDesc{int or Random State, optional field}, Pseudo
		number generator state used for random sampling. \default{None}
	\end{itemize}
	\item \textbf{Mini Batch Sparse PCA} \\
	Finds the set of sparse components that can optimally reconstruct the data. The amount
	 of sparseness is controllable by the coefficient of the L1 penalty, given by the
	parameter alpha.
	\skltype{Mini Batch Sparse PCA}{decomposition|MiniBatchSparsePCA}
	\begin{itemize}
		\item \xmlNode{n\_components}, \xmlDesc{integer, optional field}, Number of
		 sparse atoms to extract. \default{None}
		\item \xmlNode{alpha}, \xmlDesc{float, optional field}, Sparsity controlling
		parameter. Higher values lead to sparser components. \default{1.0}
		\item \xmlNode{ridge\_alpha}, \xmlDesc{float, optional field}, Amount of ridge
		 shrinkage to apply in order to improve conditioning when calling the transform
		 method. \default{0.01}
		\item \xmlNode{n\_iter}, \xmlDesc{float, optional field}, number of iterations
		to perform per mini batch. \default{100}
		\item \xmlNode{callback}, \xmlDesc{callable, optional field}, callable that
		gets invoked every five iterations. \default{None}
		\item \xmlNode{batch\_size}, \xmlDesc{int, optional field}, the number of
		features to take in each mini batch. \default{3}
		\item \xmlNode{verbose}, \xmlDesc{boolean, optional field}, Degree of verbosity
		 of the printed output. \default{False}
		\item \xmlNode{shuffle}, \xmlDesc{boolean, optional field}, whether to shuffle
		the data before splitting it in batches. \default{True}
		\item \xmlNode{n\_jobs}, \xmlDesc{integer, optional field}, Parameters (keyword
		arguments) and values for kernel passed as callable object. Ignored by other
		kernels. \default{3}
		\item \xmlNode{metho}, \xmlDesc{string, optional field}, method to use,
		options are:
		\begin{itemize}
			\item lars: uses the least angle regression method to solve the lasso
			 problem (linear\_model.lars\_path),
			\item cd: uses the coordinate descent method to compute the Lasso solution
			 (linear\_model.Lasso)
		\end{itemize}
		Lars will be faster if the estimated components are sparse. \default{lars}
		\item \xmlNode{random\_state}, \xmlDesc{integer or Random State, optional field},
		 Pseudo number generator state used for random sampling. \default{None}
	\end{itemize}
\end{itemize}

\subparagraph{Truncated singular value decomposition} \hfil \\
\label{subparagraph:TruncatedSVD}
Dimensionality reduction using truncated SVD (aka LSA).
\skltype{Truncated SVD}{decomposition|TruncatedSVD}
\begin{itemize}
	\item \xmlNode{n\_components}, \xmlDesc{integer, optional field}, Desired dimensionality
	of output data. Must be strictly less than the number of features. The default value is
	useful for visualisation. For LSA, a value of 100 is recommended. \default{2}
	\item \xmlNode{algorithm}, \xmlDesc{string, optional field}, SVD solver to use:
	\begin{itemize}
		\item Randomized: randomized algorithm
		\item Arpack: ARPACK wrapper in.
	\end{itemize}
	\default{Randomized}
	\item \xmlNode{n\_iter}, \xmlDesc{float, optional field}, number of iterations andomized
	SVD solver. Not used by ARPACK. \default{5}
	\item \xmlNode{random\_state}, \xmlDesc{int or Random State, optional field}, Pseudo number
	generator state used for random sampling. If not given, the numpy.random singleton is used.
	\default{None}
	\item \xmlNode{tol}, \xmlDesc{float, optional field}, Tolerance for ARPACK. 0 means machine
	precision. Ignored by randomized SVD solver. \default{0.0}
\end{itemize}

\subparagraph{Fast ICA} \hfil \\
\label{subparagraph:FastICA}
A fast algorithm for Independent Component Analysis.
\skltype{Fast ICA}{decomposition|FastICA}
\begin{itemize}
	\item \xmlNode{n\_components}, \xmlDesc{integer, optional field}, Number of components to
	use. If none is passed, all are used. \default{None}
	\item \xmlNode{algorithm}, \xmlDesc{string, optional field}, algorithm used in FastICA:
	\begin{itemize}
		\item parallel,
		\item deflation.
	\end{itemize}
	\default{parallel}
	\item \xmlNode{fun}, \xmlDesc{string or function, optional field}, The functional form of
	the G function used in the approximation to neg-entropy. Could be either:
	\begin{itemize}
		\item logcosh,
		\item exp, or
		\item cube.
	\end{itemize}
	One can also provide own function. It should return a tuple containing the value of the
	 function, and of its derivative, in the point. \default{logcosh}
	\item \xmlNode{fun\_args}, \xmlDesc{dictionary, optional field}, Arguments to send to the
	functional form. If empty and if fun=’logcosh’, fun\_args will take value {‘alpha’ : 1.0}.
	\default{None}
	\item \xmlNode{max\_iter}, \xmlDesc{float, optional field}, maximum number of iterations
	 during fit. \default{200}
	\item \xmlNode{tol}, \xmlDesc{float, optional field}, Tolerance on update at each iteration.
	\default{0.0001}
	\item \xmlNode{w\_init}, \xmlDesc{None or an (n\_components, n\_components) ndarray,
	optional field}, The mixing matrix to be used to initialize the algorithm. \default{None}
	\item \xmlNode{randome\_state}, \xmlDesc{int or Random State, optional field}, Pseudo number
	 generator state used for random sampling. \default{None}
\end{itemize}

\paragraph{Manifold learning}
\label{paragraph:Manifold}
A manifold is a topological space that resembles a Euclidean space locally at each point. Manifold
learning is an approach to non-linear dimensionality reduction. It assumes that the data of interest
 lie on an embedded non-linear manifold within the higher-dimensional space. If this manifold is of
 low dimension, data can be visualized in the low-dimensional space. Algorithms for this task are
based on the idea that the dimensionality of many data sets is only artificially high.
\subparagraph{Isomap} \hfil \\
\label{subparagraph:Isomap}
Non-linear dimensionality reduction through Isometric Mapping (Isomap).
\skltype{Isometric Mapping}{manifold|Isomap}
\begin{itemize}
	\item \xmlNode{n\_neighbors}, \xmlDesc{integer, optional field}, Number of neighbors to
	consider for each point. \default{5}
	\item \xmlNode{n\_components}, \xmlDesc{integer, optional field}, Number of coordinates to
	manifold. \default{2}
	\item \xmlNode{eigen\_solver}, \xmlDesc{string, optional field}, eigen solver to use:
	\begin{itemize}
		\item auto: Attempt to choose the most efficient solver for the given problem,
		\item arpack: Use Arnoldi decomposition to find the eigenvalues and eigenvectors
		\item dense: Use a direct solver (i.e. LAPACK) for the eigenvalue decomposition
	\end{itemize}
	\default{auto}
	\item \xmlNode{tol}, \xmlDesc{float, optional field}, Convergence tolerance passed to
	arpack or lobpcg. not used if eigen\_solver is ‘dense’. \default{0.0}
	\item \xmlNode{max\_iter}, \xmlDesc{float, optional field}, Maximum number of iterations
	for the arpack solver. not used if eigen\_solver == ‘dense’. \default{None}
	\item \xmlNode{path\_method}, \xmlDesc{string, optional field}, Method to use in finding
	 shortest path. Could be either:
	\begin{itemize}
		\item Auto: attempt to choose the best algorithm
		\item FW: Floyd-Warshall algorithm
		\item D: Dijkstra algorithm with Fibonacci Heaps
	\end{itemize}
	\default{auto}
	\item \xmlNode{neighbors\_algorithm}, \xmlDesc{string, optional field}, Algorithm to use
	 for nearest neighbors search, passed to neighbors.NearestNeighbors instance.
	\begin{itemize}
		\item auto,
		\item brute
		\item kd\_tree
		\item ball\_tree
	\end{itemize}
	\default{auto}
\end{itemize}

\textbf{Example:}
\begin{lstlisting}[style=XML,morekeywords={subType}]
<Simulation>
  ...
  <Models>
    ...
      <PostProcessor name='PostProcessorName' subType='DataMining'>
          <KDD lib='SciKitLearn'>
              <Features>input</Features>
              <SKLtype>manifold|Isomap</SKLtype>
              <n_neighbors>5</n_neighbors>
	      <n_components>3</n_components>
	      <eigen_solver>arpack</eigen_solver>
	      <neighbors_algorithm>kd_tree</neighbors_algorithm>
            </KDD>
      </PostProcessor>
    ...
  <Models>
  ...
<Simulation>
\end{lstlisting}


\subparagraph{Locally Linear Embedding} \hfil \\
\label{subparagraph:LLE}
\skltype{Locally Linear Embedding}{manifold|LocallyLinearEmbedding}
\begin{itemize}
	\item \xmlNode{n\_neighbors}, \xmlDesc{integer, optional field}, Number of neighbors to
	consider for each point. \default{5}
	\item \xmlNode{n\_components}, \xmlDesc{integer, optional field}, Number of coordinates to
	 manifold. \default{2}
	\item \xmlNode{reg}, \xmlDesc{float, optional field}, regularization constant, multiplies
	the trace of the local covariance matrix of the distances. \default{0.01}
	\item \xmlNode{eigen\_solver}, \xmlDesc{string, optional field}, eigen solver to use:
	\begin{itemize}
		\item auto: Attempt to choose the most efficient solver for the given problem,
		\item arpack: use arnoldi iteration in shift-invert mode.
		\item dense: use standard dense matrix operations for the eigenvalue
	\end{itemize}
	\default{auto}
	\item \xmlNode{tol}, \xmlDesc{float, optional field}, Convergence tolerance passed to arpack.
	 not used if eigen\_solver is ‘dense’. \default{1E-06}
	\item \xmlNode{max\_iter}, \xmlDesc{int, optional field}, Maximum number of iterations for the
	 arpack solver. not used if eigen\_solver == ‘dense’. \default{100}
	\item \xmlNode{method}, \xmlDesc{string, optional field}, Method to use. Could be either:
	\begin{itemize}
		\item Standard: use the standard locally linear embedding algorithm
		\item hessian: use the Hessian eigenmap method
		\item itsa: use local tangent space alignment algorithm
	\end{itemize}
	\default{standard}
	\item \xmlNode{hessian\_tol}, \xmlDesc{float, optional field}, Tolerance for Hessian eigenmapping
	 method. Only used if method == 'hessian' \default{0.0001}
	\item \xmlNode{modified\_tol}, \xmlDesc{float, optional field}, Tolerance for modified LLE method.
	 Only used if method == 'modified' \default{0.0001}
	\item \xmlNode{neighbors\_algorithm}, \xmlDesc{string, optional field}, Algorithm to use for nearest
	 neighbors search, passed to neighbors.NearestNeighbors instance.
	\begin{itemize}
		\item auto,
		\item brute
		\item kd\_tree
		\item ball\_tree
	\end{itemize}
	\default{auto}
	\item \xmlNode{random\_state}, \xmlDesc{int or numpy random state, optional field}, the generator
	or seed used to determine the starting vector for arpack iterations. \default{None}
\end{itemize}
\subparagraph{Spectral Embedding} \hfil \\
\label{subparagraph:Spectral}
Spectral embedding for non-linear dimensionality reduction, it forms an affinity matrix given by the
 specified function and applies spectral decomposition to the corresponding graph laplacian. The resulting
 transformation is given by the value of the eigenvectors for each data point
\skltype{Spectral Embedding}{manifold|SpectralEmbedding}
\begin{itemize}
	\item \xmlNode{n\_components}, \xmlDesc{integer, optional field}, the dimension of projected
	sub-space. \default{2}
	\item \xmlNode{eigen\_solver}, \xmlDesc{string, optional field}, the eigen value decomposition
	 strategy to use:
	\begin{itemize}
		\item none,
		\item arpack.
		\item lobpcg,
		\item amg
	\end{itemize}
	\default{none}
	\item \xmlNode{random\_state}, \xmlDesc{integer or numpy random state, optional field}, A
	pseudo random number generator used for the initialization of the lobpcg eigen vectors
	decomposition when eigen\_solver == ‘amg. \default{None}
	\item \xmlNode{affinity}, \xmlDesc{string or callable, optional field}, How to construct
	 the affinity matrix:
	\begin{itemize}
		\item \textit{nearest\_neighbors} : construct affinity matrix by knn graph
		\item \textit{rbf} : construct affinity matrix by rbf kernel
		\item \textit{precomputed} : interpret X as precomputed affinity matrix
		\item \textit{callable} : use passed in function as affinity the function takes
		 in data matrix (n\_samples, n\_features) and return affinity matrix (n\_samples, n\_samples).
	\end{itemize}
	\default{nearest\_neighbor}
	\item \xmlNode{gamma}, \xmlDesc{float, optional field}, Kernel coefficient for rbf kernel.
	\default{None}
	\item \xmlNode{n\_neighbors}, \xmlDesc{int, optional field}, Number of nearest neighbors for
	 nearest\_neighbors graph building. \default{None}
\end{itemize}

\subparagraph{Multi-dimensional Scaling (MDS)} \hfil \\
\label{subparagraph:MDS}

\skltype{Multi Dimensional Scaling}{manifold|MDS}
\begin{itemize}
	\item \xmlNode{metric}, \xmlDesc{boolean, optional field}, compute metric or nonmetric SMACOF
	 (Scaling by Majorizing a Complicated Function) algorithm \default{True}
	\item \xmlNode{n\_components}, \xmlDesc{integer, optional field}, number of dimension in
	which to immerse the similarities overridden if initial array is provided. \default{2}
	\item \xmlNode{n\_init}, \xmlDesc{integer, optional field}, Number of time the smacof
	algorithm will be run with different initialisation. The final results will be the best
	 output of the n\_init consecutive runs in terms of stress. \default{4}
	\item \xmlNode{max\_iter}, \xmlDesc{integer, optional field}, Maximum number of iterations
	of the SMACOF algorithm for a single run \default{300}
	\item \xmlNode{verbose}, \xmlDesc{integer, optional field}, level of verbosity \default{0}
	\item \xmlNode{eps}, \xmlDesc{float, optional field}, relative tolerance with respect
	to stress to declare converge \default{1E-06}
	\item \xmlNode{n\_jobs}, \xmlDesc{integer, optional field}, The number of jobs to use for
	the computation. This works by breaking down the pairwise matrix into n\_jobs even slices
	 and computing them in parallel. If -1 all CPUs are used. If 1 is given, no parallel
	 computing code is used at all, which is useful for debugging. For n\_jobs below -1,
	(n\_cpus + 1 + n\_jobs) are used. Thus for n\_jobs = -2, all CPUs but one are used.
	\default{1}
	\item \xmlNode{random\_state}, \xmlNode{integer or numpy random state, optional field},
	The generator used to initialize the centers. If an integer is given, it fixes the seed.
	Defaults to the global numpy random number generator. \default{None}
	\item \xmlNode{dissimilarity}, \xmlDesc{string, optional field}, Which dissimilarity
	measure to use. Supported are ‘euclidean’ and ‘precomputed’. \default{euclidean}
\end{itemize}


\paragraph{Scipy}
\xmlString{Scipy} provides a Hierarchical clustering that performs clustering over PointSet and HistorySet.
This algorithm also automatically generates a dendrogram in .pdf format (i.e., dendrogram.pdf).

\begin{itemize}
  \item \xmlNode{SCIPYtype}, \xmlDesc{string, required field}, SCIPY algorithm to be employed.
  \item \xmlNode{Features}, \xmlDesc{string, required field}, defines the data to be used for training the data mining algorithm. It can be:
    \begin{itemize}
      \item the name of the variable in the defined dataObject entity
      \item the location (i.e. input or output). In this case the data mining is applied to all the variables in the defined space.
    \end{itemize}
  \item \xmlNode{method},         \xmlDesc{string, required field}, The linkage algorithm to be used  \default{single, complete, weighted, centroids, median, ward}.
  \item \xmlNode{metric},         \xmlDesc{string, required field}, The distance metric to be used \default{ ‘braycurtis’, ‘canberra’, ‘chebyshev’, ‘cityblock’,
                                                                    ‘correlation’, ‘cosine’, ‘dice’, ‘euclidean’, ‘hamming’, ‘jaccard’, ‘kulsinski’,
                                                                    ‘mahalanobis’, ‘matching’, ‘minkowski’, ‘rogerstanimoto’, ‘russellrao’, ‘seuclidean’,
                                                                    ‘sokalmichener’, ‘sokalsneath’, ‘sqeuclidean’, ‘yule’}.
  \item \xmlNode{level},          \xmlDesc{float, required field},  Clustering distance level where actual clusters are formed.
  \item \xmlNode{criterion},      \xmlDesc{string, required field}, The criterion to use in forming flat clusters. This can be any of the following values:
    \begin{itemize}
      \item   ``inconsistent''     : If a cluster node and all its descendants have an inconsistent value less than or equal to `t` then all its leaf descendants
                                     belong to the same flat cluster. When no non-singleton cluster meets this criterion, every node is assigned to its own
                                     cluster. (Default)
      \item   ``distance''         : Forms flat clusters so that the original observations in each flat cluster have no greater a cophenetic distance than $t$.
      \item   ``maxclust''         : Finds a minimum threshold ``r'' so that the cophenetic distance between any two original observations in the same flat cluster
                                     is no more than ``r'' and no more than $t$ flat clusters are formed.
      \item   ``monocrit''         : Forms a flat cluster from a cluster node c with index i when $monocrit[j] <= t$.
       \item  ``maxclust\_monocrit'' : Forms a flat cluster from a non-singleton cluster node ``c'' when $monocrit[i] <= r$ for all cluster indices ``i''
                                       below and including ``c''. ``r'' is minimized such that no more than ``t'' flat clusters are formed. monocrit must be
                                       monotonic.
    \end{itemize}
  \item \xmlNode{dendrogram},     \xmlDesc{boolean, required field}, If True the dendrogram is actually created.
  \item \xmlNode{truncationMode}, \xmlDesc{string, required field}, The dendrogram can be hard to read when the original observation matrix from which the
                                                                    linkage is derived is large. Truncation is used to condense the dendrogram. There are several
                                                                    modes:
                                                                    \begin{itemize}
                                                                      \item ``None'': No truncation is performed (Default).
                                                                      \item ``lastp'': The last p non-singleton formed in the linkage are the only non-leaf nodes
                                                                       in the linkage; they correspond to rows $Z[n-p-2:end]$ in Z. All other non-singleton
                                                                       clusters are contracted into leaf nodes.
                                                                      \item ``level''/``mtica'': No more than p levels of the dendrogram tree are displayed.
                                                                      This corresponds to Mathematica behavior.
                                                                    \end{itemize}
  \item \xmlNode{p},              \xmlDesc{int, required field},     The $p$ parameter for truncationMode.
  \item \xmlNode{leafCounts},     \xmlDesc{boolean, required field}, When True the cardinality non singleton nodes contracted into a leaf node is indiacted in
                                                                     parenthesis.
  \item \xmlNode{showContracted}, \xmlDesc{boolean, required field}, When True the heights of non singleton nodes contracted into a leaf node are plotted as
                                                                     crosses along the link connecting that leaf node.
  \item \xmlNode{annotatedAbove}, \xmlDesc{float, required field},  Clustering level above which the branching level is annotated.
\end{itemize}

\textbf{Example:}
\begin{lstlisting}[style=XML,morekeywords={subType}]
<Simulation>
  ...
  <Models>
    ...
    <PostProcessor name="hierarchical" subType="DataMining" verbosity="quiet">
      <KDD lib="Scipy" labelFeature='labels'>
        <SCIPYtype>cluster|Hierarchical</SCIPYtype>
        <Features>output</Features>
        <method>single</method>
        <metric>euclidean</metric>
        <level>75</level>
        <criterion>distance</criterion>
        <dendrogram>true</dendrogram>
        <truncationMode>lastp</truncationMode>
        <p>20</p>
        <leafCounts>True</leafCounts>
        <showContracted>True</showContracted>
        <annotatedAbove>10</annotatedAbove>
      </KDD>
    </PostProcessor>
    ...
  <Models>
  ...
<Simulation>
\end{lstlisting}




%%%%%%%%%%%%%% ParetoFrontier PP %%%%%%%%%%%%%%%%%%%

\subsubsection{ParetoFrontier}
\label{ParetoFrontierPP}
The \textbf{ParetoFrontier} post-processor is designed to identify the points lying on the Pareto Frontier in a multi-dimensional trade-space.
This post-processor receives as input a \textbf{DataObject} (a PointSet only) which contains all data points in the trade-space space and it
returns the subset of points lying in the Pareto Frontier as a PointSet.

It is here assumed that each data point of the input PointSet is a realization of the system under consideration for a
specific configuration to which corresponds several objective variables (e.g., cost and value).

%
\ppType{ParetoFrontier}{ParetoFrontier}
%
\begin{itemize}
  \item   \xmlNode{objective},\xmlDesc{string, required parameter}, ID of the objective variable that represents a dimension of the trade-space space.
          The \xmlNode{costID} requires one identifying attribute:
          \begin{itemize}
            \item \xmlAttr{goal}, \xmlDesc{string, required field}, Goal of the objective variable characteristic: minimzation (min) or maximization (max)
            \item \xmlAttr{upperLimit}, \xmlDesc{string, optional field}, Desired upper limit of the objective variable for the points in the Pareto frontier
            \item \xmlAttr{lowerLimit}, \xmlDesc{string, optional field}, Desired lower limit of the objective variable for the points in the Pareto frontier
          \end{itemize}
\end{itemize}

The following is an example where a set of realizations (the ``candidates'' PointSet) has been generated by changing two parameters
(var1 and var2) which produced two output variables: cost (which it is desired to be minimized) and value (which it is desired to be maximized).
The \textbf{ParetoFrontier} post-processor takes the ``candidates'' PointSet and populates a Point similar in structure
(the ``paretoPoints'' PointSet).

\textbf{Example:}
\begin{lstlisting}[style=XML,morekeywords={anAttribute},caption=ParetoFrontier input example (no expand)., label=lst:ParetoFrontier_PP_InputExample]
  <Models>
    <PostProcessor name="paretoPP" subType="ParetoFrontier">
      <objective goal='min' upperLimit='0.5'>cost</objective>
      <objective goal='max' lowerLimit='0.5'>value</objective>
    </PostProcessor>
  </Models>

  <Steps>
    <PostProcess name="PP">
      <Input     class="DataObjects"  type="PointSet"        >candidates</Input>
      <Model     class="Models"       type="PostProcessor"   >paretoPP</Model>
      <Output    class="DataObjects"  type="PointSet"        >paretoPoints</Output>
    </PostProcess>
  </Steps>

  <DataObjects>
    <PointSet name="candidates">
      <Input>var1,var2</Input>
      <Output>cost,value</Output>
    </PointSet>
    <PointSet name="paretoPoints">
      <Input>var1,var2</Input>
      <Output>cost,value</Output>
    </PointSet>
  </DataObjects>
\end{lstlisting}

\nb it is possible to specify both upper and lower limits for each objective variable.
When one or both of these limits are specified, then the pareto frontier is filtered such that all pareto frontier points that
satisfy those limits are preserved.

%%%%%%%%%%%%%% Metric PP %%%%%%%%%%%%%%%%%%%

\subsubsection{Metric}
\label{MetricPP}
The \textbf{Metric} post-processor is specifically used to calculate the distance values among points from PointSets and histories from HistorySets,
while the \textbf{Metrics} block (See Chapter \ref{sec:Metrics}) allows the user to specify the similarity/dissimilarity metrics to be used in this
post-processor. Both \textbf{PointSet} and \textbf{HistorySet} can be accepted by this post-processor.
If the name of given variable is unique, it can be used directly, otherwise the variable can be specified
with $DataObjectName|InputOrOutput|VariableName$ like other places in RAVEN.
Some of the Metrics also accept distributions to calculate the distance against.
These are specified by using the name of the distribution.
%
\ppType{Metric}{Metric}
%
\begin{itemize}
  \item \xmlNode{Features}, \xmlDesc{comma separated string, required field}, specifies the names of the features.
    This xml-node accepts the following attribute:
    \begin{itemize}
      \item \xmlAttr{type}, \xmlDesc{required string attribute}, the type of provided features. Currently only
        accept `variable'.
    \end{itemize}
  \item \xmlNode{Targets}, \xmlDesc{comma separated string, required field}, contains a comma separated list of
    the targets. \nb Each target is paired with a feature listed in xml node \xmlNode{Features}. In this case, the
    number of targets should be equal to the number of features.
    This xml-node accepts the following attribute:
    \begin{itemize}
      \item \xmlAttr{type}, \xmlDesc{required string attribute}, the type of provided features. Currently only
        accept `variable'.
    \end{itemize}
  \item \xmlNode{multiOutput}, \xmlDesc{optional string attribute}, only used when \textbf{HistorySet} is used as
    input. Defines aggregating of time-dependent metrics' calculations. Available options include:
    \textbf{mean, max, min, raw\_values} over the time. For example, when `mean' is used, the metrics' calculations
    will be averaged over the time. When `raw\_values' is used, the full set of  metrics' calculations will be dumped.
    \default{raw\_values}
  \item \xmlNode{weight}, \xmlDesc{comma separated floats, optional field}, when `mean' is provided for \xmlNode{multiOutput},
    the user can specify the weights that can be used for the average calculation of all outputs.
  \item \xmlNode{pivotParameter}, \xmlDesc{optional string attribute}, only used when \textbf{HistorySet}
    is used as input. The pivotParameter for given metrics' calculations.
    \default{time}
  \item \xmlNode{Metric}, \xmlDesc{string, required field}, specifies the \textbf{Metric} name that is defined via
    \textbf{Metrics} entity. In this xml-node, the following xml attributes need to be specified:
    \begin{itemize}
      \item \xmlAttr{class}, \xmlDesc{required string attribute}, the class of this metric (e.g. Metrics)
      \item \xmlAttr{type}, \xmlDesc{required string attribute}, the sub-type of this Metric (e.g. SKL, Minkowski)
    \end{itemize}
\end{itemize}

\textbf{Example:}

\begin{lstlisting}[style=XML]
<Simulation>
 ...
  <Models>
    ...
    <PostProcessor name="pp1" subType="Metric">
      <Features type="variable">ans</Features>
      <Targets type="variable">ans2</Targets>
      <Metric class="Metrics" type="SKL">euclidean</Metric>
      <Metric class="Metrics" type="SKL">cosine</Metric>
      <Metric class="Metrics" type="SKL">manhattan</Metric>
      <Metric class="Metrics" type="ScipyMetric">braycurtis</Metric>
      <Metric class="Metrics" type="ScipyMetric">canberra</Metric>
      <Metric class="Metrics" type="ScipyMetric">correlation</Metric>
      <Metric class="Metrics" type="ScipyMetric">minkowski</Metric>
    </PostProcessor>
    ...
  </Models>
 ...
</Simulation>
\end{lstlisting}

In order to access the results from this post-processor, RAVEN will define the variables as ``MetricName'' +
``\_'' + ``TargetVariableName'' + ``\_'' + ``FeatureVariableName'' to store the calculation results, and these
variables are also accessible by the users through RAVEN entities \textbf{DataObjects} and \textbf{OutStreams}.
\nb We will replace ``|'' in ``TargetVariableName'' and ``FeatureVariableName'' with ``\_''.
In previous example, variables such as \textit{euclidean\_ans2\_ans, cosine\_ans2\_ans, poly\_ans2\_ans} are accessible
by the users.

%%%%%%%%%%%%%% Cross Validation PP %%%%%%%%%%%%%%%%%%%

\subsubsection{CrossValidation}
\label{CVPP}
The \textbf{CrossValidation} post-processor is specifically used to evaluate estimator (i.e. ROMs) performance.
Cross-validation is a statistical method of evaluating and comparing learning algorithms by dividing data into
two portions: one used to `train' a surrogate model and the other used to validate the model, based on specific
scoring metrics. In typical cross-validation, the training and validation sets must crossover in successive
rounds such that each data point has a chance of being validated against the various sets. The basic form of
cross-validation is k-fold cross-validation. Other forms of cross-validation are special cases of k-fold or involve
repeated rounds of k-fold cross-validation. \nb It is important to notice that this post-processor currently can
only accept \textbf{PointSet} data object.
%
\ppType{CrossValidation}{CrossValidation}
%
\begin{itemize}
  \item \xmlNode{SciKitLearn}, \xmlDesc{string, required field}, the subnodes specifies the necessary information
    for the algorithm to be used in the post-processor. `SciKitLearn' is based on algorithms in SciKit-Learn
    library, and currently it performs cross-validation over \textbf{PointSet} only.
  \item \xmlNode{Metric}, \xmlDesc{string, required field}, specifies the \textbf{Metric} name that is defined via
    \textbf{Metrics} entity. In this xml-node, the following xml attributes need to be specified:
    \begin{itemize}
      \item \xmlAttr{class}, \xmlDesc{required string attribute}, the class of this metric (e.g. Metrics)
      \item \xmlAttr{type}, \xmlDesc{required string attribute}, the sub-type of this Metric (e.g. SKL, Minkowski)
    \end{itemize}
    \nb Currently, cross-validation post-processor only accepts \xmlNode{SKL} metrics with \xmlNode{metricType}
    \xmlString{mean\_absolute\_error}, \xmlString{explained\_variance\_score}, \xmlString{r2\_score},
    \xmlString{mean\_squared\_error}, and \xmlString{median\_absolute\_error}.
\end{itemize}

\textbf{Example:}

\begin{lstlisting}[style=XML]
<Simulation>
 ...
  <Files>
    <Input name="output_cv" type="">output_cv.xml</Input>
    <Input name="output_cv.csv" type="">output_cv.csv</Input>
  </Files>
  <Models>
    ...
    <ROM name="surrogate" subType="SciKitLearn">
      <SKLtype>linear_model|LinearRegression</SKLtype>
      <Features>x1,x2</Features>
      <Target>ans</Target>
      <fit_intercept>True</fit_intercept>
      <normalize>True</normalize>
    </ROM>
    <PostProcessor name="pp1" subType="CrossValidation">
        <SciKitLearn>
            <SKLtype>KFold</SKLtype>
            <n_splits>3</n_splits>
            <shuffle>False</shuffle>
        </SciKitLearn>
        <Metric class="Metrics" type="SKL">m1</Metric>
    </PostProcessor>
    ...
  </Models>
  <Metrics>
    <SKL name="m1">
      <metricType>mean_absolute_error</metricType>
    </SKL>
  </Metrics>
  <Steps>
    <PostProcess name="PP1">
        <Input class="DataObjects" type="PointSet">outputDataMC</Input>
        <Input class="Models" type="ROM">surrogate</Input>
        <Model class="Models" type="PostProcessor">pp1</Model>
        <Output class="Files" type="">output_cv</Output>
        <Output class="Files" type="">output_cv.csv</Output>
    </PostProcess>
  </Steps>
 ...
</Simulation>
\end{lstlisting}

In order to access the results from this post-processor, RAVEN will define the variables as ``cv'' +
``\_'' + ``MetricName'' + ``\_'' + ``ROMTargetVariable'' to store the calculation results, and these
variables are also accessible by the users through RAVEN entities \textbf{DataObjects} and \textbf{OutStreams}.
In previous example, variable \textit{cv\_m1\_ans} are accessible by the users.

\paragraph{SciKitLearn}

The algorithm for cross-validation is chosen by the subnode \xmlNode{SKLtype} under the parent node \xmlNode{SciKitLearn}.
In addition, a special subnode \xmlNode{average} can be used to obtain the average cross validation results.

\begin{itemize}
  \item \xmlNode{SKLtype}, \xmlDesc{string, required field}, contains a string that
    represents the cross-validation algorithm to be used. As mentioned, its format is:

    \xmlNode{SKLtype}algorithm\xmlNode{/SKLtype}.
  \item \xmlNode{average}, \xmlDesc{boolean, optional field}, if `True`, dump the average cross validation results into the
    output files.
\end{itemize}


Based on the \xmlNode{SKLtype} several different algorithms are available. In the following paragraphs a brief
explanation and the input requirements are reported for each of them.

\paragraph{K-fold}
\textbf{KFold} divides all the samples in $k$ groups of samples, called folds (if $k=n$, this is equivalent to the
\textbf{Leave One Out} strategy), of equal sizes (if possible). The prediction function is learned using $k-1$ folds,
and fold left out is used for test.
In order to use this algorithm, the user needs to set the subnode:
\xmlNode{SKLtype}KFold\xmlNode{/SKLtype}.
In addition to this XML node, several others are available:
\begin{itemize}
  \item \xmlNode{n\_splits}, \xmlDesc{integer, optional field}, number of folds, must be at least 2. \default{3}
  \item \xmlNode{shuffle}, \xmlDesc{boolean, optional field}, whether to shuffle the data before splitting into
    batches.
  \item \xmlNode{random\_state}, \xmlDesc{integer, optional field}, when shuffle=True,
    pseudo-random number generator state used for shuffling. If not present, use default numpy RNG for shuffling.
\end{itemize}

\paragraph{Stratified k-fold}
\textbf{StratifiedKFold} is a variation of \textit{k-fold} which returns stratified folds: each set contains approximately
the same percentage of samples of each target class as the complete set.
In order to use this algorithm, the user needs to set the subnode:

\xmlNode{SKLtype}StratifiedKFold\xmlNode{/SKLtype}.

In addition to this XML node, several others are available:
\begin{itemize}
  \item \xmlNode{labels}, \xmlDesc{list of integers, (n\_samples), required field}, contains a label for each sample.
  \item \xmlNode{n\_splits}, \xmlDesc{integer, optional field}, number of folds, must be at least 2. \default{3}
  \item \xmlNode{shuffle}, \xmlDesc{boolean, optional field}, whether to shuffle the data before splitting into
    batches.
  \item \xmlNode{random\_state}, \xmlDesc{integer, optional field}, when shuffle=True,
    pseudo-random number generator state used for shuffling. If not present, use default numpy RNG for shuffling.
\end{itemize}

\paragraph{Label k-fold}
\textbf{LabelKFold} is a variation of \textit{k-fold} which ensures that the same label is not in both testing and
training sets. This is necessary for example if you obtained data from different subjects and you want to avoid
over-fitting (i.e., learning person specific features) by testing and training on different subjects.
In order to use this algorithm, the user needs to set the subnode:

\xmlNode{SKLtype}LabelKFold\xmlNode{/SKLtype}.

In addition to this XML node, several others are available:
\begin{itemize}
  \item \xmlNode{labels}, \xmlDesc{list of integers with length (n\_samples, ), required field}, contains a label for
    each sample. The folds are built so that the same label does not appear in two different folds.
  \item \xmlNode{n\_splits}, \xmlDesc{integer, optional field}, number of folds, must be at least 2. \default{3}
\end{itemize}

\paragraph{Leave-One-Out - LOO}
\textbf{LeaveOneOut} (or LOO) is a simple cross-validation. Each learning set is created by taking all the samples
except one, the test set being the sample left out. Thus, for $n$ samples, we have $n$ different training sets and
$n$ different tests set. This is cross-validation procedure does not waste much data as only one sample is removed from
the training set.
In order to use this algorithm, the user needs to set the subnode:

\xmlNode{SKLtype}LeaveOneOut\xmlNode{/SKLtype}.

\paragraph{Leave-P-Out - LPO}
\textbf{LeavePOut} is very similar to \textbf{LeaveOneOut} as it creates all the possible training/test sets by removing
$p$ samples from the complete set. For $n$ samples, this produces $(^n_p)$ train-test pairs. Unlike \textbf{LeaveOneOut}
and \textbf{KFold}, the test sets will overlap for $p > 1$.
In order to use this algorithm, the user needs to set the subnode:

\xmlNode{SKLtype}LeavePOut\xmlNode{/SKLtype}.

In addition to this XML node, several others are available:
\begin{itemize}
  \item \xmlNode{p}, \xmlDesc{integer, required field}, size of the test sets
\end{itemize}

\paragraph{Leave-One-Label-Out - LOLO}
\textbf{LeaveOneLabelOut} (LOLO) is a cross-validation scheme which holds out the samples according to a third-party
provided array of integer labels. This label information can be used to encode arbitrary domain specific pre-defined
cross-validation folds. Each training set is thus constituted by all samples except the ones related to a specific
label.
In order to use this algorithm, the user needs to set the subnode:

\xmlNode{SKLtype}LeaveOneLabelOut\xmlNode{/SKLtype}.

In addition to this XML node, several others are available:
\begin{itemize}
  \item \xmlNode{labels}, \xmlDesc{list of integers, (n\_samples,), required field}, arbitrary
    domain-specific stratificatioin of the data to be used to draw the splits.
\end{itemize}

\paragraph{Leave-P-Label-Out}
\textbf{LeavePLabelOut} is imilar as \textit{Leave-One-Label-Out}, but removes samples related to $P$ labels for
each training/test set.
In order to use this algorithm, the user needs to set the subnode:

\xmlNode{SKLtype}LeavePLabelOut\xmlNode{/SKLtype}.

In addition to this XML node, several others are available:
\begin{itemize}
  \item \xmlNode{labels}, \xmlDesc{list of integers, (n\_samples,), required field}, arbitrary
    domain-specific stratificatioin of the data to be used to draw the splits.
  \item \xmlNode{n\_groups}, \xmlDesc{integer, optional field}, number of samples to leave out in the test split.
\end{itemize}

\paragraph{ShuffleSplit}
\textbf{ShuffleSplit} iterator will generate a user defined number of independent train/test dataset splits. Samples
are first shuffled and then split into a pair of train and test sets. it is possible to control the randomness for
reproducibility of the results by explicitly seeding the \xmlNode{random\_state} pseudo random number generator.
In order to use this algorithm, the user needs to set the subnode:

\xmlNode{SKLtype}ShuffleSplit\xmlNode{/SKLtype}.

In addition to this XML node, several others are available:
\begin{itemize}
  \item \xmlNode{n\_splits}, \xmlDesc{integer, optional field}, number of re-shuffling and splitting iterations
    \default{10}.
  \item \xmlNode{test\_size}, \xmlDesc{float or integer, optional field}, if float, should be between 0.0 and 1.0 and
    represent the proportion of the dataset to include in the test split. \default{0.1}
    If integer, represents the absolute number of test samples. If not present, the value is automatically set to
    the complement of the train size.
  \item \xmlNode{train\_size}, \xmlDesc{float or integer, optional field}, if float, should be between 0.0 and 1.0 and represent
    the proportion of the dataset to include in the train split. If integer, represents the absolute number of train
    samples. If not present, the value is automatically set to the complement of the test size.
  \item \xmlNode{random\_state}, \xmlDesc{integer, optional field}, when shuffle=True,
    pseudo-random number generator state used for shuffling. If not present, use default numpy RNG for shuffling.
\end{itemize}

\paragraph{Label-Shuffle-Split}
\textbf{LabelShuffleSplit} iterator behaves as a combination of \textbf{ShuffleSplit} and \textbf{LeavePLabelOut},
and generates a sequence of randomized partitions in which a subset of labels are held out for each split.
In order to use this algorithm, the user needs to set the subnode:

\xmlNode{SKLtype}LabelShuffleSplit\xmlNode{/SKLtype}.

In addition to this XML node, several others are available:
\begin{itemize}
  \item \xmlNode{labels}, \xmlDesc{list of integers, (n\_samples)}, labels of samples.
  \item \xmlNode{n\_splits}, \xmlDesc{integer, optional field}, number of re-shuffling and splitting iterations
    \default{10}.
  \item \xmlNode{test\_size}, \xmlDesc{float or integer, optional field}, if float, should be between 0.0 and 1.0 and
    represent the proportion of the dataset to include in the test split. \default{0.1}
    If integer, represents the absolute number of test samples. If not present, the value is automatically set to
    the complement of the train size.
  \item \xmlNode{train\_size}, \xmlDesc{float or integer, optional field}, if float, should be between 0.0 and 1.0 and represent
    the proportion of the dataset to include in the train split. If integer, represents the absolute number of train
    samples. If not present, the value is automatically set to the complement of the test size.
  \item \xmlNode{random\_state}, \xmlDesc{integer, optional field}, when shuffle=True,
    pseudo-random number generator state used for shuffling. If not present, use default numpy RNG for shuffling.
\end{itemize}

%%%%%%%%%%%%%% ValueDuration %%%%%%%%%%%%%%%%%%%
\subsubsection{ValueDuration}
\label{ValueDurationPP}
The \xmlNode{ValueDuration} postprocessor is a tool to construct a particular kind of histogram, where the
independent variable is the number of times a variable exceeds a particular value, and the dependent variable
is the values themselves.  An example of this is the Load Duration Curve in energy modeling. This approach is
similar to that used in Lebesgue integration. Note that for each realization in the input
\xmlNode{HistorySet}, a seperate load duration curve will be created for each target.

The \xmlNode{ValueDuration} postprocessor can only act on \xmlNode{HistorySet} data objects, and generates a
\xmlNode{HistorySet} in return.  Two output variables are created for each \xmlAttr{target}:
\xmlString{counts\_x} and \xmlString{bins\_x}, where \xmlString{x} is replaced by the name of the target.
These must be specified in the output data object in order to be collected.

To plot a traditional Load Duration Curve, the x-axis should be the bins variable, and the y-axis should be
the counts variable.

\ppType{ValueDuration}{ValueDuration}
%
\begin{itemize}
  \item \xmlNode{target}, \xmlDesc{comma separated strings, required field}, specifies the names of the
    target(s) for which Value Duration histograms should be generated.
  \item \xmlNode{bins}, \xmlDesc{integer, required field}, specifies the number of bins that the values of the
    targets should be counted into.
\end{itemize}

\textbf{Example:}

\begin{lstlisting}[style=XML]
<Simulation>
 ...
  <Models>
    ...
    <PostProcessor name="pp" subType="ValueDuration">
      <target>x, y</target>
      <bins>100</bins>
    </PostProcessor>
    ...
  </Models>
 ...
</Simulation>
\end{lstlisting}

%%%%%%%%%%%%%% FastFourierTransform %%%%%%%%%%%%%%%%%%%
\subsubsection{FastFourierTransform}
\label{FastFourierTransformPP}
The \xmlNode{FastFourierTransform} postprocessor provides access to the Numpy fast fourier transform function
\texttt{numpy.fft.fft}
and provides the frequencies, periods, and amplitudes from performing the transform. The periods are simply
the inverse of the frequencies, and the frequency units are the deltas between pivot values in the provided
input. For example, if data is collected every 3600 seconds, the units of frequency are per-hour.  This
postrpocessor expects uniformly-spaced pivot values. Note that for each realization in the input data object,
a separate fft will be created for each target.

The \xmlNode{FastFourierTransform} postprocessor can act on any target in a DataObject that depends on a
single index, and generates three histories per sample per target: an independent variable
\xmlString{target\_fft\_frequency}, and two dependent values \xmlString{target\_fft\_period} and
\xmlString{target\_fft\_amplitude}, which both depend on the frequency by default. In all three outputs,
\emph{target} is replaced by the name of the target for which the fft was requested.

\ppType{FastFourierTransform}{FastFourierTransform}
%
\begin{itemize}
  \item \xmlNode{target}, \xmlDesc{comma separated strings, required field}, specifies the names of the
    target(s) for which the fast Fourier transform should be calculated.
 \end{itemize}
\textbf{Example:}

\begin{lstlisting}[style=XML]
<Simulation>
 ...
  <Models>
    ...
    <PostProcessor name="pp" subType="FastFourierTransform">
      <target>x, y</target>
    </PostProcessor>
    ...
  </Models>
 ...
</Simulation>
\end{lstlisting}
%%%%%%%%%%%%%% SampleSelector %%%%%%%%%%%%%%%%%%%
\subsubsection{SampleSelector}
\label{SampleSelectorPP}
The \xmlNode{SampleSelector} postprocessor is a tool to select a row from a dataset, depending on different
criteria. The different criteria that can be used are listed below.

The \xmlNode{SampleSelector} postprocessor can  act on any \xmlNode{DataObjects}, and generates a
\xmlNode{DataObject} with a single realization in return.

\ppType{SampleSelector}{SampleSelector}
%
\begin{itemize}
  \item \xmlNode{criterion}, \xmlDesc{string, required field}, specifies the criterion to select the
    realization from the input DataObject. Options are as follows:
    \begin{itemize}
      \item \xmlString{min}, choose the realization that has the lowest value of the \xmlNode{target}
        variable. The target must be scalar.
      \item \xmlString{max}, choose the realization that has the highest value of the \xmlNode{target}
        variable. The target must be scalar.
      \item \xmlString{index}, choose the realization that has the provided index. The index must be an
        integer and is zero-based, meaning the first entry is at index 0, the second entry is at index 1, etc.
        The realization order is taken from the order in which they were entered originally into the input
        DataObject. If this option is used, the \xmlNode{criterion} node must have an \xmlAttr{value}
        attribute that gives the index.
    \end{itemize}
  \item \xmlNode{target}, \xmlDesc{string, optional field}, required if the criterion targets a particular
    variable (such as the minimum and maximum criteria). Specifies the name of the
    target for which the criterion should be evaluated.
\end{itemize}

\textbf{Example:}

\begin{lstlisting}[style=XML]
<Simulation>
 ...
  <Models>
    ...
    <PostProcessor name="select_min" subType="SampleSelector">
      <target>x</target>
      <criterion>min</criterion>
    </PostProcessor>
    ...
    <PostProcessor name="select_index" subType="SampleSelector">
      <criterion value='3'>index</criterion>
    </PostProcessor>
    ...
  </Models>
 ...
</Simulation>
\end{lstlisting}

%%%%% PP Validation %%%%%%%
\subsubsection{Validation}
\label{subsubsec:Validation}

The \xmlNode{Validation} post-processor represents a gate 
for applying a different range of algorithms to validate (e.g. compare)
dataset and/or models (e.g. Distributions). 
The post-processor is in charge of deploying a common infrastructure
for the user of  \textbf{Validation} problems. 
Several algorithms are avaialable within this post-processor:
\begin{itemize}
  \item  \textbf{Probabilistic}, for Static and Time-dependent data
  % \item  \textbf{DSS}
  % \item  \textbf{Representativity}
  % \item  \textbf{PCM}
\end{itemize}
%

The \textbf{Validation} post-processor makes use of the \textbf{Metric} system (See Chapter \ref{sec:Metrics}) to, in conjucntion with the specific algorithm chosen from the list above,
to report validation scores for both static and time-dependent data.
Indeed, Both \textbf{PointSet} and \textbf{HistorySet} can be accepted by this post-processor (depending on which algorithm is chosen).
If the name of given variable to be compared is unique, it can be used directly, otherwise the variable can be specified
with $DataObjectName|InputOrOutput|VariableName$ nomenclature.

%
\ppType{Validation}{Validation}
%
\begin{itemize}
  \item \xmlNode{Features}, \xmlDesc{comma separated string, required field}, specifies the names of the features.
  \item \xmlNode{Targets}, \xmlDesc{comma separated string, required field}, contains a comma separated list of
     targets. \nb Each target is paired with a feature listed in xml node \xmlNode{Features}. In this case, the
    number of targets should be equal to the number of features.
    \item \xmlNode{Metric}, \xmlDesc{string, required field}, specifies the \textbf{Metric} name that is defined via
    \textbf{Metrics} entity. In this xml-node, the following xml attributes need to be specified:
    \begin{itemize}
      \item \xmlAttr{class}, \xmlDesc{required string attribute}, the class of this metric (e.g. Metrics)
      \item \xmlAttr{type}, \xmlDesc{required string attribute}, the sub-type of this Metric (e.g. SKL, Minkowski)
    \end{itemize}
    The choice of the available metrics depends on the specific validation algorithm that is chosen (see table \ref{tab:validationAlgorithms})
\end{itemize}

In addition to the nodes above, the user must choose a validation algorithm:
\begin{itemize}
  \item \xmlNode{Probabilistic}, \xmlDesc{XML node, optional field}, specify that the validation needs to be performed 
  using the Probabilistic metrics: \textbf{CDFAreaDifference} (see \ref{subsubsec:metric_CDFAreaDifference})  or \textbf{PDFCommonArea} (see \ref{subsubsec:metric_PDFCommonArea}) 
  This xml-node accepts the following attribute:
    \begin{itemize}
      \item \xmlAttr{ name}, \xmlDesc{required string attribute}, the  user defined name of the validation algorithm used as prefix for the output results.
    \end{itemize}
  %\item \xmlNode{DSS}, \xmlDesc{XML node, optional field}, specify that the validation needs to be performed via DSS.  
  %This xml-node accepts the following attribute:
  %  \begin{itemize}
  %    \item \xmlAttr{ name}, \xmlDesc{required string attribute}, the  user defined name of the validation algorithm used as prefix for the output results.
  %  \end{itemize}
  %  The following subnodes must be inputted:
  %   \begin{itemize}
  %     \item \xmlNode{myNode}, \xmlDesc{comma separated string, required field}, DESCRIPTION
  %  \end{itemize}
\end{itemize}

\begin{table}[]
\caption{Validation Algorithms and respective available metrics and DataObjects}
\label{tab:validationAlgorithms}
\begin{tabular}{|c|c|c|}
\hline
\textbf{Validation Algorithm} & \textbf{DataObject}                                            & \textbf{Available Metrics}                                                   \\ \hline
Probabilistic                 & \begin{tabular}[c]{@{}c@{}}PointSet \\ HistorySet\end{tabular} & \begin{tabular}[c]{@{}c@{}}CDFAreaDifference\\ \\ PDFCommonArea\end{tabular} \\ \hline
DSS                           & HistorySet                                                     & Not Available Yet                                                            \\ \hline
\end{tabular}
\end{table}

\textbf{Example:}
\begin{lstlisting}[style=XML,morekeywords={subType}]
<Simulation>
  ...
  <Models>
    ...
    <PostProcessor name="pp1" subType="Validation">
      <Features>outputDataMC1|ans</Features>
      <Targets>outputDataMC2|ans2</Targets>
      <Metric class="Metrics" type="CDFAreaDifference">cdf_diff</Metric>
      <Metric class="Metrics" type="PDFCommonArea">pdf_area</Metric>
      <Probabilistic name="myProbMetric">
         <!--  the Probabilistic Validation does  not -->
      </Probabilistic>
    </PostProcessor>
    ...
  <Models>
  ...
<Simulation>
\end{lstlisting}

%%%%% PP EconomicRatio %%%%%%%
\input{EconomicRatio.tex}

%%%%% HistorySetDelay %%%%%%
\subsubsection{HistorySetDelay}
\label{HistorySetDelay}

This Post-Processor allows history sets to add delayed or lagged
variables. It copies a variable, but with a delay. For example, if
there a variable price that is set hourly, than new variable called
price\_prev\_hour could be set by using a delay of -1 as seen in the
listing below.  This can be useful for training a ROM or other data
analysis.

\ppType{HistorySetDelay}{HistorySetDelay}

In the \xmlNode{PostProcessor} input block, one or more of the following XML sub-nodes are required:

\begin{itemize}
\item \xmlNode{delay}, \xmlDesc{empty}, a delay node with the following required parameters:
  \begin{itemize}
  \item \xmlAttr{original}, \xmlDesc{string, required field}, the variable to start with
  \item \xmlAttr{new}, \xmlDesc{string, required field}, the new variable to create
  \item \xmlAttr{steps}, \xmlDesc{integer, required field}, the delay (if negative) or steps into the future (if positive) to use for the new variable (so -1 gives the previous, 1 gives the next)
  \item \xmlAttr{default}, \xmlDesc{float, required field}, the value to use for cases where there is no previous or next value (such as the beginning when a negative delay is used, or the end when the delay is positive).
  \end{itemize}
\end{itemize}

\begin{lstlisting}[style=XML]
<Simulation>
  ...
  <Models>
    ...
    <PostProcessor name="delayPP" subType="HistorySetDelay">
      <delay original="price" new="price_prev_hour" steps="-1" default="0.0"/>
      <delay original="price" new="price_prev_day" steps="-24" default="0.0"/>
      <delay original="price" new="price_prev_week" steps="-168" default="-1.0"/>
    </PostProcessor>
  </Models>
  ...
  <Steps>
    ...
    <PostProcess name="delay">
      <Input class="DataObjects" type="HistorySet">samples</Input>
      <Model class="Models" type="PostProcessor">delayPP</Model>
      <Output class="DataObjects" type="HistorySet">delayed_samples</Output>
    </PostProcess>
    ...
  </Steps>
  ...
  <DataObjects>
    <HistorySet name="samples">
      <Input>demand</Input>
      <Output>price</Output>
      <options>
        <pivotParameter>hour</pivotParameter>
      </options>
    </HistorySet>
    <HistorySet name="delayed_samples">
      <Input>demand</Input>
      <Output>price,price_prev_hour,price_prev_day,price_prev_week</Output>
      <options>
        <pivotParameter>hour</pivotParameter>
      </options>
    </HistorySet>
    ...
  </DataObjects>
</Simulation>
\end{lstlisting}


%%%%% HStoPSOperator %%%%%%
\subsubsection{HStoPSOperator}
\label{HStoPSOperator}

This Post-Processor performs the conversion from HistorySet to PointSet performing a projection of the output space.
\ppType{HStoPSOperator}{HStoPSOperator}
In the \xmlNode{PostProcessor} input block, the following XML sub-nodes are available:

\begin{itemize}
   \item \xmlNode{pivotParameter}, \xmlDesc{string, optional field}, ID of the temporal variable. Default is ``time''.
   \nb Used just in case the  \xmlNode{pivotValue}-based operation  is requested
    \item \xmlNode{operator}, \xmlDesc{string, optional field}, the operation to perform on the output space:
      \begin{itemize}
        \item \textbf{min}, compute the minimum of each variable along each single history
         \item \textbf{max}, compute the maximum of each variable along each single history
         \item \textbf{average}, compute the average of each variable along each single history
         \item \textbf{all}, join together all of the each variable in
           the history, and make the pivotParameter a regular
           parameter.  Unlike the min and max operators, this keeps
           all the data, just organized differently. This operator
           does this by propagating the other input parameters for
           each item of the pivotParameter.
           Table~\ref{operator_all_switch_before} shows an example
           HistorySet with input parameter x, pivot parameter t, and
           output parameter b and then
           Table~\ref{operator_all_switch_after} shows the resulting
           PointSet with input parameters x and t, and output
           parameter b. Note that which parameters are input and which
           are output in the resulting PointSet depends on the
           DataObject specification.
       \end{itemize}
        \nb This node can be inputted only if \xmlNode{pivotValue} and \xmlNode{row} are not present
     \item \xmlNode{pivotValue}, \xmlDesc{float, optional field}, the value of the pivotParameter with respect to the other outputs need to be extracted.
       \nb This node can be inputted only if \xmlNode{operator} and \xmlNode{row} are not present
     \item \xmlNode{pivotStrategy}, \xmlDesc{string, optional field}, The strategy to use for the pivotValue:
       \begin{itemize}
        \item \textbf{nearest}, find the value that is the nearest with respect the \xmlNode{pivotValue}
        \item \textbf{floor}, find the value that is the nearest with respect to the \xmlNode{pivotValue} but less then the  \xmlNode{pivotValue}
        \item \textbf{celing}, find the value that is the nearest with respect to the \xmlNode{pivotValue} but greater then the  \xmlNode{pivotValue}
        \item \textbf{interpolate}, if the exact  \xmlNode{pivotValue}  can not be found, interpolate using a linear approach
       \end{itemize}

       \nb Valid just in case \xmlNode{pivotValue} is present
     \item \xmlNode{row}, \xmlDesc{int, optional field}, the row index at which the outputs need to be extracted.
       \nb This node can be inputted only if \xmlNode{operator} and \xmlNode{pivotValue} are not present
\end{itemize}

This example will show how the XML input block would look like:

\begin{lstlisting}[style=XML,morekeywords={subType,debug,name,class,type}]
<Simulation>
  ...
  <Models>
    ...
    <PostProcessor name="HStoPSperatorRows" subType="HStoPSOperator">
      <row>-1</row>
    </PostProcessor>
    <PostProcessor name="HStoPSoperatorPivotValues" subType="HStoPSOperator">
        <pivotParameter>time</pivotParameter>
        <pivotValue>0.3</pivotValue>
    </PostProcessor>
    <PostProcessor name="HStoPSoperatorOperatorMax" subType="HStoPSOperator">
        <pivotParameter>time</pivotParameter>
        <operator>max</operator>
    </PostProcessor>
    <PostProcessor name="HStoPSoperatorOperatorMin" subType="HStoPSOperator">
        <pivotParameter>time</pivotParameter>
        <operator>min</operator>
    </PostProcessor>
    <PostProcessor name="HStoPSoperatorOperatorAverage" subType="HStoPSOperator">
        <pivotParameter>time</pivotParameter>
        <operator>average</operator>
    </PostProcessor>
    ...
  </Models>
  ...
</Simulation>
\end{lstlisting}

\begin{table}[!hbtp]
  \caption{Starting HistorySet for operator all}
  \label{operator_all_switch_before}
\begin{tabular}{l|l|l}
  x & t & b \\
  \hline
  5.0 &  &  \\
  \hline
  & 1.0 & 6.0 \\
  \hline
  & 2.0 & 7.0 \\
\end{tabular}
\end{table}

\begin{table}[!hbtp]
  \caption{Resulting PointSet after operator all}
  \label{operator_all_switch_after}
\begin{tabular}{l|l|l}
  x & t & b \\
  \hline
  5.0 & 1.0 & 6.0  \\
  \hline
  5.0 & 2.0 & 7.0 \\
\end{tabular}
\end{table}


%%%%% HistorySetSampling %%%%%%
\subsubsection{HistorySetSampling}
\label{HistorySetSampling}

This Post-Processor performs the conversion from HistorySet to HistorySet
The conversion is made so that each history H is re-sampled accordingly  to a
specific sampling strategy.
It can be used to reduce the amount of space required by the HistorySet.

\ppType{HistorySetSampling}{HistorySetSampling}

In the \xmlNode{PostProcessor} input block, the following XML sub-nodes are required,
independent of the \xmlAttr{subType} specified:

\begin{itemize}
   \item \xmlNode{samplingType}, \xmlDesc{string, required field}, specifies the type of sampling method to be used:
   \begin{itemize}
     \item uniform: the set of \xmlNode{numberOfSamples} samples are uniformly distributed along the time axis
     \item firstDerivative: the set of \xmlNode{numberOfSamples} samples are distributed along the time axis in regions with
                            higher first order derivative
     \item secondDerivative: the set of \xmlNode{numberOfSamples} samples are distributed along the time axis in regions with
                             higher second order derivative
     \item filteredFirstDerivative: samples are located where the first derivative is greater than the specified \xmlNode{tolerance} value
                                    (hence, the number of samples can vary from history to history)
     \item filteredSecondDerivative: samples are located where the second derivative is greater than the specified \xmlNode{tolerance} value
                                     (hence, the number of samples can vary from history to history)
   \end{itemize}
   \item \xmlNode{numberOfSamples}, \xmlDesc{integer, optional field}, number of samples (required only for the following sampling
                                             types: uniform, firstDerivative secondDerivative)
   \item \xmlNode{pivotParameter}, \xmlDesc{string, required field}, ID of the temporal variable
   \item \xmlNode{interpolation}, \xmlDesc{string, optional field}, type of interpolation to be employed for the history reconstruction
                                           (required only for the following sampling types: uniform, firstDerivative secondDerivative).
                                           Valid types of interpolation to specified: linear, nearest, zero, slinear, quadratic, cubic, intervalAverage
   \item \xmlNode{tolerance}, \xmlDesc{string, optional field}, tolerance level (required only for the following sampling types:
                                       filteredFirstDerivative or filteredSecondDerivative)
\end{itemize}


%%%%% HistorySetSync %%%%%%
\subsubsection{HistorySetSync}
\label{HistorySetSync}

This Post-Processor performs the conversion from HistorySet to HistorySet
The conversion is made so that all histories are synchronized in time.
It can be used to allow the histories to be sampled at the same time instant.

There are two possible synchronization methods, specified through the \xmlNode{syncMethod} node.  If the
\xmlNode{syncMethod} is \xmlString{grid}, a \xmlNode{numberOfSamples} node is specified,
which yields an equally-spaced grid of time points. The output values for these points will be linearly derived
using nearest sampled time points, and the new HistorySet will contain only the new grid points.

The other methods are used by specifying \xmlNode{syncMethod} as \xmlString{all}, \xmlString{min}, or
\xmlString{max}.  For \xmlString{all}, the postprocessor will iterate through the
existing histories, collect all the time points used in any of them, and use these as the new grid on which to
establish histories, retaining all the exact original values and interpolating linearly where necessary.
In the event of \xmlString{min} or \xmlString{max}, the postprocessor will find the smallest or largest time
history, respectively, and use those time values as nodes to interpolate between.

\ppType{HistorySetSync}{HistorySetSync}

In the \xmlNode{PostProcessor} input block, the following XML sub-nodes are required,
independent of the \xmlAttr{subType} specified:

\begin{itemize}
   \item \xmlNode{pivotParameter}, \xmlDesc{string, required field}, ID of the temporal variable
   \item \xmlNode{extension}, \xmlDesc{string, required field}, type of extension when the sync process goes outside the boundaries of the history (zeroed or extended)
   \item \xmlNode{syncMethod}, \xmlDesc{string, required field}, synchronization strategy to employ (see
     description above).  Options are \xmlString{grid}, \xmlString{all}, \xmlString{max}, \xmlString{min}.
   \item \xmlNode{numberOfSamples}, \xmlDesc{integer, optional field}, required if \xmlNode{syncMethod} is
     \xmlString{grid}, number of new time samples
\end{itemize}


%%%%% HistorySetSnapShot %%%%%%
\subsubsection{HistorySetSnapShot}
\label{HistorySetSnapShot}

This Post-Processor performs a conversion from HistorySet to PointSet.
The conversion is made so that each history $H$ is converted to a single point $P$.
There are several methods that can be employed to choose the single point from the history:
\begin{itemize}
  \item min: Take a time slice when the \xmlNode{pivotVar} is at its smallest value,
  \item max: Take a time slice when the \xmlNode{pivotVar} is at its largest value,
  \item average: Take a time slice when the \xmlNode{pivotVar} is at its time-weighted average value,
  \item value: Take a time slice when the \xmlNode{pivotVar} \emph{first passes} its specified value,
  \item timeSlice: Take a time slice index from the sampled time instance space.
\end{itemize}
To demonstrate the timeSlice, assume that each history H is a dict of n output variables $x_1=[...],
x_n=[...]$, then the resulting point P is at time instant index t: $P=[x_1[t],...,x_n[t]]$.

Choosing one the these methods for the \xmlNode{type} node will take a time slice for all the variables in the
output space based on the provided parameters.  Alternatively, a \xmlString{mixed} type can be used, in which
each output variable can use a different time slice parameter.  In other words, you can take the max of one
variable while taking the minimum of another, etc.

\ppType{HistorySetSnapShot}{HistorySetSnapShot}

In the \xmlNode{PostProcessor} input block, the following XML sub-nodes are required,
independent of the \xmlAttr{subType} specified:

\begin{itemize}
  \item \xmlNode{type}, \xmlDesc{string, required field}, type of operation: \xmlString{min}, \xmlString{max},
                        \xmlString{average}, \xmlString{value}, \xmlString{timeSlice}, or \xmlString{mixed}
   \item \xmlNode{extension}, \xmlDesc{string, required field}, type of extension when the sync process goes outside the boundaries of the history (zeroed or extended)
   \item \xmlNode{pivotParameter}, \xmlDesc{string, optional field}, name of the temporal variable.  Required for the
     \xmlString{average} and \xmlString{timeSlice} methods.
\end{itemize}

If a \xmlString{timeSlice} type is in use, the following nodes also are required:
\begin{itemize}
   \item \xmlNode{timeInstant}, \xmlDesc{integer, required field}, required and only used in the
     \xmlString{timeSlice} type.  Location of the time slice (integer index)
   \item \xmlNode{numberOfSamples}, \xmlDesc{integer, required field}, number of samples
\end{itemize}

If instead a \xmlString{min}, \xmlString{max}, \xmlString{average}, or \xmlString{value} is used, the following nodes
are also required:
\begin{itemize}
   \item \xmlNode{pivotVar}, \xmlDesc{string, required field},  Name of the chosen indexing variable (the
         variable whose min, max, average, or value is used to determine the time slice)
       \item \xmlNode{pivotVal}, \xmlDesc{float, optional field},  required for \xmlString{value} type, the value for the chosen variable
\end{itemize}

Lastly, if a \xmlString{mixed} approach is used, the following nodes apply:
\begin{itemize}
  \item \xmlNode{max}, \xmlDesc{string, optional field}, the names of variables whose output should be their
    own maximum value within the history.
  \item \xmlNode{min}, \xmlDesc{string, optional field}, the names of variables whose output should be their
    own minimum value within the history.
  \item \xmlNode{average}, \xmlDesc{string, optional field}, the names of variables whose output should be their
    own average value within the history. Note that a \xmlNode{pivotParameter} node is required to perform averages.
  \item \xmlNode{value}, \xmlDesc{string, optional field}, the names of variables whose output should be taken
    at a time slice determined by another variable.  As with the non-mixed \xmlString{value} type, the first
    time the \xmlAttr{pivotVar} crosses the specified \xmlAttr{pivotVal} will be the time slice taken.
    This node requires two attributes, if used:
    \begin{itemize}
      \item \xmlAttr{pivotVar}, \xmlDesc{string, required field}, the name of the variable on which the time
        slice will be performed.  That is, if we want the value of $y$ when $t=0.245$,
        this attribute would be \xmlString{t}.
      \item \xmlAttr{pivotVal}, \xmlDesc{float, required field}, the value of the \xmlAttr{pivotVar} on which the time
        slice will be performed.  That is, if we want the value of $y$ when $t=0.245$,
        this attribute would be \xmlString{0.245}.
    \end{itemize}
  Note that all the outputs of the \xmlNode{DataObject} output of this postprocessor must be listed under one
  of the \xmlString{mixed} node types in order for values to be returned.
\end{itemize}

\textbf{Example (mixed):}
This example will output the average value of $x$ for $x$, the value of $y$ at
time$=0.245$ for $y$, and the value of $z$ at $x=4.0$ for $z$.
\begin{lstlisting}[style=XML,morekeywords={subType,debug,name,class,type}]
<Simulation>
  ...
  <Models>
    ...
    <PostProcessor name="mampp2" subType="HistorySetSnapShot">
      <type>mixed</type>
      <average>x</average>
      <value pivotVar="time" pivotVal="0.245">y</value>
      <value pivotVar="x" pivotVal="4.0">z</value>
      <pivotParameter>time</pivotParameter>
      <extension>zeroed</extension>
    </PostProcessor>
    ...
  </Models>
  ...
</Simulation>
\end{lstlisting}


%%%%% HS2PS %%%%%%
\subsubsection{HS2PS}
\label{HS2PS}

This Post-Processor performs a conversion from HistorySet to PointSet.
The conversion is made so that each history $H$ is converted to a single point $P$.
Assume that each history $H$ is a dict of $n$ output variables $x_1=[...],x_n=[...]$, then the resulting point $P$ is $P=concat(x_1,...,x_n)$.
Note: it is here assumed that all histories have been sync so that they have the same length, start point and end point. If you are not sure, do a pre-processing the the original history set.

\ppType{HS2PS}{HS2PS}

In the \xmlNode{PostProcessor} input block, the following XML sub-nodes are required,
independent of the \xmlAttr{subType} specified (min, max, avg and value case):

\begin{itemize}
   \item \xmlNode{pivotParameter}, \xmlDesc{string, optional field}, ID of the temporal variable (only for avg)
\end{itemize}


%%%%% TypicalHistoryFromHistorySet %%%%%%
\subsubsection{TypicalHistoryFromHistorySet}
\label{TypicalHistoryFromHistorySet}

This Post-Processor performs a simplified procedure of \cite{wilcox2008users} to form a ``typical'' time series from multiple time series. The input should be a HistorySet, with each history in the HistorySet synchronized. For HistorySet that is not synchronized, use Post-Processor method \textbf{HistorySetSync}  to synchronize the data before running this method.

Each history in input HistorySet is first converted to multiple histories each has maximum time specified in \xmlNode{outputLen} (see below). Each converted history $H_i$ is divided into a set of subsequences $\{H_i^j\}$, and the division is guided by the \xmlNode{subseqLen} node specified in the input XML. The value of \xmlNode{subseqLen} should be a list of positive numbers that specify the length of each subsequence. If the number of subsequence for each history is more than the number of values given in \xmlNode{subseqLen}, the values in \xmlNode{subseqLen} would be reused.

For each variable $x$, the method first computes the empirical CDF (cumulative density function) by using all the data values of $x$ in the HistorySet. This CDF is termed as long-term CDF for $x$. Then for each subsequence $H_i^j$, the method computes the empirical CDF by using all the data values of $x$ in $H_i^j$. This CDF is termed as subsequential CDF. For the first interval window (i.e., $j=1$), the method computes the Finkelstein-Schafer (FS) statistics \cite{finkelstein1971improved} between the long term CDF and the subsequential CDF of $H_i^1$ for each $i$. The FS statistics is defined as following.
\begin{align*}
FS & = \sum_x FS_x\\
FS_x &= \frac{1}{N}\sum_{n=1}^N\delta_n
\end{align*}
where $N$ is the number of value reading in the empirical CDF and $\delta_n$ is the absolute difference between the long term CDF and the subsequential CDF at value $x_n$. The subsequence $H_i^1$ with minimal FS statistics will be selected as the typical subsequence for the interval window $j=1$. Such process repeats for $j=2,3,\dots$ until all subsequences have been processed. Then all the typical subsequences will be concatenated to form a complete history.

\ppType{TypicalHistoryFromHistorySet}{TypicalHistoryFromHistorySet}

In the \xmlNode{PostProcessor} input block, the following XML sub-nodes are required,
independent of the \xmlAttr{subType} specified:

\begin{itemize}
   \item \xmlNode{pivotParameter}, \xmlDesc{string, optional field}, ID of the temporal variable
   \default{Time}
   \item \xmlNode{subseqLen}, \xmlDesc{integers, required field}, length of the divided subsequence (see above)
   \item \xmlNode{outputLen}, \xmlDesc{integer, optional field}, maximum value of the temporal variable for the generated typical history
   \default{Maximum value of the variable with name of \xmlNode{pivotParameter}}
\end{itemize}

For example, consider history of data collected over three years in one-second increments,
where the user wants a single \emph{typical year} extracted from the data.
The user wants this data constructed by combining twelve equal \emph{typical month}
segments.  In this case, the parameter \xmlNode{outputLen} should be \texttt{31536000} (the number of seconds
in a year), while the parameter \xmlNode{subseqLen} should be \texttt{2592000} (the number of seconds in a
month).  Using a value for \xmlNode{subseqLen} that is either much, much smaller than \xmlNode{outputLen} or
of equal size to \xmlNode{outputLen} might have unexpected results.  In general, we recommend using a
\xmlNode{subseqLen} that is roughly an order of magnitude smaller than \xmlNode{outputLen}.


%%%%% dataObjectLabelFilter %%%%%%
\subsubsection{dataObjectLabelFilter}
\label{dataObjectLabelFilter}

This Post-Processor allows to filter the portion of a dataObject, either PointSet or HistorySet, with a given clustering label.
A clustering algorithm associates a unique cluster label to each element of the dataObject (PointSet or HistorySet).
This cluster label is a natural number ranging from $0$ (or $1$ depending on the algorithm) to $N$ where $N$ is the number of obtained clusters.
Recall that some clustering algorithms (e.g., K-Means) receive $N$ as input while others (e.g., Mean-Shift) determine $N$ after clustering has been performed.
Thus, this Post-Processor is naturally employed after a data-mining clustering techniques has been performed on a dataObject so that each clusters
can be analyzed separately.

\ppType{dataObjectLabelFilter}{dataObjectLabelFilter}

In the \xmlNode{PostProcessor} input block, the following XML sub-nodes are required,
independently of the \xmlAttr{subType} specified:

\begin{itemize}
   \item \xmlNode{label}, \xmlDesc{string, required field}, name of the clustering label
   \item \xmlNode{clusterIDs}, \xmlDesc{integers, required field}, ID of the selected clusters. Note that more than one ID can be provided as input
\end{itemize}


%%%%%%%%%%%%%% InterfacedPostProcessors %%%%%%%%%%%%%%%%
% To be replaced by the PostProcessor Plugin
%%%%%%%%%%%%%%%%%%%%%%%%%%%%%%%%%%%%%%%%%%%%%%%%%%%%%%%%
%\subsubsection{Interfaced}
\label{Interfaced}
The \textbf{Interfaced} post-processor is a Post-Processor that allows the user
to create its own Post-Processor. While the External Post-Processor (see
Section~\ref{External} allows the user to create case-dependent
Post-Processors, with this new class the user can create new general
purpose Post-Processors.
%

\ppType{Interfaced}{Interfaced}

\begin{itemize}
  \item \xmlNode{method}, \xmlDesc{comma separated string, required field},
  lists the method names of a method that will be computed (each
  returning a post-processing value). All available methods need to be included
  in the ``/raven/framework/PostProcessorFunctions/'' folder
\end{itemize}

\textbf{Example:}
\begin{lstlisting}[style=XML,morekeywords={subType,debug,name,class,type}]
<Simulation>
  ...
  <Models>
    ...
    <PostProcessor name="example" subType='InterfacedPostProcessor'verbosity='debug'>
       <method>testInterfacedPP</method>
       <!--Here, the xml nodes required by the chosen method have to be
       included.
        -->
    </PostProcessor>
    ...
  </Models>
  ...
</Simulation>
\end{lstlisting}

All the \textbf{Interfaced} post-processors need to be contained in the
``/raven/framework/PostProcessorFunctions/'' folder. In fact, once the
\textbf{Interfaced} post-processor is defined in the RAVEN input file, RAVEN
search that the method of the post-processor is located in such folder.

The class specified in the \textbf{Interfaced} post-processor has to inherit the
PostProcessorInterfaceBase class and the user must specify this set of
methods:
\begin{itemize}
  \item initialize: in this method, the internal parameters of the
  post-processor are initialized. Mandatory variables that needs to be
  specified are the following:
\begin{itemize}
  \item self.inputFormat: type of dataObject expected in input
  \item self.outputFormat: type of dataObject generated in output
\end{itemize}
  \item readMoreXML: this method is in charge of reading the PostProcessor xml
  node, parse it and fill the PostProcessor internal variables.
  \item run: this method performs the desired computation of the dataObject.
\end{itemize}

\begin{lstlisting}[language=python]
from PostProcessorInterfaceBaseClass import PostProcessorInterfaceBase
class testInterfacedPP(PostProcessorInterfaceBase):
  def initialize(self)
  def readMoreXML(self,xmlNode)
  def run(self,inputDic)
\end{lstlisting}

\paragraph{Data Format}
The user is not allowed to modify directly the DataObjects, however the
content of the DataObjects is available in the form of a python dictionary.
Both the dictionary give in input and the one generated in the output of the
PostProcessor are structured as follows:

\begin{lstlisting}[language=python]
inputDict = {'data':{}, 'metadata':{}}
\end{lstlisting}

where:

\begin{lstlisting}[language=python]
inputDict['data'] = {'input':{}, 'output':{}}
\end{lstlisting}

In the input dictonary, each input variable is listed as a dictionary that
contains a numpy array with its own values as shown below for a simplified
example

\begin{lstlisting}[language=python]
inputDict['data']['input'] = {'inputVar1': array([ 1.,2.,3.]),
                              'inputVar2': array([4.,5.,6.])}
\end{lstlisting}

Similarly, if the dataObject is a PointSet then the output dictionary is
structured as follows:

\begin{lstlisting}[language=python]
inputDict['data']['output'] = {'outputVar1': array([ .1,.2,.3]),
                               'outputVar2':array([.4,.5,.6])}
\end{lstlisting}

Howevers, if the dataObject is a HistorySet then the output dictionary is
structured as follows:

\begin{lstlisting}[language=python]
inputDict['data']['output'] = {'hist1': {}, 'hist2':{}}
\end{lstlisting}

where

\begin{lstlisting}[language=python]
inputDict['output']['data'][hist1] = {'time': array([ .1,.2,.3]),
                              'outputVar1':array([ .4,.5,.6])}
inputDict['output']['data'][hist2] = {'time': array([ .1,.2,.3]),
                              'outputVar1':array([ .14,.15,.16])}
\end{lstlisting}



%
%%%%%%%%%%%%%%%%%%%%%%%%%%%%%%%%%%%%%
%%%%%%  EnsembleModel  Model   %%%%%%
%%%%%%%%%%%%%%%%%%%%%%%%%%%%%%%%%%%%%
%

\subsection{EnsembleModel}
\label{subsec:models_EnsembleModel}
As already mentioned, the \textbf{EnsembleModel} is able to combine \textbf{Code}(see ~\ref{subsec:models_code}),
\textbf{ExternalModel}(see ~\ref{subsec:models_externalModel}) and \textbf{ROM}(see ~\ref{subsec:models_externalModel}) Models.
\\It is aimed to create a chain of Models (whose execution order is determined by the Input/Output relationships among them).
  If the relationships among the models evolve in a non-linear system, a Picard's Iteration scheme is employed.
\\Currently this model is able to share information (i.e. data) using \textbf{PointSet},  \textbf{HistorySet} and \textbf{DataSet}

The specifications of a EnsembleModel must be defined within the XML block
\xmlNode{EnsembleModel}.
%
This XML node needs to contain the attributes:

\vspace{-5mm}
\begin{itemize}
  \itemsep0em
  \item \xmlAttr{name}, \xmlDesc{required string attribute}, user-defined name
  of this EnsembleModel.
  %
  \nb As with the other objects, this is the name that can be used to refer to
  this specific entity from other input blocks in the XML.
  \item \xmlAttr{subType}, \xmlDesc{required string attribute}, must be kept
  empty.
  %
\end{itemize}
\vspace{-5mm}

Within the \xmlNode{EnsembleModel} XML node, the multiple Models that constitute
this EnsembleModel needs to be inputted. Each Model is specified within a \xmlNode{Model} block (\nb each model
here specified need to be inputted in the\xmlNode{Models} main XML block) :
\begin{itemize}
  \item \xmlNode{Model}, \xmlDesc{XML node, required parameter}.
  %
  The text portion of this node needs to contain the name of the Model
  %
  \\This XML node needs to contain the attributes:

\vspace{-5mm}
\begin{itemize}
  \itemsep0em
  \item \xmlAttr{class}, \xmlDesc{required string attribute}, the class of this sub-model (e.g. Models)
  %
  \item \xmlAttr{type}, \xmlDesc{required string attribute}, the sub-type of this Model (e.g. ExternalModel, ROM, Code)
  %
\end{itemize}
\vspace{-5mm}

  %
  In addition the following XML sub-nodes need to be inputted (or optionally inputted):
  \begin{itemize}
     \item \xmlNode{TargetEvaluation}, \xmlDesc{string, required field},
        represents the container where the output of this Model are stored.
        %
        From a practical point of view, this XML node must contain the name of
        a data object defined in the \xmlNode{DataObjects} block (see
        Section~\ref{sec:DataObjects}).
        %
        Currently, the  \xmlNode{EnsembleModel} accept all \textbf{\textit{DataObjects'}}  types:
        \textbf{PointSet},  \textbf{HistorySet} and \textbf{DataSet}
        \nb The  \xmlNode{TargetEvaluation} is primary used for input-output identification. If the linked
        DataObject is not placed as additional output of the Step where the EnsembleModel is used, it will
        not be filled with the data coming from the calculation and it will be kept empty.
     \item \xmlNode{Input}, \xmlDesc{string, required field},
        represents the input entities that need to be passed to this sub-model
        %
        The user can specify as many \xmlNode{Input} as required by the sub-model.
        \nb All the inputs here specified need to be listed in the Steps where the EnsembleModel
        is used.
     \item \xmlNode{Output}, \xmlDesc{string, optional field},
        represents the output entities that need to be linked to this sub-model.  \nb The \xmlNode{Output}s here
        specified are not part
        of the determination of the EnsembleModel execution but represent an additional storage of results from the
        sub-models. For example, if the \xmlNode{TargetEvaluation} is of type PointSet (since just scalar data needs to
        be transferred to other
        models) and the sub-model is able to also output history-type data, this Output can be of type HistorySet.
        Note that the structure of each Output dataObject must include only variables (either input or output) that are
        defined among the model.
        As an example, the Output dataObjects cannot contained variables that are defined at the Ensemble model
        level.
        %
        The user can specify as many \xmlNode{Output} (s) as needed. The optional \xmlNode{Output}s  can be of
        both classes ``DataObjects'' and ``Databases''
        (e.g. \textit{PointSet}, \textit{HistorySet}, \textit{DataSet}, \textit{HDF5})
        \nb \textbf{The \xmlNode{Output} (s) here specified MUST be listed in the Step in which the EnsembleModel is used.}
    \end{itemize}
  %
\end{itemize}


It is important to notice that when the EnsembleModel detects a chain of models that evolve in a non-linear system, a Picard's Iteration scheme is activated. In this case, an additional XML sub-node within the main \xmlNode{EnsembleModel} XML node needs to be specified:
\begin{itemize}
  \item \xmlNode{settings}, \xmlDesc{XML node, required parameter (if Picard's activated)}.
  %
  The body of this sub-node  contains the following XML sub-nodes:
  %
  \begin{itemize}
     \item \xmlNode{maxIterations}, \xmlDesc{integer, optional field},
        maximum number of Picard's iteration to be performed (in case the iteration scheme does
        not previously converge). \default{30};
     \item \xmlNode{tolerance}, \xmlDesc{float, optional field},
        convergence criterion. It represents the L2 norm residue below which the Picard's iterative scheme is
        considered converged. \default{0.001};
     \item \xmlNode{initialConditions}, \xmlDesc{XML node, required parameter  (if Picard's activated)},
        Within this sub-node, the initial conditions for the input variables (that are part of a loop)  need to
        be specified in sub-nodes named with the variable name (e.g. \xmlNode{varName}). The body of the
        \xmlNode{varName} contains the value of the initial conditions (scalar or arrays, depending of the
        type of variable). If an array needs to be inputted, the user can specify the attribute  \xmlAttr{repeat}
        and the code is going to repeat for  \xmlAttr{repeat}-times the value inputted in the body.
     \item \xmlNode{initialStartModels}, \xmlDesc{XML node, only required parameter when Picard's iteration is
     activated},
        specifies the list of models that will be initially executed. \nb Do not input this node for non-Picard calculations,
        otherwise an error will be raised.
  \end{itemize}
\end{itemize}

\nb \textcolor{red} { \textbf{ It is crucial to understand that the choice of the \xmlNode{DataObject} used as
 \newline \xmlNode{TargetEvaluation} determines how the data are going to be transferred from a model to
  the other. If for example the chain of models is $A \rightarrow B$:}}
\begin{itemize}
  \item \textcolor{red} { \textbf{ If model $B$ expects as input scalars and outputs time-series, the \xmlNode{TargetEvaluation}
  of the  model $B$ will be a \textit{HistorySet} and the  \xmlNode{TargetEvaluation} of the model $A$ will be either
  a \textit{PointSet} or a \textit{DataSet} (where the output variables that need to be transferred to the model $A$ are scalars) }    }
   \item \textcolor{red} { \textbf{ If model $B$ expects as input scalars and time-series and outputs time-series or scalars or both, the \xmlNode{TargetEvaluation}
  of the  model $B$ will be a \textit{DataSet} and the \newline  \xmlNode{TargetEvaluation} of the model $A$ will be either
  a \textit{HistorySet} or a \textit{DataSet}  }    }
  \item \textcolor{red} { \textbf{ If both model $A$ and $B$ expect as input scalars and output scalars, the \xmlNode{TargetEvaluation}
  of the  both models  $A$  and $B$ will be  \textit{PointSet}s  }  }
\end{itemize}

\textbf{Example (Linear System):}
\begin{lstlisting}[style=XML,morekeywords={subType,debug,name,class,type}]
<Simulation>
  ...
  <Models>
    ...
    <EnsembleModel name="heatTransferEnsembleModel" subType="">
      <Model class="Models" type="ExternalModel">
        thermalConductivityComputation
        <TargetEvaluation class="DataObjects" type="PointSet">
          thermalConductivityComputationContainer
        </TargetEvaluation>
        <Input class="DataObjects" type="PointSet">
          inputHolder
        </Input>
      </Model>
      <Model class="Models" type="ExternalModel" >
          heatTransfer
          <TargetEvaluation class="DataObjects" type="PointSet">
            heatTransferContainer
          </TargetEvaluation>
        <Input class="DataObjects" type="PointSet">
          inputHolder
        </Input>
        <Output class="DataObjects" type="HistorySet">
          thisModelLinkedOutput
        </Output>
        <Output class="Databases" type="HDF5">
          thisModelLinkedHDF5
        </Output>
      </Model>
    </EnsembleModel>
    ...
  </Models>
  ...
</Simulation>
\end{lstlisting}

\textbf{Example (Non-Linear System):}
\begin{lstlisting}[style=XML,morekeywords={subType,debug,repeat,name,class,type}]
<Simulation>
  ...
  <Models>
    ...
    <EnsembleModel name="heatTransferEnsembleModel" subType="">
      <settings>
        <maxIterations>8</maxIterations>
         <tolerance>0.01</tolerance>
         <initialConditions>
           <!-- the value 0.7 is going to be repeated 10 times in order to create  an array for var1 -->
           <var1 repeat="10">0.7</var1>
           <!-- an array for var2 has been inputted -->
           <var2> 0.5 0.3 0.4</var2>
           <!-- a scalar for var3 has been inputted -->
           <var3> 45.0</var3>
         </initialConditions>
      </settings>

      <Model class="Models" type="ExternalModel">
        thermalConductivityComputation
        <TargetEvaluation class="DataObjects" type="PointSet">
          thermalConductivityComputationContainer
        </TargetEvaluation>
        <Input class="DataObjects" type="PointSet">
          inputHolder
        </Input>
      </Model>
      <Model class="Models" type="ExternalModel" >
          heatTransfer
          <TargetEvaluation class="DataObjects" type="PointSet">
            heatTransferContainer
          </TargetEvaluation>
        <Input class="DataObjects" type="PointSet">
          inputHolder
        </Input>
      </Model>
    </EnsembleModel>
    ...
  </Models>
  ...
</Simulation>
\end{lstlisting}


%
%%%%%%%%%%%%%%%%%%%%%%%%%%%%%%%%%%%%%
%%%%%%  HybridModel  Model   %%%%%%
%%%%%%%%%%%%%%%%%%%%%%%%%%%%%%%%%%%%%
%

\subsection{HybridModel}
\label{subsec:models_HybridModel}
The \textbf{HybridModel} is a new \textit{Model} entity. This new Model is able to combine reduced order model
(ROMs) and any other high-fidelity Model (i.e. Code, ExternalModel). The ROMs will be trained based on the results
from the high-fidelity model. The accuracy of the ROMs will be evaluated based on the cross validation scores,
and the validity of the ROMs will be determined via some local validation metrics (\nb currently only one metric
is available, i.e. CrowdingDistance). After these ROMs are trained, the \textbf{HybridModel} can decide which of
the Model (i.e the ROMs or high-fidelity model) to be executed based on the accuracy and validity of the ROMs.

Currently this model is only able to share information (i.e. data) using \textbf{PointSet}.

The specifications of a HybridModel must be defined within the XML block
\xmlNode{HybridModel}.
%
This XML node needs to contain the attributes:

\vspace{-5mm}
\begin{itemize}
  \itemsep0em
  \item \xmlAttr{name}, \xmlDesc{required string attribute}, user-defined name
  of this HybridModel.
  %
  \nb As with the other objects, this is the name that can be used to refer to
  this specific entity from other input blocks in the XML.
  \item \xmlAttr{subType}, \xmlDesc{required string attribute}, must be kept
  empty.
  %
\end{itemize}
\vspace{-5mm}

Within the \xmlNode{HybridModel} XML node, the multiple entities that constitute
this HybridModel needs to be inputted.

\begin{itemize}
  \item \xmlNode{Model}, \xmlDesc{XML node, required parameter}.
  %
  The text portion of this node needs to contain the name of the Model
  %
  \assemblerAttrDescription{Model}
  %
  \item \xmlNode{ROM}, \xmlDesc{XML node, required parameter}.
  %
  The text portion of this node needs to contain the name of the ROM
  The user can specify as many \xmlNode{ROM} as required by the \xmlNode{Model}.
  \nb The outputs of each ROM should be different, and the total set of ROMs' outputs
  should be the same as the set of \textit{Model's} outputs.
  %
  \assemblerAttrDescription{Model}
  %
  \item \xmlNode{CV}, \xmlDesc{XML node, required parameter}.
  %
    The text portion of this node needs to contain the name of the \xmlNode{PostProcessor} with \xmlAttr{subType}
    ``CrossValidation``.
  %
    \assemblerAttrDescription{Model}
  %
  \item \xmlNode{TargetEvaluation}, \xmlDesc{XML node, required parameter}.
  %
    The text portion of this node needs to contain the name of a data object defined in the \xmlNode{DataObjects} block.
    \nb currently only accept data object with type ``PointSet``. The \xmlNode{TargetEvaluation} is primary used for
    training ROMs. \nb The linked DataObject should be placed as additional output of the Step where the
    \textbf{HybridModel} is used.
  %
    \assemblerAttrDescription{DataObjects}
  %
\end{itemize}

An additional XML sub-node within the main \xmlNode{HybridModel} XML node needs to be specified:
\begin{itemize}
  \item \xmlNode{settings}, \xmlDesc{XML node, optional parameter}.
  %
  The body of this sub-node  contains the following XML sub-nodes:
  %
  \begin{itemize}
     \item \xmlNode{minInitialTrainSize}, \xmlDesc{integer, optional field}, the minimum initial number of high-fidelity
       model runs before starting train the ROMs.
       \default{10};
     \item \xmlNode{tolerance}, \xmlDesc{float, optional field}, ROMs convergence criterion indicates the displacement
       from the optimum results of cross validation. In other words, small tolerance indicates tight convergence criterion
       of the ROMs, while large tolerance indicates loose convergence criterion of the ROMs.
       \nb Currently, this tolerance can be only used for cross validations with SKL Metrics: \textit{explained\_variance\_score},
       \textit{r2\_score}, \textit{median\_absolute\_error}, \textit{mean\_squared\_error} and \textit{mean\_absolute\_error}.
       \default{0.01};
     \item \xmlNode{maxTrainSize}, \xmlDesc{XML node, optional field}, the maximum size of training set of ROMs.
       \default{1.0E6}
  \end{itemize}
  \item \xmlNode{validationMethod}, \xmlDesc{XML node, optional parameter}.
  %
  The validity methods that are used to determine which model to run (i.e. ROMs or high-fidelity Model).
  This XML node needs to contain the attributes:
  %
  \begin{itemize}
    \itemsep0em
    \item \xmlAttr{name}, \xmlDesc{required string attribute}, user-defined name
      of this \xmlNode{validationMethod}.
      \nb Currently, only one method is available, ie. ``CrowdingDistance``.
  \end{itemize}
  %
  The body of this sub-node  contains the following XML sub-nodes:
  %
  \begin{itemize}
     \item \xmlNode{threshold}, \xmlDesc{XML node, required field}, the threshold that is used for ``CrowdingDistance`` method.
  \end{itemize}
\end{itemize}


\textbf{Example (ExternalModel):}
\begin{lstlisting}[style=XML,morekeywords={subType,debug,name,class,type}]
<Simulation>
  ...
  <Metrics>
    <SKL name="m1">
      <metricType>mean_absolute_error</metricType>
    </SKL>
  </Metrics>

  <Models>
    <ExternalModel ModuleToLoad="EM2linear" name="thermalConductivityComputation" subType="">
      <variables>leftTemperature,rightTemperature,k,averageTemperature</variables>
    </ExternalModel>
    <ROM name="knr" subType="SciKitLearn">
      <SKLtype>neighbors|KNeighborsRegressor</SKLtype>
      <Features>leftTemperature, rightTemperature</Features>
      <Target>k</Target>
      <n_neighbors>5</n_neighbors>
      <weights>uniform</weights>
      <algorithm>auto</algorithm>
      <leaf_size>30</leaf_size>
      <metric>minkowski</metric>
      <p>2</p>
    </ROM>
    <PostProcessor name="pp1" subType="CrossValidation">
        <SciKitLearn>
            <SKLtype>KFold</SKLtype>
            <n_splits>10</n_splits>
            <shuffle>False</shuffle>
        </SciKitLearn>
        <Metric class="Metrics" type="SKL">m1</Metric>
    </PostProcessor>
    <HybridModel name="hybrid" subType="">
        <Model class="Models" type="ExternalModel">thermalConductivityComputation</Model>
        <ROM class="Models" type="ROM">knr</ROM>
        <TargetEvaluation class="DataObjects" type="PointSet">thermalConductivityComputationContainer</TargetEvaluation>
        <CV class="Models" type="PostProcessor">pp1</CV>
        <settings>
            <tolerance>0.01</tolerance>
            <trainStep>1</trainStep>
            <maxTrainSize>1000</maxTrainSize>
            <initialTrainSize>10</initialTrainSize>
        </settings>
        <validationMethod name="CrowdingDistance">
            <threshold>0.2</threshold>
        </validationMethod>
    </HybridModel>
  </Models>
  ...
</Simulation>

\end{lstlisting}

\textbf{Example (Code):}
\begin{lstlisting}[style=XML,morekeywords={subType,debug,repeat,name,class,type}]
<Simulation>
  ...
  <Metrics>
    <SKL name="m1">
      <metricType>mean_absolute_error</metricType>
    </SKL>
  </Metrics>

  <Models>
    <Code name="poly" subType="GenericCode">
      <executable>runCode/poly_inp_io.py</executable>
      <clargs arg="python" type="prepend"/>
      <clargs arg="-i" extension=".one" type="input"/>
      <fileargs arg="aux" extension=".two" type="input"/>
      <fileargs arg="output" type="output"/>
      <prepend>python</prepend>
    </Code>
    <ROM name="knr" subType="SciKitLearn">
      <SKLtype>neighbors|KNeighborsRegressor</SKLtype>
      <Features>x, y</Features>
      <Target>poly</Target>
      <n_neighbors>5</n_neighbors>
      <weights>uniform</weights>
      <algorithm>auto</algorithm>
      <leaf_size>30</leaf_size>
      <metric>minkowski</metric>
      <p>2</p>
    </ROM>
    <PostProcessor name="pp1" subType="CrossValidation">
        <SciKitLearn>
            <SKLtype>KFold</SKLtype>
            <n_splits>10</n_splits>
            <shuffle>False</shuffle>
        </SciKitLearn>
        <Metric class="Metrics" type="SKL">m1</Metric>
    </PostProcessor>
    <HybridModel name="hybrid" subType="">
        <Model class="Models" type="Code">poly</Model>
        <ROM class="Models" type="ROM">knr</ROM>
        <TargetEvaluation class="DataObjects" type="PointSet">samples</TargetEvaluation>
        <CV class="Models" type="PostProcessor">pp1</CV>
        <settings>
            <tolerance>0.1</tolerance>
            <trainStep>1</trainStep>
            <maxTrainSize>1000</maxTrainSize>
            <initialTrainSize>10</initialTrainSize>
        </settings>
        <validationMethod name="CrowdingDistance">
            <threshold>0.2</threshold>
        </validationMethod>
    </HybridModel>
  </Models>
  ...
  <Steps>
    <MultiRun name="hybridModelCode">
      <Input class="Files" type="">gen.one</Input>
      <Input class="Files" type="">gen.two</Input>
      <Input class="DataObjects" type="PointSet">inputHolder</Input>
      <Model class="Models" type="HybridModel">hybrid</Model>
      <Sampler class="Samplers" type="Stratified">LHS</Sampler>
      <Output class="DataObjects" type="PointSet">samples</Output>
      <Output class="OutStreams" type="Print">samples</Output>
    </MultiRun>
  </Steps>
  ...
</Simulation>
\end{lstlisting}
%
\nb For this example, the user needs to provide all the inputs for the \textbf{HybridModel}, i.e. Files for the
\textbf{Code} and DataObject for the \textbf{ROM} defined in the \textbf{HybridModel}.

%%%%%%%%%%%%%%%%%%%%%%%%%%%%%%%%%%%%%%%%%%%%%%%%%%%%%
%%%%%            Logical Model                %%%%%%%
%%%%%%%%%%%%%%%%%%%%%%%%%%%%%%%%%%%%%%%%%%%%%%%%%%%%%
\input{logical_model.tex}
