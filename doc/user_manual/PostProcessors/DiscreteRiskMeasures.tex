\subsubsection{Discrete Risk Measures}
\label{DiscreteRiskMeasures}
This Post-Processor calculates a series of risk importance measures from a PointSet.
This calculation is performed for a set of input parameters given an output target.

The user is required to provide the following information:
\begin{itemize}
   \item the set of input variables. For each variable the following need to be specified:
     \begin{itemize}
       \item the set of values that imply a reliability value equal to $1$ for the input variable
       \item the set of values that imply a reliability value equal to $0$ for the input variable
     \end{itemize}
   \item the output target variable. For this variable it is needed to specify the values of
      the output target variable that defines the desired outcome.
\end{itemize}

The following variables are first determined for each input variable $i$:
\begin{itemize}
   \item $R_0$ Probability of the outcome of the output target variable (nominal value)
   \item $R^{+}_i$ Probability of the outcome of the output target variable if reliability of the input variable is equal to $0$
   \item $R^{-}_i$ Probability of the outcome of the output target variable if reliability of the input variable is equal to $1$
\end{itemize}

Available measures are:
\begin{itemize}
   \item Risk Achievement Worth (RAW): $RAW = R^{+}_i / R_0 $
   \item Risk Achievement Worth (RRW): $RRW = R_0 / R^{-}_i$
   \item Fussell-Vesely (FV): $FV = (R_0 - R^{-}_i) / R_0$
   \item Birnbaum (B): $B = R^{+}_i - R^{-}_i$
\end{itemize}

\ppType{RiskMeasureDiscrete}{RiskMeasureDiscrete}

In the \xmlNode{PostProcessor} input block, the following XML sub-nodes are required,
independent of the \xmlAttr{subType} specified:

\begin{itemize}
   \item \xmlNode{measures}, \xmlDesc{string, required field}, desired risk importance measures
      that have to be computed (RRW, RAW, FV, B)
   \item \xmlNode{variable}, \xmlDesc{string, required field}, ID of the input variable. This
      node is provided for each input variable. This nodes needs to contain also these attributes:
     \begin{itemize}
       \item \xmlAttr{R0values}, \xmlDesc{float, required field}, interval of values (comma separated values)
          that implies a reliability value equal to $0$ for the input variable
       \item \xmlAttr{R1values}, \xmlDesc{float, required field}, interval of values (comma separated values)
          that implies a reliability value equal to $1$ for the input variable
     \end{itemize}
   \item \xmlNode{target}, \xmlDesc{string, required field}, ID of the output variable. This nodes needs to
      contain also the attribute \xmlAttr{values}, \xmlDesc{string, required field}, interval of
      values of the output target variable that defines the desired outcome
\end{itemize}

\textbf{Example:}
This example shows an example where it is desired to calculate all available risk importance
measures for two input variables (i.e., pumpTime and valveTime)
given an output target variable (i.e., Tmax).
A value of the input variable pumpTime in the interval $[0,240]$ implies a reliability
value of the input variable pumpTime equal to $0$.
A value of the input variable valveTime in the interval $[0,60]$ implies a reliability
value of the input variable valveTime equal to $0$.
A value of the input variables valveTime and pumpTime in the interval $[1441,2880]$ implies a
reliability value of the input variables equal to $1$.
The desired outcome of the output variable Tmax occurs in the interval $[2200,2500]$.
\begin{lstlisting}[style=XML,morekeywords={subType,debug,name,class,type}]
<Simulation>
  ...
  <Models>
    ...
    <PostProcessor name="riskMeasuresDiscrete" subType="RiskMeasuresDiscrete">
      <measures>B,FV,RAW,RRW</measures>
      <variable R0values='0,240' R1values='1441,2880'>pumpTime</variable>
      <variable R0values='0,60'  R1values='1441,2880'>valveTime</variable>
      <target   values='2200,2500'                  >Tmax</target>
    </PostProcessor>
    ...
  </Models>
  ...
</Simulation>
\end{lstlisting}

This Post-Processor allows the user to consider also multiple datasets (a data set for each initiating event)
and calculate the global risk importance measures.
This can be performed by:
\begin{itemize}
  \item Including all datasets in the step
\begin{lstlisting}[style=XML,morekeywords={subType,debug,name,class,type}]
<Simulation>
  ...
  </Steps>
    ...
    <PostProcess name="PP">
      <Input   class="DataObjects"  type="PointSet"        >outRun1</Input>
      <Input   class="DataObjects"  type="PointSet"        >outRun2</Input>
      <Model   class="Models"       type="PostProcessor"   >riskMeasuresDiscrete</Model>
      <Output  class="DataObjects"  type="PointSet"        >outPPS</Output>
      <Output  class="OutStreams"   type="Print"           >PrintPPS_dump</Output>
    </PostProcess>
  </Steps>
  ...
</Simulation>
\end{lstlisting}
  \item Adding in the Post-processor the frequency of the initiating event associated to each dataset
\begin{lstlisting}[style=XML,morekeywords={subType,debug,name,class,type}]
<Simulation>
  ...
  <Models>
    ...
    <PostProcessor name="riskMeasuresDiscrete" subType="RiskMeasuresDiscrete">
      <measures>FV,RAW</measures>
      <variable R1values='-0.1,0.1' R0values='0.9,1.1'>Astatus</variable>
      <variable R1values='-0.1,0.1' R0values='0.9,1.1'>Bstatus</variable>
      <variable R1values='-0.1,0.1' R0values='0.9,1.1'>Cstatus</variable>
      <variable R1values='-0.1,0.1' R0values='0.9,1.1'>Dstatus</variable>
      <target   values='0.9,1.1'>outcome</target>
      <data     freq='0.01'>outRun1</data>
      <data     freq='0.02'>outRun2</data>
    </PostProcessor>
    ...
  </Models>
  ...
</Simulation>
\end{lstlisting}

\end{itemize}

This post-processor can be made time dependent if a single HistorySet is provided among the other data objects.
The HistorySet contains the temporal profiles of a subset of the input variables. This temporal profile can be only
boolean, i.e., 0 (component offline) or 1 (component online).
Note that the provided history set must contains a single History; multiple Histories are not allowed.
When this post-processor is in a dynamic configuration (i.e., time-dependent), the user is required to specify an xml
node \xmlNode{temporalID} that indicates the ID of the temporal variable.
For each time instant, this post-processor determines the temporal profiles of the desired risk importance measures.
Thus, in this case, an HistorySet must be chosen as an output data object.
An example is shown below:
\begin{lstlisting}[style=XML,morekeywords={subType,debug,name,class,type}]
<Simulation>
  ...
  <Models>
    ...
    <PostProcessor name="riskMeasuresDiscrete" subType="RiskMeasuresDiscrete">
      <measures>B,FV,RAW,RRW,R0</measures>
      <variable R1values='-0.1,0.1' R0values='0.9,1.1'>Astatus</variable>
      <variable R1values='-0.1,0.1' R0values='0.9,1.1'>Bstatus</variable>
      <variable R1values='-0.1,0.1' R0values='0.9,1.1'>Cstatus</variable>
      <target   values='0.9,1.1'>outcome</target>
      <data     freq='1.0'>outRun1</data>
      <temporalID>time</temporalID>
    </PostProcessor>
    ...
  </Models>
  ...
  <Steps>
    ...
    <PostProcess name="PP">
      <Input     class="DataObjects"  type="PointSet"        >outRun1</Input>
      <Input     class="DataObjects"  type="HistorySet"      >timeDepProfiles</Input>
      <Model     class="Models"       type="PostProcessor"   >riskMeasuresDiscrete</Model>
      <Output    class="DataObjects"  type="HistorySet"      >outHS</Output>
      <Output    class="OutStreams"   type="Print"           >PrintHS</Output>
    </PostProcess>
    ...
  </Steps>
  ...
</Simulation>
\end{lstlisting}
